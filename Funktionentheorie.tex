\documentclass[11pt,titlepage]{article}
\usepackage{amsmath,amssymb,amstext,mathtools,amsthm}
\usepackage{amssymb}
\usepackage{xcolor}
\usepackage[utf8]{inputenc}
\usepackage[ngerman]{babel}
\usepackage[paper=a4paper,left=25mm,right=25mm,top=25mm,bottom=25mm]{geometry}

\usepackage{dsfont}
\usepackage{xfrac}
\usepackage{tikz}

\usetikzlibrary{positioning}
\usetikzlibrary{arrows}

\theoremstyle{definition}
\newtheorem{theorem}{Satz}[section]
\newtheorem{corollary}[theorem]{Folgerung}
\newtheorem{proposition}[theorem]{Proposition}
\newtheorem{lemma}[theorem]{Lemma}
\newtheorem{definition}[theorem]{Definition}
\newtheorem{example}[theorem]{Beispiel}
\newtheorem*{axiom}{Axiom}
\newtheorem{remark}{Bemerkung}

\theoremstyle{remark}
\newtheorem*{repetition}{Wiederholung}
\newtheorem*{remind}{Erinnerung}

\title{Funktionentheorie}
\author{Jannis Klingler}
\date{\today}

\begin{document}

\maketitle

\section{Holomorphe und analytische Funktionen}

	\subsection{Analytische Funktionen}
	
	\begin{repetition}
		Setze $\mathbb{C}=\mathbb{R}^2$. Für $z=(x,y)$, $w=(u,v)$ definiere:
		\begin{eqnarray*}
			z+w&=&(x+u,y+v) \quad\text{Vektoraddition} \\
			z\cdot w&=&(x\cdot u-y\cdot v,x\cdot v+y\cdot u) \\
			0&=&(0,0) \qquad\text{neutrales Element $(+)$}\\
			1&=&(1,1) \qquad\text{neutrales Element $(\cdot )$}\\
			i&=&(0,1)
		\end{eqnarray*}
		Komplexe Konjugation: $z\to \overline{z} =(x,-y)$ ist ein Automorphismus, dh.
		\begin{eqnarray*}
			\overline{z+w} &=& \overline{z} +\overline{w} \\
			\overline{z\cdot w} &=& \overline{z} \cdot \overline{w} \\
			\overline{0} &=& 0\\
			\overline{1} &=& 1 \\
			\overline{i} &=& (0,1)
		\end{eqnarray*}
		Mit diesen Operationen ist $\mathbb{C}$ ein Körper.
		\begin{eqnarray*}
			-z=(-x,-y) \qquad \qquad \frac{1}{z}=\frac{\overline{z}}{z\cdot \overline{z}}=\left(
			\frac{x}{x^2 +y^2}-\frac{y}{x^2 +y^2} \right)
		\end{eqnarray*}
		wir definieren einen Absolutbetrag $|z| = \sqrt{z\overline{z}}\in \mathbb{R}$, denn 
		$z\cdot\overline{z} \in \mathbb{R} =\{ z\in \mathbb{C} \mid z=\overline{z} \} =
		\{ (x,0)\mid x\in\mathbb{R} \} \subset \mathbb{C}$ \\
		Jetzt können wir schreiben $z=(x,y)=(x,0)+(y,0)=(x,0)+i\cdot (y,0)=x+iy$ \\
		Graphische Darstellung ("Gaußsche Zahlenebene").
	\end{repetition}
	
	\textbf{\underline{Zur Erinnerung}:}
	
	\begin{definition}[Topologischer Raum]
		Ein topologischer Raum heißt zusammenhängend, wenn er nicht als disjunkte Vereinigung zweier 
		nichtleerer, offener Teilmengen geschrieben werden kann.
	\end{definition}
	
	\begin{definition}[Wegzusammenhängend]
		Ein topologischer Raum $X$ heißt wegzusammenhängend, wenn es zu je zwei Punkten 
		$p,q\in X$ eine stetige Abbildung $\gamma :[0,1]\to X$ mit $\gamma (0)=p$, $\gamma (1)=q$ 
		gibt.
	\end{definition}
	
	\begin{theorem}
		Eine offene Teilmenge von $\mathbb{C}$ ist genau dann zusammenhängend, wenn sie 
		wegzusammenhängend ist.
	\end{theorem}
	\begin{proof}
		$"\Leftarrow"$: Sei $X$ wegzusammenhängend. Seien $U,V\subset X$ offen, $X=U\cup V$, 
		$p\in U$, $q\in V$ (also $U,V$ nicht leer). Dann existiert $\gamma :[0,1]\to X$ stetig mit 
		$\gamma (0)=p$, $\gamma (1)=q$. Dann sind $\gamma ^{-1}(U),\ \gamma^{-1}(V)\subset 
    		[0,1]$ offen. Da $[0,1]$ zusammenhängend ist und $0\in \gamma^{-1}(U)$, 
		$1\in \gamma^{-1}(V)$, \\
		$\gamma^{-1}(U)\cup\gamma^{-1}(V)=\gamma^{-1}(U\cup V)=\gamma^{-1}(X)=[0,1]$ folgt 
		$\gamma^{-1}(U)\cap \gamma^{-1}(V) \neq \emptyset$. \\Also existiert $t\in \gamma^{-1}(U)
		\cap\gamma^{-1}(V)$ und $\gamma(t)\in U\cap V$. Da das für alle offenen, nichtleeren 
		Teilmengen $U,V$ mit $U\cup V=X$ gilt, ist X zusammenhängend.\\
		Einfacher: \\
		Angenommen $X$ ist nicht zusammenhängend. Dann existieren offene, nicht-leere Teilmengen 
		$U,V\subset X$ mit $U\cup V=X$, $U\cap V=\emptyset$. Dann existiert eine stetige Funktion 
		$f:X\to \mathbb{R}$ mit 
		\[ f(x)= \begin{dcases} 0 & x\in U \\ 1& x\in V \end{dcases} \]
		Wähle jetzt $p\in U$, $q\in V$. Gäbe es einen Weg $\gamma : [0,1]\to X$ mit $\gamma(0)=p$, 
		$\gamma(1)=q$, dann wäre $f\circ \gamma :[0,1]\to \mathbb{R}$ stetig, im Widerspruch zum 
		Zwischenwertsatz. \\
		$"\Rightarrow"$: Sei $X\subset \mathbb{C}$ (offen) zusammenhängend. \\Sei $p\in X$ und sei 
		$U=\{ q\in X \mid \exists \gamma :[0,1]\to X \text{ stetig}:\gamma(0)=p,\ \gamma(1)=q \}$\\
		Behauptung: $U$ ist offen, also existiert $\varepsilon>0$, sd. $B_{\varepsilon}(q)\subset X$. 
		Sei $q'\in B_{\varepsilon}(q)$. Dann existiert $\gamma':[0,1]\to X$, sd. 
		\[ \gamma'(t)= \begin{dcases} \gamma(2t) & 0\leq t\leq \frac{1}{2} \\ (2-2t)q + (2t-1)q' & 
			\frac{1}{2} \leq t\leq 1 \end{dcases} \]
		$\Rightarrow$ $B_{\varepsilon}(q)\subset U$ $\Rightarrow$ $U$ offen.\\
		Behauptung: $X\setminus U$ ist offen: \\
		Sei $q\in X\setminus U$. Da $X$ offen, existiert $\varepsilon >0$ mit 
		$B_{\varepsilon}(q)\subset X$. Wäre $B_{\varepsilon}(q)\cap U \neq \emptyset$, so existiert 
		$q'\in B_{\varepsilon}(q)\cap U$, ein Weg $\gamma$ von $p$ nach $q$ in $X$ und mit einer 
		ähnlichen Konstruktion auch eine Kurve $\gamma'$ von $p$ nach $q$. Also auch 
		$X\setminus U = \emptyset$.\\
		$\Rightarrow$ $X$ ist wegzusammenhängend.	
	\end{proof}
	
	\begin{definition}[Gebiet]
		Ein Gebiet ist eine offene, zusammenhängende Teilmenge von $\mathbb{C}$.
	\end{definition}
	
	\begin{remind}
		Eine (komplexe) Potenzreihe ist ein Ausdruck der Form $R(z)=\sum^{\infty}_{n=0} a_n z^n$ mit 
		$a_n \in \mathbb{C}$ für alle $n$. Sie hat den Konvergenzradius $\rho = \left( limsup_{n\to\infty}
		\sqrt[n]{|a_n |}\right)^{-1} \in [0,\infty]$. Dann:
		\begin{eqnarray*}
			R(z)\text{ konvergiert für alle $z$ mit }|z|< \rho \\
			R(z)\text{ divergiert für alle $z$ mit }|z|> \rho 
		\end{eqnarray*}
		wenn $\rho>0$ ist, heißt $R(z)$ konvergent und $B_{\rho}(0)\subset \mathbb{C}$ der 
		Konvergenzkreis.
	\end{remind}
	
	\begin{definition}[Analytische Funktion]
		Es sei $\Omega\in\mathbb{C}$ ein Gebiet und $f:\Omega\to\mathbb{C}$ eine Abbildung. 
		Dann heißt $f$ eine analytische Funktion (auf $\Omega$), wenn es zu jedem Punkt 
		$z_0 \in \Omega$ eine Potenzreihe $R(z)$ mit Konvergenzradius $\rho>0$ existiert, sd. 
		$f(z)=R(z-z_0 )$ für alle $z\in \Omega\cap B_{\rho}(z_0)$.
	\end{definition}
	
	\begin{example}
		Betrachte die Exponentialreihe
		\begin{eqnarray*}
			e^z = \sum_{n=0}^{\infty} \frac{z^n}{n!} 
		\end{eqnarray*}
		$\limsup \sqrt[n]{|\frac{1}{n!}|} =0 \quad \Rightarrow$ Konvergenzradius ist $\rho=\infty$.
		Mit dem Umordnungssatz zeigt man 
		\begin{eqnarray*}
			e^{z+w} =e^z \cdot e^w
		\end{eqnarray*}
		Da die Exponentialreihe reelle Koeffizienten hat, gilt
		\[ \overline{e^z} =\sum_{n=0}^{\infty} \overline{\left( \frac{z^n}{n!} \right)} = 
		\sum_{n=0}^{\infty} \frac{ \overline{z}^n}{n!} = e^{\overline{z}} \]
		Sei jetzt $z=x+iy$, dann gilt 
		\[e^z = e^x \cdot e^{iy} \]
		und $|e^{iy}|^2 = e^{iy} \cdot \overline{e^{iy}} = e^{iy} \cdot e^{-iy} = e^0 = 1$. \\
		Also definiere $e^{iy}=\cos(y)+i\sin (y)$.\\
		Jetzt kann man komplexe Multiplikation in Polarkoordinaten verstehen. \\
		Schreibe $z=r\cdot e^{i\varphi}$, $w=s\cdot e^{i\varphi}$ dann heißt $r=|z|$ der Absolutbetrag
		 und 
		$\varphi\in \mathbb{R}\setminus 2\pi \mathbb{Z}$ das Argument. \\
		Wir repräsentieren $\varphi$ durch die Funktion $arg:\mathbb{C}^{\times} =\mathbb{C}\setminus 
		\{ 0\} \to (-\pi ,\pi]$. \\
		$z\cdot w= r\cdot e^{i\varphi} \cdot s \cdot e^{i\psi} = (rs)\cdot e^{i(\varphi+\psi)}$.
	\end{example}
	
	\begin{theorem}[Identitätssatz für Potenzreihen]
		Es sei $\Omega \subset \mathbb{C}$ Gebiet und $f:\Omega\to \mathbb{C}$ analytisch. 
		Falls es $z_0 \in \Omega$ und eine Folge $(z_n)_{n\in \mathbb{N}}$ in $\Omega\setminus\{
		z_0\}$ mit $\lim_{n\to \infty} z_n = z_0$ gibt, sd. $f(z_n)=0$ für alle $n$, dann ist $f=0$ konstant.
	\end{theorem}
	
	\begin{corollary}
		Seien $f,g$ zwei analytische Funktionen auf $\Omega$, $z_0$, $(z_n)_{n\in\mathbb{N}}$ wie 
		oben, aber mit $f(z_n)=g(z_n)$ für alle $n$, dann folgt $f=g$ auf ganz $\Omega$.
	\end{corollary}
	
	\begin{definition}
		$f$ heißt analytisch auf $\Omega$, wenn es zu jedem Punkt $z\in \Omega$ eine Umgebung 
		$U\subset\Omega$ von $z$ und eine Potenzreihe $R$ um $z$ gibt, die auf ganz $U$ 
		konvergiert, sd. $R(\omega)=f(\omega)$ für alle $\omega\in\Omega$.
	\end{definition}
	
	\begin{proof}
		Sei zunächst $U$ Umgebung von $z$, auf der $f$ mit einer Potenzreihe $R(z)=\sum_{n=0}^
		{\infty} a_n (z-z_n)$ übereinstimmt. \\
		Ohne Einschränkung sei $z_0 =0$. Da $R$ konvergiert, gilt $\rho >0$, also $\infty > \frac{1}{\rho}
		=\limsup_{n\to\infty} \sqrt[n]{|a_n|}$. Also existiert $n_0\in \mathbb{N}_0$ und $C>\frac{1}{\rho}$, 
		sd. $|a_n|< C^n$ für alle $n\geq n_0$. Da nur endlich viele $n\leq n_0$ existieren, können wir 
		$C$ ggf. etwas größer wählen, sd. $|a_n|<C^n$ für alle $n$. Wir beweisen indirekt, dass alle 
		$a_n =0$ sind, dh. wir nehmen an, es gäbe $n$ mit $a_n \neq 0$. Es sei $n_0$ das kleinste $n$ 
		mit $a_{n_{0}}\neq 0$, dh. $a_n =0$ für $n<n_0$. 
		Wir suchen $r>0$, sd. $|a_n z^{n_0} | > \sum_{n=n_0 +1}^{\infty} |a_n z^n | \left(\geq | \sum_{n=
		n_0 +1}^{\infty} a_n z^n |\right)$ für alle $z\in \mathbb{C}$ mit $0<|z|<r$. Denn dann folgt 
		$R(z)=a_{n_0} z^{n_0} +\sum_{a_n}^{z^n} \neq 0$ für $z$ wie oben, also auch für unendlich 
		viele der Folgenglieder $z_n$ aus unserer Annahme.
		\[ \sum_{n=n_0 +1}^{\infty} |a_n z^n| \leq \sum_{n=n_0 +1}^{\infty} C^n |z^n| \underset{\text{
		geometrische Reihe}}{=} \frac{C^{n+1}|z|^{n+1}}{1-C|z|} \]
		Wir suchen also $r>0$, sd.
		\begin{eqnarray*}
			|a_{n_0}|r^{n_0} > \underbrace{\frac{C^{n+1}|z|^{n+1}}{1-Cr}}_{\text{$>0$, für 
			$r>\frac{1}{C}$}} &\Leftrightarrow & |a_n| (r^{n_0} - Cr^{n_0 +1} ) > C^{n_0 +1} r^{n_0 +1} \\
			& \Leftrightarrow & |a_{n_0}| > r (C^{n_0 +1}+|a_{n_0}|C) \\
			& \Leftrightarrow & r>\frac{|a_{n_0}|}{C^{n_0 +1} + |a_{n_0}|C}
		\end{eqnarray*}
		Jetzt folgt für alle $z$ mit $0< |z|<r$, dass $R(z)\neq 0$ wie gewünscht, Widerspruch! \\
		Also folgt $R=0$ und somit $f|_U =0$.
		Definiere $W=\{ z\in\Omega \mid z\text{ hat Umgebung $U$ mit $f|_U =0$} \}$ \\
		$\Rightarrow$ $W$ ist offen und nichtleer. \\
		Behauptung: $W$ ist auch abgeschlossen. Falls nicht, existiert ein Häufungspunkt $z_0$ von 
		$W$ in $\Omega$ mit $z_0 \in W$. Dann existiert $(z_n)_n$ Folge in $W\setminus \{z_0\}$ mit 
		$\lim_{n\to\infty} z_n =z_0$ und $f(z_n)=0$ für alle $n$. Mit den obigen Argumenten folgt: 
		$z_0$ hat Umgebung $U\subset \Omega$ mit $f|_U =0$, somit $z_0 \in W$. \\
		$W$ offen, abgeschlossen und nichtleer $\Rightarrow$ (da $\Omega$ zusammenhängend ist) 
		$\Omega = W$, also $f=0$.
	\end{proof}
	
	(Proposition im Kurzskript zum Rechnen mit Potenzreihen)...
	
	\subsection{Komplexe Differenzierbarkeit}
	
	\begin{definition}
		Eine $\mathbb{R}$-lineare Abbildung $A:\mathbb{C}\to \mathbb{C}$ heißt $\mathbb{C}$-
		antilinear, wenn
		\[ A(zw)=\overline{z} \cdot A(w) \quad \forall w,z\in\mathbb{C}. \]
		Jede $\mathbb{R}$-lineare Abbildung lässt sich zerlegen als $A=A'+A''$ mit 
		$A'(z)=a'\cdot z$ und $A''(z)=a''\cdot\overline{z}$, dabei heißen $A'$ der Linearteil und $A''$ 
		der Antilinearteil von $A$.\\
		Insbesondere ist $A$ genau dann $\mathbb{C}$-linear, wenn $A''=0$.
	\end{definition}
	
	\begin{proof}
		Setze $A'(z)=\frac{A(z)-i\cdot A(iz)}{2}$, $A''(z)=\frac{A(z)+i\cdot A(iz)}{2}$. Daraus folgt 
		\[ A'(z) + A''(z)=\frac{A(z)-i\cdot A(iz)}{2}+\frac{A(z)+i\cdot A(iz)}{2}=A(z) \]
		\begin{eqnarray*}
			A'((u+iv)\cdot z)&=&\frac{A(uz)+A(ivz)-iA(iuz)-iA(-vz)}{2} \\
			&=& \frac{uA(z)\overbrace{-iviA(iz)}^{=+vA(iz)}-iuA(iz)+ivA(z)}{2} \\
			&=&\frac{(u+iv)(A(z)-iA(iz))}{2} \\
			&=& (u+iv)A'(z)
		\end{eqnarray*}
		Analog dazu ist $A''$ $\mathbb{C}$-antilinear. Es folgt $A'(z)=A'(z\cdot 1)=z\cdot \underbrace{
		A'(1)}_{a'}$,\\$A''(z)=A''(z\cdot 1)=\overline{z}\cdot\underbrace{A''(1)}_{a''}$.
	\end{proof}
	
	\begin{remind}
		Sei $U\subset\mathbb{C}$ offen, $f:U\to\mathbb{C}\sim\mathbb{R}^2$ eine Funktion. $f$ 
		heißt total differenzierbar bei $z_0\in U$, falls eine $\mathbb{R}$-lineare Abbildung 
		$A:\mathbb{C}\to\mathbb{C}$ existiert, sd. \[ \lim_{z\to z_0} \frac{f(z)-f(z_0)-A(z-z_0)}
		{|z-z_0 |}=0. \]
		Dann ist $f$ auch partiell differenzierbar und die partiellen Ableitungen sind gerade die Einträge 
		der reellen $2\times 2$-Matrix $A$.
	\end{remind}
	
	\begin{definition}[Komplexe Differenzierbarkeit]
		Es sei $U\subset \mathbb{C}$ offen. Eine Funktion $f:U\to \mathbb{C}$ heißt komplex 
		differenzierbar bei $z_0\in U$, falls $\lim_{z\to z_0} \frac{f(z)-f(z_0)}{z-z_0}$ existiert. Dieser 
		Grenzwert heißt dann die komplexe Ableitung $f'(z_0)\in\mathbb{C}$. Wenn $f$ auf ganz 
		$U$ differenzierbar ist, heißt $f$ auch holomorph auf $U$.
	\end{definition}
	
	\begin{definition}
		Sei $f:U\to\mathbb{C}$ eine Funktion, $U\subset\mathbb{C}$ offen. Schreibe $f=u+iv$ für 
		Funktionen $u,v:U\to\mathbb{R}$, sowie $z=x+iy$. \\
		Definiere die Wirtinger-Ableitungen 
		\[ \frac{\partial f}{\partial z}=\frac{1}{2}\frac{\partial f}{\partial x} -\frac{i}{2}\frac{\partial f}{\partial y} 
		=\frac{1}{2} \left(\frac{\partial u}{\partial x}+\frac{\partial v}{\partial y}\right)+\frac{i}{2}\left(
		\frac{\partial v}{\partial x}-\frac{\partial u}{\partial y}\right) \]
		\[ \frac{\partial f}{\partial \overline{z}}=\frac{1}{2} \frac{\partial f}{\partial x} +\frac{i}{2}
		\frac{\partial f}{\partial y}=\frac{1}{2}\left( \frac{\partial u}{\partial x}-\frac{\partial v}{\partial y}\right)
		+\frac{i}{2}\left(\frac{\partial v}{\partial x}+\frac{\partial u}{\partial y}\right) \]
	\end{definition}
	
	\begin{example}
		$\frac{\partial z}{\partial z}=1$, $\frac{\partial z}{\partial \overline{z}}=0$, 
		$\frac{\partial\overline{z}}{\partial z}=0$, $\frac{\partial\overline{z}}{\partial\overline{z}}=1$
	\end{example}
	
	\begin{lemma}
		Es sei $U\subset\mathbb{C}$ offen, $f:U\to\mathbb{C}$ eine Funktion, $z_0\in U$. Dann sind 
		äquivalent
		\begin{enumerate}
			\item $f$ ist komplex differenzierbar bei $z_0$
			\item Es existiert eine stetige Funktion $\varphi:U\to\mathbb{C}$ mit $f(z)=f(z_0)+\varphi(z)
			\cdot (z-z_0)$
			\item $f$ ist bei $z_0$ reell, total differenzierbar mit $\mathbb{C}$-linearer Ableitung
			\item $f$ ist bei $z_0$ reell, total differenzierbar und $\frac{\partial f}{\partial \overline{z}}|_
			{z_0}=0$
			\item $f$ ist bei $z_0$ reell, total differenzierbar und es gelten die Cauchy-Riemann-
			Differentialgleichungen: $\frac{\partial u}{\partial x}|_{z_0}=\frac{\partial v}{\partial y}|_{z_0}$ 
			und $\frac{\partial u}{\partial y}|_{z_0}=-\frac{\partial v}{\partial x}|_{z_0}$, wobei wieder 
			$f=u+iv$ gelte.
		\end{enumerate}
		Insbesondere ist $f$ dann auch bei $z_0$ stetig.
	\end{lemma}
\end{document}