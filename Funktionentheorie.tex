\documentclass[11pt,titlepage]{article}
\usepackage{amsmath,amssymb,amstext,mathtools,amsthm}
\usepackage{amssymb}
\usepackage{xcolor}
\usepackage[utf8]{inputenc}
\usepackage[ngerman]{babel}
\usepackage[paper=a4paper,left=25mm,right=25mm,top=25mm,bottom=25mm]{geometry}

\usepackage{dsfont}
\usepackage{xfrac}
\usepackage{tikz}

\usetikzlibrary{positioning}
\usetikzlibrary{arrows}

\theoremstyle{definition}
\newtheorem{theorem}{Satz}[section]
\newtheorem{corollary}[theorem]{Folgerung}
\newtheorem{proposition}[theorem]{Proposition}
\newtheorem{lemma}[theorem]{Lemma}
\newtheorem{definition}[theorem]{Definition}
\newtheorem{example}[theorem]{Beispiel}
\newtheorem*{axiom}{Axiom}
\newtheorem{remark}{Bemerkung}

\theoremstyle{remark}
\newtheorem*{repetition}{Wiederholung}
\newtheorem*{remind}{Erinnerung}

\title{Funktionentheorie}
\author{Jannis Klingler}
\date{\today}

\begin{document}

\maketitle

\section{Holomorphe und analytische Funktionen}

	\subsection{Analytische Funktionen}
	
	\begin{repetition}
		Setze $\mathbb{C}=\mathbb{R}^2$. Für $z=(x,y)$, $w=(u,v)$ definiere:
		\begin{eqnarray*}
			z+w&=&(x+u,y+v) \quad\text{Vektoraddition} \\
			z\cdot w&=&(x\cdot u-y\cdot v,x\cdot v+y\cdot u) \\
			0&=&(0,0) \qquad\text{neutrales Element $(+)$}\\
			1&=&(1,1) \qquad\text{neutrales Element $(\cdot )$}\\
			i&=&(0,1)
		\end{eqnarray*}
		Komplexe Konjugation: $z\to \overline{z} =(x,-y)$ ist ein Automorphismus, dh.
		\begin{eqnarray*}
			\overline{z+w} &=& \overline{z} +\overline{w} \\
			\overline{z\cdot w} &=& \overline{z} \cdot \overline{w} \\
			\overline{0} &=& 0\\
			\overline{1} &=& 1 \\
			\overline{i} &=& (0,1)
		\end{eqnarray*}
		Mit diesen Operationen ist $\mathbb{C}$ ein Körper.
		\begin{eqnarray*}
			-z=(-x,-y) \qquad \qquad \frac{1}{z}=\frac{\overline{z}}{z\cdot \overline{z}}=\left(
			\frac{x}{x^2 +y^2}-\frac{y}{x^2 +y^2} \right)
		\end{eqnarray*}
		wir definieren einen Absolutbetrag $|z| = \sqrt{z\overline{z}}\in \mathbb{R}$, denn 
		$z\cdot\overline{z} \in \mathbb{R} =\{ z\in \mathbb{C} \mid z=\overline{z} \} =
		\{ (x,0)\mid x\in\mathbb{R} \} \subset \mathbb{C}$ \\
		Jetzt können wir schreiben $z=(x,y)=(x,0)+(y,0)=(x,0)+i\cdot (y,0)=x+iy$ \\
		Graphische Darstellung ("Gaußsche Zahlenebene").
	\end{repetition}
	
	\textbf{\underline{Zur Erinnerung}:}
	
	\begin{definition}[Topologischer Raum]
		Ein topologischer Raum heißt zusammenhängend, wenn er nicht als disjunkte Vereinigung zweier 
		nichtleerer, offener Teilmengen geschrieben werden kann.
	\end{definition}
	
	\begin{definition}[Wegzusammenhängend]
		Ein topologischer Raum $X$ heißt wegzusammenhängend, wenn es zu je zwei Punkten 
		$p,q\in X$ eine stetige Abbildung $\gamma :[0,1]\to X$ mit $\gamma (0)=p$, $\gamma (1)=q$ 
		gibt.
	\end{definition}
	
	\begin{theorem}
		Eine offene Teilmenge von $\mathbb{C}$ ist genau dann zusammenhängend, wenn sie 
		wegzusammenhängend ist.
	\end{theorem}
	\begin{proof}
		$"\Leftarrow"$: Sei $X$ wegzusammenhängend. Seien $U,V\subset X$ offen, $X=U\cup V$, 
		$p\in U$, $q\in V$ (also $U,V$ nicht leer). Dann existiert $\gamma :[0,1]\to X$ stetig mit 
		$\gamma (0)=p$, $\gamma (1)=q$. Dann sind $\gamma ^{-1}(U),\ \gamma^{-1}(V)\subset 
    		[0,1]$ offen. Da $[0,1]$ zusammenhängend ist und $0\in \gamma^{-1}(U)$, 
		$1\in \gamma^{-1}(V)$, \\
		$\gamma^{-1}(U)\cup\gamma^{-1}(V)=\gamma^{-1}(U\cup V)=\gamma^{-1}(X)=[0,1]$ folgt 
		$\gamma^{-1}(U)\cap \gamma^{-1}(V) \neq \emptyset$. \\Also existiert $t\in \gamma^{-1}(U)
		\cap\gamma^{-1}(V)$ und $\gamma(t)\in U\cap V$. Da das für alle offenen, nichtleeren 
		Teilmengen $U,V$ mit $U\cup V=X$ gilt, ist X zusammenhängend.\\
		Einfacher: \\
		Angenommen $X$ ist nicht zusammenhängend. Dann existieren offene, nicht-leere Teilmengen 
		$U,V\subset X$ mit $U\cup V=X$, $U\cap V=\emptyset$. Dann existiert eine stetige Funktion 
		$f:X\to \mathbb{R}$ mit 
		\[ f(x)= \begin{dcases} 0 & x\in U \\ 1& x\in V \end{dcases} \]
		Wähle jetzt $p\in U$, $q\in V$. Gäbe es einen Weg $\gamma : [0,1]\to X$ mit $\gamma(0)=p$, 
		$\gamma(1)=q$, dann wäre $f\circ \gamma :[0,1]\to \mathbb{R}$ stetig, im Widerspruch zum 
		Zwischenwertsatz. \\
		$"\Rightarrow"$: Sei $X\subset \mathbb{C}$ (offen) zusammenhängend. \\Sei $p\in X$ und sei 
		$U=\{ q\in X \mid \exists \gamma :[0,1]\to X \text{ stetig}:\gamma(0)=p,\ \gamma(1)=q \}$\\
		Behauptung: $U$ ist offen, also existiert $\varepsilon>0$, sd. $B_{\varepsilon}(q)\subset X$. 
		Sei $q'\in B_{\varepsilon}(q)$. Dann existiert $\gamma':[0,1]\to X$, sd. 
		\[ \gamma'(t)= \begin{dcases} \gamma(2t) & 0\leq t\leq \frac{1}{2} \\ (2-2t)q + (2t-1)q' & 
			\frac{1}{2} \leq t\leq 1 \end{dcases} \]
		$\Rightarrow$ $B_{\varepsilon}(q)\subset U$ $\Rightarrow$ $U$ offen.\\
		Behauptung: $X\setminus U$ ist offen: \\
		Sei $q\in X\setminus U$. Da $X$ offen, existiert $\varepsilon >0$ mit 
		$B_{\varepsilon}(q)\subset X$. Wäre $B_{\varepsilon}(q)\cap U \neq \emptyset$, so existiert 
		$q'\in B_{\varepsilon}(q)\cap U$, ein Weg $\gamma$ von $p$ nach $q$ in $X$ und mit einer 
		ähnlichen Konstruktion auch eine Kurve $\gamma'$ von $p$ nach $q$. Also auch 
		$X\setminus U = \emptyset$.\\
		$\Rightarrow$ $X$ ist wegzusammenhängend.	
	\end{proof}
	
	\begin{definition}[Gebiet]
		Ein Gebiet ist eine offene, zusammenhängende Teilmenge von $\mathbb{C}$.
	\end{definition}
	
	\begin{remind}
		Eine (komplexe) Potenzreihe ist ein Ausdruck der Form $R(z)=\sum^{\infty}_{n=0} a_n z^n$ mit 
		$a_n \in \mathbb{C}$ für alle $n$. Sie hat den Konvergenzradius $\rho = \left( \limsup_{n\to\infty}
		\sqrt[n]{|a_n |}\right)^{-1} \in [0,\infty]$. Dann:
		\begin{eqnarray*}
			R(z)\text{ konvergiert für alle $z$ mit }|z|< \rho \\
			R(z)\text{ divergiert für alle $z$ mit }|z|> \rho 
		\end{eqnarray*}
		wenn $\rho>0$ ist, heißt $R(z)$ konvergent und $B_{\rho}(0)\subset \mathbb{C}$ der 
		Konvergenzkreis.
	\end{remind}
	
	\begin{definition}[Analytische Funktion]
		Es sei $\Omega\in\mathbb{C}$ ein Gebiet und $f:\Omega\to\mathbb{C}$ eine Abbildung. 
		Dann heißt $f$ eine analytische Funktion (auf $\Omega$), wenn es zu jedem Punkt 
		$z_0 \in \Omega$ eine Potenzreihe $R(z)$ mit Konvergenzradius $\rho>0$ existiert, sd. 
		$f(z)=R(z-z_0 )$ für alle $z\in \Omega\cap B_{\rho}(z_0)$.
	\end{definition}
	
	\begin{example}
		Betrachte die Exponentialreihe
		\begin{eqnarray*}
			e^z = \sum_{n=0}^{\infty} \frac{z^n}{n!} 
		\end{eqnarray*}
		$\limsup \sqrt[n]{|\frac{1}{n!}|} =0 \quad \Rightarrow$ Konvergenzradius ist $\rho=\infty$.
		Mit dem Umordnungssatz zeigt man 
		\begin{eqnarray*}
			e^{z+w} =e^z \cdot e^w
		\end{eqnarray*}
		Da die Exponentialreihe reelle Koeffizienten hat, gilt
		\[ \overline{e^z} =\sum_{n=0}^{\infty} \overline{\left( \frac{z^n}{n!} \right)} = 
		\sum_{n=0}^{\infty} \frac{ \overline{z}^n}{n!} = e^{\overline{z}} \]
		Sei jetzt $z=x+iy$, dann gilt 
		\[e^z = e^x \cdot e^{iy} \]
		und $|e^{iy}|^2 = e^{iy} \cdot \overline{e^{iy}} = e^{iy} \cdot e^{-iy} = e^0 = 1$. \\
		Also definiere $e^{iy}=\cos(y)+i\sin (y)$.\\
		Jetzt kann man komplexe Multiplikation in Polarkoordinaten verstehen. \\
		Schreibe $z=r\cdot e^{i\varphi}$, $w=s\cdot e^{i\varphi}$ dann heißt $r=|z|$ der Absolutbetrag
		 und 
		$\varphi\in \mathbb{R}\setminus 2\pi \mathbb{Z}$ das Argument. \\
		Wir repräsentieren $\varphi$ durch die Funktion $arg:\mathbb{C}^{\times} =\mathbb{C}\setminus 
		\{ 0\} \to (-\pi ,\pi]$. \\
		$z\cdot w= r\cdot e^{i\varphi} \cdot s \cdot e^{i\psi} = (rs)\cdot e^{i(\varphi+\psi)}$.
	\end{example}
	
	\begin{theorem}[Identitätssatz für Potenzreihen]
		Es sei $\Omega \subset \mathbb{C}$ Gebiet und $f:\Omega\to \mathbb{C}$ analytisch. 
		Falls es $z_0 \in \Omega$ und eine Folge $(z_n)_{n\in \mathbb{N}}$ in $\Omega\setminus\{
		z_0\}$ mit $\lim_{n\to \infty} z_n = z_0$ gibt, sd. $f(z_n)=0$ für alle $n$, dann ist $f=0$ konstant.
	\end{theorem}
	
	\begin{corollary}
		Seien $f,g$ zwei analytische Funktionen auf $\Omega$, $z_0$, $(z_n)_{n\in\mathbb{N}}$ wie 
		oben, aber mit $f(z_n)=g(z_n)$ für alle $n$, dann folgt $f=g$ auf ganz $\Omega$.
	\end{corollary}
	
	\begin{definition}
		$f$ heißt analytisch auf $\Omega$, wenn es zu jedem Punkt $z\in \Omega$ eine Umgebung 
		$U\subset\Omega$ von $z$ und eine Potenzreihe $R$ um $z$ gibt, die auf ganz $U$ 
		konvergiert, sd. $R(\omega)=f(\omega)$ für alle $\omega\in\Omega$.
	\end{definition}
	
	\begin{proof}
		Sei zunächst $U$ Umgebung von $z$, auf der $f$ mit einer Potenzreihe $R(z)=\sum_{n=0}^
		{\infty} a_n (z-z_n)$ übereinstimmt. \\
		Ohne Einschränkung sei $z_0 =0$. Da $R$ konvergiert, gilt $\rho >0$, also $\infty > \frac{1}{\rho}
		=\limsup_{n\to\infty} \sqrt[n]{|a_n|}$. Also existiert $n_0\in \mathbb{N}_0$ und $C>\frac{1}{\rho}$, 
		sd. $|a_n|< C^n$ für alle $n\geq n_0$. Da nur endlich viele $n\leq n_0$ existieren, können wir 
		$C$ ggf. etwas größer wählen, sd. $|a_n|<C^n$ für alle $n$. Wir beweisen indirekt, dass alle 
		$a_n =0$ sind, dh. wir nehmen an, es gäbe $n$ mit $a_n \neq 0$. Es sei $n_0$ das kleinste $n$ 
		mit $a_{n_{0}}\neq 0$, dh. $a_n =0$ für $n<n_0$. 
		Wir suchen $r>0$, sd. $|a_n z^{n_0} | > \sum_{n=n_0 +1}^{\infty} |a_n z^n | \left(\geq | \sum_{n=
		n_0 +1}^{\infty} a_n z^n |\right)$ für alle $z\in \mathbb{C}$ mit $0<|z|<r$. Denn dann folgt 
		$R(z)=a_{n_0} z^{n_0} +\sum_{a_n}^{z^n} \neq 0$ für $z$ wie oben, also auch für unendlich 
		viele der Folgenglieder $z_n$ aus unserer Annahme.
		\[ \sum_{n=n_0 +1}^{\infty} |a_n z^n| \leq \sum_{n=n_0 +1}^{\infty} C^n |z^n| \underset{\text{
		geometrische Reihe}}{=} \frac{C^{n+1}|z|^{n+1}}{1-C|z|} \]
		Wir suchen also $r>0$, sd.
		\begin{eqnarray*}
			|a_{n_0}|r^{n_0} > \underbrace{\frac{C^{n+1}|z|^{n+1}}{1-Cr}}_{\text{$>0$, für 
			$r>\frac{1}{C}$}} &\Leftrightarrow & |a_n| (r^{n_0} - Cr^{n_0 +1} ) > C^{n_0 +1} r^{n_0 +1} \\
			& \Leftrightarrow & |a_{n_0}| > r (C^{n_0 +1}+|a_{n_0}|C) \\
			& \Leftrightarrow & r>\frac{|a_{n_0}|}{C^{n_0 +1} + |a_{n_0}|C}
		\end{eqnarray*}
		Jetzt folgt für alle $z$ mit $0< |z|<r$, dass $R(z)\neq 0$ wie gewünscht, Widerspruch! \\
		Also folgt $R=0$ und somit $f|_U =0$.
		Definiere $W=\{ z\in\Omega \mid z\text{ hat Umgebung $U$ mit $f|_U =0$} \}$ \\
		$\Rightarrow$ $W$ ist offen und nichtleer. \\
		Behauptung: $W$ ist auch abgeschlossen. Falls nicht, existiert ein Häufungspunkt $z_0$ von 
		$W$ in $\Omega$ mit $z_0 \in W$. Dann existiert $(z_n)_n$ Folge in $W\setminus \{z_0\}$ mit 
		$\lim_{n\to\infty} z_n =z_0$ und $f(z_n)=0$ für alle $n$. Mit den obigen Argumenten folgt: 
		$z_0$ hat Umgebung $U\subset \Omega$ mit $f|_U =0$, somit $z_0 \in W$. \\
		$W$ offen, abgeschlossen und nichtleer $\Rightarrow$ (da $\Omega$ zusammenhängend ist) 
		$\Omega = W$, also $f=0$.
	\end{proof}
	
	(Proposition im Kurzskript zum Rechnen mit Potenzreihen)...
	
	\subsection{Komplexe Differenzierbarkeit}
	
	\begin{definition}
		Eine $\mathbb{R}$-lineare Abbildung $A:\mathbb{C}\to \mathbb{C}$ heißt $\mathbb{C}$-
		antilinear, wenn
		\[ A(zw)=\overline{z} \cdot A(w) \quad \forall w,z\in\mathbb{C}. \]
		Jede $\mathbb{R}$-lineare Abbildung lässt sich zerlegen als $A=A'+A''$ mit 
		$A'(z)=a'\cdot z$ und $A''(z)=a''\cdot\overline{z}$, dabei heißen $A'$ der Linearteil und $A''$ 
		der Antilinearteil von $A$.\\
		Insbesondere ist $A$ genau dann $\mathbb{C}$-linear, wenn $A''=0$.
	\end{definition}
	
	\begin{proof}
		Setze $A'(z)=\frac{A(z)-i\cdot A(iz)}{2}$, $A''(z)=\frac{A(z)+i\cdot A(iz)}{2}$. Daraus folgt 
		\[ A'(z) + A''(z)=\frac{A(z)-i\cdot A(iz)}{2}+\frac{A(z)+i\cdot A(iz)}{2}=A(z) \]
		\begin{eqnarray*}
			A'((u+iv)\cdot z)&=&\frac{A(uz)+A(ivz)-iA(iuz)-iA(-vz)}{2} \\
			&=& \frac{uA(z)\overbrace{-iviA(iz)}^{=+vA(iz)}-iuA(iz)+ivA(z)}{2} \\
			&=&\frac{(u+iv)(A(z)-iA(iz))}{2} \\
			&=& (u+iv)A'(z)
		\end{eqnarray*}
		Analog dazu ist $A''$ $\mathbb{C}$-antilinear. Es folgt $A'(z)=A'(z\cdot 1)=z\cdot \underbrace{
		A'(1)}_{a'}$,\\$A''(z)=A''(z\cdot 1)=\overline{z}\cdot\underbrace{A''(1)}_{a''}$.
	\end{proof}
	
	\begin{repetition}
		Sei $U\subset\mathbb{C}$ offen, $f:U\to\mathbb{C}\sim\mathbb{R}^2$ eine Funktion. $f$ 
		heißt total differenzierbar bei $z_0\in U$, falls eine $\mathbb{R}$-lineare Abbildung 
		$A:\mathbb{C}\to\mathbb{C}$ existiert, sd. \[ \lim_{z\to z_0} \frac{f(z)-f(z_0)-A(z-z_0)}
		{|z-z_0 |}=0. \]
		Dann ist $f$ auch partiell differenzierbar und die partiellen Ableitungen sind gerade die Einträge 
		der reellen $2\times 2$-Matrix $A$.
	\end{repetition}
	
	\begin{definition}[Komplexe Differenzierbarkeit]
		Es sei $U\subset \mathbb{C}$ offen. Eine Funktion $f:U\to \mathbb{C}$ heißt komplex 
		differenzierbar bei $z_0\in U$, falls $\lim_{z\to z_0} \frac{f(z)-f(z_0)}{z-z_0}$ existiert. Dieser 
		Grenzwert heißt dann die komplexe Ableitung $f'(z_0)\in\mathbb{C}$. Wenn $f$ auf ganz 
		$U$ differenzierbar ist, heißt $f$ auch holomorph auf $U$.
	\end{definition}
	
	\begin{definition}
		Sei $f:U\to\mathbb{C}$ eine Funktion, $U\subset\mathbb{C}$ offen. Schreibe $f=u+iv$ für 
		Funktionen $u,v:U\to\mathbb{R}$, sowie $z=x+iy$. \\
		Definiere die Wirtinger-Ableitungen 
		\[ \frac{\partial f}{\partial z}=\frac{1}{2}\frac{\partial f}{\partial x} -\frac{i}{2}\frac{\partial f}{\partial y} 
		=\frac{1}{2} \left(\frac{\partial u}{\partial x}+\frac{\partial v}{\partial y}\right)+\frac{i}{2}\left(
		\frac{\partial v}{\partial x}-\frac{\partial u}{\partial y}\right) \]
		\[ \frac{\partial f}{\partial \overline{z}}=\frac{1}{2} \frac{\partial f}{\partial x} +\frac{i}{2}
		\frac{\partial f}{\partial y}=\frac{1}{2}\left( \frac{\partial u}{\partial x}-\frac{\partial v}{\partial y}\right)
		+\frac{i}{2}\left(\frac{\partial v}{\partial x}+\frac{\partial u}{\partial y}\right) \]
	\end{definition}
	
	\begin{example}
		$\frac{\partial z}{\partial z}=1$, $\frac{\partial z}{\partial \overline{z}}=0$, 
		$\frac{\partial\overline{z}}{\partial z}=0$, $\frac{\partial\overline{z}}{\partial\overline{z}}=1$
	\end{example}
	
	\begin{lemma}[Definition]
		Es sei $U\subset\mathbb{C}$ offen, $f:U\to\mathbb{C}$ eine Funktion, $z_0\in U$. Dann sind 
		äquivalent
		\begin{enumerate}
			\item $f$ ist komplex differenzierbar bei $z_0$
			\item Es existiert eine stetige Funktion $\varphi:U\to\mathbb{C}$ mit $f(z)=f(z_0)+\varphi(z)
			\cdot (z-z_0)$
			\item $f$ ist bei $z_0$ reell, total differenzierbar mit $\mathbb{C}$-linearer Ableitung
			\item $f$ ist bei $z_0$ reell, total differenzierbar und $\frac{\partial f}{\partial \overline{z}}|_
			{z_0}=0$
			\item $f$ ist bei $z_0$ reell, total differenzierbar und es gelten die Cauchy-Riemann-
			Differentialgleichungen (C-R-DGL): 
			$\frac{\partial u}{\partial x}|_{z_0}=\frac{\partial v}{\partial y}|_{z_0}$ 
			und $\frac{\partial u}{\partial y}|_{z_0}=-\frac{\partial v}{\partial x}|_{z_0}$, wobei wieder 
			$f=u+iv$ gelte.
		\end{enumerate}
		Insbesondere ist $f$ dann auch bei $z_0$ stetig.
	\end{lemma}
	
	\begin{proof}
		(1)$\Rightarrow$(2): Setze 
		\[ \varphi(z)=\begin{dcases} \frac{f(z)-f(z_0)}{z-z_0} & z\neq z_0 \\ f'(z_0) & z=z_0 \end{dcases}\]
		Stetigkeit bei $z_0$ folgt aus der komplexen Differenzierbarkeit.\\
		(2)$\Rightarrow$(3): Schreibe
		\begin{eqnarray*}
			\lim_{z\to z_0} \frac{f(z)-f(z_0)-\varphi(z_0)\cdot (z-z_0)}{|z-z_0|} = 
			\lim_{z\to z_0} \underbrace{(\varphi(z)-\varphi(z_0))}_{\to \ 0\text{, da $\varphi$ stetig in 
			$z_0$.}} \underbrace{\frac{z-z_0}{|z-z_0|}}_{\text{beschränkt (Norm $1$)}} =0
		\end{eqnarray*}
		$\Rightarrow$ $f$ ist bei $z_0$ total-reell-differenzierbar. Die Ableitung ist die $\mathbb{C}$-
		lineare Abbildung $\omega\mapsto \varphi(z_0)\cdot \omega$.\\
		(3)$\Rightarrow$(4): Da die reelle Ableitung $\mathbb{C}$-linear ist, folgt 
		$\frac{\partial f}{\partial \overline{z}}(z_0)=0$ ( was nach Definition gerade der Antilinearteil 
		der Ableitung ist)\\
		(4)$\Rightarrow$(5):
		\begin{eqnarray*}
			0=\underbrace{\frac{\partial f}{\partial \overline{z}}(z_0)}_{\in \mathbb{C}} \underset{\text{
			Def. 1.12}}{=} \underbrace{\frac{1}{2} \left( \frac{\partial u}{\partial x} -
			\frac{\partial v}{\partial y} \right)}_{\text{Realteil}}(z_0) + \frac{i}{2} \underbrace{
			\left( \frac{\partial v}{\partial x}+\frac{\partial u}{\partial y}\right)}_{\text{Imaginärteil}}(z_0)
		\end{eqnarray*}
		hieraus lassen sich die C-R-DGL direkt ablesen.\\
		(5)$\Rightarrow$(1): Schreibe $z=z_0 +x+iy$ dann gilt
		\begin{eqnarray*}
			f(z) &=& f(z_0) +\frac{\partial u}{\partial x} x +\frac{\partial u}{\partial y} y +
			 \frac{\partial v}{\partial x} ix + \frac{\partial v}{\partial y} iy +R(x,y) \\
			&\overset{\text{C-R-DGL}}{=}& f(z_0) +\frac{\partial u}{\partial x} (x+iy) -\frac{\partial v}
			{\partial x}(y-ix)+R(x,y) \\
			&=& f(z_0) +\left(\frac{\partial u}{\partial x}+i \frac{\partial v}{\partial x}\right)\cdot
			(x+iy)+R(x,y)
		\end{eqnarray*}
		mit $R(x,y)=o(|(x,y)|)$, das heißt $\lim_{(x,y)\to 0} \frac{R(x,y)}{|(x,y)|}=0$ (der Restterm geht 
		schneller gegen Null als $(x,y)$). Es folgt 
		\begin{eqnarray*}
			\lim_{z\to z_0}\frac{f(z)-f(z_0)}{z-z_0} &=& \lim_{z\to z_0}\frac{\left(\frac{\partial u}{\partial x}
			+i\frac{\partial v}{\partial x}\right)\cdot (z-z_0)+R(x,y)}{z-z_0} \\
			&=&\frac{\partial u}{\partial x}+i\frac{\partial v}{\partial x} +\lim_{z\to z_=}
			\underbrace{\frac{R(x,y)}{|z-z_0|}}_{\to\  0}\cdot \underbrace{\frac{|z-z_0|}{z-z_=}}_{\text{
			beschränkt}}\\
			&=& \frac{\partial u}{\partial x}+i\frac{\partial v}{\partial x} \in\mathbb{C}
		\end{eqnarray*}
	\end{proof}
	
	\begin{example}
		Die komplexe Exponentialfunktion ist holomorph auf ganz $\mathbb{C}$ (Begründung folgt)
	\end{example}
	
	\begin{proposition}
		Es gelten folgende Differentiationsregeln:
		\begin{enumerate}
			\item \underline{Linearität:} Seien $f,g:\Omega\to\mathbb{C}$ holomorph, $a,b\in\mathbb{C}
			$, dann ist $a\cdot f+b\cdot g$ holomorph mit $(a\cdot f+b\cdot g)'(z)=a\cdot f'(z)+b\cdot 
			g'(z)$.
			\item \underline{Kettenregel:} Sei $f:\Omega\to\Omega'$, $g:\Omega'\to\mathbb{C}$ 
			holomorph, dann ist $g\circ f:\Omega\to\mathbb{C}$ holomorph mit 
			$(g\circ f)'(z)=f'(g(z))\cdot g'(z)$.
			\item \underline{Produktregel:} Seien $f,g:\Omega\to\mathbb{C}$ holomorph, dann ist 
			$f\cdot g$ holomorph mit Ableitung $(f\cdot g)'(z)= f'(z)\cdot g(z)+f(z)\cdot g'(z)$.
		\end{enumerate}
	\end{proposition}
	
	\begin{proof}
		(1): Additivität ist klar. Multiplikativität siehe (3) \\
		(2): Übung\\
		(3): Schreibe $f=u+iv$, $g=r+is$, $u,v,r,s:\Omega\to\mathbb{R}$, dann ist 
		$f\cdot g=(u\cdot r-v\cdot s)+i\cdot(u\cdot s+v\cdot r)$. Jetzt setzen wir mit den reellen 
		Produktregeln fort und sind fertig.
	\end{proof}
	
	\begin{theorem}
		Es sei $R(z)=\sum_{n=0}^{\infty} a_n \cdot z^n $ konvergente Potenzreihe mit 
		Konvergenzradius $\rho >0$, dann ist $R(z)$ auf $B_{\rho}(0)$ holomorph mit Ableitung 
		\[R'(z)=\sum_{n=0}^{\infty} a_n \cdot n\cdot z^{n-1} =\sum_{m=0}^{\infty}a_{m+1}(m+1)z^m,\quad
		n-1=m.\]
	\end{theorem}
	
	\begin{proof}
		Siehe Analysis, beruht auf folgendem Satz: Sei $(f_n)_{n\in\mathbb{N}}$ Folge differenzierbarer 
		Funktionen auf $U$, sd. $(f_n)_n$ punktweise und $(f'_n)_n$ lokal-gleichmäßig konvergiert. \\
		$\Rightarrow$ $(\lim_{n\to\infty} f_n)'=\lim_{n\to\infty}f'_n$
	\end{proof}
	
	\subsection{Das komplexe Kurvenintegral}
	
	\begin{definition}
		Eine stückweise $C^1$-Kurve $\gamma:[a,b]\to\mathbb{C}$ ist eine stetige Abbildung, sd. 
		$a=t_0 < t_1 <\ldots<t_n=b$ existieren, für die $\gamma |_{ [t_{i-1},t_i]}\in C^1$ für 
		$i=1,\ldots ,n$. \\
		Für $t\neq t_i$, $t\in [a,b]$, sei $\dot{\gamma}(t)=\frac{\mathrm{d}\gamma}{\mathrm{d}t}(t)$ der 
		Geschwindigkeitsvektor. $\gamma$ heißt geschlossen, wenn \[\gamma(a)=\gamma(b).\]
	\end{definition}
	
	\begin{definition}[Kurvenintegral]
		Sei $\Omega\subset\mathbb{C}$ ein Gebiet, $\gamma:[a,b]\to\Omega$ stückweise $C^1$, sei 
		$f:\Omega\to\mathbb{C}$ stetig. Definiere das komplexe Kurvenintegral
		\[ \int_{\gamma} f(z)\mathrm{d}z :=\int_a^b f(\gamma(t))\cdot \dot{\gamma}(t) 
		\mathrm{d}t \in \mathbb{C}. \]
		Dazu bilden wir rechts Real- und Imaginärteil des Integranden und 
		integrieren diese seperat mit dem Riemann-/ Regel-/ Lesegueintegral.
		\[ \int_{\gamma} f(z)\mathrm{d}z =\int_a^b \underbrace{(u(\gamma(t))+i\cdot v(\gamma(t)))}_{f(z)}
		\cdot \underbrace{(\dot{x}(t)+i\cdot \dot{y}(t))\mathrm{d}t}_{\mathrm{d}z} \]
		,wobei $f=u+iv$, $u,v:\Omega\to\mathbb{R}$ und $\gamma =x+iy$, $x,y:[a,b]\to\mathbb{R}$ 
		(ausmultiplizieren vgl Kurzskript).  Mithin ist $f(z)=f(\gamma(t))$ der $"$Integrand$"$ und 
		$\mathrm{d}z=\mathrm{d}(\gamma(t))=\dot{\gamma}(t)\mathrm{d}t$ das 
		$"$Tangentenelement$"$ des Kurvenintegrals.
	\end{definition}
	
	\begin{remark}
		\[\overset{\text{ausmult.}}{=} \int_a^b (u(\gamma(t))\cdot \dot{x}(t)-v(\gamma(t))\cdot\dot{y}(t)
		\mathrm{d}t+i\cdot \int_a^b (u(\gamma(t))\cdot \dot{y}(t)+v(\gamma(t))\cdot\dot{x}(t))\mathrm{d}t\]
		Der Realteil ist das Kurvenintegral über $\overline{f}=u-iv$ aus der Analysis (aufgefasst als 
		Vektorfeld $\begin{pmatrix} \text{Re}\ f \\ \text{Im}\ f \end{pmatrix}$) und der Imaginärteil 
		das entsprechende $"$normale$"$ Kurvenintegral.
	\end{remark}
	
	\begin{proposition}
		Es sei $\gamma:[a,b]\to\Omega$ eine stückweise $C^1$-Kurve und $\varphi:[c,d]\to[a,b]$ ein 
		stückweiser $C^1$-Diffeomorphismus, dann ist $\gamma\circ\varphi:[c,d]\to\Omega$ eine 
		stückweise $C^1$-Kurve. $sign(\dot{\varphi})$ lässt sich zu einer konstanten Funktion auf 
		$[c,d]$ fortsetzen und für alle stetigen Funktionen $f:\Omega\to\mathbb{C}$ gilt
		\[ \int_{\gamma} f(z)\mathrm{d}z =sign(\dot{\varphi})\int_{\gamma\circ\varphi} f(\omega)
		\mathrm{d}\omega \]
	\end{proposition}
	
	\begin{proof}
		Übung mit Substitutionsformel. \\
		(Ein stückweise $C^1$-Diffeomorphismus $\varphi:[c,d]\to[a,b]$ ist ein Homomorphismus, sd. 
		ein $m\in\mathbb{N}$ und $c=s_0 <s_1 <\ldots <s_m =d$ existieren mit 
		$\varphi |_{[s_{i-1},s_i]}\in C^1 ([s_{i-1},s_i])$ für $i=1,\ldots ,m$)
	\end{proof}
	
	\begin{corollary}[aus Hauptsatz der Differential- und Integralrechnung]
		Es sei $\Omega\subset\mathbb{C}$ ein Gebiet, $\gamma:[a,b]\to\Omega$ eine stückweise 
		$C^1$-Kurve und $f:\Omega\to\mathbb{C}$ holomorph. Dann gilt der 
		Hauptsatz der Differential- und Integralrechnung
		\[ \int_{\gamma} f'(z)\mathrm{d}z=f(\gamma(t))|_{t=a}^b =f(\gamma(b))-f(\gamma(a)) \]
	\end{corollary}
	
	\begin{proof}
		Es sei $a=t_0 <t_1 <\ldots <t_n =b$, sd. $\gamma_i =\gamma|_{[t_{i-1},t_i]}\in C^1 
		([t_{i-q},t_i])$ für $i=1,\ldots ,n$.
		\[ [t_{i-1},t_i]\overset{\gamma_i}{\to}\Omega\overset{f}{\to}\mathbb{C} \]
		\begin{eqnarray*}
			\int_{\gamma_i} f'(z)\mathrm{d}z = \int_{t_{i-1}}^{t_i}f'(\gamma_i (t))\cdot \dot{\gamma_i}(t)
			\mathrm{d}t &=& \int_{t_{i-1}}^{t_i} (f\circ \gamma_i)'(t)\mathrm{d}t \\
			&=&(f\circ \gamma_i)(t_i)-(f\circ \gamma_i)(t_{i-1})
		\end{eqnarray*}
		\begin{eqnarray*}
			\Rightarrow\ \int_{\gamma} f'(z)\mathrm{d}z 
			&=& (f(\gamma(t_1))-f(\gamma(t_0)))+(f(\gamma(t_2))-f(\gamma(t_1)))+\ldots +
			(f(\gamma(t_n))-f(\gamma(t_{n-1}))) \\
			&=& f(\gamma(b))-f(\gamma(a))
		\end{eqnarray*}
	\end{proof}
	
	\begin{remark}
		Wir möchten uns das komplexe Kurvenintegral als Umkehrung der komplexen Ableitung 
		vorstellen. Wir sehen im nächsten Abschnitt, für welche Funktionen das geht.
	\end{remark}
	
	\subsection{Der Cauchy-Integralsatz}
	
	\begin{definition}[stückweise $C^1$-Homotopie]
		Eine stückweise $C^1$-Homotopie, in einem Gebiet $\Omega\subset\mathbb{C}$, zwischen 
		zwei stückweisen $C^1$-Kurven $\gamma_0,\gamma_1:[a,b]\to\Omega$ mit $\gamma_0 (a)=
		\gamma_1 (a)=p$, $\gamma_0 (b)=\gamma_1 (b)=q$ ist eine stetige Abbildung 
		$h:[a,b]\times [0,1]\to\Omega$, sd. $m,n\in\mathbb{N}$, $a=t_0 <t_1 <\ldots <t_n =b$, 
		$0=s_0 <\ldots <s_m =1$ existieren, sd. $h|_{[t_{j-1},t_j]\times [s_{k-1},s_k]}\in C^1$ ist 
		(auch auf den jeweiligen Randstücken) und $h(t,l)=\gamma_l (t)$ für $l\in [0,1]$, $t\in [a,b]$ und 
		$h(a,s)=p$, $h(b,s)=q$ für alle $s\in [0,1]$.
	\end{definition}
	
	\begin{definition}[homotope Kurve]
		Eine (stückweise $C^1$-) Kurve $\gamma_0 :[a,b]\to\Omega$ heißt zu einer 
		(stückweisen $C^1$-) Kurve $\gamma_1 :[a,b]\to\Omega$, mit gleichem Anfangs- und Endpunkt, 
		(stückweise $C^1$-) homotop in $\Omega$, wenn es eine stückweise $C^1$-Homotopie 
		zwischen ihnen in $\Omega$ gibt.
	\end{definition}
	
	\begin{definition}[nullhomotope Kurve]
		Eine geschlossene (stückweise $C^1$-) Kurve $\gamma$ heißt (stückweise $C^1$-) nullhomotop 
		in $\Omega$, wenn sie $C^1$-homotop zu einer konstanten Kurve ist.
	\end{definition}
	
	\begin{definition}[einfach zusammenhängend]
		Das Gebiet $\Omega$ heißt einfach zusammenhängend, wenn jede geschlossene (stückweise 
		$C^1$-) Kurve in $\Omega$ (stückweise $C^1$-) nullhomotop in $\Omega$ ist.
	\end{definition}
	
	\begin{remark}[Einschub zu Kurvenintegral]
		Sei $\gamma:[a,b]\to\mathbb{C}$ eine stückweise $C^1$-Kurve, dann definieren wir die 
		Bogenlänge (bzw. Länge) als
		\[ L(\gamma)=\int_a^b |\dot{\gamma}(t)|\mathrm{d}t =\sup_{n,a=t_0 <\ldots <t_n =b} 
		\sum_{j=1}^n |\gamma(t_j)-\gamma(t_{j-1})| \]
		Dann gilt
		\begin{eqnarray*}
			\left|\int_{\gamma} f(z)\mathrm{d}z \right| = \left|\int_a^b f(\gamma(t))\cdot 
			\dot{\gamma}(t)\mathrm{d}t \right| 
			&\leq& \int_a^b |f(\gamma(t))\cdot \dot{\gamma}(t) |\mathrm{d}t \\
			&=& \int_a^b |f(\gamma(t))|\cdot \dot{\gamma}(t)\mathrm{d}t \\
			&\leq& \sup_{t\in [a,b]} |f(\gamma(t))| \cdot L(\gamma).
		\end{eqnarray*}
	\end{remark}
	
	\begin{theorem}[Cauchy-Integralsatz]
		Es sei $\Omega\subset\mathbb{C}$ ein Gebiet, $f:\Omega\to\mathbb{C}$ holomorph und 
		$\gamma:[a,b]\to\Omega$ eine stückweise $C^1$-Kurve, die in $\Omega$ stückweise 
		$C^1$-nullhomotop ist. Dann gilt 
		\[ \int_{\gamma} f(z) \mathrm{d}z =0 \]
	\end{theorem}
	
	\begin{proof}
		Es reich zu zeigen, dass für jede stückweise $C^1$-Abbildung $h:\underbrace{[a,b]\times [0,1]}_
		{R}\to\Omega$ gilt
		\[ \int_{h(\partial R)} f(z) \mathrm{d}z =0. \]
		Dabei ist $\int_{h(\partial R)}\quad$ eine Abkürzung für $\int_{h(\partial R)}\quad =
		\int_{h_1}\quad +\int_{h_2}\quad +\int_{h_3}\quad +\int_{h_4}\quad$, wobei 
		$h_1 (t)=h(t,0)$, $h_2 (s)=h(b,s)$, $h_3 (t)=h(a+b-t,1)$, $h_4 (s)=h(a,1-s)$. \\
		Setze das zu einer stückweisen $C^1$-Kurve mit Namen $h(\partial R)$ zusammen. \\
		\underline{Annahme:} Es gebe eine solche Abbildung $h:[a,b]\times [0,1]\to \Omega$, sd. 
		$\int_{h(\partial R)} f(z)\mathrm{d}z \neq 0$. \\
		Wir zerlegen das Rechteck $R$ in vier gleich große Teile $R_1 ,\ldots,R_4$ und sehen, dass
		\[ \int_{h(\partial R)} f(z)\mathrm{d}z = \int_{h(\partial R_1)} f(z)\mathrm{d}z +\ldots +
		\int_{h(\partial R_4)} f(z)\mathrm{d}z. \]
		Da sich die zusätzlichen Integrale über Strecken im Inneren von $R$ wegen Proposition 1.21 
		wegheben. \\
		Jetzt wählen wir das Teilrechteck aus, für den das jeweilige Kurvenintegral über den Rand den 
		größten Absolutbetrag hat, nenne es $R_1$. Es folgt 
		\[ \left| \int_{h(\partial R_1}f(z)\mathrm{d}z\right| \geq \frac{1}{4} \left|\int_{h(\partial R)}f(z)
		\mathrm{d}z \right| \]
		Wir zerlegen weiter und erhalten so eine Folge von Rechtecken $R_1 \supset R_2 \supset\ldots 
		,R_n$ mit Seitenlängen von $R_n$ proportional zu $2^{-n}$, sd.
		\[ \left| \int_{h(\partial R_n)} f(z)\mathrm{d}z \right| \geq 2^{-n} \left| \int_{h(\partial R)}f(z)
		\mathrm{d}z \right|. \]
		Nach dem Satz über die Invervallverschachtelung (Analysis) existiert ein eindeutiger Punkt 
		$(t_0,s_0)\in\mathbb{R}^2$ mit $(t_0,s_0)\in  \bigcap_{n\in\mathbb{N}}R_n$. \\
		Es sei $z_0 =h(t_0,s_0)\in\Omega$.\\
		\underline{Beachte:} Da $h$ stückweise $C^1$ ist, erhalten wir für jedes der endlich vielen 
		Rechtecke aus Definition 1.22 eine obere Schranke für $|\frac{\partial h}{\partial t}|$, 
		$|\frac{\partial h}{\partial s}|$ (wegen der Kompaktheit). 
		Da es nur endlich viele dieser Rechtecke gibt, folgt $|\frac{\partial h}{\partial t}|\leq C$, 
		$|\frac{\partial h}{\partial s}| \leq C$ auf ganz $R=R_0$, für ein festes $C>0$. \\
		Schreibe nahe $z_0$ die Funktion $f$ als $f(z)=f(z_0)+f'(z_0)\cdot (z-z_0)+r(z-z_0)$, wobei \\
		$\lim_{z\to z_0} |\frac{r(z-z_0)}{z-z_0}|=0$, da $f$ holomorph ist (vgl. Lemma 1.14).\\
		Da $f(z_0)+f'(z_0)\cdot (z-z_0)$ das Differential der holomorphen Funktion
		$z\mapsto f(z_0)\cdot (z-z_0)+\frac{1}{2}f'(z_0)\cdot (z-z_0)^2$ ist, folgt mit Bemerkung 2, dass 
		das Integral von $f(z_0)+f'(z_0)\cdot (z-z_0)$ über die geschlossenen Kurven $h(\partial R_n)$ 
		verschwindet. Die Länge $L$ von $h(\partial R_n)$, $L(h(\partial R_n))$ können wir abschätzen 
		durch $4\cdot 2^{-n}\cdot C$. Es folgt
		\begin{eqnarray*}
			\left| \int_{h(\partial R)}f(z)\mathrm{d}z \right| 
			&\leq& \lim_{n\to\infty} 2^{2n} \left| \int_{h(\partial R_n)}f(z)\mathrm{d}z \right| \\
			&\overset{(I)}{=}& \lim_{n\to\infty} 2^{2n} \left|\int_{h(\partial R_n)}
			r(z-z_0)\mathrm{d}z \right| \\
			&\leq& \lim_{n\to\infty} \left( 2^{2n} \cdot \sup_{h(\partial R_n)}|r(z-z_0)|\cdot 
			\underbrace{L(h(\partial R_n))}_{\leq 4\cdot C\cdot 
			2^{-n}}\right) \\
			&\leq& \lim_{n\to\infty} \left(2^n \cdot 4\cdot C \cdot \sup_{h(\partial R_n)} |r(z-z_0)| \cdot 
			\frac{|z-z_0|}{|z-z_0|}\right) \\
			&\underset{|z-z_0|\leq 2^{-n}\cdot 2\cdot C}{\leq}& \lim_{n\to\infty} \left(8\cdot C^2 \cdot
			\sup_{h(\partial R_n)} \frac{|r(z-z_0)|}{|z-z_0|}\right) \\
			&=& \underbrace{\lim_{z\to z_0} \frac{|r(z-z_0)|}{|z-z_0|}}_{=0}\cdot 8\cdot C^2 \\
			&=&0.
		\end{eqnarray*}
		Also gilt $| \int_{h(\partial R)}f(z)\mathrm{d}z |=0$ im Widerspruch zur Annahme.
	\end{proof}
\end{document}