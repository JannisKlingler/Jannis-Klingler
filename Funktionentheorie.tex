\documentclass[11pt,titlepage]{article}
\usepackage{amsmath,amssymb,amstext,mathtools,amsthm}
\usepackage{amssymb}
\usepackage{xcolor}
\usepackage[utf8]{inputenc}
\usepackage[ngerman]{babel}
\usepackage[paper=a4paper,left=25mm,right=25mm,top=25mm,bottom=25mm]{geometry}
\usepackage{hyperref}
\hypersetup{bookmarksnumbered}

\usepackage{dsfont}
\usepackage{xfrac}
\usepackage{tikz}

\usetikzlibrary{positioning}
\usetikzlibrary{arrows}

\theoremstyle{definition}
\newtheorem{theorem}{Satz}[section]
\newtheorem{corollary}[theorem]{Folgerung}
\newtheorem{proposition}[theorem]{Proposition}
\newtheorem{lemma}[theorem]{Lemma}
\newtheorem{definition}[theorem]{Definition}
\newtheorem{example}[theorem]{Beispiel}
\newtheorem*{axiom}{Axiom}
\newtheorem{remark}{Bemerkung}

\theoremstyle{remark}
\newtheorem*{repetition}{Wiederholung}
\newtheorem*{remind}{Erinnerung}

\title{Funktionentheorie}
\author{Jannis Klingler}
\date{\today}

\begin{document}

\maketitle

\section{Holomorphe und analytische Funktionen}

	\subsection{Analytische Funktionen}
	
	\begin{repetition}
		Setze $\mathbb{C}=\mathbb{R}^2$. Für $z=(x,y)$, $w=(u,v)$ definiere:
		\begin{eqnarray*}
			z+w&=&(x+u,y+v) \quad\text{Vektoraddition} \\
			z\cdot w&=&(x\cdot u-y\cdot v,x\cdot v+y\cdot u) \\
			0&=&(0,0) \qquad\text{neutrales Element $(+)$}\\
			1&=&(1,1) \qquad\text{neutrales Element $(\cdot )$}\\
			i&=&(0,1)
		\end{eqnarray*}
		Komplexe Konjugation: $z\to \overline{z} =(x,-y)$ ist ein Automorphismus, dh.
		\begin{eqnarray*}
			\overline{z+w} &=& \overline{z} +\overline{w} \\
			\overline{z\cdot w} &=& \overline{z} \cdot \overline{w} \\
			\overline{0} &=& 0\\
			\overline{1} &=& 1 \\
			\overline{i} &=& (0,1)
		\end{eqnarray*}
		Mit diesen Operationen ist $\mathbb{C}$ ein Körper.
		\begin{eqnarray*}
			-z=(-x,-y) \qquad \qquad \frac{1}{z}=\frac{\overline{z}}{z\cdot \overline{z}}=\left(
			\frac{x}{x^2 +y^2}-\frac{y}{x^2 +y^2} \right)
		\end{eqnarray*}
		wir definieren einen Absolutbetrag $|z| = \sqrt{z\overline{z}}\in \mathbb{R}$, denn 
		$z\cdot\overline{z} \in \mathbb{R} =\{ z\in \mathbb{C} \mid z=\overline{z} \} =
		\{ (x,0)\mid x\in\mathbb{R} \} \subset \mathbb{C}$ \\
		Jetzt können wir schreiben $z=(x,y)=(x,0)+(y,0)=(x,0)+i\cdot (y,0)=x+iy$ \\
		Graphische Darstellung ("Gaußsche Zahlenebene").
	\end{repetition}
	
	\textbf{\underline{Zur Erinnerung}:}
	
	\begin{definition}[Topologischer Raum]
		Ein topologischer Raum heißt zusammenhängend, wenn er nicht als disjunkte Vereinigung zweier 
		nichtleerer, offener Teilmengen geschrieben werden kann.
	\end{definition}
	
	\begin{definition}[Wegzusammenhängend]
		Ein topologischer Raum $X$ heißt wegzusammenhängend, wenn es zu je zwei Punkten 
		$p,q\in X$ eine stetige Abbildung $\gamma :[0,1]\to X$ mit $\gamma (0)=p$, $\gamma (1)=q$ 
		gibt.
	\end{definition}
	
	\begin{theorem}
		Eine offene Teilmenge von $\mathbb{C}$ ist genau dann zusammenhängend, wenn sie 
		wegzusammenhängend ist.
	\end{theorem}
	\begin{proof}
		$"\Leftarrow"$: Sei $X$ wegzusammenhängend. Seien $U,V\subset X$ offen, $X=U\cup V$, 
		$p\in U$, $q\in V$ (also $U,V$ nicht leer). Dann existiert $\gamma :[0,1]\to X$ stetig mit 
		$\gamma (0)=p$, $\gamma (1)=q$. Dann sind $\gamma ^{-1}(U),\ \gamma^{-1}(V)\subset 
    		[0,1]$ offen. Da $[0,1]$ zusammenhängend ist und $0\in \gamma^{-1}(U)$, 
		$1\in \gamma^{-1}(V)$, \\
		$\gamma^{-1}(U)\cup\gamma^{-1}(V)=\gamma^{-1}(U\cup V)=\gamma^{-1}(X)=[0,1]$ folgt 
		$\gamma^{-1}(U)\cap \gamma^{-1}(V) \neq \emptyset$. \\Also existiert $t\in \gamma^{-1}(U)
		\cap\gamma^{-1}(V)$ und $\gamma(t)\in U\cap V$. Da das für alle offenen, nichtleeren 
		Teilmengen $U,V$ mit $U\cup V=X$ gilt, ist X zusammenhängend.\\
		Einfacher: \\
		Angenommen $X$ ist nicht zusammenhängend. Dann existieren offene, nicht-leere Teilmengen 
		$U,V\subset X$ mit $U\cup V=X$, $U\cap V=\emptyset$. Dann existiert eine stetige Funktion 
		$f:X\to \mathbb{R}$ mit 
		\[ f(x)= \begin{dcases} 0 & x\in U \\ 1& x\in V \end{dcases} \]
		Wähle jetzt $p\in U$, $q\in V$. Gäbe es einen Weg $\gamma : [0,1]\to X$ mit $\gamma(0)=p$, 
		$\gamma(1)=q$, dann wäre $f\circ \gamma :[0,1]\to \mathbb{R}$ stetig, im Widerspruch zum 
		Zwischenwertsatz. \\
		$"\Rightarrow"$: Sei $X\subset \mathbb{C}$ (offen) zusammenhängend. \\Sei $p\in X$ und sei 
		$U=\{ q\in X \mid \exists \gamma :[0,1]\to X \text{ stetig}:\gamma(0)=p,\ \gamma(1)=q \}$\\
		Behauptung: $U$ ist offen, also existiert $\varepsilon>0$, sd. $B_{\varepsilon}(q)\subset X$. 
		Sei $q'\in B_{\varepsilon}(q)$. Dann existiert $\gamma':[0,1]\to X$, sd. 
		\[ \gamma'(t)= \begin{dcases} \gamma(2t) & 0\leq t\leq \frac{1}{2} \\ (2-2t)q + (2t-1)q' & 
			\frac{1}{2} \leq t\leq 1 \end{dcases} \]
		$\Rightarrow$ $B_{\varepsilon}(q)\subset U$ $\Rightarrow$ $U$ offen.\\
		Behauptung: $X\setminus U$ ist offen: \\
		Sei $q\in X\setminus U$. Da $X$ offen, existiert $\varepsilon >0$ mit 
		$B_{\varepsilon}(q)\subset X$. Wäre $B_{\varepsilon}(q)\cap U \neq \emptyset$, so existiert 
		$q'\in B_{\varepsilon}(q)\cap U$, ein Weg $\gamma$ von $p$ nach $q$ in $X$ und mit einer 
		ähnlichen Konstruktion auch eine Kurve $\gamma'$ von $p$ nach $q$. Also auch 
		$X\setminus U = \emptyset$.\\
		$\Rightarrow$ $X$ ist wegzusammenhängend.	
	\end{proof}
	
	\begin{definition}[Gebiet]
		Ein Gebiet ist eine offene, zusammenhängende Teilmenge von $\mathbb{C}$.
	\end{definition}
	
	\begin{remind}
		Eine (komplexe) Potenzreihe ist ein Ausdruck der Form $R(z)=\sum^{\infty}_{n=0} a_n z^n$ mit 
		$a_n \in \mathbb{C}$ für alle $n$. Sie hat den Konvergenzradius $\rho = \left( \limsup_{n\to\infty}
		\sqrt[n]{|a_n |}\right)^{-1} \in [0,\infty]$. Dann:
		\begin{eqnarray*}
			R(z)\text{ konvergiert für alle $z$ mit }|z|< \rho \\
			R(z)\text{ divergiert für alle $z$ mit }|z|> \rho 
		\end{eqnarray*}
		wenn $\rho>0$ ist, heißt $R(z)$ konvergent und $B_{\rho}(0)\subset \mathbb{C}$ der 
		Konvergenzkreis.
	\end{remind}
	
	\begin{definition}[Analytische Funktion]
		Es sei $\Omega\in\mathbb{C}$ ein Gebiet und $f:\Omega\to\mathbb{C}$ eine Abbildung. 
		Dann heißt $f$ eine analytische Funktion (auf $\Omega$), wenn es zu jedem Punkt 
		$z_0 \in \Omega$ eine Potenzreihe $R(z)$ mit Konvergenzradius $\rho>0$ existiert, sd. 
		$f(z)=R(z-z_0 )$ für alle $z\in \Omega\cap B_{\rho}(z_0)$.
	\end{definition}
	
	\begin{example}
		Betrachte die Exponentialreihe
		\begin{eqnarray*}
			e^z = \sum_{n=0}^{\infty} \frac{z^n}{n!} 
		\end{eqnarray*}
		$\limsup \sqrt[n]{|\frac{1}{n!}|} =0 \quad \Rightarrow$ Konvergenzradius ist $\rho=\infty$.
		Mit dem Umordnungssatz zeigt man 
		\begin{eqnarray*}
			e^{z+w} =e^z \cdot e^w
		\end{eqnarray*}
		Da die Exponentialreihe reelle Koeffizienten hat, gilt
		\[ \overline{e^z} =\sum_{n=0}^{\infty} \overline{\left( \frac{z^n}{n!} \right)} = 
		\sum_{n=0}^{\infty} \frac{ \overline{z}^n}{n!} = e^{\overline{z}} \]
		Sei jetzt $z=x+iy$, dann gilt 
		\[e^z = e^x \cdot e^{iy} \]
		und $|e^{iy}|^2 = e^{iy} \cdot \overline{e^{iy}} = e^{iy} \cdot e^{-iy} = e^0 = 1$. \\
		Also definiere $e^{iy}=\cos(y)+i\sin (y)$.\\
		Jetzt kann man komplexe Multiplikation in Polarkoordinaten verstehen. \\
		Schreibe $z=r\cdot e^{i\varphi}$, $w=s\cdot e^{i\varphi}$ dann heißt $r=|z|$ der Absolutbetrag
		 und 
		$\varphi\in \mathbb{R}\setminus 2\pi \mathbb{Z}$ das Argument. \\
		Wir repräsentieren $\varphi$ durch die Funktion $arg:\mathbb{C}^{\times} =\mathbb{C}\setminus 
		\{ 0\} \to (-\pi ,\pi]$. \\
		$z\cdot w= r\cdot e^{i\varphi} \cdot s \cdot e^{i\psi} = (rs)\cdot e^{i(\varphi+\psi)}$.
	\end{example}
	
	\begin{theorem}[Identitätssatz für Potenzreihen]
		Es sei $\Omega \subset \mathbb{C}$ Gebiet und $f:\Omega\to \mathbb{C}$ analytisch. 
		Falls es $z_0 \in \Omega$ und eine Folge $(z_n)_{n\in \mathbb{N}}$ in $\Omega\setminus\{
		z_0\}$ mit $\lim_{n\to \infty} z_n = z_0$ gibt, sd. $f(z_n)=0$ für alle $n$, dann ist $f=0$ konstant.
	\end{theorem}
	
	\begin{corollary}
		Seien $f,g$ zwei analytische Funktionen auf $\Omega$, $z_0$, $(z_n)_{n\in\mathbb{N}}$ wie 
		oben, aber mit $f(z_n)=g(z_n)$ für alle $n$, dann folgt $f=g$ auf ganz $\Omega$.
	\end{corollary}
	
	\begin{definition}
		$f$ heißt analytisch auf $\Omega$, wenn es zu jedem Punkt $z\in \Omega$ eine Umgebung 
		$U\subset\Omega$ von $z$ und eine Potenzreihe $R$ um $z$ gibt, die auf ganz $U$ 
		konvergiert, sd. $R(\omega)=f(\omega)$ für alle $\omega\in\Omega$.
	\end{definition}
	
	\begin{proof}
		Sei zunächst $U$ Umgebung von $z$, auf der $f$ mit einer Potenzreihe $R(z)=\sum_{n=0}^
		{\infty} a_n (z-z_n)$ übereinstimmt. \\
		Ohne Einschränkung sei $z_0 =0$. Da $R$ konvergiert, gilt $\rho >0$, also $\infty > \frac{1}{\rho}
		=\limsup_{n\to\infty} \sqrt[n]{|a_n|}$. Also existiert $n_0\in \mathbb{N}_0$ und $C>\frac{1}{\rho}$, 
		sd. $|a_n|< C^n$ für alle $n\geq n_0$. Da nur endlich viele $n\leq n_0$ existieren, können wir 
		$C$ ggf. etwas größer wählen, sd. $|a_n|<C^n$ für alle $n$. Wir beweisen indirekt, dass alle 
		$a_n =0$ sind, dh. wir nehmen an, es gäbe $n$ mit $a_n \neq 0$. Es sei $n_0$ das kleinste $n$ 
		mit $a_{n_{0}}\neq 0$, dh. $a_n =0$ für $n<n_0$. 
		Wir suchen $r>0$, sd. $|a_n z^{n_0} | > \sum_{n=n_0 +1}^{\infty} |a_n z^n | \left(\geq | \sum_{n=
		n_0 +1}^{\infty} a_n z^n |\right)$ für alle $z\in \mathbb{C}$ mit $0<|z|<r$. Denn dann folgt 
		$R(z)=a_{n_0} z^{n_0} +\sum_{a_n}^{z^n} \neq 0$ für $z$ wie oben, also auch für unendlich 
		viele der Folgenglieder $z_n$ aus unserer Annahme.
		\[ \sum_{n=n_0 +1}^{\infty} |a_n z^n| \leq \sum_{n=n_0 +1}^{\infty} C^n |z^n| \underset{\text{
		geometrische Reihe}}{=} \frac{C^{n+1}|z|^{n+1}}{1-C|z|} \]
		Wir suchen also $r>0$, sd.
		\begin{eqnarray*}
			|a_{n_0}|r^{n_0} > \underbrace{\frac{C^{n+1}|z|^{n+1}}{1-Cr}}_{\text{$>0$, für 
			$r>\frac{1}{C}$}} &\Leftrightarrow & |a_n| (r^{n_0} - Cr^{n_0 +1} ) > C^{n_0 +1} r^{n_0 +1} \\
			& \Leftrightarrow & |a_{n_0}| > r (C^{n_0 +1}+|a_{n_0}|C) \\
			& \Leftrightarrow & r>\frac{|a_{n_0}|}{C^{n_0 +1} + |a_{n_0}|C}
		\end{eqnarray*}
		Jetzt folgt für alle $z$ mit $0< |z|<r$, dass $R(z)\neq 0$ wie gewünscht, Widerspruch! \\
		Also folgt $R=0$ und somit $f|_U =0$.
		Definiere $W=\{ z\in\Omega \mid z\text{ hat Umgebung $U$ mit $f|_U =0$} \}$ \\
		$\Rightarrow$ $W$ ist offen und nichtleer. \\
		Behauptung: $W$ ist auch abgeschlossen. Falls nicht, existiert ein Häufungspunkt $z_0$ von 
		$W$ in $\Omega$ mit $z_0 \in W$. Dann existiert $(z_n)_n$ Folge in $W\setminus \{z_0\}$ mit 
		$\lim_{n\to\infty} z_n =z_0$ und $f(z_n)=0$ für alle $n$. Mit den obigen Argumenten folgt: 
		$z_0$ hat Umgebung $U\subset \Omega$ mit $f|_U =0$, somit $z_0 \in W$. \\
		$W$ offen, abgeschlossen und nichtleer $\Rightarrow$ (da $\Omega$ zusammenhängend ist) 
		$\Omega = W$, also $f=0$.
	\end{proof}
	
	(Proposition im Kurzskript zum Rechnen mit Potenzreihen)...
	
	\subsection{Komplexe Differenzierbarkeit}
	
	\begin{definition}
		Eine $\mathbb{R}$-lineare Abbildung $A:\mathbb{C}\to \mathbb{C}$ heißt $\mathbb{C}$-
		antilinear, wenn
		\[ A(zw)=\overline{z} \cdot A(w) \quad \forall w,z\in\mathbb{C}. \]
		Jede $\mathbb{R}$-lineare Abbildung lässt sich zerlegen als $A=A'+A''$ mit 
		$A'(z)=a'\cdot z$ und $A''(z)=a''\cdot\overline{z}$, dabei heißen $A'$ der Linearteil und $A''$ 
		der Antilinearteil von $A$.\\
		Insbesondere ist $A$ genau dann $\mathbb{C}$-linear, wenn $A''=0$.
	\end{definition}
	
	\begin{proof}
		Setze $A'(z)=\frac{A(z)-i\cdot A(iz)}{2}$, $A''(z)=\frac{A(z)+i\cdot A(iz)}{2}$. Daraus folgt 
		\[ A'(z) + A''(z)=\frac{A(z)-i\cdot A(iz)}{2}+\frac{A(z)+i\cdot A(iz)}{2}=A(z) \]
		\begin{eqnarray*}
			A'((u+iv)\cdot z)&=&\frac{A(uz)+A(ivz)-iA(iuz)-iA(-vz)}{2} \\
			&=& \frac{uA(z)\overbrace{-iviA(iz)}^{=+vA(iz)}-iuA(iz)+ivA(z)}{2} \\
			&=&\frac{(u+iv)(A(z)-iA(iz))}{2} \\
			&=& (u+iv)A'(z)
		\end{eqnarray*}
		Analog dazu ist $A''$ $\mathbb{C}$-antilinear. Es folgt $A'(z)=A'(z\cdot 1)=z\cdot \underbrace{
		A'(1)}_{a'}$,\\$A''(z)=A''(z\cdot 1)=\overline{z}\cdot\underbrace{A''(1)}_{a''}$.
	\end{proof}
	
	\begin{repetition}
		Sei $U\subset\mathbb{C}$ offen, $f:U\to\mathbb{C}\sim\mathbb{R}^2$ eine Funktion. $f$ 
		heißt total differenzierbar bei $z_0\in U$, falls eine $\mathbb{R}$-lineare Abbildung 
		$A:\mathbb{C}\to\mathbb{C}$ existiert, sd. \[ \lim_{z\to z_0} \frac{f(z)-f(z_0)-A(z-z_0)}
		{|z-z_0 |}=0. \]
		Dann ist $f$ auch partiell differenzierbar und die partiellen Ableitungen sind gerade die Einträge 
		der reellen $2\times 2$-Matrix $A$.
	\end{repetition}
	
	\begin{definition}[Komplexe Differenzierbarkeit]
		Es sei $U\subset \mathbb{C}$ offen. Eine Funktion $f:U\to \mathbb{C}$ heißt komplex 
		differenzierbar bei $z_0\in U$, falls $\lim_{z\to z_0} \frac{f(z)-f(z_0)}{z-z_0}$ existiert. Dieser 
		Grenzwert heißt dann die komplexe Ableitung $f'(z_0)\in\mathbb{C}$. Wenn $f$ auf ganz 
		$U$ differenzierbar ist, heißt $f$ auch holomorph auf $U$.
	\end{definition}
	
	\begin{definition}
		Sei $f:U\to\mathbb{C}$ eine Funktion, $U\subset\mathbb{C}$ offen. Schreibe $f=u+iv$ für 
		Funktionen $u,v:U\to\mathbb{R}$, sowie $z=x+iy$. \\
		Definiere die Wirtinger-Ableitungen 
		\[ \frac{\partial f}{\partial z}=\frac{1}{2}\frac{\partial f}{\partial x} -\frac{i}{2}\frac{\partial f}{\partial y} 
		=\frac{1}{2} \left(\frac{\partial u}{\partial x}+\frac{\partial v}{\partial y}\right)+\frac{i}{2}\left(
		\frac{\partial v}{\partial x}-\frac{\partial u}{\partial y}\right) \]
		\[ \frac{\partial f}{\partial \overline{z}}=\frac{1}{2} \frac{\partial f}{\partial x} +\frac{i}{2}
		\frac{\partial f}{\partial y}=\frac{1}{2}\left( \frac{\partial u}{\partial x}-\frac{\partial v}{\partial y}\right)
		+\frac{i}{2}\left(\frac{\partial v}{\partial x}+\frac{\partial u}{\partial y}\right) \]
	\end{definition}
	
	\begin{example}
		$\frac{\partial z}{\partial z}=1$, $\frac{\partial z}{\partial \overline{z}}=0$, 
		$\frac{\partial\overline{z}}{\partial z}=0$, $\frac{\partial\overline{z}}{\partial\overline{z}}=1$
	\end{example}
	
	\begin{lemma}[Definition]\label{lem:komplexdiffbar}
		Es sei $U\subset\mathbb{C}$ offen, $f:U\to\mathbb{C}$ eine Funktion, $z_0\in U$. Dann sind 
		äquivalent
		\begin{enumerate}
			\item $f$ ist komplex differenzierbar bei $z_0$
			\item Es existiert eine stetige Funktion $\varphi:U\to\mathbb{C}$ mit $f(z)=f(z_0)+\varphi(z)
			\cdot (z-z_0)$
			\item $f$ ist bei $z_0$ reell, total differenzierbar mit $\mathbb{C}$-linearer Ableitung
			\item $f$ ist bei $z_0$ reell, total differenzierbar und $\frac{\partial f}{\partial \overline{z}}|_
			{z_0}=0$
			\item $f$ ist bei $z_0$ reell, total differenzierbar und es gelten die Cauchy-Riemann-
			Differentialgleichungen (C-R-DGL): 
			$\frac{\partial u}{\partial x}|_{z_0}=\frac{\partial v}{\partial y}|_{z_0}$ 
			und $\frac{\partial u}{\partial y}|_{z_0}=-\frac{\partial v}{\partial x}|_{z_0}$, wobei wieder 
			$f=u+iv$ gelte.
		\end{enumerate}
		Insbesondere ist $f$ dann auch bei $z_0$ stetig.
	\end{lemma}
	
	\begin{proof}
		(1)$\Rightarrow$(2): Setze 
		\[ \varphi(z)=\begin{dcases} \frac{f(z)-f(z_0)}{z-z_0} & z\neq z_0 \\ f'(z_0) & z=z_0 \end{dcases}\]
		Stetigkeit bei $z_0$ folgt aus der komplexen Differenzierbarkeit.\\
		(2)$\Rightarrow$(3): Schreibe
		\begin{eqnarray*}
			\lim_{z\to z_0} \frac{f(z)-f(z_0)-\varphi(z_0)\cdot (z-z_0)}{|z-z_0|} = 
			\lim_{z\to z_0} \underbrace{(\varphi(z)-\varphi(z_0))}_{\to \ 0\text{, da $\varphi$ stetig in 
			$z_0$.}} \underbrace{\frac{z-z_0}{|z-z_0|}}_{\text{beschränkt (Norm $1$)}} =0
		\end{eqnarray*}
		$\Rightarrow$ $f$ ist bei $z_0$ total-reell-differenzierbar. Die Ableitung ist die $\mathbb{C}$-
		lineare Abbildung $\omega\mapsto \varphi(z_0)\cdot \omega$.\\
		(3)$\Rightarrow$(4): Da die reelle Ableitung $\mathbb{C}$-linear ist, folgt 
		$\frac{\partial f}{\partial \overline{z}}(z_0)=0$ ( was nach Definition gerade der Antilinearteil 
		der Ableitung ist)\\
		(4)$\Rightarrow$(5):
		\begin{eqnarray*}
			0=\underbrace{\frac{\partial f}{\partial \overline{z}}(z_0)}_{\in \mathbb{C}} \underset{\text{
			Def. 1.12}}{=} \underbrace{\frac{1}{2} \left( \frac{\partial u}{\partial x} -
			\frac{\partial v}{\partial y} \right)}_{\text{Realteil}}(z_0) + \frac{i}{2} \underbrace{
			\left( \frac{\partial v}{\partial x}+\frac{\partial u}{\partial y}\right)}_{\text{Imaginärteil}}(z_0)
		\end{eqnarray*}
		hieraus lassen sich die C-R-DGL direkt ablesen.\\
		(5)$\Rightarrow$(1): Schreibe $z=z_0 +x+iy$ dann gilt
		\begin{eqnarray*}
			f(z) &=& f(z_0) +\frac{\partial u}{\partial x} x +\frac{\partial u}{\partial y} y +
			 \frac{\partial v}{\partial x} ix + \frac{\partial v}{\partial y} iy +R(x,y) \\
			&\overset{\text{C-R-DGL}}{=}& f(z_0) +\frac{\partial u}{\partial x} (x+iy) -\frac{\partial v}
			{\partial x}(y-ix)+R(x,y) \\
			&=& f(z_0) +\left(\frac{\partial u}{\partial x}+i \frac{\partial v}{\partial x}\right)\cdot
			(x+iy)+R(x,y)
		\end{eqnarray*}
		mit $R(x,y)=o(|(x,y)|)$, das heißt $\lim_{(x,y)\to 0} \frac{R(x,y)}{|(x,y)|}=0$ (der Restterm geht 
		schneller gegen Null als $(x,y)$). Es folgt 
		\begin{eqnarray*}
			\lim_{z\to z_0}\frac{f(z)-f(z_0)}{z-z_0} &=& \lim_{z\to z_0}\frac{\left(\frac{\partial u}{\partial x}
			+i\frac{\partial v}{\partial x}\right)\cdot (z-z_0)+R(x,y)}{z-z_0} \\
			&=&\frac{\partial u}{\partial x}+i\frac{\partial v}{\partial x} +\lim_{z\to z_=}
			\underbrace{\frac{R(x,y)}{|z-z_0|}}_{\to\  0}\cdot \underbrace{\frac{|z-z_0|}{z-z_=}}_{\text{
			beschränkt}}\\
			&=& \frac{\partial u}{\partial x}+i\frac{\partial v}{\partial x} \in\mathbb{C}
		\end{eqnarray*}
	\end{proof}
	
	\begin{example}
		Die komplexe Exponentialfunktion ist holomorph auf ganz $\mathbb{C}$ (Begründung folgt)
	\end{example}
	
	\begin{proposition}\label{prop:Rechenregeln}
		Es gelten folgende Differentiationsregeln:
		\begin{enumerate}
			\item \underline{Linearität:} Seien $f,g:\Omega\to\mathbb{C}$ holomorph, $a,b\in\mathbb{C}
			$, dann ist $a\cdot f+b\cdot g$ holomorph mit $(a\cdot f+b\cdot g)'(z)=a\cdot f'(z)+b\cdot 
			g'(z)$.
			\item \underline{Kettenregel:} Sei $f:\Omega\to\Omega'$, $g:\Omega'\to\mathbb{C}$ 
			holomorph, dann ist $g\circ f:\Omega\to\mathbb{C}$ holomorph mit 
			$(g\circ f)'(z)=f'(g(z))\cdot g'(z)$.
			\item \underline{Produktregel:} Seien $f,g:\Omega\to\mathbb{C}$ holomorph, dann ist 
			$f\cdot g$ holomorph mit Ableitung $(f\cdot g)'(z)= f'(z)\cdot g(z)+f(z)\cdot g'(z)$.
		\end{enumerate}
	\end{proposition}
	
	\begin{proof}
		(1): Additivität ist klar. Multiplikativität siehe (3) \\
		(2): Übung\\
		(3): Schreibe $f=u+iv$, $g=r+is$, $u,v,r,s:\Omega\to\mathbb{R}$, dann ist 
		$f\cdot g=(u\cdot r-v\cdot s)+i\cdot(u\cdot s+v\cdot r)$. Jetzt setzen wir mit den reellen 
		Produktregeln fort und sind fertig.
	\end{proof}
	
	\begin{theorem}
		Es sei $R(z)=\sum_{n=0}^{\infty} a_n \cdot z^n $ konvergente Potenzreihe mit 
		Konvergenzradius $\rho >0$, dann ist $R(z)$ auf $B_{\rho}(0)$ holomorph mit Ableitung 
		\[R'(z)=\sum_{n=0}^{\infty} a_n \cdot n\cdot z^{n-1} =\sum_{m=0}^{\infty}a_{m+1}(m+1)z^m,\quad
		n-1=m.\]
	\end{theorem}
	
	\begin{proof}
		Siehe Analysis, beruht auf folgendem Satz: Sei $(f_n)_{n\in\mathbb{N}}$ Folge differenzierbarer 
		Funktionen auf $U$, sd. $(f_n)_n$ punktweise und $(f'_n)_n$ lokal-gleichmäßig konvergiert. \\
		$\Rightarrow$ $(\lim_{n\to\infty} f_n)'=\lim_{n\to\infty}f'_n$
	\end{proof}
	
	\subsection{Das komplexe Kurvenintegral}
	
	\begin{definition}
		Eine stückweise $C^1$-Kurve $\gamma:[a,b]\to\mathbb{C}$ ist eine stetige Abbildung, sd. 
		$a=t_0 < t_1 <\ldots<t_n=b$ existieren, für die $\gamma |_{ [t_{i-1},t_i]}\in C^1$ für 
		$i=1,\ldots ,n$. \\
		Für $t\neq t_i$, $t\in [a,b]$, sei $\dot{\gamma}(t)=\frac{\mathrm{d}\gamma}{\mathrm{d}t}(t)$ der 
		Geschwindigkeitsvektor. $\gamma$ heißt geschlossen, wenn \[\gamma(a)=\gamma(b).\]
	\end{definition}
	
	\begin{definition}[Kurvenintegral]
		Sei $\Omega\subset\mathbb{C}$ ein Gebiet, $\gamma:[a,b]\to\Omega$ stückweise $C^1$, sei 
		$f:\Omega\to\mathbb{C}$ stetig. Definiere das komplexe Kurvenintegral
		\[ \int_{\gamma} f(z)\mathrm{d}z :=\int_a^b f(\gamma(t))\cdot \dot{\gamma}(t) 
		\mathrm{d}t \in \mathbb{C}. \]
		Dazu bilden wir rechts Real- und Imaginärteil des Integranden und 
		integrieren diese seperat mit dem Riemann-/ Regel-/ Lesegueintegral.
		\[ \int_{\gamma} f(z)\mathrm{d}z =\int_a^b \underbrace{(u(\gamma(t))+i\cdot v(\gamma(t)))}_{f(z)}
		\cdot \underbrace{(\dot{x}(t)+i\cdot \dot{y}(t))\mathrm{d}t}_{\mathrm{d}z} \]
		,wobei $f=u+iv$, $u,v:\Omega\to\mathbb{R}$ und $\gamma =x+iy$, $x,y:[a,b]\to\mathbb{R}$ 
		(ausmultiplizieren vgl Kurzskript).  Mithin ist $f(z)=f(\gamma(t))$ der $"$Integrand$"$ und 
		$\mathrm{d}z=\mathrm{d}(\gamma(t))=\dot{\gamma}(t)\mathrm{d}t$ das 
		$"$Tangentenelement$"$ des Kurvenintegrals.
	\end{definition}
	
	\begin{remark}
		\[\overset{\text{ausmult.}}{=} \int_a^b (u(\gamma(t))\cdot \dot{x}(t)-v(\gamma(t))\cdot\dot{y}(t)
		\mathrm{d}t+i\cdot \int_a^b (u(\gamma(t))\cdot \dot{y}(t)+v(\gamma(t))\cdot\dot{x}(t))\mathrm{d}t\]
		Der Realteil ist das Kurvenintegral über $\overline{f}=u-iv$ aus der Analysis (aufgefasst als 
		Vektorfeld $\begin{pmatrix} \text{Re}\ f \\ \text{Im}\ f \end{pmatrix}$) und der Imaginärteil 
		das entsprechende $"$normale$"$ Kurvenintegral.
	\end{remark}
	
	\begin{proposition}
		Es sei $\gamma:[a,b]\to\Omega$ eine stückweise $C^1$-Kurve und $\varphi:[c,d]\to[a,b]$ ein 
		stückweiser $C^1$-Diffeomorphismus, dann ist $\gamma\circ\varphi:[c,d]\to\Omega$ eine 
		stückweise $C^1$-Kurve. $sign(\dot{\varphi})$ lässt sich zu einer konstanten Funktion auf 
		$[c,d]$ fortsetzen und für alle stetigen Funktionen $f:\Omega\to\mathbb{C}$ gilt
		\[ \int_{\gamma} f(z)\mathrm{d}z =sign(\dot{\varphi})\int_{\gamma\circ\varphi} f(\omega)
		\mathrm{d}\omega \]
	\end{proposition}
	
	\begin{proof}
		Übung mit Substitutionsformel. \\
		(Ein stückweise $C^1$-Diffeomorphismus $\varphi:[c,d]\to[a,b]$ ist ein Homomorphismus, sd. 
		ein $m\in\mathbb{N}$ und $c=s_0 <s_1 <\ldots <s_m =d$ existieren mit 
		$\varphi |_{[s_{i-1},s_i]}\in C^1 ([s_{i-1},s_i])$ für $i=1,\ldots ,m$)
	\end{proof}
	
	\begin{corollary}[aus Hauptsatz der Differential- und Integralrechnung]\label{Haupsatz}
		Es sei $\Omega\subset\mathbb{C}$ ein Gebiet, $\gamma:[a,b]\to\Omega$ eine stückweise 
		$C^1$-Kurve und $f:\Omega\to\mathbb{C}$ holomorph. Dann gilt der 
		Hauptsatz der Differential- und Integralrechnung
		\[ \int_{\gamma} f'(z)\mathrm{d}z=f(\gamma(t))|_{t=a}^b =f(\gamma(b))-f(\gamma(a)) \]
	\end{corollary}
	
	\begin{proof}
		Es sei $a=t_0 <t_1 <\ldots <t_n =b$, sd. $\gamma_i =\gamma|_{[t_{i-1},t_i]}\in C^1 
		([t_{i-q},t_i])$ für $i=1,\ldots ,n$.
		\[ [t_{i-1},t_i]\xrightarrow{\gamma_i}\Omega\xrightarrow{f}\mathbb{C} \]
		\begin{eqnarray*}
			\int_{\gamma_i} f'(z)\mathrm{d}z = \int_{t_{i-1}}^{t_i}f'(\gamma_i (t))\cdot \dot{\gamma_i}(t)
			\mathrm{d}t &=& \int_{t_{i-1}}^{t_i} (f\circ \gamma_i)'(t)\mathrm{d}t \\
			&=&(f\circ \gamma_i)(t_i)-(f\circ \gamma_i)(t_{i-1})
		\end{eqnarray*}
		\begin{eqnarray*}
			\Rightarrow\ \int_{\gamma} f'(z)\mathrm{d}z 
			&=& (f(\gamma(t_1))-f(\gamma(t_0)))+(f(\gamma(t_2))-f(\gamma(t_1)))+\ldots +
			(f(\gamma(t_n))-f(\gamma(t_{n-1}))) \\
			&=& f(\gamma(b))-f(\gamma(a))
		\end{eqnarray*}
	\end{proof}
	
	\begin{remark}\label{Bemerkung2}
		Wir möchten uns das komplexe Kurvenintegral als Umkehrung der komplexen Ableitung 
		vorstellen. Wir sehen im nächsten Abschnitt, für welche Funktionen das geht.
	\end{remark}
	
	\subsection{Der Cauchy-Integralsatz}
	
	\begin{definition}[stückweise $C^1$-Homotopie]\label{def:Homotopie}
		Eine stückweise $C^1$-Homotopie, in einem Gebiet $\Omega\subset\mathbb{C}$, zwischen 
		zwei stückweisen $C^1$-Kurven $\gamma_0,\gamma_1:[a,b]\to\Omega$ mit $\gamma_0 (a)=
		\gamma_1 (a)=p$, $\gamma_0 (b)=\gamma_1 (b)=q$ ist eine stetige Abbildung 
		$h:[a,b]\times [0,1]\to\Omega$, sd. $m,n\in\mathbb{N}$, $a=t_0 <t_1 <\ldots <t_n =b$, 
		$0=s_0 <\ldots <s_m =1$ existieren, sd. $h|_{[t_{j-1},t_j]\times [s_{k-1},s_k]}\in C^1$ ist 
		(auch auf den jeweiligen Randstücken) und $h(t,l)=\gamma_l (t)$ für $l\in [0,1]$, $t\in [a,b]$ und 
		$h(a,s)=p$, $h(b,s)=q$ für alle $s\in [0,1]$.
	\end{definition}
	
	\begin{definition}[homotope Kurve]
		Eine (stückweise $C^1$-) Kurve $\gamma_0 :[a,b]\to\Omega$ heißt zu einer 
		(stückweisen $C^1$-) Kurve $\gamma_1 :[a,b]\to\Omega$, mit gleichem Anfangs- und Endpunkt, 
		(stückweise $C^1$-) homotop in $\Omega$, wenn es eine stückweise $C^1$-Homotopie 
		zwischen ihnen in $\Omega$ gibt.
	\end{definition}
	
	\begin{definition}[nullhomotope Kurve]
		Eine geschlossene (stückweise $C^1$-) Kurve $\gamma$ heißt (stückweise $C^1$-) nullhomotop 
		in $\Omega$, wenn sie $C^1$-homotop zu einer konstanten Kurve ist.
	\end{definition}
	
	\begin{definition}[einfach zusammenhängend]
		Das Gebiet $\Omega$ heißt einfach zusammenhängend, wenn jede geschlossene (stückweise 
		$C^1$-) Kurve in $\Omega$ (stückweise $C^1$-) nullhomotop in $\Omega$ ist.
	\end{definition}
	
	\begin{remark}[Einschub zu Kurvenintegral]
		Sei $\gamma:[a,b]\to\mathbb{C}$ eine stückweise $C^1$-Kurve, dann definieren wir die 
		Bogenlänge (bzw. Länge) als
		\[ L(\gamma)=\int_a^b |\dot{\gamma}(t)|\mathrm{d}t =\sup_{n,a=t_0 <\ldots <t_n =b} 
		\sum_{j=1}^n |\gamma(t_j)-\gamma(t_{j-1})| \]
		Dann gilt
		\begin{eqnarray*}
			\left|\int_{\gamma} f(z)\mathrm{d}z \right| = \left|\int_a^b f(\gamma(t))\cdot 
			\dot{\gamma}(t)\mathrm{d}t \right| 
			&\leq& \int_a^b |f(\gamma(t))\cdot \dot{\gamma}(t) |\mathrm{d}t \\
			&=& \int_a^b |f(\gamma(t))|\cdot \dot{\gamma}(t)\mathrm{d}t \\
			&\leq& \sup_{t\in [a,b]} |f(\gamma(t))| \cdot L(\gamma).
		\end{eqnarray*}
	\end{remark}
	
	\begin{theorem}[Cauchy-Integralsatz]
		Es sei $\Omega\subset\mathbb{C}$ ein Gebiet, $f:\Omega\to\mathbb{C}$ holomorph und 
		$\gamma:[a,b]\to\Omega$ eine stückweise $C^1$-Kurve, die in $\Omega$ stückweise 
		$C^1$-nullhomotop ist. Dann gilt 
		\[ \int_{\gamma} f(z) \mathrm{d}z =0 \]
	\end{theorem}
	
	\begin{proof}
		Es reicht zu zeigen, dass für jede stückweise $C^1$-Abbildung $h:\underbrace{[a,b]\times [0,1]}_
		{R}\to\Omega$ gilt
		\[ \int_{h(\partial R)} f(z) \mathrm{d}z =0. \]
		Dabei ist $\int_{h(\partial R)}\quad$ eine Abkürzung für $\int_{h(\partial R)}\quad =
		\int_{h_1}\quad +\int_{h_2}\quad +\int_{h_3}\quad +\int_{h_4}\quad$, wobei 
		$h_1 (t)=h(t,0)$, $h_2 (s)=h(b,s)$, $h_3 (t)=h(a+b-t,1)$, $h_4 (s)=h(a,1-s)$. \\
		Setze das zu einer stückweisen $C^1$-Kurve mit Namen $h(\partial R)$ zusammen. \\
		\underline{Annahme:} Es gebe eine solche Abbildung $h:[a,b]\times [0,1]\to \Omega$, sd. 
		$\int_{h(\partial R)} f(z)\mathrm{d}z \neq 0$. \\
		Wir zerlegen das Rechteck $R$ in vier gleich große Teile $R_1 ,\ldots,R_4$ und sehen, dass
		\[ \int_{h(\partial R)} f(z)\mathrm{d}z = \int_{h(\partial R_1)} f(z)\mathrm{d}z +\ldots +
		\int_{h(\partial R_4)} f(z)\mathrm{d}z. \]
		Da sich die zusätzlichen Integrale über Strecken im Inneren von $R$ wegen Proposition 
		\ref{Hauptsatz} wegheben. \\
		Jetzt wählen wir das Teilrechteck aus, für den das jeweilige Kurvenintegral über den Rand den 
		größten Absolutbetrag hat, nenne es $R_1$. Es folgt 
		\[ \left| \int_{h(\partial R_1)}f(z)\mathrm{d}z\right| \geq \frac{1}{4} \left|\int_{h(\partial R)}f(z)
		\mathrm{d}z \right| \]
		Wir zerlegen weiter und erhalten so eine Folge von Rechtecken $R_1 \supset R_2 \supset\ldots 
		,R_n$ mit Seitenlängen von $R_n$ proportional zu $2^{-n}$, sd.
		\[ \left| \int_{h(\partial R_n)} f(z)\mathrm{d}z \right| \geq 2^{-n} \left| \int_{h(\partial R)}f(z)
		\mathrm{d}z \right|. \]
		Nach dem Satz über die Invervallverschachtelung (Analysis) existiert ein eindeutiger Punkt 
		$(t_0,s_0)\in\mathbb{R}^2$ mit $(t_0,s_0)\in  \bigcap_{n\in\mathbb{N}}R_n$. \\
		Es sei $z_0 =h(t_0,s_0)\in\Omega$.\\
		\underline{Beachte:} Da $h$ stückweise $C^1$ ist, erhalten wir für jedes der endlich vielen 
		Rechtecke aus Definition \ref{def:Homotopie} eine obere Schranke für 
		$|\frac{\partial h}{\partial t}|$, $|\frac{\partial h}{\partial s}|$ (wegen der Kompaktheit). 
		Da es nur endlich viele dieser Rechtecke gibt, folgt $|\frac{\partial h}{\partial t}|\leq C$, 
		$|\frac{\partial h}{\partial s}| \leq C$ auf ganz $R=R_0$, für ein festes $C>0$. \\
		Schreibe nahe $z_0$ die Funktion $f$ als $f(z)=f(z_0)+f'(z_0)\cdot (z-z_0)+r(z-z_0)$, wobei \\
		$\lim_{z\to z_0} |\frac{r(z-z_0)}{z-z_0}|=0$, da $f$ holomorph ist 
		(vgl. Lemma \ref{lem:komplexdiffbar}).\\
		Da $f(z_0)+f'(z_0)\cdot (z-z_0)$ das Differential der holomorphen Funktion
		$z\mapsto f(z_0)\cdot (z-z_0)+\frac{1}{2}f'(z_0)\cdot (z-z_0)^2$ ist, folgt mit Bemerkung 
		\ref{Bemerkung2}, dass 
		das Integral von $f(z_0)+f'(z_0)\cdot (z-z_0)$ über die geschlossenen Kurven $h(\partial R_n)$ 
		verschwindet. Die Länge $L$ von $h(\partial R_n)$, $L(h(\partial R_n))$ können wir abschätzen 
		durch $4\cdot 2^{-n}\cdot C$. Es folgt
		\begin{eqnarray*}
			\left| \int_{h(\partial R)}f(z)\mathrm{d}z \right| 
			&\leq& \lim_{n\to\infty} 2^{2n} \left| \int_{h(\partial R_n)}f(z)\mathrm{d}z \right| \\
			&\overset{(I)}{=}& \lim_{n\to\infty} 2^{2n} \left|\int_{h(\partial R_n)}
			r(z-z_0)\mathrm{d}z \right| \\
			&\leq& \lim_{n\to\infty} \left( 2^{2n} \cdot \sup_{h(\partial R_n)}|r(z-z_0)|\cdot 
			\underbrace{L(h(\partial R_n))}_{\leq 4\cdot C\cdot 
			2^{-n}}\right) \\
			&\leq& \lim_{n\to\infty} \left(2^n \cdot 4\cdot C \cdot \sup_{h(\partial R_n)} |r(z-z_0)| \cdot 
			\frac{|z-z_0|}{|z-z_0|}\right) \\
			&\underset{|z-z_0|\leq 2^{-n}\cdot 2\cdot C}{\leq}& \lim_{n\to\infty} \left(8\cdot C^2 \cdot
			\sup_{h(\partial R_n)} \frac{|r(z-z_0)|}{|z-z_0|}\right) \\
			&=& \underbrace{\lim_{z\to z_0} \frac{|r(z-z_0)|}{|z-z_0|}}_{=0}\cdot 8\cdot C^2 \\
			&=&0.
		\end{eqnarray*}
		Also gilt $| \int_{h(\partial R)}f(z)\mathrm{d}z |=0$ im Widerspruch zur Annahme.
	\end{proof}
	
	\begin{corollary}\label{coroll:homotop}
		Es sei $\Omega\subset\mathbb{C}$ Gebiet, $\gamma_0$, $\gamma_1$ zwei stückweise 
		$C^1$-Kurven in $\Omega$ von $p$ nach $q$, die stückweise $C^1$-homotop sind. 
		Dann gilt
		\[ \int_{\gamma_0} f(z)\mathrm{d}z=\int_{\gamma_1}f(z)\mathrm{d}z \]
	\end{corollary}
	
	\begin{proof}
		Sei $h$ eine stückweise $C^1$-Homotopie zwischen $\gamma_0$ und $\gamma_1$ in 
		$\Omega$. \\
		Betrachte $k:[0,1]^2 \to[a,b]\times[0,1]$ mit
		\[ k(u,v)=\begin{dcases} (a+(1-v)4u(b-a),0) & u\in[0,\frac{1}{4}] \\
		(a+(1-v)(b-a),4u-1) & u\in[\frac{1}{4},\frac{1}{2}] \\
		(a+(1-v)(3-4u)(b-a),1) & u\in [\frac{1}{2},\frac{3}{4}] \\
		(a,4-4u)& u\in[\frac{3}{4},1]
		\end{dcases} \]
		Die Kurve $(h\circ k)(\cdot,0):[0,1]\to\Omega$ ist geschlossen und wegen der Invarianz des 
		Kurvenintegrals unter Umparametrisierung erhalten wir 
		\begin{eqnarray*}
			\int_{(h\circ k)(\cdot,0)}f(z)\mathrm{d}z &=& \int_{\gamma_0}f(z)\mathrm{d}z +
			\underbrace{\int_q f(z)\mathrm{d}z}_{0} +\int_{\gamma_1 (-\cdot)}f(z)\mathrm{d}z +
			\int_p f(z)\mathrm{d}z \\
			&=&\int_{\gamma_0}f(z)\mathrm{d}z-\int_{\gamma_1}f(z)\mathrm{d}z
		\end{eqnarray*}
		$(h\circ k)$ ist eine Nullhomotopie dieser Kurve. also verschwindet der obige Ausdruck.
	\end{proof}
	
	\begin{theorem}[erweiterter Cauchy-Integralsatz]
		Sei $f:\Omega\to\mathbb{C}$ stetig differenzierbar und $\gamma:[0,1]\to\Omega$ umlaufe eine 
		einfach zusammenhängende Teilmenge $A\subset\Omega$ im mathematischen Drehsinn.
		Dann gilt
		\[\int_{\gamma}f(z)\mathrm{d}z =2i \int_A \frac{\partial f}{\partial \overline{z}}(z)
		\underbrace{\mathrm{d}A(z)}_{\text{Flächenelement}} \]
		(Vergleiche mit dem Satz von Stokes oder dem Gaußschen Divergenzsatz)
	\end{theorem}
	
	\begin{proof}
		Beweisskizze: Da $A$ einfach zusammenhängend ist, ist $\gamma$ in $A$ nullhomotop. \\
		Sei $h:[0,1]^2 \to A\subset\Omega$ eine Nullhomotopie. Annahme:
		\[ \left| \int_{\gamma}f(z)\mathrm{d}z -2i \int_A \frac{\partial f}{\partial \overline{z}}\mathrm{d} A(z)
		\right| =\varepsilon >0. \]
		Zerlege $[0,1]^2$ in vier gleich große Quadrate $R' ,\ldots ,R''''$. Dann gilt für eins der 
		Quadrate:
		\[ \left| \int_{h(\partial R^? )} f(z)\mathrm{d}z -2i \int_{h(R^?)}
		\frac{\partial f}{\partial \overline{z}}\mathrm{d} A(z) \right| \geq \frac{\varepsilon}{4} \]
		Nenne es $R_1$ und zerlege weiter. Erhalte eine Intervallverschachtelung mit Grenzpunkt 
		$(t_0, s_0)\in[0,1]^2$; sei $h(t_0,s_0) =:z_0\in\Omega$. Schreibe
		\[f(z)=f(z_0)+\frac{\partial f}{\partial z} (z_0)\cdot (z-z_0) +\frac{\partial f}{\partial \overline{z}}
		(z_0)\cdot \overline{(z-z_0)} +r(z-z_0). \]
		mit $\lim_{z\to z_0} \frac{r(z-z_0)}{|z-z_0|}=0$. Wir wissen, dass 
		\[ \int_{h(\partial R^n)} \left( f(z_0)+\frac{\partial f}{\partial z} (z_0)(z-z_0) \right) \mathrm{d}z =0. \]
		In einer Übung berechnen wir 
		\[ \int_{h(\partial R^n)}\frac{\partial f}{\partial \overline{z}}(z_0)\overline{(z-z_0)}\mathrm{d}z =
		2i\cdot A(h(R^n)) \]
		(falls $h(R^n)$ ein Parallelogramm ist — da $h$ stückweise $C^1$ ist, ist $h(R^n)$ $"$fast$"$ 
		ein Parallelogramm, sd. die obige Behauptung bis auf einen ausreichend kleinen Rest stimmt.)
		Außerdem gilt
		\[ \lim_{n\to\infty} \left| \int_{h(R^n)} \frac{\partial f}{\partial \overline{z}} \mathrm{d} A(z) - 
		\int_{h(R^n)} \frac{\partial f}{\partial \overline{z}}(z_0)\mathrm{d}A(z) \right| \cdot 2^{2n} =0 \]
		da $\frac{\partial f}{\partial \overline{z}}$ stetig ist. $\frac{\partial f}{\partial \overline{z}}(z_0) 
		\cdot A(h(R^n))$ Also erhalten wir einen Widerspruch genau wie im Beweis des Integralsatzes.
	\end{proof}
	
	\subsection{Die Potenzreihendarstellung}
	
	\underline{Ziel:} \begin{itemize} \item $"$holomorph$"$ und $"$ analytisch$"$ sind gleichbedeutend. 
	\item Man kann Ableitungen als Integrale schreiben.
	\item Funktionen haben Stammfunktionen genau dann, wenn sie holomorph sind.
	\end{itemize}
	
	\begin{theorem}[Cauchy-Formel]
		Es sei $\Omega\subset\mathbb{C}$ ein Gebiet. $f:\Omega\to\mathbb{C}$ holomorph, 
		$z_0\in\Omega$, $r>0$ sei so gewählt, dass $\overline{B_r (z_0)}\subset\Omega$. 
		$\gamma$ beschreibe den Rand von $B_r (z_0)$ im mathematische Drehsinn. Dann gilt 
		für all $z\in B_r (z_0)$, dass 
		\[ f(z)= \frac{1}{2 \pi i} \int_{\gamma} \frac{f(\zeta)}{\zeta -z}\mathrm{d}\zeta. \]
	\end{theorem}
	
	\begin{proof}
		$\frac{f(\zeta )}{\zeta -z}$ ist in $\zeta$ holomorph in $\Omega\setminus \{ z\}$. 
		Wähle $\varepsilon >0$ hinreichend klein, sd. $B_{\varepsilon}(z)\subset B_r (z_0)$. 
		Dann lässt sich eine in $\Omega\setminus \{ z \}$ nullhomotope Kurve $\varphi$ finden, sd.
		\[ 0= \int_{\varphi} \frac{f(\zeta)}{\zeta -z}\mathrm{d}\zeta = 
		\int_{\gamma}\frac{f(\zeta)}{\zeta -z} \mathrm{d}\zeta - \int_{\partial B_{\varepsilon}(z) }
		\frac{f(\zeta)}{\zeta - z}\mathrm{d}\zeta \]
		Berechne jetzt für $\varepsilon >0$ klein 
		\begin{eqnarray*}
			\int_{\partial B_{\varepsilon}(z)} \frac{f(\zeta)}{\zeta -z}\mathrm{d}\zeta &=& 
			\int_0^1 f(\underbrace{z+\varepsilon \cdot e^{2\pi it} }_{\zeta})\cdot 
			\frac{1}{\varepsilon \cdot e^{2\pi it}} \underbrace{2\pi i \varepsilon \cdot e^{2 \pi it}}_{=
			\dot{\varphi}(t)} \mathrm{d}t \\
			&=& 2\pi i \int_0^1 f(\underbrace{f(z)+R(\varepsilon e^{2\pi i t})}_{\text{da $f$ stetig ist, gilt 
			$R\to0$ für $\varepsilon\to 0$}})\mathrm{d}t
		\end{eqnarray*}
		$\lim_{\varepsilon\to 0} \int_{\partial B_{\varepsilon} (z)} \frac{f(\zeta)}{\zeta -z} \mathrm{d}\zeta
		=2\pi i f(z)$.
	\end{proof}
	
	\begin{corollary}[Mittelwertsatz]
		Es seien $\Omega$, $f$, $z_0$, $r$ wie oben, dann gilt 
		\[ f(z_0)= \int_0^1 f(z_0+r\cdot e^{2\pi it})\mathrm{d}t \]
		Kein Kurvenintegral und das hier ist nicht der Mittelwertsatz aus Ana 1.
	\end{corollary}
	
	\begin{proof}
		Setze $z=z_0$ in der Integralformel 
		\[ f(z_0)= \frac{1}{2\pi i} \int_0^1 f(z_0+r\cdot e^{2\pi it})\cdot \frac{1}{r e^{2\pi it}} r\cdot 2\pi i \cdot 
		e^{2\pi it}\mathrm{d}t \]
	\end{proof}
	
	\begin{example}
		Wähle $\Omega=\mathbb{C}$, $f(z)=e^z$ $z_0 =0$, $r=1$. Dann gilt
		\begin{eqnarray*}
			1&=& e^0 = \int_0^1 e^{\cos(2\pi t)+i\sin (2\pi t)}\mathrm{d}t \quad \varphi=2\pi t \\
			&=& \frac{1}{2\pi} \underbrace{\int_0^{2\pi} e^{\cos(\varphi)}\cdot \cos(\sin(\varphi)) 
			\mathrm{d}\varphi }_{2\pi} +\frac{1}{2\pi i} \underbrace{\int_0^{2\pi}e^{\cos(\varphi)}\cdot 
			\sin(\sin(\varphi))\mathrm{d}\varphi}_{0}
		\end{eqnarray*}
	\end{example}
	
	\begin{theorem}[Potenzreihenentwicklung]
		Es sei $f:\Omega\to\mathbb{C}$ holomorph und $z_0\in\Omega$. Dann konvergiert die 
		Potenzreihe $\sum_{n=0}^{\infty} a_n \cdot (z-z_0)^n$ mit $a_n = \frac{1}{2\pi i} \int_{\partial B_r 
		(z_0)} \frac{f(z)}{(z-z_0)^{n+1}}\mathrm{d}z$ (für ein $r>0$, sd. $\overline{B_r (z_0)}
		\subset\Omega$) mit Konvergenzradius $\varphi \geq \sup\{r|\overline{B_r (z_0)}
		\subset\Omega\}$ und stellt auf $B_r (z_0)$ die Funktion $f$ dar.
	\end{theorem}
	
	\begin{proof}
		\begin{eqnarray*}
			f(z) &=& \frac{1}{2\pi i} \int_{\partial B_r (z_0)} \frac{f(\zeta)}{\zeta -z}\mathrm{d}\zeta 
			= \frac{1}{2\pi i}\int_{\partial B_r (z_0)} \frac{f(\zeta)}{(\zeta - z_0)-(z-z_0)}\mathrm{d}\zeta \\
			&=& \frac{1}{2\pi i} \int_{\partial B_r (z_0)} f(\zeta)\cdot \underbrace{\sum_{n=0}^{\infty} 
			\frac{(z-z_0)^n}{(\zeta - z_0)^{n+1}}}_{\frac{1}{\zeta -z_0}\cdot\frac{1}{1-\frac{z-z_0}
			{\zeta -z_0}}}\mathrm{d}\zeta 
			= \sum_{n=0}^{\infty} (z-z_0)^n \cdot \frac{1}{2\pi i} \int_{\partial B_r (z_0)} 
			\frac{f(\zeta)}{(\zeta -z_0)^{n+1}} \mathrm{d}\zeta.
		\end{eqnarray*}
		Wir dürfen Summation und Integration vertauschen, falls $|z-z_0|<|\zeta -z_0|=r$, da dann 
		Summe und Integral absolut konvergieren. Der Konvergenzradius ist daher mindestens $r$. 
		Und zwar für jedes $r$ mit $\overline{B_r (z_0)}\subset\Omega$.
	\end{proof}
	
	\begin{corollary}\label{coroll:holo}
		Holomorphe Funktionen sind (komplex) analytisch, insbesondere $C^{\infty}$.
	\end{corollary}
	
	\begin{proof}
		Sei $f:\Omega\to\mathbb{C}$ holomorph, $z_0 \in\Omega$, $r>0$, sd. $\overline{B_r (z_0)}
		\subset\Omega$. Dann können wir $f$ auf $B_r (z_0)$ durch eine Potenzreihe darstellen.
		Insbesondere ist $f$ auf $B_r (z_0)$ analytisch (und $C^{\infty}$). 
		Da das für alle $z_0 \in\Omega$ geht, folgt die Behauptung.
	\end{proof}
	
	Somit: $"$holomorph$"$ und $"$analytisch$"$ sind gleichbedeutend. \\
	\underline{Grund:} $"$Holomorphie$"$ ist gleichbedeutend mit den 
	Cauchy-Rieman-Differentialgleichungen (Lemma \ref{lem:komplexdiffbar}). 
	Diese sind $"$elliptisch$"$ und Lösungen 
	elliptischer Differentialgleichungen sind mindestens so oft differenzierbar, wie ihre 
	Koeffizienten und ihre rechte Seite.\\
	\underline{Zur Erinnerung:} Wir haben die Rechenregeln für Potenzreihen aus Proposition 1.7 
	(Kurzskript)	
	nicht bewiesen. Mit Folgerung \ref{coroll:holo} und Proposition \ref{prop:Rechenregeln} geht der 
	Beweis recht einfach.
	
	\begin{corollary}\label{coroll:stammfkt}
		Es sei $\Omega$ einfach zusammenhängend. Dann ist $f:\Omega\to\mathbb{C}$ genau dann 
		holomorph, wenn $f$ eine Stammfunktion $F$ besitzt (das heißt $F$ ist holomorph mit 
		$F'=f$).
	\end{corollary}
	
	\begin{proof}
		$"\Leftarrow"$ Sei $F$ Stammfunktion. Da $F$ holomorph ist, ist $F$ beliebig oft komplex 
		differenzierbar, siehe Folgerung \ref{coroll:holo}. Also ist auch $f=F'$ beliebig oft komplex 
		differenzierbar, also insbesondere auch holomorph. \\
		$"\Rightarrow"$ Da $\Omega$ einfach zusammenhängend ist, sind je zwei Kurven 
		$\gamma_0$, $\gamma_1$ von $z_0\in\Omega$ nach $z\in\Omega$ homotop. Somit gilt 
		\[ \int_{\gamma_0} f(\zeta)\mathrm{d}\zeta =\int_{\gamma_1}f(\zeta)\mathrm{d}\zeta 
		\quad \text{nach Folgerung \ref{coroll:homotop}} \]
		Fixiere also $z_0$ und definiere $F=\int_{\gamma} f(\zeta)\mathrm{d}\zeta$ für eine Kurve 
		$\gamma:[0,1]\to\Omega$ mit $\gamma(0)=z_0$ und $\gamma(1)=z$. 
		Um $F'(z)$ zu berechnen, betrachte $\omega$ nahe $z$ und eine Kurve $\gamma$ von $z_0$ 
		nach $\omega$ der Form (siehe Skizze Skriptum Niklas) \\
		Dann gilt:
		\begin{eqnarray*}
			\lim_{\omega\to z}\frac{F(\omega)-F(z)}{\omega -z} &=&
			\lim_{\omega\to z}\frac{1}{\omega -z}\left( \int_{\gamma_{\omega}}f(\zeta)\mathrm{d}\zeta -
			\int_{\gamma_z}f(\zeta)\mathrm{d}\zeta \right) \\
			&=&\lim_{\omega\to z}\frac{1}{\omega -z}\int_{\gamma_{\omega -z}}
			f(\zeta)\mathrm{d}\zeta \\
			&=&\lim_{\omega\to z}\frac{1}{\omega -z}\int_0^1 f(\gamma_{\omega -z}(t))
			\underbrace{\omega -z}_{=\dot{\gamma}_{\omega -z}(t)}\mathrm{d}t \\
			&=& f(z).
		\end{eqnarray*}
		$\Rightarrow$ $F$ ist eine Stammfunktion.
	\end{proof}
	
	Zur Erinnerung: Identitätssatz für Potenzreihen.
	
	\begin{corollary}[Identitätssatz für holomorphe Funktionen]
		Es sei $\Omega\subset\mathbb{C}$ ein Gebiet, $f,g:\Omega\to\mathbb{C}$ holomorph. Falls 
		eine Teilmenge $A\subset\Omega$ mit Häufungspunkt $z\in\Omega$ existiert, sd. 
		$f|_A =g|_A$, dann gilt $f=g$ auf ganz $\Omega$.
	\end{corollary}
	
	\begin{proof} 
		Nach Folgerung \ref{coroll:holo} sind $f$ und $g$ analytisch. \\
		$"A$ hat Häufungspunkt $z"$ $\Leftrightarrow$ Es existiert eine Folge $(z_n)_{n\in\mathbb{N}}
		\subset A\setminus\{ z\}$, sd. $z_n\xrightarrow{n\to\infty} z$. \\
		Es folgt $(g-f)(z_n)=0$ für alle $n$ und nach dem Identitätssatz für Potenzreihen bzw. 
		analytische Funktionen gilt somit $g-f=0$ auf ganz $\Omega$.
	\end{proof}
	
	Der Identitätssatz ermöglicht es manche aus dem reellen bekannten Funktionen auf $\mathbb{C}$ 
	zu übertragen und ihre Eigenschaften zu verstehen.
	
	\begin{example}\label{bsp:sin}
		Es gilt für $x\in\mathbb{R}$, dass
		\[ \sin(x)=\frac{e^{ix}-e^{-ix}}{2i}=\sum_{n=0}^{\infty}(-1)^n \frac{x^{2n+1}}{(2n+1)!} \]
		Wir können $z\in\mathbb{C}$ in die Potenzreihenentwicklung einsetzen. Da der 
		Konvergenzradius $\infty$ ist, erhalten wir eine Funktion $\sin:\mathbb{C}\to\mathbb{C}$. 
		Da die Identitäten 
		\begin{eqnarray}
			\sin(z)&=&\frac{e^{iz}-e^{-iz}}{2i} \nonumber \\
			\sin(z+w)&=&\sin(z)\cos(w)+\cos(z)\sin(w) \\
			\sin''(z)&=&-\sin(z) \nonumber
		\end{eqnarray}
		für alle $x\in\mathbb{R}$ gelten, gelten sie nach dem Identitätssatz für alle $z$, $w$ aus 
		$\mathbb{C}$. \\
		Zu (1) [Additionstheorem]: Nehme zunächst $w\in\mathbb{R}$ als Konstante an, dann folgt das 
		Additionstheorem für alle $z\in\mathbb{C}$, $w\in\mathbb{R}$. Nehme nun $z\in\mathbb{C}$ 
		konstant an, erhalte Additionstheorem für alle $z$, $w\in\mathbb{C}$. \\
		Definiere die Hyperbelfunktion $\cosh$, $\sinh$ durch
		\begin{eqnarray*}
			\cosh(z)&=&\cos(iz)=\frac{e^{-z}+e^z}{2} \\
			\sinh(z)&=&\frac{\sin(iz)}{i}=\frac{e^{-z}-e^z}{-2} =\frac{e^z -e^{-z}}{2}
		\end{eqnarray*}
		Auf der anderen Seite verhindert der Identitätssatz die Existenz holomorpher Fortsetzungen 
		von reellen Funktionen mit bestimmten Eigenschaften.
	\end{example}
	
	\begin{example}
		\begin{enumerate}
			\item Es gibt kein Gebiet $\Omega$ mit $\mathbb{R}\setminus\{0\}\subset\Omega$ 
			und sich die Funktion $x\mapsto |x|$ auf $\Omega$ fortsetzen ließe. \\
			Denn: wäre $f$ eine Fortsetzung, dann wäre $f(z)=z$ auf $(0,\infty)\subset\mathbb{R}$ 
			und daher auf ganz $\Omega$.
			\item Betrachte 
			\[ f(x)=\begin{dcases} 0 & x\leq 0 \\ e^{-\frac{1}{x}} & x>0. \end{dcases} \]
			Diese Funktion ist $C^{\infty}$ und bei $x=0$ verschwinden alle Ableitungen. Sie ist nicht 
			analytisch bei $x=0$ und hat daher keine holomorphe Fortsetzung.
		\end{enumerate}
	\end{example}
	
	\begin{theorem}[Morera]\label{thm:morera}
		Es sei $\Omega\subset\mathbb{C}$ ein Gebiet und $f:\Omega\to\mathbb{C}$ stetig, sd. das 
		Kurvenintegral von $f$ über den Rand eines jeden Dreiecks, das ganz in $\Omega$ liegt 
		verschwindet. Dann ist $f$ holomorph.
	\end{theorem}
	
	\begin{proof}
		Benutze Folgerung \ref{coroll:stammfkt} auf kleinen Bällen $B_r (z_0)\subset\Omega$ für 
		$z_0\in\Omega$ und $r>0$ ausreichend klein. \\
		Definiere jetzt $F(z)=\int_{\gamma_z}f(\zeta)\mathrm{d}\zeta$, wobei $z\in B_r (z_0)$ und 
		$\gamma (t)=z+t(\omega-z)$. Argumentiere wie in Folgerung \ref{coroll:stammfkt}, dass 
		$F'(z)=f(z)$, allerdings benutzen wir diesmal:
		\[ \int_{\gamma_{\omega}}f(\zeta)\mathrm{d}\zeta -\int_{\gamma_z}f(\zeta)\mathrm{d}\zeta 
		=\int_{\gamma_z}f(\zeta)\mathrm{d}\zeta +\int_{\gamma_{\omega -z}}f(\zeta)\mathrm{d}\zeta 
		-\int_{\gamma_z}f(\zeta)\mathrm{d}\zeta =\int_{\gamma_{\omega -z}}f(\zeta)\mathrm{d}\zeta. \]
		$\Rightarrow$ $F'=f$ auf $B_r (z_0)$. \\
		Da $z_0\in\Omega$ und $r>0$ beliebig waren, ist $f$ auf $\Omega$ holomorph.
	\end{proof}
	
	\begin{theorem}[Schwarzsches Spiegelungsprinzip]
		Es sei $\Omega\subset\mathbb{C}$ symmetrisch bezüglich $\mathbb{R}$ (dh. $z\in\Omega\ 
		\Leftrightarrow\ \overline{z}\in\Omega$). Schreibe $\Omega_+ =\{z\in\Omega | \text{Im } z >0\}$, 
		$\Omega_0 =\Omega\cap\mathbb{R}$ und $\Omega_- =\{z\in\Omega | \text{Im } z<0\}$.
		Sei $f:\Omega_+ \cup \Omega_0 \to\mathbb{C}$ stetig, sd. $f|_{\Omega_+}$ holomorph 
		und $f|_{\Omega_0}$ reellwertig ist. Dann existiert eine holomorphe Fortsetzung 
		$f:\Omega\to\mathbb{C}$ mit $f(\overline{z})=\overline{f(z)}$.
	\end{theorem}
	
	\begin{proof}
		Definiere $f(z)=\overline{f(z)}$ für $z\in\Omega$, dann ist $f$ auf ganz $\Omega$ stetig. 
		Zeige jetzt, dass die Voraussetzungen des Satzes von Morera gelten.
		\begin{enumerate}
			\item Für jedes Dreieck in $\Omega_+$ stimmt die Behauptung
			\item Sei $\triangle\subset\Omega_+ \cup\Omega_0$ ein Dreieck. Dann betrachte 
			Dreiecke $\triangle_n\subset\Omega_+$, die dagegen konvergieren. Da das Integral 
			stetig vom Integranden abhängt (glm. stetig gilt, da $\triangle$-Fläche kompakt ist), ist 
			auch das Integral über den Rand von $\triangle$ gleich $0$.
			\item Falls $\triangle\subset\Omega\subset\Omega_- \cup\Omega_0$ liegt, berechne 
			\begin{eqnarray*}
				\int_{\gamma}f(z)\mathrm{d}z&=&\int_a^b f(\gamma(t))\dot{\gamma}(t)\mathrm{d}t =
				\int_a^b \overline{f(\overline{\gamma(t)})} \ 
				\overline{\dot{\overline{\gamma}}(t)}\mathrm{d}t \\
				&=& \overline{\int_a^b f(\overline{\gamma(t)}) \dot{\overline{\gamma}}(t)\mathrm{d}t}
				=0,
			\end{eqnarray*}
			falls $\gamma$ den Rand von $\triangle$ beschreibt.
			\item $\triangle$ erstreckt sich über alle Dreiecke. Dann zerfällt $\triangle$ in höchstens 
			3 Dreiecke vom Typ (1)-(3). Jetzt folgt Homotopie aus Satz \ref{thm:morera}.
		\end{enumerate}
	\end{proof}
	
	\begin{example}
		$\sin$ aus Beispiel \ref{bsp:sin}.
	\end{example}
	
	\begin{remark}
		Es sei $g$ auf $\partial B_r (z_0)$ stetig. Dan können wir 
		\[ f(z)=\frac{1}{2\pi i}\int_{\partial B_r (z_0)} \frac{g(\zeta)}{\zeta -z}\mathrm{d}\zeta \]
		für alle $z\in B_r(z_0)$ definieren. \\
		Frage: Setzt $f$ die Funktion $g$ stetig fort? \\
		(Beachte: $\partial B_r(z_0)$ ist im schlimmsten Fall der Rand des Konvergenzkreises...) \\
		Falls ja, wäre auch $f(z)\cdot(z-z_0)^k$ holomorph für alle $k\geq 0$ und somit hätten wir 
		nach dem Integralsatz
		\[\frac{1}{2\pi i}\int_{\partial B_r(z_0)}f(\zeta)\underbrace{(\zeta-z_0)^k}_{z_0 +re^{it}}
		\mathrm{d}\zeta =0. \]
		Das bedeutet, dass $"$ungefähr die Hälte$"$ der Fourierzerlegung von $t\mapsto 
		g(z_0 +r\cdot e^{it})$ verschwindet.
	\end{remark}
	
	\section{Abbildungsverhalten holomorpher Funktionen}
	
	Aus der reellen Analysis: Zwischenwertsatz (Bilder von Invervallen sind Intervalle) lokaler 
	Umkehrsatz für $f:U\to\mathbb{R}^n$, $U\subset\mathbb{R}^n$ 
	\begin{enumerate}
		\item Funktionen auf $\Omega\subset\{z_0\}$
		\item Maximumsprinzip \& Satz von Liouville
		\item lokaler Umkehrsatz / Blättersatz
	\end{enumerate}
	$\Omega$ ist stets ein Gebiet in $\mathbb{C}$ und $f$ (falls nicht anders gesagt) stets holomorph.
	
	\subsection{Nullstellen und isolierte Singularitäten}
	
	\begin{definition}[Nullstellen-Ordnung]
		Für $z_0\in\Omega$ wende $f:\Omega\to\mathbb{C}$ in einer Umgebung $U\subset\Omega$ 
		von $z_0$ dargestellt durch
		\[ f(z)=\sum_{n=0}^{\infty} a_n (z-z_0)^n .\]
		Die (Nullstellen-) Ordnung von $f$ bei $z_0$ ist die kleinste Natürliche $n_0 =ord_{z_0}(l)$, sd. 
		$a_{n_0} \neq 0$ und $a_n =0$ für alle $0\leq n<n_0$. \\
		Falls $ord_{z_0} (f)>0$ ist, hat $f$ bei $z_0$ eine Nullstelle der Ordnung $ord_{z_0} (f)$.
	\end{definition}
	
	\begin{example}
		\begin{enumerate}
			\item Die Sinus-Funktion hat um $0$ die Entwicklung
			\[ \sin(z)=\sum_{n=0}^{\infty} (-1)^n \frac{ z^{2n+1}}{(2n+1)!},\ \text{also $ord_0 (\sin)=1$.}\]
			also $ord_0 (\sin)=1$. Da $\sin(\pi -z)=\sin(z)$ folgt $ord_{\pi}(sin)=1$. \\
			Da $\sin(2\pi +z)=\sin(z)$ folgt $ord_{k\pi}(\sin)=1$ für alle $k\in\mathbb{Z}$ (Ansonsten 
			hat der Sinus keine Nullstellen – siehe Übung zu $\cos$).
			\item Der Cosinus hat Nullstellen der Ordnung $1$ bei $(k+\frac{1}{\varepsilon})\cdot \pi$, 
			$k\in\mathbb{Z}$.
		\end{enumerate}
	\end{example}
	
	\begin{definition}[isolierte bzw. hebbare/, wesentliche Singularität]
		Es sei $z_0\in\Omega$ und $f:\Omega\setminus\{z_0\}\to\mathbb{C}$ holomorph, dann heißt 
		$z_0$ eine isolierte Singularität von $f$.
		\begin{enumerate}
			\item Wenn sich $f$ zu einer holomorphen Funktion auf ganz  $\Omega$ forsetzen lässt, 
			heißt $z_0$ eine hebbare Singularität.
			\item Wenn es $m>1$ und Zahlen $a_n ,\ldots ,a_m \in\mathbb{C}$ mit $a_m \neq 0$ gibt, 
			sd. \[f(z)=\sum_{n=1}^m \frac{a_n}{(z-z_0)^n} \] bei $z_0$ eine hebbare Singularität hat, 
			dann hat $f$ bei $z_0$ eine %...
			\item Wenn für alle $r>0$ mit $B_r (z_0)\subset\Omega$ das Bild 
			$im (f|_{B_r(z_0)\setminus\{z_0\}})$ dicht in $\mathbb{C}$ liegt, heißt $z_0$ eine 
			wesentliche Singularität von $f$ und wir setzen $ord_{z_0}(f)=-\infty$.
		\end{enumerate}
		(Der Vollständigkeit halber sei $ord_{z_0}(0)=\infty$).
	\end{definition}
	
	\begin{example}
		\begin{enumerate}
			\item Der Tangens $\tan(z)=\frac{\sin(z)}{\cos(z)}$ hat Nullstellen der Ordnung $1$ bei 
			$z=k\pi$, $k\in\mathbb{N}$. Bei $z_0 =\frac{\pi}{2}$ schreiben wir
			\begin{eqnarray*}
				-\sin(z-\frac{\pi}{2})=\sin(\frac{\pi}{2}-z) =\cos(z)=-(z-\frac{\pi}{2})\cdot 
				\underbrace{\sum_{n=0}^{\infty} (-1)^n \frac{(z-\frac{\pi}{2})^{2n}}{(2n+1)!}}_{=f(z)
				\text{ holom. }f(\frac{\pi}{2})\neq 0}
			\end{eqnarray*}
			Da $\tan(z+\pi)=\tan(z)$, hat $\tan$ bei $(k+\frac{1}{2})\pi$ ebenfalls einen Pol der 
			Ordnung 1.
			\[ \tan(z) =-\frac{\sin(z)}{(z-\frac{\pi}{2})-f(z)}=-\frac{1}{z-\frac{\pi}{2}}+\underbrace{ 
			\frac{f(z)-\sin(z)}{(z-\frac{\pi}{2})\cdot f(z)}}_{g(z)} \]
			Da $f(z)-\sin(z)$ bei $z=\frac{\pi}{2}$ den Wert $0$ hat, gilt
			\[ f(z)-\sin(z)=\sum_{n=1}^{\infty} b_n \cdot (z-\frac{\pi}{2})^n\]
			und den obigen Bruch kürzen, somit hat $g(z)$ bei $\frac{\pi}{2}$ eine hebbare 
			Singularität. 
			Also hat $\tan(z)$ bei $z=\frac{\pi}{2}$ einen Pol der Ordnung 1 mit Hauptteil 
			$-\frac{1}{z-\frac{\pi}{2}}$ und daher $ord_{\frac{\pi}{2}}(\tan)=1$.
			\item Die Funktion $z\mapsto e^{-\frac{1}{z}}$ hat bei $z=0$ eine wesentliche 
			Singularität. Sei etwa $r>0$, dann ex. $n\in\mathbb{N}$ sd. $\frac{1}{2\pi n}<r$. Dann 
			betrachte $U=\{ \omega\in\mathbb{C} | Im (\omega)\in (2\pi n,2\pi (n+1)]\}$. 
			Aus $\omega\in U$ folgt $|\frac{1}{\omega}|<r$. Auf $U$ nimmt die Exponentialfunktion 
			alle Werte in $\mathbb{C}\setminus\{ 0\}$ an: jeder der Werte hat die Form 
			$s\cdot e^{i\varphi}=e^{\log s+i\varphi}$, $\OE \varphi\in (2\pi n,2\pi (n+1)]$. 
			Da $-\frac{1}{\omega}\in B_r(0)$ für alle $\omega\in U$ nimmt $e^{-\frac{1}{z}}$ auf 
			$B_r^{\times}(0)=B_r(0)\setminus\{0\}$ alle Werte in $\mathbb{C}^{\times} =\mathbb{C} 
			\setminus\{0\}$ an. Also ist $0$ wesentliche Singularität.
			\item Das gleiche gilt für $e^{-\frac{1}{z^2}}$, obwohl diese Funktion auf 
			$\mathbb{R}$ eine hebbare Singularität bei $0$ hat.
		\end{enumerate}
	\end{example}
	
	\begin{theorem}[Riemannscher Hebbarkeitssatz]
		Wenn $f:\Omega\setminus\{z_0\}\to\mathbb{C}$ die Eigenschaft 
		\[\lim_{z\to z_0} (z-z_0)\cdot f(z) =0\]
		hat, dann hat $f$ bei $z_0$ eine hebbare Singularität.
	\end{theorem}
	
	\begin{proof}
		Betrachte die Funktion $g(z)=(z-z_0)^2 \cdot f(z)$ auf $\Omega\setminus\{z_0\}$. Dann ist 
		$g$ stetig auf $\Omega$ fortsetzbar durch $g(z_0)=0$. Außerdem ist $g$ auf 
		$\Omega\setminus\{z_0\}$ holomorph und sogar bei $z_0$ mit $g'(z_0)=0$, denn:
		\[ \lim_{z\to z_0} \frac{g(z)-g(z_0)}{z-z_0} = \lim_{z\to z_0} (z-z_0)\cdot f(z) =0. \]
		Also ist $g$ holomorph und hat daher bei $z_0$ eine Potenzreihendarstellung 
		\[g(z)=\sum_{n=0}^{\infty} a_n (z-z_0)^n \] mit $a_0=a_1=0$. Somit lässt sich $f$ bei $z_0$ 
		zu einer holomorphen Funktion mit Potenzreihe
		\[f(z)=\sum_{n=0}^{\infty} a_{n+2}(z-z_0)^n \]
		fortsetzen.
	\end{proof}
	
	\begin{example}
		Es sei $r>0$. Dann gibt es keine holomorphe Funktion $f$ auf $B_r^{\times}(0)$ mit $f(z)^2 =z$. 
		Denn wäre $f$ eine solche Funktion, dann wäre $|f(z)|=\sqrt{|z|}$, also 
		\[\lim_{z\to z_0} (z-z_0)\cdot f(z) =0. \]
		Das heißt $f$ müsste sich holomorph auf $B_r(0)$ fortsetzen lassen, aber das geht nicht, da die 
		reelle Wurzelfunktion bei $x=0$ bereits nicht differenzierbar ist.
	\end{example}
	
	\begin{remark}
		Wir können $ord_{z_0}(f)$ wie folgt charakterisieren: $n=ord_{z_0}(f)$ ist die kleinste ganze 
		Zahl, sd.
		\[ g(z)= (z-z_0)^{-n} f(z) \]
		bei $z_0$ eine hebbare Singularität hat. 
		\begin{enumerate}
			\item $f$ hat hebbare Singularität $\Rightarrow$ $ord_{z_0}(f)\geq 0$ und $g$ ist nahe 
			$z_0$ beschränkt für $n=ord_{z_0}(f)$, (siehe Potenzreihenentwicklung), hat also 
			hebbare Singularität, wohingegen für $n=ord_{z_0}(f)+1$ die Funktion $g$ nahe 
			$z_0$ nich einmal beschränkt ist.
			\item Wenn $f$ einen Pol hat, habe
			\[ h(z)=f(z)-\sum_{n=1}^m \frac{a_n}{(z-z_0)^n} \]
			mit $m=-ord_{z_0}(f)$ und $a_m \neq 0$ eine hebbare Singularität. Also hat 
			$(z-z_0)^m \cdot f(z)$ bei $z_0$ hebbare Singularität, $(z-z_0)^{m-1}\cdot f(z)$ jedoch 
			nicht.
		\end{enumerate}
	\end{remark}
	
	\begin{theorem}[Casorati-Weierstraß]
		Sei $f:\Omega\setminus\{z_0\}\to\mathbb{C}$ holomorph, dann trifft genau eine der folgenden 
		Aussagen zu.
		\begin{enumerate}
			\item $f$ hat eine hebbare Singularität bei $z_0$
			\item $f$ hat eine Polstelle bei $z_0$
			\item $f$ hat eine wesentliche Singularität bei $z_0$
		\end{enumerate}
	\end{theorem}
	
	\begin{proof}
		Klar: (1) und (2) schließen einander aus. \\
		(3) schließt (1) und (2) aus: \\
		(1)$\Rightarrow$ $\lim_{z\to z_0} f(z)=a\in\mathbb{C}$ existiert, zu jedem $\delta >0$ existiert 
		also ein $\varepsilon >0$, sd. $im(f|_{B_{\varepsilon}^{\times}(z_0)})\subset B_{\delta}(a)$ 
		Insbesondere liegt dieses Bild nicht dicht in $\mathbb{C}$. \\
		(2)$\Rightarrow$ $\lim_{z\to z_0} |f(z)| =\infty$, dh. zu jedem $\delta >0$ ex. $\varepsilon >0$, 
		sd. $im(f|_{B_{\varepsilon}^{\times}(z_0)})\subset\mathbb{C}\setminus B_{\frac{1}{\delta}}(0)$ 
		Insbesondere liegt das Bild nicht dicht in $\mathbb{C}$. \\
		Noch zeigen: Wenn das Bild von $f|_{B_r^{\times} (z_0)}$ nicht dicht liegt, hat $f$ einen Pol 
		oder eine hebbare Singularität. Wenn $im(f|_{B_r^{\times}(z_0)})$ nicht dicht in 
		$\mathbb{C}$ ist, ex. $b\in\mathbb{C}\setminus\overline{im(f|_{B_r^{\times}(z_0)})}$, dh. es ex. 
		$\varepsilon >0$, sd. $B_{\varepsilon}(b)\cap im(f|_{B_r^{\times}(z_0)})=\emptyset.$ 
		Betrachte die Funktion $g:\Omega\setminus\{z_0\}\to\mathbb{C}$ mit 
		\[ g(z) =\frac{1}{f(z)-b} \]
		Dann ist $|g(z)| < \frac{1}{\varepsilon}$ auf $B_r^{\times}(z_0)$, somit hat $g$ eine holomorphe 
		Fortsetzung auf ganz $B_r (z_0)$. Also hat $f(z)=\frac{1}{g(z)}+b$ einen Pol oder eine 
		hebbare Singularität.
	\end{proof}
\end{document}