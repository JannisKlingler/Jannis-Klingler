\documentclass[11pt,titlepage]{article}
\usepackage{amsmath,amssymb,amstext,mathtools,amsthm}
\usepackage{amssymb}
\usepackage{xcolor}
\usepackage[utf8]{inputenc}
\usepackage[ngerman]{babel}
\usepackage[paper=a4paper,left=25mm,right=25mm,top=25mm,bottom=25mm]{geometry}
\usepackage{hyperref}
\hypersetup{bookmarksnumbered}

\usepackage{dsfont}
\usepackage{xfrac}
\usepackage{tikz}
\usepackage{bigints}

\usetikzlibrary{positioning}
\usetikzlibrary{arrows}

\theoremstyle{definition}
\newtheorem{theorem}{Satz}[section]
\newtheorem{corollary}[theorem]{Folgerung}
\newtheorem{proposition}[theorem]{Proposition}
\newtheorem{lemma}[theorem]{Lemma}
\newtheorem{definition}[theorem]{Definition}
\newtheorem{example}[theorem]{Beispiel}
\newtheorem*{axiom}{Axiom}
\newtheorem{remark}[theorem]{Bemerkung}

\theoremstyle{remark}
\newtheorem*{repetition}{Wiederholung}
\newtheorem*{remind}{Erinnerung}

\title{Funktionentheorie}
\author{Jannis Klingler}
\date{\today}

\begin{document}

	\maketitle
	
	\newpage
	\tableofcontents
	\newpage

	\section{Holomorphe und analytische Funktionen}

	\subsection{Analytische Funktionen}
	
	\begin{repetition}
		Setze $\mathbb{C}=\mathbb{R}^2$. Für $z=(x,y)$, $w=(u,v)$ definiere:
		\begin{eqnarray*}
			z+w&=&(x+u,y+v) \quad\text{Vektoraddition} \\
			z\cdot w&=&(x\cdot u-y\cdot v,x\cdot v+y\cdot u) \\
			0&=&(0,0) \qquad\text{neutrales Element $(+)$}\\
			1&=&(1,1) \qquad\text{neutrales Element $(\cdot )$}\\
			i&=&(0,1)
		\end{eqnarray*}
		Komplexe Konjugation: $z\to \overline{z} =(x,-y)$ ist ein Automorphismus, dh.
		\begin{eqnarray*}
			\overline{z+w} &=& \overline{z} +\overline{w} \\
			\overline{z\cdot w} &=& \overline{z} \cdot \overline{w} \\
			\overline{0} &=& 0\\
			\overline{1} &=& 1 \\
			\overline{i} &=& (0,1)
		\end{eqnarray*}
		Mit diesen Operationen ist $\mathbb{C}$ ein Körper.
		\begin{eqnarray*}
			-z=(-x,-y) \qquad \qquad \frac{1}{z}=\frac{\overline{z}}{z\cdot \overline{z}}=\left(
			\frac{x}{x^2 +y^2}-\frac{y}{x^2 +y^2} \right)
		\end{eqnarray*}
		wir definieren einen Absolutbetrag $|z| = \sqrt{z\overline{z}}\in \mathbb{R}$, denn 
		$z\cdot\overline{z} \in \mathbb{R} =\{ z\in \mathbb{C} \mid z=\overline{z} \} =
		\{ (x,0)\mid x\in\mathbb{R} \} \subset \mathbb{C}$ \\
		Jetzt können wir schreiben $z=(x,y)=(x,0)+(y,0)=(x,0)+i\cdot (y,0)=x+iy$ \\
		Graphische Darstellung ("Gaußsche Zahlenebene").
	\end{repetition}
	
	\textbf{\underline{Zur Erinnerung}:}
	
	\begin{definition}[Topologischer Raum]
		Ein topologischer Raum heißt zusammenhängend, wenn er nicht als disjunkte Vereinigung zweier 
		nichtleerer, offener Teilmengen geschrieben werden kann.
	\end{definition}
	
	\begin{definition}[Wegzusammenhängend]
		Ein topologischer Raum $X$ heißt wegzusammenhängend, wenn es zu je zwei Punkten 
		$p,q\in X$ eine stetige Abbildung $\gamma :[0,1]\to X$ mit $\gamma (0)=p$, $\gamma (1)=q$ 
		gibt.
	\end{definition}
	
	\begin{theorem}
		Eine offene Teilmenge von $\mathbb{C}$ ist genau dann zusammenhängend, wenn sie 
		wegzusammenhängend ist.
	\end{theorem}
	\begin{proof}
		$"\Leftarrow"$: Sei $X$ wegzusammenhängend. Seien $U,V\subset X$ offen, $X=U\cup V$, 
		$p\in U$, $q\in V$ (also $U,V$ nicht leer). Dann existiert $\gamma :[0,1]\to X$ stetig mit 
		$\gamma (0)=p$, $\gamma (1)=q$. Dann sind $\gamma ^{-1}(U),\ \gamma^{-1}(V)\subset 
    		[0,1]$ offen. Da $[0,1]$ zusammenhängend ist und $0\in \gamma^{-1}(U)$, 
		$1\in \gamma^{-1}(V)$, \\
		$\gamma^{-1}(U)\cup\gamma^{-1}(V)=\gamma^{-1}(U\cup V)=\gamma^{-1}(X)=[0,1]$ folgt 
		$\gamma^{-1}(U)\cap \gamma^{-1}(V) \neq \emptyset$. \\Also existiert $t\in \gamma^{-1}(U)
		\cap\gamma^{-1}(V)$ und $\gamma(t)\in U\cap V$. Da das für alle offenen, nichtleeren 
		Teilmengen $U,V$ mit $U\cup V=X$ gilt, ist X zusammenhängend.\\
		Einfacher: \\
		Angenommen $X$ ist nicht zusammenhängend. Dann existieren offene, nicht-leere Teilmengen 
		$U,V\subset X$ mit $U\cup V=X$, $U\cap V=\emptyset$. Dann existiert eine stetige Funktion 
		$f:X\to \mathbb{R}$ mit 
		\[ f(x)= \begin{dcases} 0 & x\in U \\ 1& x\in V \end{dcases} \]
		Wähle jetzt $p\in U$, $q\in V$. Gäbe es einen Weg $\gamma : [0,1]\to X$ mit $\gamma(0)=p$, 
		$\gamma(1)=q$, dann wäre $f\circ \gamma :[0,1]\to \mathbb{R}$ stetig, im Widerspruch zum 
		Zwischenwertsatz. \\
		$"\Rightarrow"$: Sei $X\subset \mathbb{C}$ (offen) zusammenhängend. \\Sei $p\in X$ und sei 
		$U=\{ q\in X \mid \exists \gamma :[0,1]\to X \text{ stetig}:\gamma(0)=p,\ \gamma(1)=q \}$\\
		Behauptung: $U$ ist offen, also existiert $\varepsilon>0$, sd. $B_{\varepsilon}(q)\subset X$. 
		Sei $q'\in B_{\varepsilon}(q)$. Dann existiert $\gamma':[0,1]\to X$, sd. 
		\[ \gamma'(t)= \begin{dcases} \gamma(2t) & 0\leq t\leq \frac{1}{2} \\ (2-2t)q + (2t-1)q' & 
			\frac{1}{2} \leq t\leq 1 \end{dcases} \]
		$\Rightarrow$ $B_{\varepsilon}(q)\subset U$ $\Rightarrow$ $U$ offen.\\
		Behauptung: $X\setminus U$ ist offen: \\
		Sei $q\in X\setminus U$. Da $X$ offen, existiert $\varepsilon >0$ mit 
		$B_{\varepsilon}(q)\subset X$. Wäre $B_{\varepsilon}(q)\cap U \neq \emptyset$, so existiert 
		$q'\in B_{\varepsilon}(q)\cap U$, ein Weg $\gamma$ von $p$ nach $q$ in $X$ und mit einer 
		ähnlichen Konstruktion auch eine Kurve $\gamma'$ von $p$ nach $q$. Also auch 
		$X\setminus U = \emptyset$.\\
		$\Rightarrow$ $X$ ist wegzusammenhängend.	
	\end{proof}
	
	\begin{definition}[Gebiet]
		Ein Gebiet ist eine offene, zusammenhängende Teilmenge von $\mathbb{C}$.
	\end{definition}
	
	\begin{remind}
		Eine (komplexe) Potenzreihe ist ein Ausdruck der Form $R(z)=\sum^{\infty}_{n=0} a_n z^n$ mit 
		$a_n \in \mathbb{C}$ für alle $n$. Sie hat den Konvergenzradius $\rho = \left( \limsup_{n\to\infty}
		\sqrt[n]{|a_n |}\right)^{-1} \in [0,\infty]$. Dann:
		\begin{eqnarray*}
			R(z)\text{ konvergiert für alle $z$ mit }|z|< \rho \\
			R(z)\text{ divergiert für alle $z$ mit }|z|> \rho 
		\end{eqnarray*}
		wenn $\rho>0$ ist, heißt $R(z)$ konvergent und $B_{\rho}(0)\subset \mathbb{C}$ der 
		Konvergenzkreis.
	\end{remind}
	
	\begin{definition}[Analytische Funktion]
		Es sei $\Omega\in\mathbb{C}$ ein Gebiet und $f:\Omega\to\mathbb{C}$ eine Abbildung. 
		Dann heißt $f$ eine analytische Funktion (auf $\Omega$), wenn es zu jedem Punkt 
		$z_0 \in \Omega$ eine Potenzreihe $R(z)$ mit Konvergenzradius $\rho>0$ existiert, sd. 
		$f(z)=R(z-z_0 )$ für alle $z\in \Omega\cap B_{\rho}(z_0)$.
	\end{definition}
	
	\begin{example}
		Betrachte die Exponentialreihe
		\begin{eqnarray*}
			e^z = \sum_{n=0}^{\infty} \frac{z^n}{n!} 
		\end{eqnarray*}
		$\limsup \sqrt[n]{|\frac{1}{n!}|} =0 \quad \Rightarrow$ Konvergenzradius ist $\rho=\infty$.
		Mit dem Umordnungssatz zeigt man 
		\begin{eqnarray*}
			e^{z+w} =e^z \cdot e^w
		\end{eqnarray*}
		Da die Exponentialreihe reelle Koeffizienten hat, gilt
		\[ \overline{e^z} =\sum_{n=0}^{\infty} \overline{\left( \frac{z^n}{n!} \right)} = 
		\sum_{n=0}^{\infty} \frac{ \overline{z}^n}{n!} = e^{\overline{z}} \]
		Sei jetzt $z=x+iy$, dann gilt 
		\[e^z = e^x \cdot e^{iy} \]
		und $|e^{iy}|^2 = e^{iy} \cdot \overline{e^{iy}} = e^{iy} \cdot e^{-iy} = e^0 = 1$. \\
		Also definiere $e^{iy}=\cos(y)+i\sin (y)$.\\
		Jetzt kann man komplexe Multiplikation in Polarkoordinaten verstehen. \\
		Schreibe $z=r\cdot e^{i\varphi}$, $w=s\cdot e^{i\varphi}$ dann heißt $r=|z|$ der Absolutbetrag
		 und 
		$\varphi\in \mathbb{R}\setminus 2\pi \mathbb{Z}$ das Argument. \\
		Wir repräsentieren $\varphi$ durch die Funktion $arg:\mathbb{C}^{\times} =\mathbb{C}\setminus 
		\{ 0\} \to (-\pi ,\pi]$. \\
		$z\cdot w= r\cdot e^{i\varphi} \cdot s \cdot e^{i\psi} = (rs)\cdot e^{i(\varphi+\psi)}$.
	\end{example}
	
	\begin{theorem}[Identitätssatz für Potenzreihen]
		Es sei $\Omega \subset \mathbb{C}$ Gebiet und $f:\Omega\to \mathbb{C}$ analytisch. 
		Falls es $z_0 \in \Omega$ und eine Folge $(z_n)_{n\in \mathbb{N}}$ in $\Omega\setminus\{
		z_0\}$ mit $\lim_{n\to \infty} z_n = z_0$ gibt, sd. $f(z_n)=0$ für alle $n$, dann ist $f=0$ konstant.
	\end{theorem}
	
	\begin{corollary}
		Seien $f,g$ zwei analytische Funktionen auf $\Omega$, $z_0$, $(z_n)_{n\in\mathbb{N}}$ wie 
		oben, aber mit $f(z_n)=g(z_n)$ für alle $n$, dann folgt $f=g$ auf ganz $\Omega$.
	\end{corollary}
	
	\begin{definition}
		$f$ heißt analytisch auf $\Omega$, wenn es zu jedem Punkt $z\in \Omega$ eine Umgebung 
		$U\subset\Omega$ von $z$ und eine Potenzreihe $R$ um $z$ gibt, die auf ganz $U$ 
		konvergiert, sd. $R(\omega)=f(\omega)$ für alle $\omega\in\Omega$.
	\end{definition}
	
	\begin{proof}
		Sei zunächst $U$ Umgebung von $z$, auf der $f$ mit einer Potenzreihe $R(z)=\sum_{n=0}^
		{\infty} a_n (z-z_n)$ übereinstimmt. \\
		Ohne Einschränkung sei $z_0 =0$. Da $R$ konvergiert, gilt $\rho >0$, also $\infty > \frac{1}{\rho}
		=\limsup_{n\to\infty} \sqrt[n]{|a_n|}$. Also existiert $n_0\in \mathbb{N}_0$ und $C>\frac{1}{\rho}$, 
		sd. $|a_n|< C^n$ für alle $n\geq n_0$. Da nur endlich viele $n\leq n_0$ existieren, können wir 
		$C$ ggf. etwas größer wählen, sd. $|a_n|<C^n$ für alle $n$. Wir beweisen indirekt, dass alle 
		$a_n =0$ sind, dh. wir nehmen an, es gäbe $n$ mit $a_n \neq 0$. Es sei $n_0$ das kleinste $n$ 
		mit $a_{n_{0}}\neq 0$, dh. $a_n =0$ für $n<n_0$. 
		Wir suchen $r>0$, sd. $|a_n z^{n_0} | > \sum_{n=n_0 +1}^{\infty} |a_n z^n | \left(\geq | \sum_{n=
		n_0 +1}^{\infty} a_n z^n |\right)$ für alle $z\in \mathbb{C}$ mit $0<|z|<r$. Denn dann folgt 
		$R(z)=a_{n_0} z^{n_0} +\sum_{a_n}^{z^n} \neq 0$ für $z$ wie oben, also auch für unendlich 
		viele der Folgenglieder $z_n$ aus unserer Annahme.
		\[ \sum_{n=n_0 +1}^{\infty} |a_n z^n| \leq \sum_{n=n_0 +1}^{\infty} C^n |z^n| \underset{\text{
		geometrische Reihe}}{=} \frac{C^{n+1}|z|^{n+1}}{1-C|z|} \]
		Wir suchen also $r>0$, sd.
		\begin{eqnarray*}
			|a_{n_0}|r^{n_0} > \underbrace{\frac{C^{n+1}|z|^{n+1}}{1-Cr}}_{\text{$>0$, für 
			$r>\frac{1}{C}$}} &\Leftrightarrow & |a_n| (r^{n_0} - Cr^{n_0 +1} ) > C^{n_0 +1} r^{n_0 +1} \\
			& \Leftrightarrow & |a_{n_0}| > r (C^{n_0 +1}+|a_{n_0}|C) \\
			& \Leftrightarrow & r>\frac{|a_{n_0}|}{C^{n_0 +1} + |a_{n_0}|C}
		\end{eqnarray*}
		Jetzt folgt für alle $z$ mit $0< |z|<r$, dass $R(z)\neq 0$ wie gewünscht, Widerspruch! \\
		Also folgt $R=0$ und somit $f|_U =0$.
		Definiere $W=\{ z\in\Omega \mid z\text{ hat Umgebung $U$ mit $f|_U =0$} \}$ \\
		$\Rightarrow$ $W$ ist offen und nichtleer. \\
		Behauptung: $W$ ist auch abgeschlossen. Falls nicht, existiert ein Häufungspunkt $z_0$ von 
		$W$ in $\Omega$ mit $z_0 \in W$. Dann existiert $(z_n)_n$ Folge in $W\setminus \{z_0\}$ mit 
		$\lim_{n\to\infty} z_n =z_0$ und $f(z_n)=0$ für alle $n$. Mit den obigen Argumenten folgt: 
		$z_0$ hat Umgebung $U\subset \Omega$ mit $f|_U =0$, somit $z_0 \in W$. \\
		$W$ offen, abgeschlossen und nichtleer $\Rightarrow$ (da $\Omega$ zusammenhängend ist) 
		$\Omega = W$, also $f=0$.
	\end{proof}
	
	(Proposition im Kurzskript zum Rechnen mit Potenzreihen)...
	
	\subsection{Komplexe Differenzierbarkeit}
	
	\begin{definition}
		Eine $\mathbb{R}$-lineare Abbildung $A:\mathbb{C}\to \mathbb{C}$ heißt $\mathbb{C}$-
		antilinear, wenn
		\[ A(zw)=\overline{z} \cdot A(w) \quad \forall w,z\in\mathbb{C}. \]
		Jede $\mathbb{R}$-lineare Abbildung lässt sich zerlegen als $A=A'+A''$ mit 
		$A'(z)=a'\cdot z$ und $A''(z)=a''\cdot\overline{z}$, dabei heißen $A'$ der Linearteil und $A''$ 
		der Antilinearteil von $A$.\\
		Insbesondere ist $A$ genau dann $\mathbb{C}$-linear, wenn $A''=0$.
	\end{definition}
	
	\begin{proof}
		Setze $A'(z)=\frac{A(z)-i\cdot A(iz)}{2}$, $A''(z)=\frac{A(z)+i\cdot A(iz)}{2}$. Daraus folgt 
		\[ A'(z) + A''(z)=\frac{A(z)-i\cdot A(iz)}{2}+\frac{A(z)+i\cdot A(iz)}{2}=A(z) \]
		\begin{eqnarray*}
			A'((u+iv)\cdot z)&=&\frac{A(uz)+A(ivz)-iA(iuz)-iA(-vz)}{2} \\
			&=& \frac{uA(z)\overbrace{-iviA(iz)}^{=+vA(iz)}-iuA(iz)+ivA(z)}{2} \\
			&=&\frac{(u+iv)(A(z)-iA(iz))}{2} \\
			&=& (u+iv)A'(z)
		\end{eqnarray*}
		Analog dazu ist $A''$ $\mathbb{C}$-antilinear. Es folgt $A'(z)=A'(z\cdot 1)=z\cdot \underbrace{
		A'(1)}_{a'}$,\\$A''(z)=A''(z\cdot 1)=\overline{z}\cdot\underbrace{A''(1)}_{a''}$.
	\end{proof}
	
	\begin{repetition}
		Sei $U\subset\mathbb{C}$ offen, $f:U\to\mathbb{C}\sim\mathbb{R}^2$ eine Funktion. $f$ 
		heißt total differenzierbar bei $z_0\in U$, falls eine $\mathbb{R}$-lineare Abbildung 
		$A:\mathbb{C}\to\mathbb{C}$ existiert, sd. \[ \lim_{z\to z_0} \frac{f(z)-f(z_0)-A(z-z_0)}
		{|z-z_0 |}=0. \]
		Dann ist $f$ auch partiell differenzierbar und die partiellen Ableitungen sind gerade die Einträge 
		der reellen $2\times 2$-Matrix $A$.
	\end{repetition}
	
	\begin{definition}[Komplexe Differenzierbarkeit]
		Es sei $U\subset \mathbb{C}$ offen. Eine Funktion $f:U\to \mathbb{C}$ heißt komplex 
		differenzierbar bei $z_0\in U$, falls $\lim_{z\to z_0} \frac{f(z)-f(z_0)}{z-z_0}$ existiert. Dieser 
		Grenzwert heißt dann die komplexe Ableitung $f'(z_0)\in\mathbb{C}$. Wenn $f$ auf ganz 
		$U$ differenzierbar ist, heißt $f$ auch holomorph auf $U$.
	\end{definition}
	
	\begin{definition}
		Sei $f:U\to\mathbb{C}$ eine Funktion, $U\subset\mathbb{C}$ offen. Schreibe $f=u+iv$ für 
		Funktionen $u,v:U\to\mathbb{R}$, sowie $z=x+iy$. \\
		Definiere die Wirtinger-Ableitungen 
		\[ \frac{\partial f}{\partial z}=\frac{1}{2}\frac{\partial f}{\partial x} -\frac{i}{2}\frac{\partial f}{\partial y} 
		=\frac{1}{2} \left(\frac{\partial u}{\partial x}+\frac{\partial v}{\partial y}\right)+\frac{i}{2}\left(
		\frac{\partial v}{\partial x}-\frac{\partial u}{\partial y}\right) \]
		\[ \frac{\partial f}{\partial \overline{z}}=\frac{1}{2} \frac{\partial f}{\partial x} +\frac{i}{2}
		\frac{\partial f}{\partial y}=\frac{1}{2}\left( \frac{\partial u}{\partial x}-\frac{\partial v}{\partial y}\right)
		+\frac{i}{2}\left(\frac{\partial v}{\partial x}+\frac{\partial u}{\partial y}\right) \]
	\end{definition}
	
	\begin{example}
		$\frac{\partial z}{\partial z}=1$, $\frac{\partial z}{\partial \overline{z}}=0$, 
		$\frac{\partial\overline{z}}{\partial z}=0$, $\frac{\partial\overline{z}}{\partial\overline{z}}=1$
	\end{example}
	
	\begin{lemma}[Definition]\label{lem:komplexdiffbar}
		Es sei $U\subset\mathbb{C}$ offen, $f:U\to\mathbb{C}$ eine Funktion, $z_0\in U$. Dann sind 
		äquivalent
		\begin{enumerate}
			\item $f$ ist komplex differenzierbar bei $z_0$
			\item Es existiert eine stetige Funktion $\varphi:U\to\mathbb{C}$ mit $f(z)=f(z_0)+\varphi(z)
			\cdot (z-z_0)$
			\item $f$ ist bei $z_0$ reell, total differenzierbar mit $\mathbb{C}$-linearer Ableitung
			\item $f$ ist bei $z_0$ reell, total differenzierbar und $\frac{\partial f}{\partial \overline{z}}|_
			{z_0}=0$
			\item $f$ ist bei $z_0$ reell, total differenzierbar und es gelten die Cauchy-Riemann-
			Differentialgleichungen (C-R-DGL): 
			$\frac{\partial u}{\partial x}|_{z_0}=\frac{\partial v}{\partial y}|_{z_0}$ 
			und $\frac{\partial u}{\partial y}|_{z_0}=-\frac{\partial v}{\partial x}|_{z_0}$, wobei wieder 
			$f=u+iv$ gelte.
		\end{enumerate}
		Insbesondere ist $f$ dann auch bei $z_0$ stetig.
	\end{lemma}
	
	\begin{proof}
		(1)$\Rightarrow$(2): Setze 
		\[ \varphi(z)=\begin{dcases} \frac{f(z)-f(z_0)}{z-z_0} & z\neq z_0 \\ f'(z_0) & z=z_0 \end{dcases}\]
		Stetigkeit bei $z_0$ folgt aus der komplexen Differenzierbarkeit.\\
		(2)$\Rightarrow$(3): Schreibe
		\begin{eqnarray*}
			\lim_{z\to z_0} \frac{f(z)-f(z_0)-\varphi(z_0)\cdot (z-z_0)}{|z-z_0|} = 
			\lim_{z\to z_0} \underbrace{(\varphi(z)-\varphi(z_0))}_{\to \ 0\text{, da $\varphi$ stetig in 
			$z_0$.}} \underbrace{\frac{z-z_0}{|z-z_0|}}_{\text{beschränkt (Norm $1$)}} =0
		\end{eqnarray*}
		$\Rightarrow$ $f$ ist bei $z_0$ total-reell-differenzierbar. Die Ableitung ist die $\mathbb{C}$-
		lineare Abbildung $\omega\mapsto \varphi(z_0)\cdot \omega$.\\
		(3)$\Rightarrow$(4): Da die reelle Ableitung $\mathbb{C}$-linear ist, folgt 
		$\frac{\partial f}{\partial \overline{z}}(z_0)=0$ ( was nach Definition gerade der Antilinearteil 
		der Ableitung ist)\\
		(4)$\Rightarrow$(5):
		\begin{eqnarray*}
			0=\underbrace{\frac{\partial f}{\partial \overline{z}}(z_0)}_{\in \mathbb{C}} \underset{\text{
			Def. 1.12}}{=} \underbrace{\frac{1}{2} \left( \frac{\partial u}{\partial x} -
			\frac{\partial v}{\partial y} \right)}_{\text{Realteil}}(z_0) + \frac{i}{2} \underbrace{
			\left( \frac{\partial v}{\partial x}+\frac{\partial u}{\partial y}\right)}_{\text{Imaginärteil}}(z_0)
		\end{eqnarray*}
		hieraus lassen sich die C-R-DGL direkt ablesen.\\
		(5)$\Rightarrow$(1): Schreibe $z=z_0 +x+iy$ dann gilt
		\begin{eqnarray*}
			f(z) &=& f(z_0) +\frac{\partial u}{\partial x} x +\frac{\partial u}{\partial y} y +
			 \frac{\partial v}{\partial x} ix + \frac{\partial v}{\partial y} iy +R(x,y) \\
			&\overset{\text{C-R-DGL}}{=}& f(z_0) +\frac{\partial u}{\partial x} (x+iy) -\frac{\partial v}
			{\partial x}(y-ix)+R(x,y) \\
			&=& f(z_0) +\left(\frac{\partial u}{\partial x}+i \frac{\partial v}{\partial x}\right)\cdot
			(x+iy)+R(x,y)
		\end{eqnarray*}
		mit $R(x,y)=o(|(x,y)|)$, das heißt $\lim_{(x,y)\to 0} \frac{R(x,y)}{|(x,y)|}=0$ (der Restterm geht 
		schneller gegen Null als $(x,y)$). Es folgt 
		\begin{eqnarray*}
			\lim_{z\to z_0}\frac{f(z)-f(z_0)}{z-z_0} &=& \lim_{z\to z_0}\frac{\left(\frac{\partial u}{\partial x}
			+i\frac{\partial v}{\partial x}\right)\cdot (z-z_0)+R(x,y)}{z-z_0} \\
			&=&\frac{\partial u}{\partial x}+i\frac{\partial v}{\partial x} +\lim_{z\to z_=}
			\underbrace{\frac{R(x,y)}{|z-z_0|}}_{\to\  0}\cdot \underbrace{\frac{|z-z_0|}{z-z_=}}_{\text{
			beschränkt}}\\
			&=& \frac{\partial u}{\partial x}+i\frac{\partial v}{\partial x} \in\mathbb{C}
		\end{eqnarray*}
	\end{proof}
	
	\begin{example}
		Die komplexe Exponentialfunktion ist holomorph auf ganz $\mathbb{C}$ (Begründung folgt)
	\end{example}
	
	\begin{proposition}\label{prop:Rechenregeln}
		Es gelten folgende Differentiationsregeln:
		\begin{enumerate}
			\item \underline{Linearität:} Seien $f,g:\Omega\to\mathbb{C}$ holomorph, $a,b\in\mathbb{C}
			$, dann ist $a\cdot f+b\cdot g$ holomorph mit $(a\cdot f+b\cdot g)'(z)=a\cdot f'(z)+b\cdot 
			g'(z)$.
			\item \underline{Kettenregel:} Sei $f:\Omega\to\Omega'$, $g:\Omega'\to\mathbb{C}$ 
			holomorph, dann ist $g\circ f:\Omega\to\mathbb{C}$ holomorph mit 
			$(g\circ f)'(z)=f'(g(z))\cdot g'(z)$.
			\item \underline{Produktregel:} Seien $f,g:\Omega\to\mathbb{C}$ holomorph, dann ist 
			$f\cdot g$ holomorph mit Ableitung $(f\cdot g)'(z)= f'(z)\cdot g(z)+f(z)\cdot g'(z)$.
		\end{enumerate}
	\end{proposition}
	
	\begin{proof}
		(1): Additivität ist klar. Multiplikativität siehe (3) \\
		(2): Übung\\
		(3): Schreibe $f=u+iv$, $g=r+is$, $u,v,r,s:\Omega\to\mathbb{R}$, dann ist 
		$f\cdot g=(u\cdot r-v\cdot s)+i\cdot(u\cdot s+v\cdot r)$. Jetzt setzen wir mit den reellen 
		Produktregeln fort und sind fertig.
	\end{proof}
	
	\begin{theorem}
		Es sei $R(z)=\sum_{n=0}^{\infty} a_n \cdot z^n $ konvergente Potenzreihe mit 
		Konvergenzradius $\rho >0$, dann ist $R(z)$ auf $B_{\rho}(0)$ holomorph mit Ableitung 
		\[R'(z)=\sum_{n=0}^{\infty} a_n \cdot n\cdot z^{n-1} =\sum_{m=0}^{\infty}a_{m+1}(m+1)z^m,\quad
		n-1=m.\]
	\end{theorem}
	
	\begin{proof}
		Siehe Analysis, beruht auf folgendem Satz: Sei $(f_n)_{n\in\mathbb{N}}$ Folge differenzierbarer 
		Funktionen auf $U$, sd. $(f_n)_n$ punktweise und $(f'_n)_n$ lokal-gleichmäßig konvergiert. \\
		$\Rightarrow$ $(\lim_{n\to\infty} f_n)'=\lim_{n\to\infty}f'_n$
	\end{proof}
	
	\subsection{Das komplexe Kurvenintegral}
	
	\begin{definition}
		Eine stückweise $C^1$-Kurve $\gamma:[a,b]\to\mathbb{C}$ ist eine stetige Abbildung, sd. 
		$a=t_0 < t_1 <\ldots<t_n=b$ existieren, für die $\gamma |_{ [t_{i-1},t_i]}\in C^1$ für 
		$i=1,\ldots ,n$. \\
		Für $t\neq t_i$, $t\in [a,b]$, sei $\dot{\gamma}(t)=\frac{\mathrm{d}\gamma}{\mathrm{d}t}(t)$ der 
		Geschwindigkeitsvektor. $\gamma$ heißt geschlossen, wenn \[\gamma(a)=\gamma(b).\]
	\end{definition}
	
	\begin{definition}[Kurvenintegral]
		Sei $\Omega\subset\mathbb{C}$ ein Gebiet, $\gamma:[a,b]\to\Omega$ stückweise $C^1$, sei 
		$f:\Omega\to\mathbb{C}$ stetig. Definiere das komplexe Kurvenintegral
		\[ \int_{\gamma} f(z)\mathrm{d}z :=\int_a^b f(\gamma(t))\cdot \dot{\gamma}(t) 
		\mathrm{d}t \in \mathbb{C}. \]
		Dazu bilden wir rechts Real- und Imaginärteil des Integranden und 
		integrieren diese seperat mit dem Riemann-/ Regel-/ Lesegueintegral.
		\[ \int_{\gamma} f(z)\mathrm{d}z =\int_a^b \underbrace{(u(\gamma(t))+i\cdot v(\gamma(t)))}_{f(z)}
		\cdot \underbrace{(\dot{x}(t)+i\cdot \dot{y}(t))\mathrm{d}t}_{\mathrm{d}z} \]
		,wobei $f=u+iv$, $u,v:\Omega\to\mathbb{R}$ und $\gamma =x+iy$, $x,y:[a,b]\to\mathbb{R}$ 
		(ausmultiplizieren vgl Kurzskript).  Mithin ist $f(z)=f(\gamma(t))$ der $"$Integrand$"$ und 
		$\mathrm{d}z=\mathrm{d}(\gamma(t))=\dot{\gamma}(t)\mathrm{d}t$ das 
		$"$Tangentenelement$"$ des Kurvenintegrals.
	\end{definition}
	
	\begin{remark}
		\[\overset{\text{ausmult.}}{=} \int_a^b (u(\gamma(t))\cdot \dot{x}(t)-v(\gamma(t))\cdot\dot{y}(t)
		\mathrm{d}t+i\cdot \int_a^b (u(\gamma(t))\cdot \dot{y}(t)+v(\gamma(t))\cdot\dot{x}(t))\mathrm{d}t\]
		Der Realteil ist das Kurvenintegral über $\overline{f}=u-iv$ aus der Analysis (aufgefasst als 
		Vektorfeld $\begin{pmatrix} \text{Re}\ f \\ \text{Im}\ f \end{pmatrix}$) und der Imaginärteil 
		das entsprechende $"$normale$"$ Kurvenintegral.
	\end{remark}
	
	\begin{proposition}
		Es sei $\gamma:[a,b]\to\Omega$ eine stückweise $C^1$-Kurve und $\varphi:[c,d]\to[a,b]$ ein 
		stückweiser $C^1$-Diffeomorphismus, dann ist $\gamma\circ\varphi:[c,d]\to\Omega$ eine 
		stückweise $C^1$-Kurve. $sign(\dot{\varphi})$ lässt sich zu einer konstanten Funktion auf 
		$[c,d]$ fortsetzen und für alle stetigen Funktionen $f:\Omega\to\mathbb{C}$ gilt
		\[ \int_{\gamma} f(z)\mathrm{d}z =sign(\dot{\varphi})\int_{\gamma\circ\varphi} f(\omega)
		\mathrm{d}\omega \]
	\end{proposition}
	
	\begin{proof}
		Übung mit Substitutionsformel. \\
		(Ein stückweise $C^1$-Diffeomorphismus $\varphi:[c,d]\to[a,b]$ ist ein Homomorphismus, sd. 
		ein $m\in\mathbb{N}$ und $c=s_0 <s_1 <\ldots <s_m =d$ existieren mit 
		$\varphi |_{[s_{i-1},s_i]}\in C^1 ([s_{i-1},s_i])$ für $i=1,\ldots ,m$)
	\end{proof}
	
	\begin{corollary}[aus Hauptsatz der Differential- und Integralrechnung]\label{Hauptsatz}
		Es sei $\Omega\subset\mathbb{C}$ ein Gebiet, $\gamma:[a,b]\to\Omega$ eine stückweise 
		$C^1$-Kurve und $f:\Omega\to\mathbb{C}$ holomorph. Dann gilt der 
		Hauptsatz der Differential- und Integralrechnung
		\[ \int_{\gamma} f'(z)\mathrm{d}z=f(\gamma(t))|_{t=a}^b =f(\gamma(b))-f(\gamma(a)) \]
	\end{corollary}
	
	\begin{proof}
		Es sei $a=t_0 <t_1 <\ldots <t_n =b$, sd. $\gamma_i =\gamma|_{[t_{i-1},t_i]}\in C^1 
		([t_{i-q},t_i])$ für $i=1,\ldots ,n$.
		\[ [t_{i-1},t_i]\xrightarrow{\gamma_i}\Omega\xrightarrow{f}\mathbb{C} \]
		\begin{eqnarray*}
			\int_{\gamma_i} f'(z)\mathrm{d}z = \int_{t_{i-1}}^{t_i}f'(\gamma_i (t))\cdot \dot{\gamma_i}(t)
			\mathrm{d}t &=& \int_{t_{i-1}}^{t_i} (f\circ \gamma_i)'(t)\mathrm{d}t \\
			&=&(f\circ \gamma_i)(t_i)-(f\circ \gamma_i)(t_{i-1})
		\end{eqnarray*}
		\begin{eqnarray*}
			\Rightarrow\ \int_{\gamma} f'(z)\mathrm{d}z 
			&=& (f(\gamma(t_1))-f(\gamma(t_0)))+(f(\gamma(t_2))-f(\gamma(t_1)))+\ldots +
			(f(\gamma(t_n))-f(\gamma(t_{n-1}))) \\
			&=& f(\gamma(b))-f(\gamma(a))
		\end{eqnarray*}
	\end{proof}
	
	\begin{remark}\label{Bemerkung2}
		Wir möchten uns das komplexe Kurvenintegral als Umkehrung der komplexen Ableitung 
		vorstellen. Wir sehen im nächsten Abschnitt, für welche Funktionen das geht.
	\end{remark}
	
	\subsection{Der Cauchy-Integralsatz}
	
	\begin{definition}[stückweise $C^1$-Homotopie]\label{def:Homotopie}
		Eine stückweise $C^1$-Homotopie, in einem Gebiet $\Omega\subset\mathbb{C}$, zwischen 
		zwei stückweisen $C^1$-Kurven $\gamma_0,\gamma_1:[a,b]\to\Omega$ mit $\gamma_0 (a)=
		\gamma_1 (a)=p$, $\gamma_0 (b)=\gamma_1 (b)=q$ ist eine stetige Abbildung 
		$h:[a,b]\times [0,1]\to\Omega$, sd. $m,n\in\mathbb{N}$, $a=t_0 <t_1 <\ldots <t_n =b$, 
		$0=s_0 <\ldots <s_m =1$ existieren, sd. $h|_{[t_{j-1},t_j]\times [s_{k-1},s_k]}\in C^1$ ist 
		(auch auf den jeweiligen Randstücken) und $h(t,l)=\gamma_l (t)$ für $l\in [0,1]$, $t\in [a,b]$ und 
		$h(a,s)=p$, $h(b,s)=q$ für alle $s\in [0,1]$.
	\end{definition}
	
	\begin{definition}[homotope Kurve]
		Eine (stückweise $C^1$-) Kurve $\gamma_0 :[a,b]\to\Omega$ heißt zu einer 
		(stückweisen $C^1$-) Kurve $\gamma_1 :[a,b]\to\Omega$, mit gleichem Anfangs- und Endpunkt, 
		(stückweise $C^1$-) homotop in $\Omega$, wenn es eine stückweise $C^1$-Homotopie 
		zwischen ihnen in $\Omega$ gibt.
	\end{definition}
	
	\begin{definition}[nullhomotope Kurve]
		Eine geschlossene (stückweise $C^1$-) Kurve $\gamma$ heißt (stückweise $C^1$-) nullhomotop 
		in $\Omega$, wenn sie $C^1$-homotop zu einer konstanten Kurve ist.
	\end{definition}
	
	\begin{definition}[einfach zusammenhängend]
		Das Gebiet $\Omega$ heißt einfach zusammenhängend, wenn jede geschlossene (stückweise 
		$C^1$-) Kurve in $\Omega$ (stückweise $C^1$-) nullhomotop in $\Omega$ ist.
	\end{definition}
	
	\begin{remark}[Einschub zu Kurvenintegral]
		Sei $\gamma:[a,b]\to\mathbb{C}$ eine stückweise $C^1$-Kurve, dann definieren wir die 
		Bogenlänge (bzw. Länge) als
		\[ L(\gamma)=\int_a^b |\dot{\gamma}(t)|\mathrm{d}t =\sup_{n,a=t_0 <\ldots <t_n =b} 
		\sum_{j=1}^n |\gamma(t_j)-\gamma(t_{j-1})| \]
		Dann gilt
		\begin{eqnarray*}
			\left|\int_{\gamma} f(z)\mathrm{d}z \right| = \left|\int_a^b f(\gamma(t))\cdot 
			\dot{\gamma}(t)\mathrm{d}t \right| 
			&\leq& \int_a^b |f(\gamma(t))\cdot \dot{\gamma}(t) |\mathrm{d}t \\
			&=& \int_a^b |f(\gamma(t))|\cdot \dot{\gamma}(t)\mathrm{d}t \\
			&\leq& \sup_{t\in [a,b]} |f(\gamma(t))| \cdot L(\gamma).
		\end{eqnarray*}
	\end{remark}
	
	\begin{theorem}[Cauchy-Integralsatz] \label{thm:CI}
		Es sei $\Omega\subset\mathbb{C}$ ein Gebiet, $f:\Omega\to\mathbb{C}$ holomorph und 
		$\gamma:[a,b]\to\Omega$ eine stückweise $C^1$-Kurve, die in $\Omega$ stückweise 
		$C^1$-nullhomotop ist. Dann gilt 
		\[ \int_{\gamma} f(z) \mathrm{d}z =0 \]
	\end{theorem}
	
	\begin{proof}
		Es reicht zu zeigen, dass für jede stückweise $C^1$-Abbildung $h:\underbrace{[a,b]\times [0,1]}_
		{R}\to\Omega$ gilt
		\[ \int_{h(\partial R)} f(z) \mathrm{d}z =0. \]
		Dabei ist $\int_{h(\partial R)}\quad$ eine Abkürzung für $\int_{h(\partial R)}\quad =
		\int_{h_1}\quad +\int_{h_2}\quad +\int_{h_3}\quad +\int_{h_4}\quad$, wobei 
		$h_1 (t)=h(t,0)$, $h_2 (s)=h(b,s)$, $h_3 (t)=h(a+b-t,1)$, $h_4 (s)=h(a,1-s)$. \\
		Setze das zu einer stückweisen $C^1$-Kurve mit Namen $h(\partial R)$ zusammen. \\
		\underline{Annahme:} Es gebe eine solche Abbildung $h:[a,b]\times [0,1]\to \Omega$, sd. 
		$\int_{h(\partial R)} f(z)\mathrm{d}z \neq 0$. \\
		Wir zerlegen das Rechteck $R$ in vier gleich große Teile $R_1 ,\ldots,R_4$ und sehen, dass
		\[ \int_{h(\partial R)} f(z)\mathrm{d}z = \int_{h(\partial R_1)} f(z)\mathrm{d}z +\ldots +
		\int_{h(\partial R_4)} f(z)\mathrm{d}z. \]
		Da sich die zusätzlichen Integrale über Strecken im Inneren von $R$ wegen Proposition 
		\ref{Hauptsatz} wegheben. \\
		Jetzt wählen wir das Teilrechteck aus, für den das jeweilige Kurvenintegral über den Rand den 
		größten Absolutbetrag hat, nenne es $R_1$. Es folgt 
		\[ \left| \int_{h(\partial R_1)}f(z)\mathrm{d}z\right| \geq \frac{1}{4} \left|\int_{h(\partial R)}f(z)
		\mathrm{d}z \right| \]
		Wir zerlegen weiter und erhalten so eine Folge von Rechtecken $R_1 \supset R_2 \supset\ldots 
		,R_n$ mit Seitenlängen von $R_n$ proportional zu $2^{-n}$, sd.
		\[ \left| \int_{h(\partial R_n)} f(z)\mathrm{d}z \right| \geq 2^{-n} \left| \int_{h(\partial R)}f(z)
		\mathrm{d}z \right|. \]
		Nach dem Satz über die Invervallverschachtelung (Analysis) existiert ein eindeutiger Punkt 
		$(t_0,s_0)\in\mathbb{R}^2$ mit $(t_0,s_0)\in  \bigcap_{n\in\mathbb{N}}R_n$. \\
		Es sei $z_0 =h(t_0,s_0)\in\Omega$.\\
		\underline{Beachte:} Da $h$ stückweise $C^1$ ist, erhalten wir für jedes der endlich vielen 
		Rechtecke aus Definition \ref{def:Homotopie} eine obere Schranke für 
		$|\frac{\partial h}{\partial t}|$, $|\frac{\partial h}{\partial s}|$ (wegen der Kompaktheit). 
		Da es nur endlich viele dieser Rechtecke gibt, folgt $|\frac{\partial h}{\partial t}|\leq C$, 
		$|\frac{\partial h}{\partial s}| \leq C$ auf ganz $R=R_0$, für ein festes $C>0$. \\
		Schreibe nahe $z_0$ die Funktion $f$ als $f(z)=f(z_0)+f'(z_0)\cdot (z-z_0)+r(z-z_0)$, wobei \\
		$\lim_{z\to z_0} |\frac{r(z-z_0)}{z-z_0}|=0$, da $f$ holomorph ist 
		(vgl. Lemma \ref{lem:komplexdiffbar}).\\
		Da $f(z_0)+f'(z_0)\cdot (z-z_0)$ das Differential der holomorphen Funktion
		$z\mapsto f(z_0)\cdot (z-z_0)+\frac{1}{2}f'(z_0)\cdot (z-z_0)^2$ ist, folgt mit Bemerkung 
		\ref{Bemerkung2}, dass 
		das Integral von $f(z_0)+f'(z_0)\cdot (z-z_0)$ über die geschlossenen Kurven $h(\partial R_n)$ 
		verschwindet. Die Länge $L$ von $h(\partial R_n)$, $L(h(\partial R_n))$ können wir abschätzen 
		durch $4\cdot 2^{-n}\cdot C$. Es folgt
		\begin{eqnarray*}
			\left| \int_{h(\partial R)}f(z)\mathrm{d}z \right| 
			&\leq& \lim_{n\to\infty} 2^{2n} \left| \int_{h(\partial R_n)}f(z)\mathrm{d}z \right| \\
			&\overset{(I)}{=}& \lim_{n\to\infty} 2^{2n} \left|\int_{h(\partial R_n)}
			r(z-z_0)\mathrm{d}z \right| \\
			&\leq& \lim_{n\to\infty} \left( 2^{2n} \cdot \sup_{h(\partial R_n)}|r(z-z_0)|\cdot 
			\underbrace{L(h(\partial R_n))}_{\leq 4\cdot C\cdot 
			2^{-n}}\right) \\
			&\leq& \lim_{n\to\infty} \left(2^n \cdot 4\cdot C \cdot \sup_{h(\partial R_n)} |r(z-z_0)| \cdot 
			\frac{|z-z_0|}{|z-z_0|}\right) \\
			&\underset{|z-z_0|\leq 2^{-n}\cdot 2\cdot C}{\leq}& \lim_{n\to\infty} \left(8\cdot C^2 \cdot
			\sup_{h(\partial R_n)} \frac{|r(z-z_0)|}{|z-z_0|}\right) \\
			&=& \underbrace{\lim_{z\to z_0} \frac{|r(z-z_0)|}{|z-z_0|}}_{=0}\cdot 8\cdot C^2 \\
			&=&0.
		\end{eqnarray*}
		Also gilt $| \int_{h(\partial R)}f(z)\mathrm{d}z |=0$ im Widerspruch zur Annahme.
	\end{proof}
	
	\begin{corollary}\label{coroll:homotop}
		Es sei $\Omega\subset\mathbb{C}$ Gebiet, $\gamma_0$, $\gamma_1$ zwei stückweise 
		$C^1$-Kurven in $\Omega$ von $p$ nach $q$, die stückweise $C^1$-homotop sind. 
		Dann gilt
		\[ \int_{\gamma_0} f(z)\mathrm{d}z=\int_{\gamma_1}f(z)\mathrm{d}z \]
	\end{corollary}
	
	\begin{proof}
		Sei $h$ eine stückweise $C^1$-Homotopie zwischen $\gamma_0$ und $\gamma_1$ in 
		$\Omega$. \\
		Betrachte $k:[0,1]^2 \to[a,b]\times[0,1]$ mit
		\[ k(u,v)=\begin{dcases} (a+(1-v)4u(b-a),0) & u\in[0,\frac{1}{4}] \\
		(a+(1-v)(b-a),4u-1) & u\in[\frac{1}{4},\frac{1}{2}] \\
		(a+(1-v)(3-4u)(b-a),1) & u\in [\frac{1}{2},\frac{3}{4}] \\
		(a,4-4u)& u\in[\frac{3}{4},1]
		\end{dcases} \]
		Die Kurve $(h\circ k)(\cdot,0):[0,1]\to\Omega$ ist geschlossen und wegen der Invarianz des 
		Kurvenintegrals unter Umparametrisierung erhalten wir 
		\begin{eqnarray*}
			\int_{(h\circ k)(\cdot,0)}f(z)\mathrm{d}z &=& \int_{\gamma_0}f(z)\mathrm{d}z +
			\underbrace{\int_q f(z)\mathrm{d}z}_{0} +\int_{\gamma_1 (-\cdot)}f(z)\mathrm{d}z +
			\int_p f(z)\mathrm{d}z \\
			&=&\int_{\gamma_0}f(z)\mathrm{d}z-\int_{\gamma_1}f(z)\mathrm{d}z
		\end{eqnarray*}
		$(h\circ k)$ ist eine Nullhomotopie dieser Kurve. also verschwindet der obige Ausdruck.
	\end{proof}
	
	\begin{theorem}[erweiterter Cauchy-Integralsatz]
		Sei $f:\Omega\to\mathbb{C}$ stetig differenzierbar und $\gamma:[0,1]\to\Omega$ umlaufe eine 
		einfach zusammenhängende Teilmenge $A\subset\Omega$ im mathematischen Drehsinn.
		Dann gilt
		\[\int_{\gamma}f(z)\mathrm{d}z =2i \int_A \frac{\partial f}{\partial \overline{z}}(z)
		\underbrace{\mathrm{d}A(z)}_{\text{Flächenelement}} \]
		(Vergleiche mit dem Satz von Stokes oder dem Gaußschen Divergenzsatz)
	\end{theorem}
	
	\begin{proof}
		Beweisskizze: Da $A$ einfach zusammenhängend ist, ist $\gamma$ in $A$ nullhomotop. \\
		Sei $h:[0,1]^2 \to A\subset\Omega$ eine Nullhomotopie. Annahme:
		\[ \left| \int_{\gamma}f(z)\mathrm{d}z -2i \int_A \frac{\partial f}{\partial \overline{z}}\mathrm{d} A(z)
		\right| =\varepsilon >0. \]
		Zerlege $[0,1]^2$ in vier gleich große Quadrate $R' ,\ldots ,R''''$. Dann gilt für eins der 
		Quadrate:
		\[ \left| \int_{h(\partial R^? )} f(z)\mathrm{d}z -2i \int_{h(R^?)}
		\frac{\partial f}{\partial \overline{z}}\mathrm{d} A(z) \right| \geq \frac{\varepsilon}{4} \]
		Nenne es $R_1$ und zerlege weiter. Erhalte eine Intervallverschachtelung mit Grenzpunkt 
		$(t_0, s_0)\in[0,1]^2$; sei $h(t_0,s_0) =:z_0\in\Omega$. Schreibe
		\[f(z)=f(z_0)+\frac{\partial f}{\partial z} (z_0)\cdot (z-z_0) +\frac{\partial f}{\partial \overline{z}}
		(z_0)\cdot \overline{(z-z_0)} +r(z-z_0). \]
		mit $\lim_{z\to z_0} \frac{r(z-z_0)}{|z-z_0|}=0$. Wir wissen, dass 
		\[ \int_{h(\partial R^n)} \left( f(z_0)+\frac{\partial f}{\partial z} (z_0)(z-z_0) \right) \mathrm{d}z =0. \]
		In einer Übung berechnen wir 
		\[ \int_{h(\partial R^n)}\frac{\partial f}{\partial \overline{z}}(z_0)\overline{(z-z_0)}\mathrm{d}z =
		2i\cdot A(h(R^n)) \]
		(falls $h(R^n)$ ein Parallelogramm ist — da $h$ stückweise $C^1$ ist, ist $h(R^n)$ $"$fast$"$ 
		ein Parallelogramm, sd. die obige Behauptung bis auf einen ausreichend kleinen Rest stimmt.)
		Außerdem gilt
		\[ \lim_{n\to\infty} \left| \int_{h(R^n)} \frac{\partial f}{\partial \overline{z}} \mathrm{d} A(z) - 
		\int_{h(R^n)} \frac{\partial f}{\partial \overline{z}}(z_0)\mathrm{d}A(z) \right| \cdot 2^{2n} =0 \]
		da $\frac{\partial f}{\partial \overline{z}}$ stetig ist. $\frac{\partial f}{\partial \overline{z}}(z_0) 
		\cdot A(h(R^n))$ Also erhalten wir einen Widerspruch genau wie im Beweis des Integralsatzes.
	\end{proof}
	
	\subsection{Die Potenzreihendarstellung}
	
	\underline{Ziel:} \begin{itemize} \item $"$holomorph$"$ und $"$ analytisch$"$ sind gleichbedeutend. 
	\item Man kann Ableitungen als Integrale schreiben.
	\item Funktionen haben Stammfunktionen genau dann, wenn sie holomorph sind.
	\end{itemize}
	
	\begin{theorem}[Cauchy-Formel] \label{thm:CF}
		Es sei $\Omega\subset\mathbb{C}$ ein Gebiet. $f:\Omega\to\mathbb{C}$ holomorph, 
		$z_0\in\Omega$, $r>0$ sei so gewählt, dass $\overline{B_r (z_0)}\subset\Omega$. 
		$\gamma$ beschreibe den Rand von $B_r (z_0)$ im mathematische Drehsinn. Dann gilt 
		für all $z\in B_r (z_0)$, dass 
		\[ f(z)= \frac{1}{2 \pi i} \int_{\gamma} \frac{f(\zeta)}{\zeta -z}\mathrm{d}\zeta. \]
	\end{theorem}
	
	\begin{proof}
		$\frac{f(\zeta )}{\zeta -z}$ ist in $\zeta$ holomorph in $\Omega\setminus \{ z\}$. 
		Wähle $\varepsilon >0$ hinreichend klein, sd. $B_{\varepsilon}(z)\subset B_r (z_0)$. 
		Dann lässt sich eine in $\Omega\setminus \{ z \}$ nullhomotope Kurve $\varphi$ finden, sd.
		\[ 0= \int_{\varphi} \frac{f(\zeta)}{\zeta -z}\mathrm{d}\zeta = 
		\int_{\gamma}\frac{f(\zeta)}{\zeta -z} \mathrm{d}\zeta - \int_{\partial B_{\varepsilon}(z) }
		\frac{f(\zeta)}{\zeta - z}\mathrm{d}\zeta \]
		Berechne jetzt für $\varepsilon >0$ klein 
		\begin{eqnarray*}
			\int_{\partial B_{\varepsilon}(z)} \frac{f(\zeta)}{\zeta -z}\mathrm{d}\zeta &=& 
			\int_0^1 f(\underbrace{z+\varepsilon \cdot e^{2\pi it} }_{\zeta})\cdot 
			\frac{1}{\varepsilon \cdot e^{2\pi it}} \underbrace{2\pi i \varepsilon \cdot e^{2 \pi it}}_{=
			\dot{\varphi}(t)} \mathrm{d}t \\
			&=& 2\pi i \int_0^1 f(\underbrace{f(z)+R(\varepsilon e^{2\pi i t})}_{\text{da $f$ stetig ist, gilt 
			$R\to0$ für $\varepsilon\to 0$}})\mathrm{d}t
		\end{eqnarray*}
		$\lim_{\varepsilon\to 0} \int_{\partial B_{\varepsilon} (z)} \frac{f(\zeta)}{\zeta -z} \mathrm{d}\zeta
		=2\pi i f(z)$.
	\end{proof}
	
	\begin{corollary}[Mittelwertsatz]
		Es seien $\Omega$, $f$, $z_0$, $r$ wie oben, dann gilt 
		\[ f(z_0)= \int_0^1 f(z_0+r\cdot e^{2\pi it})\mathrm{d}t \]
		Kein Kurvenintegral und das hier ist nicht der Mittelwertsatz aus Ana 1.
	\end{corollary}
	
	\begin{proof}
		Setze $z=z_0$ in der Integralformel 
		\[ f(z_0)= \frac{1}{2\pi i} \int_0^1 f(z_0+r\cdot e^{2\pi it})\cdot \frac{1}{r e^{2\pi it}} r\cdot 2\pi i \cdot 
		e^{2\pi it}\mathrm{d}t \]
	\end{proof}
	
	\begin{example}
		Wähle $\Omega=\mathbb{C}$, $f(z)=e^z$ $z_0 =0$, $r=1$. Dann gilt
		\begin{eqnarray*}
			1&=& e^0 = \int_0^1 e^{\cos(2\pi t)+i\sin (2\pi t)}\mathrm{d}t \quad \varphi=2\pi t \\
			&=& \frac{1}{2\pi} \underbrace{\int_0^{2\pi} e^{\cos(\varphi)}\cdot \cos(\sin(\varphi)) 
			\mathrm{d}\varphi }_{2\pi} +\frac{1}{2\pi i} \underbrace{\int_0^{2\pi}e^{\cos(\varphi)}\cdot 
			\sin(\sin(\varphi))\mathrm{d}\varphi}_{0}
		\end{eqnarray*}
	\end{example}
	
	\begin{theorem}[Potenzreihenentwicklung]
		Es sei $f:\Omega\to\mathbb{C}$ holomorph und $z_0\in\Omega$. Dann konvergiert die 
		Potenzreihe $\sum_{n=0}^{\infty} a_n \cdot (z-z_0)^n$ mit $a_n = \frac{1}{2\pi i} \int_{\partial B_r 
		(z_0)} \frac{f(z)}{(z-z_0)^{n+1}}\mathrm{d}z$ (für ein $r>0$, sd. $\overline{B_r (z_0)}
		\subset\Omega$) mit Konvergenzradius $\varphi \geq \sup\{r|\overline{B_r (z_0)}
		\subset\Omega\}$ und stellt auf $B_r (z_0)$ die Funktion $f$ dar.
	\end{theorem}
	
	\begin{proof}
		\begin{eqnarray*}
			f(z) &=& \frac{1}{2\pi i} \int_{\partial B_r (z_0)} \frac{f(\zeta)}{\zeta -z}\mathrm{d}\zeta 
			= \frac{1}{2\pi i}\int_{\partial B_r (z_0)} \frac{f(\zeta)}{(\zeta - z_0)-(z-z_0)}\mathrm{d}\zeta \\
			&=& \frac{1}{2\pi i} \int_{\partial B_r (z_0)} f(\zeta)\cdot \underbrace{\sum_{n=0}^{\infty} 
			\frac{(z-z_0)^n}{(\zeta - z_0)^{n+1}}}_{\frac{1}{\zeta -z_0}\cdot\frac{1}{1-\frac{z-z_0}
			{\zeta -z_0}}}\mathrm{d}\zeta 
			= \sum_{n=0}^{\infty} (z-z_0)^n \cdot \frac{1}{2\pi i} \int_{\partial B_r (z_0)} 
			\frac{f(\zeta)}{(\zeta -z_0)^{n+1}} \mathrm{d}\zeta.
		\end{eqnarray*}
		Wir dürfen Summation und Integration vertauschen, falls $|z-z_0|<|\zeta -z_0|=r$, da dann 
		Summe und Integral absolut konvergieren. Der Konvergenzradius ist daher mindestens $r$. 
		Und zwar für jedes $r$ mit $\overline{B_r (z_0)}\subset\Omega$.
	\end{proof}
	
	\begin{corollary}\label{coroll:holo}
		Holomorphe Funktionen sind (komplex) analytisch, insbesondere $C^{\infty}$.
	\end{corollary}
	
	\begin{proof}
		Sei $f:\Omega\to\mathbb{C}$ holomorph, $z_0 \in\Omega$, $r>0$, sd. $\overline{B_r (z_0)}
		\subset\Omega$. Dann können wir $f$ auf $B_r (z_0)$ durch eine Potenzreihe darstellen.
		Insbesondere ist $f$ auf $B_r (z_0)$ analytisch (und $C^{\infty}$). 
		Da das für alle $z_0 \in\Omega$ geht, folgt die Behauptung.
	\end{proof}
	
	Somit: $"$holomorph$"$ und $"$analytisch$"$ sind gleichbedeutend. \\
	\underline{Grund:} $"$Holomorphie$"$ ist gleichbedeutend mit den 
	Cauchy-Rieman-Differentialgleichungen (Lemma \ref{lem:komplexdiffbar}). 
	Diese sind $"$elliptisch$"$ und Lösungen 
	elliptischer Differentialgleichungen sind mindestens so oft differenzierbar, wie ihre 
	Koeffizienten und ihre rechte Seite.\\
	\underline{Zur Erinnerung:} Wir haben die Rechenregeln für Potenzreihen aus Proposition 1.7 
	(Kurzskript)	
	nicht bewiesen. Mit Folgerung \ref{coroll:holo} und Proposition \ref{prop:Rechenregeln} geht der 
	Beweis recht einfach.
	
	\begin{corollary}\label{coroll:stammfkt}
		Es sei $\Omega$ einfach zusammenhängend. Dann ist $f:\Omega\to\mathbb{C}$ genau dann 
		holomorph, wenn $f$ eine Stammfunktion $F$ besitzt (das heißt $F$ ist holomorph mit 
		$F'=f$).
	\end{corollary}
	
	\begin{proof}
		$"\Leftarrow"$ Sei $F$ Stammfunktion. Da $F$ holomorph ist, ist $F$ beliebig oft komplex 
		differenzierbar, siehe Folgerung \ref{coroll:holo}. Also ist auch $f=F'$ beliebig oft komplex 
		differenzierbar, also insbesondere auch holomorph. \\
		$"\Rightarrow"$ Da $\Omega$ einfach zusammenhängend ist, sind je zwei Kurven 
		$\gamma_0$, $\gamma_1$ von $z_0\in\Omega$ nach $z\in\Omega$ homotop. Somit gilt 
		\[ \int_{\gamma_0} f(\zeta)\mathrm{d}\zeta =\int_{\gamma_1}f(\zeta)\mathrm{d}\zeta 
		\quad \text{nach Folgerung \ref{coroll:homotop}} \]
		Fixiere also $z_0$ und definiere $F=\int_{\gamma} f(\zeta)\mathrm{d}\zeta$ für eine Kurve 
		$\gamma:[0,1]\to\Omega$ mit $\gamma(0)=z_0$ und $\gamma(1)=z$. 
		Um $F'(z)$ zu berechnen, betrachte $\omega$ nahe $z$ und eine Kurve $\gamma$ von $z_0$ 
		nach $\omega$ der Form (siehe Skizze Skriptum Niklas) \\
		Dann gilt:
		\begin{eqnarray*}
			\lim_{\omega\to z}\frac{F(\omega)-F(z)}{\omega -z} &=&
			\lim_{\omega\to z}\frac{1}{\omega -z}\left( \int_{\gamma_{\omega}}f(\zeta)\mathrm{d}\zeta -
			\int_{\gamma_z}f(\zeta)\mathrm{d}\zeta \right) \\
			&=&\lim_{\omega\to z}\frac{1}{\omega -z}\int_{\gamma_{\omega -z}}
			f(\zeta)\mathrm{d}\zeta \\
			&=&\lim_{\omega\to z}\frac{1}{\omega -z}\int_0^1 f(\gamma_{\omega -z}(t))
			\underbrace{\omega -z}_{=\dot{\gamma}_{\omega -z}(t)}\mathrm{d}t \\
			&=& f(z).
		\end{eqnarray*}
		$\Rightarrow$ $F$ ist eine Stammfunktion.
	\end{proof}
	
	Zur Erinnerung: Identitätssatz für Potenzreihen.
	
	\begin{corollary}[Identitätssatz für holomorphe Funktionen]
		Es sei $\Omega\subset\mathbb{C}$ ein Gebiet, $f,g:\Omega\to\mathbb{C}$ holomorph. Falls 
		eine Teilmenge $A\subset\Omega$ mit Häufungspunkt $z\in\Omega$ existiert, sd. 
		$f|_A =g|_A$, dann gilt $f=g$ auf ganz $\Omega$.
	\end{corollary}
	
	\begin{proof} 
		Nach Folgerung \ref{coroll:holo} sind $f$ und $g$ analytisch. \\
		$"$$A$ hat Häufungspunkt $z"$ $\Leftrightarrow$ Es existiert eine Folge 
		$(z_n)_{n\in\mathbb{N}}
		\subset A\setminus\{ z\}$, sd. $z_n\xrightarrow{n\to\infty} z$. \\
		Es folgt $(g-f)(z_n)=0$ für alle $n$ und nach dem Identitätssatz für Potenzreihen bzw. 
		analytische Funktionen gilt somit $g-f=0$ auf ganz $\Omega$.
	\end{proof}
	
	Der Identitätssatz ermöglicht es manche aus dem reellen bekannten Funktionen auf $\mathbb{C}$ 
	zu übertragen und ihre Eigenschaften zu verstehen.
	
	\begin{example}\label{bsp:sin}
		Es gilt für $x\in\mathbb{R}$, dass
		\[ \sin(x)=\frac{e^{ix}-e^{-ix}}{2i}=\sum_{n=0}^{\infty}(-1)^n \frac{x^{2n+1}}{(2n+1)!} \]
		Wir können $z\in\mathbb{C}$ in die Potenzreihenentwicklung einsetzen. Da der 
		Konvergenzradius $\infty$ ist, erhalten wir eine Funktion $\sin:\mathbb{C}\to\mathbb{C}$. 
		Da die Identitäten 
		\begin{eqnarray}
			\sin(z)&=&\frac{e^{iz}-e^{-iz}}{2i} \nonumber \\
			\sin(z+w)&=&\sin(z)\cos(w)+\cos(z)\sin(w) \\
			\sin''(z)&=&-\sin(z) \nonumber
		\end{eqnarray}
		für alle $x\in\mathbb{R}$ gelten, gelten sie nach dem Identitätssatz für alle $z$, $w$ aus 
		$\mathbb{C}$. \\
		Zu (1) [Additionstheorem]: Nehme zunächst $w\in\mathbb{R}$ als Konstante an, dann folgt das 
		Additionstheorem für alle $z\in\mathbb{C}$, $w\in\mathbb{R}$. Nehme nun $z\in\mathbb{C}$ 
		konstant an, erhalte Additionstheorem für alle $z$, $w\in\mathbb{C}$. \\
		Definiere die Hyperbelfunktion $\cosh$, $\sinh$ durch
		\begin{eqnarray*}
			\cosh(z)&=&\cos(iz)=\frac{e^{-z}+e^z}{2} \\
			\sinh(z)&=&\frac{\sin(iz)}{i}=\frac{e^{-z}-e^z}{-2} =\frac{e^z -e^{-z}}{2}
		\end{eqnarray*}
		Auf der anderen Seite verhindert der Identitätssatz die Existenz holomorpher Fortsetzungen 
		von reellen Funktionen mit bestimmten Eigenschaften.
	\end{example}
	
	\begin{example}
		\begin{enumerate}
			\item Es gibt kein Gebiet $\Omega$ mit $\mathbb{R}\setminus\{0\}\subset\Omega$ 
			und sich die Funktion $x\mapsto |x|$ auf $\Omega$ fortsetzen ließe. \\
			Denn: wäre $f$ eine Fortsetzung, dann wäre $f(z)=z$ auf $(0,\infty)\subset\mathbb{R}$ 
			und daher auf ganz $\Omega$.
			\item Betrachte 
			\[ f(x)=\begin{dcases} 0 & x\leq 0 \\ e^{-\frac{1}{x}} & x>0. \end{dcases} \]
			Diese Funktion ist $C^{\infty}$ und bei $x=0$ verschwinden alle Ableitungen. Sie ist nicht 
			analytisch bei $x=0$ und hat daher keine holomorphe Fortsetzung.
		\end{enumerate}
	\end{example}
	
	\begin{theorem}[Morera]\label{thm:morera}
		Es sei $\Omega\subset\mathbb{C}$ ein Gebiet und $f:\Omega\to\mathbb{C}$ stetig, sd. das 
		Kurvenintegral von $f$ über den Rand eines jeden Dreiecks, das ganz in $\Omega$ liegt 
		verschwindet. Dann ist $f$ holomorph.
	\end{theorem}
	
	\begin{proof}
		Benutze Folgerung \ref{coroll:stammfkt} auf kleinen Bällen $B_r (z_0)\subset\Omega$ für 
		$z_0\in\Omega$ und $r>0$ ausreichend klein. \\
		Definiere jetzt $F(z)=\int_{\gamma_z}f(\zeta)\mathrm{d}\zeta$, wobei $z\in B_r (z_0)$ und 
		$\gamma (t)=z+t(\omega-z)$. Argumentiere wie in Folgerung \ref{coroll:stammfkt}, dass 
		$F'(z)=f(z)$, allerdings benutzen wir diesmal:
		\[ \int_{\gamma_{\omega}}f(\zeta)\mathrm{d}\zeta -\int_{\gamma_z}f(\zeta)\mathrm{d}\zeta 
		=\int_{\gamma_z}f(\zeta)\mathrm{d}\zeta +\int_{\gamma_{\omega -z}}f(\zeta)\mathrm{d}\zeta 
		-\int_{\gamma_z}f(\zeta)\mathrm{d}\zeta =\int_{\gamma_{\omega -z}}f(\zeta)\mathrm{d}\zeta. \]
		$\Rightarrow$ $F'=f$ auf $B_r (z_0)$. \\
		Da $z_0\in\Omega$ und $r>0$ beliebig waren, ist $f$ auf $\Omega$ holomorph.
	\end{proof}
	
	\begin{theorem}[Schwarzsches Spiegelungsprinzip]
		Es sei $\Omega\subset\mathbb{C}$ symmetrisch bezüglich $\mathbb{R}$ (dh. $z\in\Omega\ 
		\Leftrightarrow\ \overline{z}\in\Omega$). Schreibe $\Omega_+ =\{z\in\Omega | \text{Im } z >0\}$, 
		$\Omega_0 =\Omega\cap\mathbb{R}$ und $\Omega_- =\{z\in\Omega | \text{Im } z<0\}$.
		Sei $f:\Omega_+ \cup \Omega_0 \to\mathbb{C}$ stetig, sd. $f|_{\Omega_+}$ holomorph 
		und $f|_{\Omega_0}$ reellwertig ist. Dann existiert eine holomorphe Fortsetzung 
		$f:\Omega\to\mathbb{C}$ mit $f(\overline{z})=\overline{f(z)}$.
	\end{theorem}
	
	\begin{proof}
		Definiere $f(z)=\overline{f(z)}$ für $z\in\Omega$, dann ist $f$ auf ganz $\Omega$ stetig. 
		Zeige jetzt, dass die Voraussetzungen des Satzes von Morera gelten.
		\begin{enumerate}
			\item Für jedes Dreieck in $\Omega_+$ stimmt die Behauptung
			\item Sei $\triangle\subset\Omega_+ \cup\Omega_0$ ein Dreieck. Dann betrachte 
			Dreiecke $\triangle_n\subset\Omega_+$, die dagegen konvergieren. Da das Integral 
			stetig vom Integranden abhängt (glm. stetig gilt, da $\triangle$-Fläche kompakt ist), ist 
			auch das Integral über den Rand von $\triangle$ gleich $0$.
			\item Falls $\triangle\subset\Omega\subset\Omega_- \cup\Omega_0$ liegt, berechne 
			\begin{eqnarray*}
				\int_{\gamma}f(z)\mathrm{d}z&=&\int_a^b f(\gamma(t))\dot{\gamma}(t)\mathrm{d}t =
				\int_a^b \overline{f(\overline{\gamma(t)})} \ 
				\overline{\dot{\overline{\gamma}}(t)}\mathrm{d}t \\
				&=& \overline{\int_a^b f(\overline{\gamma(t)}) \dot{\overline{\gamma}}(t)\mathrm{d}t}
				=0,
			\end{eqnarray*}
			falls $\gamma$ den Rand von $\triangle$ beschreibt.
			\item $\triangle$ erstreckt sich über alle Dreiecke. Dann zerfällt $\triangle$ in höchstens 
			3 Dreiecke vom Typ (1)-(3). Jetzt folgt Homotopie aus Satz \ref{thm:morera}.
		\end{enumerate}
	\end{proof}
	
	\begin{example}
		$\sin$ aus Beispiel \ref{bsp:sin}.
	\end{example}
	
	\begin{remark}
		Es sei $g$ auf $\partial B_r (z_0)$ stetig. Dan können wir 
		\[ f(z)=\frac{1}{2\pi i}\int_{\partial B_r (z_0)} \frac{g(\zeta)}{\zeta -z}\mathrm{d}\zeta \]
		für alle $z\in B_r(z_0)$ definieren. \\
		Frage: Setzt $f$ die Funktion $g$ stetig fort? \\
		(Beachte: $\partial B_r(z_0)$ ist im schlimmsten Fall der Rand des Konvergenzkreises...) \\
		Falls ja, wäre auch $f(z)\cdot(z-z_0)^k$ holomorph für alle $k\geq 0$ und somit hätten wir 
		nach dem Integralsatz
		\[\frac{1}{2\pi i}\int_{\partial B_r(z_0)}f(\zeta)\underbrace{(\zeta-z_0)^k}_{z_0 +re^{it}}
		\mathrm{d}\zeta =0. \]
		Das bedeutet, dass $"$ungefähr die Hälte$"$ der Fourierzerlegung von $t\mapsto 
		g(z_0 +r\cdot e^{it})$ verschwindet.
	\end{remark}
	
	\section{Abbildungsverhalten holomorpher Funktionen}
	
	Aus der reellen Analysis: Zwischenwertsatz (Bilder von Invervallen sind Intervalle) lokaler 
	Umkehrsatz für $f:U\to\mathbb{R}^n$, $U\subset\mathbb{R}^n$ 
	\begin{enumerate}
		\item Funktionen auf $\Omega\subset\{z_0\}$
		\item Maximumprinzip \& Satz von Liouville
		\item lokaler Umkehrsatz / Blättersatz
	\end{enumerate}
	$\Omega$ ist stets ein Gebiet in $\mathbb{C}$ und $f$ (falls nicht anders gesagt) stets holomorph.
	
	\subsection{Nullstellen und isolierte Singularitäten}
	
	\begin{definition}[Nullstellen-Ordnung]
		Für $z_0\in\Omega$ wende $f:\Omega\to\mathbb{C}$ in einer Umgebung $U\subset\Omega$ 
		von $z_0$ dargestellt durch
		\[ f(z)=\sum_{n=0}^{\infty} a_n (z-z_0)^n .\]
		Die (Nullstellen-) Ordnung von $f$ bei $z_0$ ist die kleinste Natürliche $n_0 =ord_{z_0}(f)$, sd. 
		$a_{n_0} \neq 0$ und $a_n =0$ für alle $0\leq n<n_0$. \\
		Falls $ord_{z_0} (f)>0$ ist, hat $f$ bei $z_0$ eine Nullstelle der Ordnung $ord_{z_0} (f)$.
	\end{definition}
	
	\begin{example}
		\begin{enumerate}
			\item Die Sinus-Funktion hat um $0$ die Entwicklung
			\[ \sin(z)=\sum_{n=0}^{\infty} (-1)^n \frac{ z^{2n+1}}{(2n+1)!},\ \text{also $ord_0 (\sin)=1$.}\]
			also $ord_0 (\sin)=1$. Da $\sin(\pi -z)=\sin(z)$ folgt $ord_{\pi}(sin)=1$. \\
			Da $\sin(2\pi +z)=\sin(z)$ folgt $ord_{k\pi}(\sin)=1$ für alle $k\in\mathbb{Z}$ (Ansonsten 
			hat der Sinus keine Nullstellen – siehe Übung zu $\cos$).
			\item Der Cosinus hat Nullstellen der Ordnung $1$ bei $(k+\frac{1}{\varepsilon})\cdot \pi$, 
			$k\in\mathbb{Z}$.
		\end{enumerate}
	\end{example}
	
	\begin{definition}[isolierte bzw. hebbare/, wesentliche Singularität]
		Es sei $z_0\in\Omega$ und $f:\Omega\setminus\{z_0\}\to\mathbb{C}$ holomorph, dann heißt 
		$z_0$ eine isolierte Singularität von $f$.
		\begin{enumerate}
			\item Wenn sich $f$ zu einer holomorphen Funktion auf ganz  $\Omega$ forsetzen lässt, 
			heißt $z_0$ eine hebbare Singularität.
			\item Wenn es $m>1$ und Zahlen $a_n ,\ldots ,a_m \in\mathbb{C}$ mit $a_m \neq 0$ gibt, 
			sd. \[f(z)=\sum_{n=1}^m \frac{a_n}{(z-z_0)^n} \] bei $z_0$ eine hebbare Singularität hat, 
			dann hat $f$ bei $z_0$ eine Polstelle der Ordnung $m$ mit Hauptteil 
			$\sum_{n=1}^m \frac{a_n}{(z-z_0)^n}$. Wir setzen $ord_{z_0}(f)=-m$.
			\item Wenn für alle $r>0$ mit $B_r (z_0)\subset\Omega$ das Bild 
			$im (f|_{B_r(z_0)\setminus\{z_0\}})$ dicht in $\mathbb{C}$ liegt, heißt $z_0$ eine 
			wesentliche Singularität von $f$ und wir setzen $ord_{z_0}(f)=-\infty$.
		\end{enumerate}
		(Der Vollständigkeit halber sei $ord_{z_0}(0)=\infty$).
	\end{definition}
	
	\begin{example}
		\begin{enumerate}
			\item Der Tangens $\tan(z)=\frac{\sin(z)}{\cos(z)}$ hat Nullstellen der Ordnung $1$ bei 
			$z=k\pi$, $k\in\mathbb{N}$. Bei $z_0 =\frac{\pi}{2}$ schreiben wir
			\begin{eqnarray*}
				-\sin(z-\frac{\pi}{2})=\sin(\frac{\pi}{2}-z) =\cos(z)=-(z-\frac{\pi}{2})\cdot 
				\underbrace{\sum_{n=0}^{\infty} (-1)^n \frac{(z-\frac{\pi}{2})^{2n}}{(2n+1)!}}_{=f(z)
				\text{ holom. }f(\frac{\pi}{2})\neq 0}
			\end{eqnarray*}
			Da $\tan(z+\pi)=\tan(z)$, hat $\tan$ bei $(k+\frac{1}{2})\pi$ ebenfalls einen Pol der 
			Ordnung 1.
			\[ \tan(z) =-\frac{\sin(z)}{(z-\frac{\pi}{2})-f(z)}=-\frac{1}{z-\frac{\pi}{2}}+\underbrace{ 
			\frac{f(z)-\sin(z)}{(z-\frac{\pi}{2})\cdot f(z)}}_{g(z)} \]
			Da $f(z)-\sin(z)$ bei $z=\frac{\pi}{2}$ den Wert $0$ hat, gilt
			\[ f(z)-\sin(z)=\sum_{n=1}^{\infty} b_n \cdot (z-\frac{\pi}{2})^n\]
			und den obigen Bruch kürzen, somit hat $g(z)$ bei $\frac{\pi}{2}$ eine hebbare 
			Singularität. 
			Also hat $\tan(z)$ bei $z=\frac{\pi}{2}$ einen Pol der Ordnung 1 mit Hauptteil 
			$-\frac{1}{z-\frac{\pi}{2}}$ und daher $ord_{\frac{\pi}{2}}(\tan)=1$.
			\item Die Funktion $z\mapsto e^{-\frac{1}{z}}$ hat bei $z=0$ eine wesentliche 
			Singularität. Sei etwa $r>0$, dann ex. $n\in\mathbb{N}$ sd. $\frac{1}{2\pi n}<r$. Dann 
			betrachte $U=\{ \omega\in\mathbb{C} | Im (\omega)\in (2\pi n,2\pi (n+1)]\}$. 
			Aus $\omega\in U$ folgt $|\frac{1}{\omega}|<r$. Auf $U$ nimmt die Exponentialfunktion 
			alle Werte in $\mathbb{C}\setminus\{ 0\}$ an: jeder der Werte hat die Form 
			$s\cdot e^{i\varphi}=e^{\log s+i\varphi}$, $\OE \varphi\in (2\pi n,2\pi (n+1)]$. 
			Da $-\frac{1}{\omega}\in B_r(0)$ für alle $\omega\in U$ nimmt $e^{-\frac{1}{z}}$ auf 
			$B_r^{\times}(0)=B_r(0)\setminus\{0\}$ alle Werte in $\mathbb{C}^{\times} =\mathbb{C} 
			\setminus\{0\}$ an. Also ist $0$ wesentliche Singularität.
			\item Das gleiche gilt für $e^{-\frac{1}{z^2}}$, obwohl diese Funktion auf 
			$\mathbb{R}$ eine hebbare Singularität bei $0$ hat.
		\end{enumerate}
	\end{example}
	
	\begin{theorem}[Riemannscher Hebbarkeitssatz]
		Wenn $f:\Omega\setminus\{z_0\}\to\mathbb{C}$ die Eigenschaft 
		\[\lim_{z\to z_0} (z-z_0)\cdot f(z) =0\]
		hat, dann hat $f$ bei $z_0$ eine hebbare Singularität.
	\end{theorem}
	
	\begin{proof}
		Betrachte die Funktion $g(z)=(z-z_0)^2 \cdot f(z)$ auf $\Omega\setminus\{z_0\}$. Dann ist 
		$g$ stetig auf $\Omega$ fortsetzbar durch $g(z_0)=0$. Außerdem ist $g$ auf 
		$\Omega\setminus\{z_0\}$ holomorph und sogar bei $z_0$ mit $g'(z_0)=0$, denn:
		\[ \lim_{z\to z_0} \frac{g(z)-g(z_0)}{z-z_0} = \lim_{z\to z_0} (z-z_0)\cdot f(z) =0. \]
		Also ist $g$ holomorph und hat daher bei $z_0$ eine Potenzreihendarstellung 
		\[g(z)=\sum_{n=0}^{\infty} a_n (z-z_0)^n \] mit $a_0=a_1=0$. Somit lässt sich $f$ bei $z_0$ 
		zu einer holomorphen Funktion mit Potenzreihe
		\[f(z)=\sum_{n=0}^{\infty} a_{n+2}(z-z_0)^n \]
		fortsetzen.
	\end{proof}
	
	\begin{example}
		Es sei $r>0$. Dann gibt es keine holomorphe Funktion $f$ auf $B_r^{\times}(0)$ mit $f(z)^2 =z$. 
		Denn wäre $f$ eine solche Funktion, dann wäre $|f(z)|=\sqrt{|z|}$, also 
		\[\lim_{z\to z_0} (z-z_0)\cdot f(z) =0. \]
		Das heißt $f$ müsste sich holomorph auf $B_r(0)$ fortsetzen lassen, aber das geht nicht, da die 
		reelle Wurzelfunktion bei $x=0$ bereits nicht differenzierbar ist.
	\end{example}
	
	\begin{remark}
		Wir können $ord_{z_0}(f)$ wie folgt charakterisieren: $n=ord_{z_0}(f)$ ist die kleinste ganze 
		Zahl, sd.
		\[ g(z)= (z-z_0)^{-n} f(z) \]
		bei $z_0$ eine hebbare Singularität hat. 
		\begin{enumerate}
			\item $f$ hat hebbare Singularität $\Rightarrow$ $ord_{z_0}(f)\geq 0$ und $g$ ist nahe 
			$z_0$ beschränkt für $n=ord_{z_0}(f)$, (siehe Potenzreihenentwicklung), hat also 
			hebbare Singularität, wohingegen für $n=ord_{z_0}(f)+1$ die Funktion $g$ nahe 
			$z_0$ nich einmal beschränkt ist.
			\item Wenn $f$ einen Pol hat, habe
			\[ h(z)=f(z)-\sum_{n=1}^m \frac{a_n}{(z-z_0)^n} \]
			mit $m=-ord_{z_0}(f)$ und $a_m \neq 0$ eine hebbare Singularität. Also hat 
			$(z-z_0)^m \cdot f(z)$ bei $z_0$ hebbare Singularität, $(z-z_0)^{m-1}\cdot f(z)$ jedoch 
			nicht.
		\end{enumerate}
	\end{remark}
	
	\begin{theorem}[Casorati-Weierstraß]
		Sei $f:\Omega\setminus\{z_0\}\to\mathbb{C}$ holomorph, dann trifft genau eine der folgenden 
		Aussagen zu.
		\begin{enumerate}
			\item $f$ hat eine hebbare Singularität bei $z_0$
			\item $f$ hat eine Polstelle bei $z_0$
			\item $f$ hat eine wesentliche Singularität bei $z_0$
		\end{enumerate}
	\end{theorem}
	
	\begin{proof}
		Klar: (1) und (2) schließen einander aus. \\
		(3) schließt (1) und (2) aus: \\
		(1)$\Rightarrow$ $\lim_{z\to z_0} f(z)=a\in\mathbb{C}$ existiert, zu jedem $\delta >0$ existiert 
		also ein $\varepsilon >0$, sd. $im(f|_{B_{\varepsilon}^{\times}(z_0)})\subset B_{\delta}(a)$ 
		Insbesondere liegt dieses Bild nicht dicht in $\mathbb{C}$. \\
		(2)$\Rightarrow$ $\lim_{z\to z_0} |f(z)| =\infty$, dh. zu jedem $\delta >0$ ex. $\varepsilon >0$, 
		sd. $im(f|_{B_{\varepsilon}^{\times}(z_0)})\subset\mathbb{C}\setminus B_{\frac{1}{\delta}}(0)$ 
		Insbesondere liegt das Bild nicht dicht in $\mathbb{C}$. \\
		Noch zeigen: Wenn das Bild von $f|_{B_r^{\times} (z_0)}$ nicht dicht liegt, hat $f$ einen Pol 
		oder eine hebbare Singularität. Wenn $im(f|_{B_r^{\times}(z_0)})$ nicht dicht in 
		$\mathbb{C}$ ist, ex. $b\in\mathbb{C}\setminus\overline{im(f|_{B_r^{\times}(z_0)})}$, dh. es ex. 
		$\varepsilon >0$, sd. $B_{\varepsilon}(b)\cap im(f|_{B_r^{\times}(z_0)})=\emptyset.$ 
		Betrachte die Funktion $g:\Omega\setminus\{z_0\}\to\mathbb{C}$ mit 
		\[ g(z) =\frac{1}{f(z)-b} \]
		Dann ist $|g(z)| < \frac{1}{\varepsilon}$ auf $B_r^{\times}(z_0)$, somit hat $g$ eine holomorphe 
		Fortsetzung auf ganz $B_r (z_0)$. Also hat $f(z)=\frac{1}{g(z)}+b$ einen Pol oder eine 
		hebbare Singularität. \\
	\end{proof}
	
	\subsection{Das Maximumprinzip und der Satz von Liouville}
	
	Frage: Kann $|f(z)|$ lokale Maxima haben? auf ganz $\mathbb{C}$ beschränkt sein? \\
	Für $f:\mathbb{R}\to\mathbb{R}$ geht das, z. B. $\cos$, $\sin$. 
	
	\begin{theorem}[Maximumprinzip] \label{thm:Maxprin}
		Sei $f:\Omega\to\mathbb{C}$ holomorph. Wenn $z_0\in\Omega$ existiert, sd.
		\[ |f(z)|\leq |f(z_0)| \]
		für alle $z\in B_r (z_0)\subset\Omega$ ($r>0$ klein genug), dann ist $f$ auf ganz $\Omega$ 
		konstant.
	\end{theorem}
	
	\begin{proof}
		Nach dem Identitätssatz reicht es zu zeigen, dass $f$ auf $S_r (z_0)$ konstant ist. \\
		Wir nehmen an, dass $\overline{B_r (z_0)}\subset\Omega$. Nach dem Mittelwertsatz gilt dann:
		\[ f(z_0)=\int_0^1 \underbrace{f(z_0 +re^{2\pi it})}_{|\%|\leq |f(z_0)|}\mathrm{d}t \]
		Schreibe $f(z_0)=re^{i\varphi}$, dann gilt 
		\[ |f(z_0)|= Re(e^{i\varphi}f(z_0)) \]
		Aus $|\omega|\leq|f(z_0)|$ folgt also 
		\begin{eqnarray}
			Re(\underbrace{e^{i\varphi}\omega}_{|\%|\leq |f(z_0)|})\leq |f(z_0)| \label{stern}
		\end{eqnarray}
		Es folgt 
		\begin{eqnarray*}
			|f(z_0)|=Re(e^{-i\varphi}\cdot f(z_0)) &=& 
			Re\left(e^{-i\varphi}\cdot \int_0^1 f(z_0 +re^{2\pi it})\mathrm{d}t\right) \\
			&=& \int_0^1 \underbrace{Re(e^{-i\varphi}\cdot f(z_0 +re^{2\pi it}))}_{\leq |f(z_0)|}
			\mathrm{d}t \leq |f(z_0)|
		\end{eqnarray*}
		Da Gleichheit gilt und der Integrand stetig ist, folgt
		\[ Re\left( e^{-i\varphi} \cdot f(z_0 +re^{2\pi it})\right) =|f(z_0)| \]
		für alle $t$. Gleichheit in \ref{stern} gilt genau dann, wenn $|\omega|=|f(z_0)|$ und das 
		Argument (die Phase) von $\omega$ gerade $\varphi$ ist. Das heißt, wenn $\omega =f(z_0)$. 
		Also gilt $f(z_0 +re^{2\pi it})=f(z_0)$ für alle $t$ und somit ist $f$ nach dem Identitätssatz 
		konstant.
	\end{proof}
	
	\begin{example}
		$\cos$ hat zwar auf $\mathbb{R}$ ein lokales Maximum bei $0$, aber da 
		$\cos (iy)=\cosh(iy)$ hat $\cos$ kein lokales Maximum in $\mathbb{C}$.
	\end{example}
	
	\begin{corollary}[Schwarz Lemma] \label{coroll:SL}
		Es sei $f:B_r (0)\to B_r(0)$ holomorph mit $f(0)=0$. Dann gilt
		\begin{enumerate}
			\item $|f(z)|\leq |z|$ für alle $z$ \label{coroll:1}
			\item $|f'(0)|\leq 1$ \label{coroll:2}
		\end{enumerate}
		Wenn in \ref{coroll:1} für $z\neq 0$ oder in \ref{coroll:2} Gleichheit gilt, existiert $\lambda\in
		\mathbb{C}$ mit $|\lambda|=1$, sd. $f(z)=\lambda\cdot z$.
	\end{corollary}
	
	\begin{proof}
		Betrachte $g(z)=\frac{f(z)}{z}$, $g:B_r(0)\to\mathbb{C}$. Da $f(0)=0$, hat $g$ bei $0$ 
		eine hebbare Singularität, $g$ ist also auf ganz $B_r(0)$ holomorph mit 
		\[ g(0)=\lim_{z\to 0}\frac{f(z)-f(0)}{z-0}=f'(0). \]
		Sei $s<r$. Da $|f(z)|\leq r$ für alle $z$ mit $|z|=s$, folgt nach dem Maximumprinzip
		\[ |g(z)| =\frac{|f(z)|}{|z|}\leq \frac{r}{s} \]
		für alle $z$ mit $|z|=s$. (Denn stetige Funktionen auf einem Kompaktum haben stehts ein 
		Maximum.) Wegen des Maximumprinzip muss das Maximum auf dem Rand liegen. 
		Für $s\to r$ erhalten wir $|g(z)|\leq 1$ für alle $ z\in B_r(0)$. Es folgen \ref{coroll:1} \& 
		\ref{coroll:2}. Falls Gleichheit gilt, hat $|g|$ ein lokales Maximum $|g(z)|=1$ bei $z\neq 0$ 
		(im Falle \ref{coroll:1}) oder bei $0$ (im Falle \ref{coroll:2}), ist also konstant. Setze 
		$\lambda=g(z)$.
	\end{proof}
	
	\begin{definition}[biholomorphe Funktion]
		Es seien $\Omega_0 ,\Omega_1\subset\mathbb{C}$ Gebiete. Eine holomorphe, bijektive 
		Abbildung $f:\Omega_0\to\Omega_1$ heißt biholomorph. \\
		Wir sehen später (Satz 2.23)
		
		%...
		
		, dass auch die Umkehrabbildung holomorph ist. Im reellen hingegen haben wir z. B. 
		$x\mapsto x^3$, die bijektiv und differenzierbar, aber kein Diffeomorphismus ist.
	\end{definition}
	
	\begin{example} \label{exp:moebius}
		Ziel: Finde alle biholomorphen Abbildungen $B_1(0)\to B_1(0)$. \\Es sei $A=\left(
		\begin{smallmatrix} a&b \\ c&d \end{smallmatrix}\right)\in M_2 (\mathbb{C})$ mit 
		$(c,d)\neq (0,0)$, dann definieren wir die Möbiustransformation 
		$M_A (z)=\frac{az+b}{cz+d}$ für alle $z$ mit $cz+d \neq 0$ (siehe Übung). \\
		Es gilt $M_A \cdot M_B =M_{A\cdot B}$ (siehe Übung). \\
		Insbesondere hat $M_A$ eine Umkehrabbildung, wenn $A$ invertierbar ist (denn 
		$M_{E_2}(z)=z$). Sei $A\in U(1,1)=\left\{ A=\left(\begin{smallmatrix} a&b\\c&d \end{smallmatrix}
		\right)\in M_2 (\mathbb{C})\  \vert\  |a|^2 -|c|^2 =|d|^2 -|b|^2 =1 \land a\overline{b}-c\overline{d}=0 
		\right\}$. Die Inverse matrix $A^{-1} =\left( \begin{smallmatrix} \overline{a}&-\overline{c} \\
		-\overline{b}&\overline{d} \end{smallmatrix}\right)$ liegt ebenfalls in $U(1,1)$
		\[ \begin{pmatrix} \overline{a}&-\overline{c} \\
		 -\overline{b}&\overline{d} \end{pmatrix}
		\begin{pmatrix} a&b \\
		c&d \end{pmatrix}
		=\begin{pmatrix}|a|^2 -|c|^2 & \overline{a}b-\overline{c}d \\
		-a\overline{b}+c\overline{d} & |a|^2 -|b|^2 \end{pmatrix}
		=\begin{pmatrix} 1&0\\0&1 \end{pmatrix} \]
		Für $A\in U(1,1)$ gilt $M_A :B_1(0)\to B_1(0)$. Sei also $|z|<1$. 
		\begin{eqnarray*}
			|az+b|^2 &=& (az+b)(\overline{a}\overline{z}+\overline{b})=|a|^2 |z|^2 +|b|^2 +az\overline{b}
			+\overline{a}\overline{z}b \\
			&=& |c|^2|z|^2 +|d|^2 +cz\overline{d} +\overline{c}\overline{z}d+(|\overline{a}|^2 -
			|\overline{c}|^2)\cdot |z|^2 -(|d|^2 -|b|^2) \\
			&=& |cz+d|^2 +|z|^2 -1
		\end{eqnarray*}
		\[\Rightarrow \quad |M_A(z)|^2 = \frac{|az+b|^2}{|cz+d|^2}=1-\overbrace{\frac{1-|z|^2}{|cz-d|^2}}^{
		>0}<1 \]
		Beachte: $|z|<1\quad\Rightarrow\quad |cz|^2 <|c|^2 =|d|^2 -1
		\quad\Rightarrow\quad cz+d\neq 0$. Dazu benutze die Behauptung
		\[ \left(|a|^2 -|c|^2 =|d|^2 -|b|^2 =1 \land a\overline{b} =c\overline{d}\right) 
		\quad\Leftrightarrow\quad 
		\left(|a|^2 -|b|^2 =|d|^2 -|c|^2 =1 \land a\overline{c}=b\overline{d}\right) \]
		Also ist $M_A$ eine holomorphe Abbildung von $B_1(0)\to B_1(0)$. Sie ist biholomorph mit 
		Umkehrabbildung $M_{A^{-1}}$.
	\end{example}
	
	\begin{corollary}[aus dem Schwarz-Lemma]
		Es sei $f: B_1(0)\to B_1(0)$ biholomorph. Dann gilt $f=M_A$ für ein $A\in U(1,1)$. 
		($U(1,1)$ ist reell dreidimensional)
	\end{corollary}
	
	\begin{definition}
		$U(1,1)=\left\{ A=\left(\begin{smallmatrix} a&b\\c&d \end{smallmatrix}
		\right)\in M_2 (\mathbb{C})\  \vert\  |a|^2 -|c|^2 =|d|^2 -|b|^2 =1 \land a\overline{b}-c\overline{d}=0 
		\right\}$.
		\[ \begin{pmatrix}a&b\\c&d\end{pmatrix}\in U(1,1)\ \Rightarrow\ 
		\begin{pmatrix}a&b\\c&d\end{pmatrix}^{-1}=\begin{pmatrix}\overline{a}&-\overline{c} \\
		-\overline{b}&\overline{d}\end{pmatrix}\overset{!}{\in}U(1,1) \]
	\end{definition}
	
	\begin{remark}
		$U(1,1)$ ist die Menge der linearen Abbildungen von $\mathbb{C}^2\to\mathbb{C}^2$, 
		die die hermitesche Form $\langle \begin{smallmatrix}z\\w\end{smallmatrix},
		\begin{smallmatrix}u\\v\end{smallmatrix}\rangle =\overline{z}u-\overline{w}v$ erfüllt. 
		$\left( \begin{smallmatrix}a&b\\c&d\end{smallmatrix} \right) \in U(1,1)$, dann folgt 
		$|a|,|d|\geq 1$, also existiert $\lambda\in\mathbb{C}$, sd. $d=\lambda\overline{a}$. 
		Aus $a\overline{b}=c\overline{d}=c\overline{\lambda} a$ folgt $b=\lambda\overline{c}$ 
		$1=|d|^2 -|b|^2 =|\lambda |^2 (|a|^2 -|c|^2 )=|\lambda |^2 \quad \Rightarrow \quad |\lambda|=1$ 
		Also gilt $|a|^2 -|b|^2 =|a|^2 -|c|^2 =1$ und $|d|^2 -|c|^2 =|a|^2 -|c|^2 =1$ und $a\overline{c} -
		b\overline{d}=a\overline{c} -\lambda \overline{c}\overline{\lambda}a=a\overline{c} - 
		\overline{c}a=0$. 
		Hieraus folgt, dass 
		\[ \begin{pmatrix} a&b \\ c&d \end{pmatrix}^{-1} = \begin{pmatrix} \overline{a} & -\overline{c} \\
		-\overline{b} & \overline{d} \end{pmatrix} \in U(1,1) \]
		Analog zeige, dass $U(1,1)$ unter Multiplikation abgeschlossen ist.
	\end{remark}
	
	\begin{repetition}
		$M_{A}^{-1} =M_{A^{-1}} \quad\Rightarrow\quad M_A :B_1(0)\to B_1(0)$ biholomorph.
	\end{repetition}
	
	\begin{corollary} \label{coroll:moebius}
		Es sei $f:B_1(0)\to B_1(0)$ biholomorph. Dann gilt $f=M_A$ für ein $A\in U(1,1)$.
	\end{corollary}
	
	\begin{proof}
		\textsl{Idee:} Benutze Schawarzlemma. \\
		Dazu brauchen wir eine Abbildung $f$, sd. $f(0)=0$. Sei zunächst $f$ wie in der Beh. 
		$z_0 =f(0)=r\cdot e^{i\varphi}$. Es gilt
		\[ B=\begin{pmatrix}1&0\\0&e^{i\varphi}\end{pmatrix}\in U(1,1) \text{   und   } 
		(M_B \circ f)(0)=\frac{1\cdot f(0)+0}{0+e^{i\varphi}}=r\in\mathbb{R}. \]
		Da $M_B$ und $f$ biholomorph sind, ist auch $M_B \circ f$ biholomorph. Für alle 
		$t\in\mathbb{R}$ gilt 
		\[C_t =\begin{pmatrix}\cosh(t)&\sinh(t)\\ \sinh(t)&\cosh(t)\end{pmatrix}\in U(1,1),\]
		denn $\cosh(t)^2 -\sinh(t)^2 =1$, da für $s\in\mathbb{R}$ gilt 
		$1=\cos(s)^2+\sin(s)^2=\cosh(is)^2+(\pm i\cdot \sinh(is))^2 =\cosh(is)^2-\sinh(is)^2$, für alle 
		$t\in\mathbb{C}$.
		\[ M_{C_t}(r)=\frac{r \cosh(t)+\sinh(t)}{r\sinh(t)+\cosh(t)}=0 \quad\Leftrightarrow\quad 
		r=-\frac{\sinh(t)}{\cosh(t)}=-\tanh(t) \]
		\[\tanh' =\frac{\cosh(t)^2 -\sinh(t)^2}{\cosh(t)^2}=\frac{1}{\cosh(t)^2}>0 \]
		für alle $t\in\mathbb{R}$. Also ist $\tanh :\mathbb{R}\to I$ umkehrbar mit Bildbereich 
		$I=(\lim_{t\to\infty}\tanh(t) ,\lim_{t\to\infty} \tanh(t))=(-1,1)$, denn 
		\[ \lim_{t\to \pm \infty} \frac{e^t -e{-t}}{e^t +e^{-t}}=\pm 1. \]
		Also existiert ein $t_0$, sd. $r=-\tanh(t_0)$, denn $r\in (-1,1)$. Es folgt
		\[(M_{C_{t_0}}\circ M_B \circ f)(0)=\frac{r\cosh(t_0)+\sinh(t_0)}{r\sinh(t_0)+\cosh(t_0)}=0 \]
		und $M_{C_{t_0}}\circ M_B\circ f:B_1(0)\to B_1(0)$ ist biholomorph. \\
		Umgekehrt erhalten wir $f$ zurück als $M_{B^{-1}}\circ M_{C_{t_0}^{-1}}\circ 
		(M_{C_{t_0}}\circ M_B \circ f)=f$. Also sei ohne Einschränkung $f:B_1(0)\to B_1(0)$ 
		biholomorph mit $f(0)=0$. Es sei $g:B_1(0)\to B_1(0)$ die Umkehrfunktion, dann gilt auch 
		$g(0)=0$. Wir leiten $f\circ g=id$ bei $z=0$ ab und erhalten $f'(0)=g'(0)=1$. 
		Aus dem Schwarzlemma folgt $1\geq |f'(0)|\cdot |g'(0)|=1$. Also existiert ein $\lambda\in 
		\mathbb{C}$ mit $|\lambda|=1$, sd. $f(z)=\lambda \cdot z$ für alle $z\in B_1(0)$. 
		Also ist $f=M_A$ mit $A=\left( \begin{smallmatrix}\lambda&0\\0&1\end{smallmatrix}\right) 
		\in U(1,1)$, denn $M_A =\frac{\lambda z+0}{0+1}=\lambda z$.
		\[M_{B^{-1}}\circ M_{C_{t_0}^{-1}}\circ M_A \underset{\text{Übung}}{=} M_{B^{-1}C_{t_0}^{-1}A}\]
		\begin{eqnarray*}
			B^{-1}C_{t_0}^{-1} A &=& \begin{pmatrix}1&0\\0&e^{-i\varphi}\end{pmatrix}\cdot 
			\begin{pmatrix}\cosh(t_0)&-\sinh(t_0)\\-\sinh(t_0)&\cosh(t_0)\end{pmatrix}
			\begin{pmatrix}\lambda&0\\0&1\end{pmatrix}\\
			&=&\begin{pmatrix} \lambda \cosh(t_0)&-\sinh(t_0)\\-e^{-i\varphi}\lambda\sinh(t_0) &
			e^{-i\varphi}\cosh(t_0)\end{pmatrix}\in U(1,1)
		\end{eqnarray*}
	\end{proof}
	
	\begin{remark}
		Im Gegensatz dazu gibt es sehr viele Diffeomorphismen $f:(0,1)\overset{\sim}{\to}(0,1)$ 
		z. B. alle Polynome $P$ mit $P(0)=0$, $P(1)=1$, sd. $P'$ auf $(0,1)$ keine Nullstelle hat.
	\end{remark}
	
	\begin{theorem}[Satz von Liouville] \label{thm:LV}
		Es sei $f:\mathbb{C}\to\mathbb{C}$ holomorph und beschränkt, dann ist $f$ konstant.
	\end{theorem}
	
	\begin{proof}
		$f$ beschränkt, das heißt es existiert $C\in\mathbb{R}$, sd. $|f(z)|\leq C$ für alle 
		$z\in\mathbb{C}$. Schreibe $f(z)=\sum_{n=0}^{\infty} a_n z^n$. Für $r>0$ gilt
		\[a_n=\frac{1}{2\pi i}\int_{S_r^1}\frac{f(z)}{z^{n+1}}\mathrm{d}z =
		\frac{1}{2\pi i}\int_0^{1\pi}\frac{f(r\cdot e^{i\varphi})}{r^{n+1}e^{i(n+1)\varphi}}ire^{i\varphi}
		\mathrm{d}\varphi \]
		\[ \Rightarrow\quad |a_n|\leq \frac{1}{2\pi i}\int_0^{2\pi}\frac{|f(re^{i\varphi})|}{r^{n+1}}r 
		\mathrm{d}\varphi\leq C\cdot \frac{1}{r_n} \]
		Für $r\to\infty$ erhalten wir $a_n =0$ für alle $n\geq 1$. Somit ist $f(z)=a_0$ konstant.
	\end{proof}
	
	\begin{repetition}
		Ein Polynom $P(z)=a_n z^n+\ldots +a_0$ mit $a_n\neq 0$ hat höchstens $n=deg\ P$ viele 
		Nullstellen.
	\end{repetition}
	
	\begin{corollary}[Fundamentalsatz der Algebra]
		Jedes nichtkonstante Polynom über $\mathbb{C}$ hat eine Nullstelle in $\mathbb{C}$. 
		(Daraus folgt induktiv, dass jedes normierte Polynom in ein Produkt von Linearfaktoren 
		zerfällt.)
	\end{corollary}
	
	\begin{proof}
		Es sei $P=a_n z^n +\ldots +a_0$ ein Polynom ohne Nullstellen in $\mathbb{C}$. Es gelte 
		$a_n \neq 0$. Dann ist $f=\frac{1}{P}$ eine holomorphe Funktion. \\
		\textsl{Behauptung:} $f$ ist beschränkt. \\
		Schreibe dazu $P(z)=a_n z^n \cdot (1+\ldots +\frac{a_0}{a_n} z^{-n})$. Sei 
		$b=|\frac{a_{n-1}}{a_n}|+\ldots +|\frac{a_0}{a_n}|$. Für $|z|>2b$ folgt dann
		\[ |1+\frac{a_{n-1}}{a_n}\cdot z^{-1}+\ldots +\frac{a_0}{a_n} z^{-n}|\geq\frac{1}{2} \]
		Somit ist $f=\frac{1}{P}$ auf $\mathbb{C}\setminus\overline{B_{2b}(0)}$ durch 
		$\frac{2}{a_n \cdot (2b)^n}$ beschränkt. Da $\overline{B_{2b}(0)}$ kompakt ist und 
		$|f|$ stetig ist, ist $|f|$ auch auf $\overline{B_{2b}(0)}$ beschränkt, also auf ganz $\mathbb{C}$. 
		Nach dem Satz von Liouville ist $f$ und $P=\frac{1}{f}$ konstant.
	\end{proof}
	
	\begin{remark}
		Es ist durchaus legitim, den $"$Fundamentalsatz der Algebra$"$ mit analytischen 
		Methoden zu beweisen, denn $\mathbb{C}$ wurde aus $\mathbb{R}$ konstruiert und 
		$\mathbb{R}$ aus $\mathbb{Q}$ durch Vervollständigung. Dadurch sind die reellen Zahlen 
		kein algebraisches, sondern ein analytisches Konstrukt.
	\end{remark}
	
	\subsection{Das lokale Abbildungsverhalten holomorpher Funktionen}
	
	\begin{repetition}
		Es sei $F:U\to \mathbb{R}^N$ eine $C^1$-Funktion, $U\subset \mathbb{R}^N$ offen, 
		bei $x_0 \in U$ sei $\mathrm{d}F(x_0)\in M_N (\mathbb{R})$ invertierbar. Dann gibt es 
		Umgebungen $V\subset U$ von $x_0$ und $W\subset \mathbb{R}^N$ von $y_0=F(x_0)$, 
		sd. $F|_V :V\to W$ ein Diffeomorphismus ist. Sei $G:W\to V$ die Umkehrabbildung, dann 
		gilt $\mathrm{d}F(G(y))\cdot \mathrm{d}G(y)=E_N$ für alle $y\in W$.
	\end{repetition}
	
	\begin{theorem}[lokaler Umkehrsatz]
		Es sei $f:\Omega\to\mathbb{C}$, $z_0\in\Omega$ mit $f'(z_0)\neq 0$. Dann existieren 
		Umgebungen $U\subset\Omega$ von $z_0$ und $V\subset\mathbb{C}$ von $w_0=f(z_0)$, 
		sd. $f|_U :U\to V$ biholomorph ist.
	\end{theorem}
	
	\begin{proof}
		Schreibe $f'=u+iv \neq 0$ nach Voraussetzung mit $u,v:\Omega\to\mathbb{R}$. Dann ist 
		als $2\times 2$ Matrix die relle Ableitung von $f$ bei $z_0$ gegeben durch 
		\[ \mathrm{d}f(z_0) =\begin{pmatrix} u(z_0)&-v(z_0)\\v(z_0)&u(z_0)\end{pmatrix} \]
		und diese Matrix ist invertierbar, denn ihre Determinante ist $u^2 +v^2 =|f'|^2$. 
		Der lokale Umkehrsatz aus Analysis II liefert $U,V$ und eine $C^1$-Umkehrfunktion 
		$g:V\to U$. Da gilt
		\[ \mathrm{d}g(w) =\mathrm{d} f(g(w))^{-1} =\frac{1}{|f'(g(w))|^2} 
		\begin{pmatrix} u(g(w))&v(g(w))\\-v(g(w))&u(g(w))\end{pmatrix}, \]
		ist $g$ komplex differenzierbar mit Ableitung $g'(w)=\frac{1}{f'(g(w))}$.
	\end{proof}
	
	\section{Der Residuensatz}
	
	\textsl{Frage:} Was passiert mit dem Kurvenintegral über einer geschlossenen Kurve, die ein Gebiet 
	umläuft, wenn der Integrand im Inneren isolierte Singularitäten hat?
	\[ \int_{\gamma} f(z)\mathrm{d}z =\sum_{j=1}^k \frac{1}{2\pi i} \underbrace{n_{\gamma}(z_j)}_{\in\mathbb{Z}}
	\cdot Res_{z_j}(f) \]
	Mit diesem Satz lassen sich viele reelle Integrale bestimmen.
	
	\subsection{Umlaufzahl und Homologie}
	
	\textsl{Ziel:} Verstehe die Zahl $n_{\gamma}(z)$, die Umlaufzahl von $\gamma$ um $z$. \\
	\textsl{Motivation:} Das Kurvenintegral ist invariant unter Umparametrisierung und man kann 
	Integrationsbereiche zerstückeln und neu zusammensetzen.
	
	\begin{definition}[formale Linearkombination]
		Sei $M$ eine Menge. Eine formale-($\mathbb{Z}$-)Linearkombination von Elementen aus $M$ 
		ist ein Ausdruck der Form $\sum_{m\in M} a_m \cdot m$, wobei $a_m =0$ für alle bis auf endlich 
		$m\in M$. Diese bilden eine abelsche Gruppe unter Addition und lassen sich mit $n\in\mathbb{Z}$ 
		multiplizieren.
	\end{definition}
	
	\begin{definition} \label{def:Kette}
		Es sei $\Omega\subset\mathbb{C}$ ein Gebiet. 
		\begin{enumerate}
			\item \label{def:Kette1}
			Zwei formale Linearkombinationen von stückweisen ($C^1$-Kurven) in $\Omega$ 
			heißen als Kette äquivalent, wenn ihre Differenz eine formale $\mathbb{Z}$-
			Linearkombination von Ausdrücken der Formen 
			\begin{eqnarray*}
				\gamma -sign(\dot{\varphi})(\gamma \circ\varphi) \\
				\gamma -\gamma|_{[a,b]} -\gamma|_{[c,d]} 
			\end{eqnarray*}
			ist, wobei $\gamma:[a,b]\to\Omega$ stückweise $C^1$-Kurve sei, 
			$\varphi:[c,d]\to [a,b]$ stückweiser $C^1$-Diffeomorphismus ist und $c\in (a,b)$. 
			Eine Äquivalenzklasse von stückweisen $C^1$-Kurven heißt 
			(ganzzahlige 1-)Kette in $\Omega$, die Menge aller Ketten bezeichnen wir mit 
			$C(\Omega)$.
			
			\item Der Rand einer Kette $c=n_1 \gamma_1 +\ldots+n_k\gamma_k$ ist die formale 
			Linearkombination von Punkten in $\Omega$ 
			\[ \partial c=n_1 [\gamma_1(b_1)]-n_1[\gamma_1(a_1)] \pm\ldots +n_k[\gamma_k(b_k)]-
			n_k[\gamma_k(a_k)], \]
			wobei $\gamma_i :[a_i,b_i]\to\Omega$ stückweise $C^1$-Kurve ist. 
			Eine Kette heißt geschlossen oder Zykel, wenn $\partial c=0$. Die Menge aller 
			(ganzzahligen 1-)Zykel in $\Omega$ bezeichnen wir mit $Z(\Omega)$.
		\end{enumerate}
	\end{definition}
	
	\begin{example}
		\begin{enumerate}
			\item Sei $\gamma_[a,b]\to\Omega$ geschlossene Kurve, sei $[\gamma]$ die 
			zugehörige Kette, damit ist $c=[\gamma]$ ein Zykel, denn 
			$\partial c =[\gamma(b)]-[\gamma(a)]=0$. \\
			\textsl{Beachte:} Wir rechnen hier nicht in $\mathbb{C}$, das heißt es ist z. B. 
			$2[i]\neq [2i]$.
			
			\item Betrachte $\gamma_1 ,\gamma_2 ,\gamma_3 :[0,1]\to\mathbb{C}$ mit 
			\[\gamma_1(t)=e^{2\pi it},\qquad\quad \gamma_2(t)=-e^{2\pi it},
			\qquad\quad\gamma_3(t)=e^{-2\pi it}.\] 
			Dann gilt
			\[ [\gamma_1]=[\gamma_2]=-[\gamma_3]\]
			(werden später wichtig)
		\end{enumerate}
	\end{example}
	
	\begin{remark} \label{rem:3.3}
		\begin{enumerate}
			\item	$\partial c$ hängt nicht von den Repräsentanten von $c$ ab. Dazu betrachte die 
			Ränder der Ketten in Definition \ref{def:Kette} in \ref{def:Kette1}, z. B.: \\
			Sei $\tilde{\gamma}$ die rückwärts durchlaufende Kurve
			\[\gamma:[a,b]\to\Omega,\ \tilde{\gamma}(t)=\gamma(-t),\ \tilde{\gamma}:[-b,-a]\to\Omega \]
			Dann gilt
			\[\partial([\gamma]+[\tilde{\gamma}])=[\gamma(b)]-[\gamma(a)]+[\tilde{\gamma}(-a)]-
			[\tilde{\gamma}(-b)]=0 \]
			
			\item Äquivalent sind:
			\begin{enumerate}
				\item die Kette $c$ ist geschlossen
				\item $c$ wird durch eine Linearkombination geschlossene Kurve repräsentiert
				\item $c$ wird durch eine geschlossene Kurve repräsentiert
			\end{enumerate}
			\textsl{Begründung:} Es sei $c=n_1 [\gamma_1]+\ldots+n_k[\gamma_k]$ \\
			(a)$\Rightarrow$(b): Wir wollen $\gamma_1$ zu einer geschlossenen Kurve ergänzen. 
			Da $\partial c=0$ gilt entweder $\gamma_1(a_1)=\gamma_1(b_1)$ und wir können induktiv 
			mit der Kette $n_2 [\gamma_2]+\ldots+n_k[\gamma_k]$ weitermachen. \\
			Falls $\gamma_1(b_1)\neq\gamma_1(a_1)$, existieren weitere Kurven, die $\gamma_1(b_1)$ 
			als Anfangs- oder Endpunkt haben. Zeige jetzt, dass sich an diesen und weiteren Kurven 
			unter den $\gamma_2,\ldots,\gamma_k$ $n_1$ geschlossene Kurven bilden lassen. 
			Danach bleibt wie oben eine Linearkombination $c'$ der Kurven $\gamma_2,\ldots,\gamma_k$ 
			mit $\partial c'=0$. Da sich die Zahl der verfügbaren Kurven Schritt für Schritt verringert, 
			ist nach endlich vielen Schritten Schluss. \\
			(b)$\Rightarrow$(c): Sei $c=n_1[\gamma_1]+\ldots+n_k[\gamma_k]$ und 
			$\gamma_i :[a_i,b_i]\to\Omega$ sei geschlossen für alle $i$. Da $\Omega$ zusammenhängend 
			ist, dürfen wir $z_0\in\Omega$ und Kurven $\alpha_i$ von $z_0$ nach $\gamma_i(a_i)=
			\gamma_i(b_i)$ wählen. Dann wird $c$ durch eine geschlossene Kurve $\gamma$ 
			repräsentiert, wobei $\gamma$ von $z_0$ entlang $\alpha_1$ zum Punkt $\gamma_1(a_1)$ 
			läuft, dann $\gamma_1$ $n_1$ durchläuft ($|n_1|$-mal rückwärts, falls $n_1<0$), dann 
			entlang $\alpha_1$ rückwärts zu $z_0$ zurück, dann entlang $\alpha_2$ zu 
			$\gamma_2(\alpha_2)$ usw. \\
			(c)$\Rightarrow$(a): klar.
		\end{enumerate}
	\end{remark}

	\begin{definition}
		Es sei $c=n_1[\gamma_1]+\ldots+n_k[\gamma_k]$ eine Kette in $\Omega$ und 
		$f:\Omega\to\mathbb{C}$ stetig. Dann definieren wir
		\[ \int_c f(z)\mathrm{d}z =n_1 \int_{\gamma_1}f(z)\mathrm{d}+\ldots+
		n_k \int_{\gamma_k}f(z)\mathrm{d}z \]
	\end{definition}
	
	\begin{remark}
		\begin{enumerate}
			\item Das ist wohldefiniert, denn: Da das Kurvenintegral parametrisierungsunabhängig ist, 
			verschwindet $\int f(z)\mathrm{d}z$ über der Kette in Definition \ref{def:Kette1}.
			
			\item Sei analog $b=m_1[z_1]+\ldots+m_l[z_l]$ eine formale Linearkombination von 
			Punkten in $\Omega$, dann definiere
			\[ f(b)=m_1f(z_1)+\ldots+m_lf(z_l) \]
			Falls $f$ holomorph mit Ableitung $f'$ ist, folgt mit Folgerung \ref{Hauptsatz}, dass
			\[ \int_c f'(z)\mathrm{d}z=f(\partial c) \]
		\end{enumerate}
	\end{remark}
	
	\begin{definition}[Umlaufzahl, nullhomolog, homolog]
		Es sei $\Omega\subset\mathbb{C}$ ein Gebiet, $c$ geschlossene Kette (Zykel) in $\Omega$ 
		und $w\in\mathbb{C}\setminus\Omega$. Dann definiere die Umlaufzahl von $c$ um $w$ durch
		\[ n_w = \frac{1}{2\pi i} \int_c \frac{1}{z-w}\mathrm{d}z. \]
		Ein Zykel $c$ heißt nullhomolog, falls $n_w(c)=0$ für alle $w\in\mathbb{C}\setminus\Omega$. \\
		Zwei Zykel heißen homolog, wenn ihre Differenz nullhomolog ist. Die Menge aller, zu einer 
		gegebenen Kette $c$, homologen Ketten bildet die Homologieklasse von $c$. 
		Die Menge der Homologieklassen bildet die (1.) Homologie von $\Omega$.
	\end{definition}
	
	\begin{remark}
		Die Umlaufzahl ist additiv, das heißt
		\begin{eqnarray*}
			n_w(c_0+c_1)&=&n_w(c_0)+n_w(c_1) \\
			n_w(l\cdot c)&=&l\cdot n_w(c)
		\end{eqnarray*}
		Es folgt, dass wir Homologieklassen addieren und mit ganzen Zahlen multiplizieren können. 
		Also ist $H(\Omega)$ eine abelsche Gruppe.
	\end{remark}
	
	\begin{example}
		Sei $\Omega=\mathbb{C}^{\times}=\mathbb{C}\setminus\{0\}$, $w=0$, $\gamma(t)=e^{2\pi it}$, 
		$t\in[0,1]$ dann ist $c=[\gamma]$ geschlossen in $\Omega$ und 
		\[ n_0(c)=\frac{1}{2\pi i} \int_{\gamma}\frac{1}{z-0}\mathrm{d}z =\frac{1}{2\pi i} \int_0^1 
		\frac{2\pi i\cdot e^{2\pi it}}{e^{2\pi it}} \mathrm{d}t=1. \]
		Sei jetzt $\Omega =B_2^{\times}(0)$, $\gamma$ wie oben, dann ist $n_2(c)=0$, denn 
		$\frac{1}{z-2}$ ist sogar auf $B_2(0)$ holomorph, also gilt der Cauchy-Integralsatz.
	\end{example}
	
	\begin{proposition}
		Sei $\Omega\subset\mathbb{C}$ ein Gebiet
		\begin{enumerate}
			\item Die Umlaufzahl ist ganzzahlig. \label{prop:umlauf1}
			\item Sie ist lokal konstant auf $\mathbb{C}\setminus\Omega$. \label{prop:umlauf2}
			\item Wenn $c$ eine formale Linearkombination nullhomotoper Kurven ist, gilt 
			$n_w(c)=0$ für alle $w\in\mathbb{C}\setminus\Omega$, d. h. $c$ ist nullhomolog. 
			\label{prop:umlauf3}
		\end{enumerate}
	\end{proposition}
	
	\begin{proof}
		Zu \ref{prop:umlauf1}: Sei ohne Einschränkung $w=0$. Zerlege $\Omega$ in offene Teilmengen 
		\[\Omega_+ =\Omega\setminus\{ x\vert x\in(-\infty,0]\},\quad \Omega_-=\Omega\setminus
		\{x\vert x\in[0,\infty)\}.\]Sei $\gamma$ eine geschlossene Kurve (keine Einschränkung nach 
		Bemerkung \ref{rem:3.3} (2)) in $\Omega$, $\gamma:[a,b]\to\Omega$. Dann bilden die 
		Zusammenhangskomponenten von $\gamma^{-1}(\Omega_{\pm})$ eine offene Überdeckung 
		von $[a,b]$, also existiert nach Heine-Borel eine endliche Teilüberdeckung von $[a,b]$. 
		Also existieren $a=t_0<t_1<\ldots<t_k=b$, sd. $\gamma|_{[t_{i-1},t_i]}\subset\Omega_+$ oder 
		$\gamma|_{[t_{i-1},t_i]}\subset\Omega_-$ für alle $1\leq i\leq k$. Auf $\Omega_+$ und $\Omega_-$ 
		hat $\frac{1}{z}$ eine Stammfunktion (auf $\Omega_+$ ist das der  Hauptzweig des Logarithmus, 
		$\log$, auf $\Omega_-$ nennen wir sie $\tilde{\log}$) \\
		Wir wählen $\tilde{\log}$ so, dass
		\begin{eqnarray*}
			\tilde{\log}(z)=\log(z),\ \text{falls $Im(z)>0$} \\
			\tilde{\log}(z)=\log(z)+2\pi i,\ \text{falls $Im(z)<0$}
		\end{eqnarray*}
		wir berechnen 
		\[ \int_{\gamma} \frac{1}{z}\mathrm{d}z \]
		einzeln auf $\gamma|_{[t_{i-1},t_i]}$ mit dem Hauptsatz 
		(Folgerung \ref{Hauptsatz}). An $t_i$ unterscheiden sich die Beträge um $\pm 2\pi i$ (oder $0$). Dito
		unterscheiden sich die Werte bei $a$ und $b$ um $0$ oder $\pm 2\pi i$, da $\gamma$ geschlossen ist.
		Da \[n_0(\gamma)=\frac{1}{2\pi i}\int_{\gamma} \frac{1}{z}\mathrm{d}z,\]
		folgt $n_0(\gamma)\in\mathbb{Z}$. \\
		
		Zu \ref{prop:umlauf2}: Nach Analyis I hängt das Integral stetig vom Integranden ab (unter 
		Voraussetzungen, die hier erfüllt sind), also hängt 
		$n_w(c)=\frac{1}{2\pi i} \int_c \frac{1}{z-w}\mathrm{d}z$ 
		stetig von $w\in\mathbb{C}\setminus\Omega$ ab.
		Da die Wertemenge $\mathbb{Z}$ diskret ist, ist $w\mapsto n_w(c)$ lokal konstant. \\
		
		Zu \ref{prop:umlauf3}: Da der Integrand $\frac{1}{z-w}$ auf $\Omega$ holomorph ist und das 
		Kurvenintegral homotopieunabhängig ist, folgt 
		\[n_w([\gamma])=0\] für nullhomotope 
		Kurven $\gamma$. Das das für alle Punkte $w\in\mathbb{C}\setminus\Omega$ gilt, ist $[\gamma]$ 
		nullhomolog.
	\end{proof}
	
	\ \\
	
	Wir wollen erhalten 
	\[ "\int_{\sum_{j=1}^l b_j [\gamma_j]}f(z)\mathrm{d}z"\  = \sum_{j=1}^l b_j \int_{\gamma_j}f(z)\mathrm{d}z \]
	Zwei formale Linearkombinationen sind als Ketten äquivalent, wenn die Integrale für jeden (stetigen) 
	Integranden gleich sind. 
	\[ \int_{\gamma}f(z)\mathrm{d}z=sign(\dot{\varphi})\int_{\gamma\circ\varphi}f(z)\mathrm{d}z \]
	$\gamma:[a,b]\to\Omega$, $\varphi:[c,d]\to[a,b]$ Diffeomorphismus. Möchte also $[\gamma]=
	sign(\dot{\varphi})\cdot [\gamma\circ\varphi]$.
	
	\ \\
	
	\begin{example}
		\begin{enumerate}
			\item $\gamma_1(t)=e^{2\pi it}$, $\gamma_2(t)=2e^{2\pi it}$ Kurven in $\mathbb{C}^{\times}
			=\mathbb{C}\setminus\{0\}$. $[\gamma_1]$ ist homolog zu $[\gamma_2]$ in $\mathbb{C}
			\setminus\{0\}$ (aber nicht in $\mathbb{C}\setminus\{0,1+i\}$)
			\[n_{1+i}(\gamma_1)=0,\qquad \qquad \qquad n_{1+i}(\gamma_2)=1.\]
			
			\item Es gibt Kurven, die in gewissen $\Omega$ nullhomolog, aber nicht nullhomotop sind 
			(Übung).
		\end{enumerate}
	\end{example}
	
	\begin{remark}
		$"$Homologie$"$ (Äquivalenzklassen homologer Ketten) zählt $"$Löcher$"$ in Gebieten mit 
		algebraischer Topologie.
	\end{remark}
	
	\subsection{Der Cauchy-Integralsatz in der Umlaufzahlversion}
	
	\textsl{Ziel:} Zykel $c_1, c_2$ sind genau dann homolog in $\Omega$, wenn alle Integrale holomorpher 
	Funktionen auf $\Omega$ über $c_1$ den gleichen Wert wir über $c_2$ haben.
	
	\begin{theorem}[Cauchy-Integralsatz; Umlaufzahlversion] \label{thm:CI;umlauf}
		Es sei $f:\Omega\to\mathbb{C}$ holomorph, $c$ nullhomologer Zykel in $\Omega$, dann gilt
		\[ \int_c f(z)\mathrm{d}z=0 \]
	\end{theorem}
	
	Der Satz folgt aus dem Cauchy-Integralsatz \ref{thm:CI} und dem folgenden Lemma.
	
	\begin{lemma}[Artin]
		Es sei $c$ Zykel in $\Omega$. Dann sind äquivalent
		\begin{enumerate}
			\item $c$ ist nullhomolog in $\Omega$; \label{artin1}
			\item $c$ lässt sich als Linearkombination nullhomotoper, geschlossener Kurven schreiben; 
			\label{artin2}
			\item $c$ wird durch eine geschlossene, nullhomotope Kurve in $\Omega$ dargestellt. 
			\label{artin3}
		\end{enumerate}
		(vergleiche Bemerkung \ref{rem:3.3}; Literatur: z. B. [Jänich], Beweis unseres Satzes 
		\ref{thm:CI;umlauf})
	\end{lemma}
	
	\begin{proof}
		$(1)\Rightarrow (2)$:
		\begin{enumerate}
			\item \textsl{Schritt:} Ersetze $c$ durch eine geschlossene Kurve $\gamma$. \label{schritt1}
			\item \textsl{Schritt:} $"$Ersetze$"$ $\gamma$ durch einen $"$Kantenweg$"$. \label{schritt2}
			\item \textsl{Schritt:} Ersetze diesen Kantenweg durch einen Weg, der keinen Punkt außerhalb 
			des Gitters durchläuft. \label{schritt3}
			\item \textsl{Schritt:} Zeige, dass dieser Zykel $0$ ist. \label{schritt4}
		\end{enumerate}
		Zu \ref{schritt2}.: Sei $\gamma:[0,1]\to\Omega$ geschlossene, nullhomologe Kurve. Zu jedem $t$ 
		existiert 
		$r(t)>0$, sd. $B_{r(t)}(\gamma(t))\subset\Omega$, da $\Omega$ offen ist. $r$ hängt stetig von $t$ ab. 
		Da $[0,1]$ kompakt ist, existiert $r_0 >0$ mit $r(t)>r_0$, d. h. der $r_0$-Ball um $\gamma(t)$ liegt 
		stets in $\Omega$. Wähle $0<\varepsilon <\frac{r_0}{3}$ und lege ein Gitternetz der 
		Machenweite $\varepsilon$ über $\mathbb{C}$. Das heißt wir schreiben $\mathbb{C}$ als 
		Vereinigung abeschlossener Quadrate der Seitenlänge $\varepsilon$, die sich höchstens in 
		einer Kante oder einer Ecke überschneiden. Wähle $n\in\mathbb{Z}$ so, dass jedes Teilstück 
		$\gamma |_{[\frac{k-1}{n},\frac{k}{n}]}$ Länge $<\varepsilon$ hat für alle $1\leq k\leq n$. 
		Zu jedem $k$ sei $z_k$ die linke untere Ecke des Quadrates, in dem $\gamma(\frac{k}{n})$ liegt. 
		Es bezeichne $\tilde{\gamma}$ einen möglichst kurzen Kantenweg durch die Punkte $z_k$. 
		Nach Wahl von $\varepsilon$ liegt $\tilde{\gamma}$ in $\Omega$ (denn Punkte von $\tilde{\gamma}$ 
		haben maximal den Abstand $\mathrm{d}(\gamma(\frac{k}{n}),z_k)+\mathrm{d}(\tilde{\gamma}(t),z_k)
		<2\cdot \sqrt{2}\varepsilon <3\varepsilon$ zur Kurve $\gamma$, für alle $t\in [\frac{k}{n},\frac{k+1}{n}]$). 
		Die Homotopie $h(t,s)=(1-s)\cdot \gamma(t) +s\cdot \tilde{\gamma}(t)$ zwischen $\gamma$ und 
		$\tilde{\gamma}$ verläuft ebenfalls in ganz $\Omega$. Der Einfachheit halber nehmen wir an, dass 
		$z_0 =\gamma(0) =\gamma(1) =\tilde{\gamma}(0) =\tilde{\gamma}(1)$ eine Gitterecke ist. Dann ist 
		die Kurve 
		\[ t\mapsto \begin{dcases} \tilde{\gamma}(2t) & 0\leq t\leq \frac{1}{2} \\
		\gamma(2-2t) & \frac{1}{2} \leq t\leq 1 \end{dcases} \] 
		nullhomotop in $\Omega$. Also ist $[\gamma]$ homolog zu $[\gamma]+([\tilde{\gamma}]-[\gamma])= 
		[\tilde{\gamma}]$ und die Differenz $[\tilde{\gamma}]-[\gamma]$ wird durch eine in $\Omega$ 
		nullhomotope Kurve dargestellt. \\
		Zu \ref{schritt3}.: Zu jedem Quadrat $Q$, das in $\Omega$ liegt, konstruieren wir einen Kantenweg in $
		\Omega$ von $z_0$ zur linken unteren Ecke. Dann umlaufen wir $Q$ einmal im mathematischen 
		Drehsinn und laufen über $-\alpha_Q$ zurück zum Punkt $z_0$. Der so enstandene Kantenweg 
		$\gamma_Q$ ist nullhomotop in $\Omega$; die zugehörige Homotopie $H_Q(t,s)$ zieht für 
		$s\leq\frac{1}{2}$ zunächst das Quadrat $Q$ auf seine linke untere Ecke. Für $s\geq \frac{1}{2}$ 
		zieht sie den Weg $\alpha_Q (-\alpha_Q)$ auf $z_0$ zusammen. Betrachte die Kette 
		\[ [\tilde{\gamma}]-\sum_Q n_Q (\tilde{\gamma})\cdot \underbrace{[\gamma_Q]}_{\text{nullhomotop}} \]
		Diese Linearkombination ist endlich, da nur Quadrate zwischen $\min_t Re(\tilde{\gamma}(t))$, 
		$\max_t Re(\tilde{\gamma}(t))$ in $\lambda$-Richtung sowie zur $\min_t Im(\gamma(t))$ und 
		$\max_t Im(\gamma(t))$ umlaufen werden; dabei sei $n_Q(\tilde{\gamma})$ die Umlaufzahl 
		um einen Punkt im Inneren von $Q$, z. B. um den Mittelpunkt (dazu ersetze für den Moment 
		$\Omega$ durch $Q$ ohne all diese Mittelpunkte). Es folgt, dass $n_Q (c_3)=n_Q(\tilde{\gamma})-
		n_Q(\tilde{\gamma})=0$. \\
		Zu \ref{schritt4}: \textsl{Behauptung}: $c_3$ durchläuft jede Kante im Gitternetz genau so oft 
		vorwärts wie rückwärts, d. h. $c_3=0$. Sei dazu $\gamma_0$ Teil einer Kurve durch eine Kante und 
		$\gamma_1$ ein Kantenweg, der stattdessen das benachbarte Quadrat $Q$ umläuft. \\
		$\Rightarrow$ $n_Q(\gamma_0)+1=n_Q(\gamma_1)$ (falls die Kante positiv durchlaufen wird) 
		beziehungsweise $n_Q(\gamma_0)-1=n_Q(\gamma_1)$ (falls die Kante negativ durchaufen wird).
		Sei $Q'$ das Quadrat auf der anderen Seite der betrachteten Kante. \\
		$\Rightarrow$ $n_{Q'}(\gamma_0)=n_Q(\gamma_0)+(\text{Koeffizient der Kante in }\gamma_0)$. \\
		Somit gilt 
		\[ [\gamma]=\underbrace{[\gamma]-[\tilde{\gamma}]}_{\text{nullhomotop}}+\sum_Q n_Q (\tilde{\gamma})
		\underbrace{[\tilde{\gamma}]}_{\text{nullhomotop}} \]
		Das zeigt \ref{artin1} $\Rightarrow$ \ref{artin2}, \ref{artin2} $\Rightarrow$ \ref{artin3} wie in 
		Bemerkung \ref{rem:3.3} und \ref{artin3} $\Rightarrow$ \ref{artin1}
	\end{proof}
	
	\begin{remark}
		Satz \ref{thm:CI;umlauf} hat eine Umkehrung: Wenn $f$ holomorph ist, $c$ ein Zykel in $\Omega$, sd. 
		das für das folgende Integral gilt
		\begin{eqnarray}
			\int_c f(z)\mathrm{d}z=0 \label{rem:CI;1}
		\end{eqnarray}
		für alle holomorphen Funktionen $f$, dann sind $c_1$ und $c_2$ homolog (Betrachte Differenz der 
		Kurven). Also misst $"$Homologie$"$ auf wievielen Zykeln Integrale holomorpher Funktionen 
		verschiedene Werte annhemen. \\
		\textsl{Begründung:} Die Funktionen $z\mapsto \frac{1}{z\cdot w}$ für $w\notin\Omega$ sind auf 
		$\Omega$ holomorph. Wenn \ref{rem:CI;1} gilt, verschwinden alle Umlaufzahlen, also ist $c$ dann 
		nullhomolog.
	\end{remark}
	
	Es bezeichne $S_r (w)$ den Zykel zur Kurve $t\mapsto w+re^{2\pi it}$, $[0,1]\to\mathbb{C}$ 
	(Kreis im mathematischen Sinne einmal um $w$ verschieben mit Radius $r$ in der komplexen Ebene).
	
	\begin{corollary}
		Es sei $\Omega\subset\mathbb{C}$ ein Gebiet und $c$ ein nullhomologer Zykel in $\Omega$. 
		Es seien $z_1,\ldots,z_k\in\Omega$ und $r_1,\ldots,r_k >0$ so, dass die abeschlossenen 
		Bälle $\overline{B_{r_j}(z_j)}$ in $\Omega$ liegen und paarweise disjunkt sind. 
		Dann ist $c$ in $\Omega\setminus\{z_1,\ldots,z_k\}$ homolog zum Zykel
		\begin{eqnarray}
			\sum_{j=1}^k n_{z_j}(c)\cdot S_{r_j}(z_j). \label{coroll:zyk;1}
		\end{eqnarray}
		(fehlende Skizze siehe Skriptum Niklas/Kai)\\
	\end{corollary}
	
	\begin{proof}
		Die Differenz von $c$ und dem Zykel \ref{coroll:zyk;1} ist nullhomolog in $\Omega\setminus\{z_1,\ldots,
		z_k\}$, denn sei $w\notin\Omega$, dann folgt
		\[ n_w(c)=n_w (S_{r_j}(z_j))=0, \]
		denn beide Zykel sind in $\Omega$ nullhomolog ($S_{r_j}(z_j)$ ist in $\Omega$ nullhomotop, da 
		$\overline{B_{r_j}(z_j)}\subset\Omega$). Für $w=z_l$ gilt
		\[ n_w(S_{r_j}(z_j))=S_{r_l} \]
		($j=l$ klar, $j\neq l:\overline{B_{r_j}(z_j)}\subset\Omega\setminus\{z_l\}$) Es folgt
		\[ n_{z_l}\left(c-\sum_{j=1}^k n_{z_j}(c)\cdot S_{r_j}(z_j)\right)=n_{z_l}(c)-n_{z_l}(c)=0. \]
		Also ist $c$ homolog zu \ref{coroll:zyk;1}.\\
		\textsl{Analog:} Falls $c$ in $\Omega\setminus\left( \overline{B_{r_1}(z_1)}\cup\ldots\cup
		\overline{B_{r_k}(z_k)}\right)$ verläuft, können wir ein $\varepsilon>0$ bestimmen, dass auch 
		$\overline{B_{r_j +\varepsilon}(z_j +\varepsilon)}\subset\Omega$ paarweise disjunkt sind. 
		Dann ist $c$ in $\Omega\setminus\left( \overline{B_{r_1}(z_1)}\cup\ldots\cup
		\overline{B_{r_k}(z_k)}\right)$ homolog zu
		\[ \sum_{j=1}^k n_{z_j}(c)\cdot S_{r_j +\varepsilon}(z_j). \]
	\end{proof}
	
	\subsection{Laurentreihen und das Residuum}
	
	Wir haben im Zusammenhang mit Singularitäten bereits Potenzreihen in $z$ betrachtet, in denen auch 
	negative Exponenten vorkommen, siehe Beispiel 2.4 (2),(3). 
	
	%...
	
	Das wollen wir jetzt systematisch machen. \\
	Wir brauchen für $0\leq r<R\leq \infty$ den Kreisring $A_{r,R}(w)$ (lat. $"$annulus$"$)
	\[ A_{r,R}(w)=\{z\in\mathbb{C} \ \vert\ r<|z-w|<R\} \]
	
	\begin{proposition}[Definition: Laurentreihe, Hauptteil, Nebenteil]
		Eine (komplexe) Laurentreihe ist ein Ausdruck der Form
		\[ L(z) =\sum_{k=-\infty}^{\infty} a_k z^k \]
		mit $a_k\in\mathbb{C}$ für alle $k\in\mathbb{Z}$. Die Summe über $k>0$ heißt Hauptteil, die Summe 
		über $k\leq 0$ heißt Nebenteil. \\
		Wir erhalten den inneren- und den äußeren Konvergenzradius
		\[ r=\limsup_{k\to\infty}\sqrt[k]{|a_{-k}|},\qquad R=\left(\limsup_{k\to\infty} \sqrt[k]{|a_k|}\right)^{-1} \]
		(beiden in $[0,\infty]$), dann konvergiert $L(z)$ auf $A_{r,R}(0)$, divergiert für $|z|<r$ oder 
		$|z|>R$ und für $|z|=r$ oder $|z|=R$ ist keine allgemeine Aussage möglich. \\
		Falls $r<R$ nennen wir $L$ konvergent, sonst divergent. \\
		Falls $r<R$, also $L$ konvergent, dann nennen wir $A_{r,R}(0)$ ihren Konvergenzring. \\
		Eine Laurentreihe $L$ um $w$ mit Konvergenzring $A_{r,R}(w)$ stellt eine holomorphe 
		Funktion $f:\Omega\to\mathbb{C}$ auf $U\subset\Omega\cap A_{r,R}(w)$ dar, falls für alle 
		$z\in U$ gilt, dass
		\[ f(z) = L(z-w) =\sum_{k=-\infty}^{\infty} a_k (z-w)^k. \]
	\end{proposition}
	
	\begin{proof}
		Der Nebenteil ist eine Potenzreihe in $z$ und konvergiert für 
		\[|z|<\left(\limsup_{k\to\infty} \sqrt[k]{|a_k|}\right)^{-1}=R.\]
		Der Hauptteil ist eine Potenzreihe in $\frac{1}{z}$ und konvergiert für
		\[ \left|\frac{1}{z}\right| < \left(\limsup_{k\to\infty} \sqrt[k]{|a_k|}\right)^{-1} \quad\Leftrightarrow\quad
		|z| > \limsup_{k\to\infty} \sqrt[k]{|a_{-k}|} =r. \]
		Also konvergiert $L(z)$ auf $A_{r,R}(0)$.
	\end{proof}
	
	\begin{example}
		\begin{enumerate}
			\item \label{exp:laur;1} Wir entwickeln die Funktion $\frac{1}{z-w}$ um $0$. Dann gilt für $|z|<|w|$
			\[ \frac{1}{z-w} = -\frac{\frac{1}{w}}{1-\frac{z}{w}}=-\frac{1}{w} \sum_{k=0}^{\infty} 
			\left(\frac{z}{w}\right)^k \qquad \qquad \text{(geom. Reihe)} \]
			Für $k\geq 0$ gilt $a_k=-\frac{1}{w^{k+1}}$ und 
			\[ \limsup_{k\to\infty}\sqrt[k]{|a_k|} =\frac{1}{|w|} \quad\Rightarrow\quad R=|w|. \]
			Für $|z|>|w|$ gilt
			\[ \frac{1}{z-w}=\frac{\frac{1}{z}}{1-\frac{w}{z}} =\frac{1}{2}\sum_{k=0}^{\infty}\left(\frac{w}{z}\right)^k 
			=\sum_{k=-\infty}^{-1} z^k \frac{1}{w^{k+1}}. \]
			Für $l>0$ gilt $a_{-l}=\frac{1}{w^{-l+1}}=w^{l-1}$ und
			\[ \limsup_{k\to\infty}\sqrt[k]{|a_k|}=|w| \quad\Rightarrow\quad r=|w|. \]
			(l=-k)
			
			\item Sei $f(z)=\frac{1}{z-w_0}+\frac{1}{z-w_1}$ mit $0<|w_0|<|w_1|$ (Skizze siehe Niklas/Kai) \\
			Indem wir passende Laurententwicklungen aus \ref{exp:laur;1} addieren, erhalten wir drei 
			Laurententwicklungen auf $A_{0,|w_0|}(0)$, $A_{|w_0|,|w_1|}(0)$, $A_{|w_1|,\infty}(0)$. 
			Auf dem mittleren Gebiet sind alle Koeffizienten null.
		\end{enumerate}
	\end{example}
	
	
	%...
	
	
	
	
	
	\subsection{Residuensatz und erste Anwendungen}
	
	\begin{theorem}
		Es sei $\Omega\subset\mathbb{C}$ Gebiet, $z_1,\ldots ,z_k\in\Omega$, $f:\Omega\setminus
		\{z_1,\ldots ,z_k\}\to\mathbb{C}$ holomorph. Sei $c$ Zykel in $\Omega\setminus\{z_1,\ldots,z_k\}$, 
		der in $\Omega$ nullholomorph ist. Dann gilt
		\[ \int_c f(z)\mathrm{d}z =\sum_{j=1}^k 2\pi i Res_{z_j}(f)\cdot n_{z_j}(c) \]
	\end{theorem}
	
	\begin{proof}
		Wir dürfen $c$ durch einen Zykel ersetzen, der in $\Omega\setminus\{z_1,\ldots,z_k\}$ zu $c$ 
		homolog ist (Satz 3.11). 
		
		%...
		
		Nach Folgerung 3.14 
		
		%...
		
		wählen wir dazu den Zykel $n_{z_1}(c)\cdot S_{\varepsilon}^1 (z_1)+\ldots+n_{z_k}(c)\cdot 
		S_{\varepsilon}^1(z_k)$ für $\varepsilon>0$ klein genug. Mit Proposition 3.19 
		
		%...
		
		folgt
		\begin{eqnarray*}
			\int_c f(z)\mathrm{d}z &=& \sum_{j=1}^k n_{z_j}(c) \int_{S_{\varepsilon}^1(z_j)}f(z)\mathrm{d}z \\
			&=& 2\pi i\sum_{j=1}^k n_{z_j}(c)\cdot Res_{z_j}(f).
		\end{eqnarray*}
	\end{proof}
	
	Zum Anwenden: versuche ein gegebenes Integral (zum Beispiel im Reellen) in ein komplexes Kurvenintegral 
	über einen Zykel umzuformen. Danach bestimme das Residuum an allen Punkten, die umlaufen werden.\\
	Ein \underline{rationaler Ausdruck} in den Größen $X_1,\ldots,X_k$ ist ein Ausdruck, der aus 
	$X_1,\ldots,X_k$ und Zahlen in $\mathbb{C}$ durch Anwenden der Grunrechenarten $(+,-,\cdot,:)$. 
	Ein \underline{Polynom} in $X_1,\ldots,X_k$ ist ein rationaler Ausdruck, in dem nicht dividiert wird. 
	Jeder rationale Ausdruck lässt sich durch geeignetes Erweitern als Quotient zweier Polynome schreiben.
	
	\begin{example}
		Der Tangens ist ein rationaler Ausdruck in $e^{iz}$ und $e^{-iz}$, denn 
		\[ \tan(z)=\frac{\sin(z)}{\cos(z)}=\frac{\frac{e^{iz}-e^{-iz}}{2i}}{\frac{e^{iz}+e^{-iz}}{2}}=-i
		\frac{e^{iz}-e^{-iz}}{e^{iz}+e^{-iz}}. \]
		Aber $e^{-z^2}=e^{(iz)^2}$ ist \underline{kein} rationaler Ausdruck in $e^{iz}$.
	\end{example}
	
	\begin{corollary}
		Es sei $R(\cos x,\sin x)$ ein rationaler Ausdruck in $\cos x$ und $\sin x$, der für alle $x\in\mathbb{R}$ 
		definiert ist. Dann hat 
		\[ z\mapsto \frac{1}{z} \cdot R(\frac{z+z^{-1}}{2},\frac{z-z^{-1}}{2i}) \]
		keine Pole auf $S^1(0)$ und es gilt:
		\begin{eqnarray*}
			\int_0^{2\pi} R(\cos x,\sin x)\mathrm{d}x &=& \frac{1}{i} \int_{S^1} \frac{R\left(\frac{z+z^{-1}}{2},
			\frac{z-z^{-1}}{2i}\right)}{z} \mathrm{d}z \\
			&=& 2\pi \sum_{z\in B_1(0)} Res_z \left(\frac{1}{z} R\left(\frac{z+z^{-1}}{2},\frac{z-z^{-1}}{2i}\right)
			\right).
		\end{eqnarray*}
		und nun endlich viele im Inneren des Einheitskreises.\\
		Gemeint ist: $R(X_1,X_2)$ ist rationaler Ausdruck. Ersetze $X_1,X_2$ durch $\cos x,\sin x$ bzw 
		$\frac{z+z^{-1}}{2},\frac{z-z^{-1}}{2i}$.
	\end{corollary}
	
	\begin{proof}
		Setze $z=e^{ix}$. Dann folgt
		\begin{itemize}
			\item $\mathrm{d}z=i e^{ix}\mathrm{d}x$, also $\mathrm{d}x=\frac{1}{i} \frac{\mathrm{d}z}{z}$
			\item $\cos x= \frac{e^{ix}+e^{-ix}}{2}=\frac{z+\frac{1}{z}}{2}$
			\item $\sin x= \frac{e^{ix}-e^{-ix}}{2i}=\frac{z-\frac{1}{z}}{2i} $
		\end{itemize}
		$x\in[0,2\pi]$ entspricht $z\in\mathbb{C}$ mit $|z|=1$. Da $R$ auf $\mathbb{R}$ definiert ist, hat 
		$\frac{1}{z} R\left(\frac{z+z^{-1}}{2},\frac{z-z^{-1}}{2i}\right)$ keine Pole auf $S^1$. Wir können 
		diesen Ausdruck als rationalen Ausdruck in $z$ auffassen. Dieser hat im Nenner ein Polynom (von 	
		endlichem Grad), also hat der Nenner auf $\mathbb{C}$ nur endlich viele Nullstellen und das sind 
		gerade die Pole von $\frac{1}{z}R(\ldots)$. Jetzt folgt die Behauptung aus der Definition des 
		Kurvenintegrals (parametrisiere $S^1$ durch $z=e^{ix}$) und dem Residuensatz. 
		(Alle Pole im Inneren von $B_1(0)$ werden von $S^1$ genau einmal umlaufen.)
	\end{proof}
	
	\begin{example}
		Sei $a\in \mathbb{R}$, $a>1$.
		\begin{eqnarray*}
			 \int_0^{2\pi} \frac{1}{a+\cos x}\mathrm{d}x &=& \frac{1}{i}\int_{S^1}\frac{1}{z} 
			 \frac{1}{a+\frac{z+z^{-1}}{2}}\mathrm{d}z \\
			 &=& \frac{1}{i}\int_{S^1} \frac{1}{az+\frac{z^2}{2}+\frac{1}{2}}\mathrm{d}z.ds
		\end{eqnarray*}
		Der Nenner hat Nullstellen bei $z_{1,2}=-a\pm \sqrt{a^2 -1}$. Da $a>1$, liegt eine davon nämlich 
		$\sqrt{a^2 -1}-a$ im Inneren von $B_1$. Wir rechnen das Residuum mit Proposition 3.21 (2). 
		
		%...
		
		Schreibe
		\[ f=\frac{g}{h} =\frac{\frac{1}{z}}{a+\frac{z+z^{-1}}{2}}. \]
		dann ist 
		\[ g(\sqrt{a^2 -1}-a)=\frac{1}{\sqrt{a^2 -1}-a} \]
		und
		\[ h'(\sqrt{a^2 -1}-a) =\frac{1}{2}-\frac{1}{2z^2}|_{\ldots}=\frac{1}{2}-\frac{1}{2(2a^2 -1-2a\sqrt{a^2 -1})}
		=\frac{(\sqrt{a^2 -1}-a)^2 -1}{2(\sqrt{a^2 -1}-a)^2} \]
		\begin{eqnarray*}
			Res_{\sqrt{a^2 -1}-a}(\frac{g}{h}) 
			&=& \frac{1}{\sqrt{a^2 -1}-a}\cdot \frac{2(\sqrt{a^2 -1}-a)^2}{(\sqrt{a^2 -1}-a)^2 -1} \\
			&=& \frac{2(\sqrt{a^2 -1}-a)}{2a^2 -2 -2a\sqrt{a^2 -1}} \\
			&=& \frac{\sqrt{a^2 -1}-a}{\sqrt{a^2 -1}(\sqrt{a^2 -1}-a)} =\frac{1}{\sqrt{a^2 -1}}
		\end{eqnarray*}
		Also gilt
		\[ \int_0^{2\pi} \frac{1}{a+\cos x}\mathrm{d}x = \frac{2\pi}{\sqrt{a^2 -1}}. \]
	\end{example}
	
	Es bezeichne $\hat{\mathbb{C}}=\mathbb{C}\cup \{\infty\}$ die Riemannsche Zahlenkugel. Sei 
	$K\subset\mathbb{C}$ kompakt. $K\subset \overline{B_R(0)}$ und $f:\mathbb{C}\setminus K\to 
	\mathbb{C}$ holomorph. Dann hat $f$ bei $z=\infty$ eine hebbare Sinuglarität/ Pol/ wesentliche 
	Singularität, wenn $f(\frac{1}{z})$ bei $z=0$ eine entsprechende Singularität hat. Dito definiere 
	Nullstellen der Ordnung $k$ bei $z=\infty$.\\
	Außerdem sei $\mathbb{H} =\{ z\in\mathbb{C} \ \vert\ Im\  z>0 \}$ der obere Halbraum. \\
	Im Folgenden arbeiten wir auf $\mathbb{R}$ mit dem Riemann-Integral.
	
	\begin{corollary}
		Es sei $R(z)$ rationaler Ausdruck, der für alle $z\in\mathbb{R}$ definiert ist und bei $\infty$ eine 
		Nullstelle der Ordnung 2 hat. Dann hat $f(z)$ nur endlich viele Polstellen in $\mathbb{H}$ und 
		\[ \int_{-\infty}^{\infty} R(x)\mathrm{d}x =2\pi i \sum_{z\in\mathbb{H}} Res_z(R(z)). \]
	\end{corollary}
	
	\begin{remark}
		Mit Hilfe der Partialbruchzerlegung können wir $R(x)$ als Summe einfacherer Ausdrücke schreiben, 
		die eine Stammfunktion besitzen.
	\end{remark}
	
	\begin{proof}
		Da der Nenner von $R$ ein Polynom ist, hat er nur endlich viele Nullstellen in $\mathbb{C}$, also 
		erst recht in $\mathbb{H}$. Betrachte $S>0$ groß und die Kontur $c$.
		
		%... Zeichnung fehlt
		
		Für $S$ groß genug liegen alle Pole von $R$ in $\mathbb{H}$ innerhalb der Kontur und werden 
		einmal umlaufen, also ist das Integral über $c$ genau die rechte Seite in der Folgerung. Da $R(z)$ 
		eine Nullstelle der Ordnung $k\geq 2$ bei $\infty$ hat, existiert $C>0$, sd.
		\[ \left|R\left(\frac{1}{z}\right)\right|\leq C\left|\frac{1}{z}\right|^k \]
		für alle $"$kleinen $\frac{1}{z}"$, dh. für $|z|\to \infty$. Also folgt 
		\[ \int_c R(z)\mathrm{d}z=\int_{-S}^S R(x)\mathrm{d}x + \underbrace{\int_0^{\pi} \underbrace{
		R(S\cdot e^{i\varphi})}_{|\cdot|\leq C\cdot S{-k}}\cdot \underbrace{iSe^{i\varphi}}_{|\cdot|=S}\mathrm{d}
		\varphi}_{|\cdot|\leq C\cdot S^{1-k} \underset{s\to\infty}{\to}0} \]
		Durch Grenzübergang $S\to \infty$ folgt die Behauptung.
	\end{proof}
	
	\begin{example}
		\[ \int_{-\infty}^{\infty} \frac{1}{1+x^2}\mathrm{d}x=2\pi i Res_i \frac{1}{1+x^2} = 
		2\pi i\cdot \frac{1}{2i}=\pi \]
		$\arctan'=\frac{1}{1+x^2}$ daraus folgt
		\[ \int_{-\infty}^{\infty} \frac{1}{1+x^2}\mathrm{d}x =\left(\lim_{x\to\infty}-\lim_{x\to\infty}\right)
		\arctan(x) =\pi. \]
	\end{example}
	
	\begin{theorem}[Laurent-Entwicklung]
		Es sei $\Omega\subset\mathbb{C}$ ein Gebiet, $w\in\mathbb{C}$, es seien $0\leq s<S\leq \infty$ 
		gegeben, sd. $A_{s,S}(w)\subset\Omega$. Sei $f:\Omega\to\mathbb{C}$ holomorph, dann 
		exisitert eine eindeutige Laurentreihe, sd.
		\[ f(z)=\sum_{k=-\infty}^{\infty} a_k (z-w)^k \]
		für alle $z\in A_{s,S}(w)$, mit
		\[ a_k=\frac{1}{2\pi i} \int_{S_{\rho}(w)}\frac{f(z)}{(z-w)^{k+1}}\mathrm{d}z, \]
		wobei $\rho\in (s,S)$. Es sei $A_{r,R}(w)$ der Konvergenzring, dann gilt
		\begin{eqnarray*}
			r &\leq& \inf \{\rho\in (0,S)\ \vert\ A_{\rho,S}(w)\subset\Omega\} \\
			R &\geq& \sup\{\rho\in (s,\infty)\ \vert\ A_{s,\rho}(w)\subset\Omega\}
		\end{eqnarray*}
		
		%... fehlende Skizze 19.6.2019
		
	\end{theorem}
	
	\begin{proof}
		Sei $z_0\in A_{s,S}(w)$, sei $\varepsilon>0$ klein genug, sd. $\overline{B_{\varepsilon}(z_0)}\subset 
		A_{s,S}(w)$. Nach der Cauchy-Integralformel (Satz \ref{thm:CF}) gilt
		\[ f(z_0)=\frac{1}{2\pi i} \int_{S_{\varepsilon}(z_0)} \frac{f(\zeta)}{\zeta -z_0}\mathrm{d}\zeta. \]
		Der Zykel $S_{\varepsilon}(z_0)$ ist in $\Omega\setminus\{z_0\}$ homolog zum Zykel 
		$S_S(w)-S_s(w)$, denn $S_{\varepsilon}(z_0)$ ist homotop zu einer Kurve, die $S_S(w)$ positiv, 
		$S_s(w)$ negativ und eine Verbindungsstrecke dazwischen je einmal in bei Richtungen durchläuft. 
		Gegebenfalls verkleinern wir $S$ und vergrößern $s$ so, dass $\overline{B_{\varepsilon}(z_0)}
		\subset A_{s,S}(w)\subset \overline{A_{s,S}(w)}\subset\Omega$. 
		Da $\frac{f(\zeta)}{\zeta -z_0}$ auf $\Omega\setminus\{z_0\}$ holomorph ist, folgt aus der 
		Homotopie-Invarianz des Kurvenintegrals, dass
		\[ f(z_0) =\frac{1}{2\pi i}\int_{S_s(w)}\frac{f(\zeta)}{\zeta -z_0}\mathrm{d}\zeta -
		\frac{1}{2\pi i}\int_{S_s(w)}\frac{f(\zeta)}{\zeta -z_0}\mathrm{d}\zeta.\]
		O. B. d. A. sei $w=0$. Schreibe jetzt
		\[ \frac{1}{\zeta -z_0}=\frac{\frac{1}{\zeta}}{1-\frac{z_0}{\zeta}}=\sum_{k=0}^{\infty} 
		\frac{z_0^k}{\zeta^{k+1}} \]
		für $|\zeta|=S>|z_0|$ und
		\[ \frac{1}{\zeta -z_0}=-\frac{\frac{1}{z_0}}{1-\frac{\zeta}{z_0}}
		=-\sum_{l=0}^{\infty} \frac{\zeta^l}{z_0^{l+1}} =-\sum_{k=-\infty}^{-1} \frac{z_0^k}{\zeta^{k+1}}.\]
		Da die geometrische Reihe absolut konvergiert, folgt
		\[ f(z_0)=\frac{1}{2\pi i}\sum_{k=0}^{\infty} z_0^k \int_{S_S(w)}\frac{f(\zeta)}{\zeta^{k+1}}\mathrm{d}\zeta
		+\frac{1}{2\pi i}\sum_{k=-\infty}^{-1} z_0^k \int_{S_s(w)}\frac{f(\zeta)}{\zeta^{k+1}}\mathrm{d}\zeta. \]
		Da die Funktion $\frac{f(\zeta)}{\zeta^{k+1}}$ auf $\Omega\setminus\{w=0\}$ holomorph ist, hängt
		\[ \int_{S_{\rho}} \frac{f(\zeta)}{\zeta^{k+1}}\mathrm{d}\zeta \]
		nicht von $\rho\in [s,S]$ ab und wir erhalten die Darstellung aus dem Satz. Das funktioniert für alle 
		$s,S$, sd. $s\leq\rho\leq S$, $\overline{A_{s,S}(w)}\subset\Omega$ und alle $z_0\in A_{s,S}(w)$, 
		also folgt die Behauptung über den Konvergenzring. \\
		Zur Eindeutigkeit nehmen wir an, dass
		\[ f(z)=\sum_{k=-\infty}^{\infty} a_k z^k = \sum_{k=-\infty}^{\infty} b_k z^k \]
		für alle $z\in A_{s,S}(w)$ und betrachten (o. B. d. A. $w=0$)
		\[ g(z)=\sum_{k=0}^{\infty} (a_k -b_k)z^k, \qquad \qquad h(z)=\sum_{k=-\infty}^{\infty} (a_{-k}-b_{-k})z^k\]
		sd. also $0=g(z)+h(\frac{1}{z})$ für alle $z\in A_{s,S}(w)$. Dann ist $g$ holomorph auf $B_R(0)$, 
		$R\geq S$ und $h$ holomorph auf $B_{\frac{1}{r}}(0)$, $r\leq s$. Da $g(z)=-h(\frac{1}{z})$ auf 
		$A_{s,S}(w)$, existiert eine holomorphe Funktion $\tilde{g}$ auf $\mathbb{C}$ mit 
		\[ \tilde{g}(z)=\begin{dcases} g(z) & |z|\leq S \\ -h(\frac{1}{z}) & |z| \geq s \end{dcases}. \]
		Sei $\rho\in (s,S)$, dann ist $g$ auf $\overline{B_{\rho}(w)}$ beschränkt und $h$ auf 
		$\overline{B_{\frac{1}{\rho}}(w)}$. Also ist $\tilde{g}$ beschränkt und somit nach dem Satz von 
		Liouville (Satz \ref{thm:LV}) konstant mit 
		\[ g(z)=\lim_{z\to\infty}\left(h\left(\frac{1}{z}\right)\right)=h(0)=0. \]
		Aus dem Eindeutigkeitssatz für Potenzreihen
		
		%...
		
		folgt $a_k =b_k$ für alle $k$.
	\end{proof}
	
	\begin{remark}
		Es sei $r=0$, dass heißt wir betrachten eine Laurentreihe
		\[ L(z)=\sum_{k=-\infty}^{\infty} a_k z^k \]
		auf $A_{0,R}(0)=B_R^*$. Betrachte die Fälle
		\begin{enumerate}
			\item \textsl{Fall:} Es existiere ein $k_0\geq 0$, sd. $a_k=0$ für alle $k<k_0$ und $a_{k_0}\neq 0$. 
			Dann ist die Singularität bei $0$ hebbar. Fall $k_0 >0$, haben wir eine Nullstelle der Ordnung 
			$k_0$ bei $0$.
			
			\item \textsl{Fall:} Es existiere $k_0 <0$, $a_k=0$ für $k\leq k_0$ und $a_{k_0}\neq0$. Dann 
			haben wir eine Polordnung $ord_g (-k_0)$ bei $0$.
			
			\item \textsl{Fall:} Wir haben eine wesentliche Singularität bei $0$.
		\end{enumerate}
		\textsl{Wichtig:} $r=0$ ist notwendig siehe Beispiel 3.16 (2), 
		
		%...
		
		wobei alle $a_k\neq 0$ waren, obwohl wir nur zwei einfache Polstellen hatte.
	\end{remark}
	
	\begin{proposition}[Definition: Residuum]
		Es $f:A_{r,R}(w)\to\mathbb{C}$ holomorph mit
		\[ f(z)=\sum_{k=-\infty}^{\infty} a_k (z-w)^k \]
		für $z\in A_{r,R}(w)$. Dann hängt das Residuum
		\[ Res_{w,\rho}(f)=\frac{1}{2\pi i}\int_{s_{\rho}(w)}f(\zeta)\mathrm{d}\zeta =a_{-1} \]
		von $f$ bei $w$ zum Radius $\rho$ nicht von $\rho\in (r,R)$ ab. Die Funktion $f(z)=-\frac{a}{z-w}$ 
		besitzt auf $A_{r,R}(w)$ genau dann eine Stammfunktion, wenn $a=a_{-1}=Res_{w,\rho}(f)$.
		Falls $r=0$, schreibe $Res_w(f)=Res_{w,\rho}(f)$.
	\end{proposition}
	
	\begin{proof}
		 Unabhängig von $\rho$ folgt aus der Homotopieinvarianz des Kurvenintegrals
		 
		 %...
		 
		 , falls $a\neq a_{-1}$
		 \[ \int_{S_{\rho}(w)} \left( f(z)-\frac{a}{z-w}\right)\mathrm{d}z=2\pi i(a_{-1}-a)\neq 0,\]
		 also existiert keine Stammfunktion (siehe Kapitel 1). Falls $a=a_{-1}$, betrachte die Funktion 
		 \[ F(z)=\sum_{k=0}^{\infty} \frac{a_k}{k+1}(z-w)^{k+1} -\sum_{k=2}^{\infty} \frac{a_{-k}}{k-1}
		 \underbrace{(z-w)^{1-k}}_{(\frac{1}{z-w})^{k-1}}. \]
		 Sie konvergiert auf $A_{r,R}(w)$ und $F'(z)=f(z)-\frac{a}{z-w}$.
	\end{proof}
	
	\begin{example}
		In Beispiel 3.16 (2)
		
		%...
		
		gilt
		\[ Res_{0,\rho}(f)=\begin{dcases} 0& \rho<|w_0| \\ 1&|w_0|<\rho<|w_1| \\ 0&\rho>|w_1| \end{dcases}\] 
		
		%... fehlende Skizze 19.6.2019
		
		Wir finden eine Stammfunktion
		\[ F(z)=\log\left(\frac{z-w_0}{z-w_1}\right) \]
		für alle $z$, die nicht auf der Strecke zwischen $w_0$ und $w_1$ liegen, denn für 
		$z\in\{w_0,w_1\}$ ist der Ausdruck $\frac{z-w_0}{z-w_1}\in\{0,\infty\}$ und es ist $\frac{z-w_0}{z-w_1}$ 
		genau dann reell und negativ, das heißt $arg\left(\frac{z-w_0}{z-w_1}\right)=\pi$  
		$mod\  2\pi\mathbb{Z}$, wenn $z$ zwischen $w_0$ und $w_1$ liegt. Und es gibt eine 
		Stammfunktion $log$ (auch $Log$) von $\frac{1}{x}$ auf $\mathbb{C}\setminus(-\infty,0]$ 
		(Übungsaufgabe).
		
		%... fehlende Skizze 19.6.2019
		
	\end{example}
	
	\begin{proposition}
		Es sei $z_0\in\Omega$ eine isolierte Singularität von $f:\Omega\setminus\{z_0\}\to\mathbb{C}$.
		\begin{enumerate}
			\item Wenn $f$ einen Pol der $ord_g\leq k$ bei $z_0$ hat, gilt 
			\[ Res_{z_0}(f)=\frac{1}{(k-1)!} \frac{\partial^{k-1}}{\partial z^{k-1}}\left((z-z_0)^kf(z)\right). \]
			
			\item Sei $f=\frac{g}{h}$, $h$ habe einfache Nullstelle bei $z_0$, dann gilt
			\[ Res_{z_0} \left(\frac{g}{h}\right)=\frac{g(z_0)}{h'(z_0)}. \]
		\end{enumerate}
	\end{proposition}
	
	\begin{proof}
		Übungsaufgabe.
	\end{proof}
	
	\textsl{Wichtig:} Wir lernen hier lieber die Methode (über welchen Zykel integrieren wir, machen wir 
	Koordinatentransformationen, etc...?). Oftmals kann man die Verfahren variieren (z. B. muss es unter 
	Umständen kein rationaler Ausdruck sein).
	
	\begin{corollary}
		Es sei $R$ ein rationaler Ausdruck, der für alle $z\in \mathbb{R}$ höchstens einfache Pole hat und bei 
		$\infty$ eine Nullstelle der Ordnung $\geq 1$. Dann hat $R$ nur endlich viele Singulariäten in 
		$\mathbb{H}$ und es konvergieren die uneigentlichen Riemann-Integrale
		\begin{enumerate}
			\item Falls die reellen Polstellen in $\pi\cdot\mathbb{Z}$ liegen:
			\[ \int_{-\infty}^{\infty} R(x)\cdot \sin x\mathrm{d}x =2\pi \cdot Re\left(\sum_{z_0 \in\mathbb{H}}
			Res_{z_0}\left(R(z)e^{iz}\right)\right) + \pi \cdot Re\left(\sum_{z_0 \in\mathbb{R}}
			Res_{z_0}\left(R(z)e^{iz}\right)\right) \]
			
			\item Falls die reellen Polstellen in $\pi \mathbb{Z}+\frac{\pi}{2}$ liegen:
			\[ \int_{-\infty}^{\infty} R(x)\cdot \cos x \mathrm{d}x =-2\pi\cdot Im\left(\sum_{z_0 \in\mathbb{H}}
			Res_{z_0}\left(R(z)e^{iz}\right)\right) - \pi \cdot Im\left(\sum_{z_0 \in\mathbb{R}}
			Res_{z_0}\left(R(z)e^{iz}\right)\right). \]
		\end{enumerate}
	\end{corollary}
	
	\begin{proof}
		Wir wählen $S$ so groß, dass alle reellen Polstellen in $(-S,S)$ und alle Polstellen in $\mathbb{H}$ 
		bereits im Rechteck $(-S,S)+(0,S)\cdot i$ liegen. 
		
		%... fehlende Skizze 26.6.2019
		
		
		Betrachte die Funktion $f(z)=R(z)e^{iz}$. Auf 
		$\mathbb{R}$ gilt:
		\[ Re\left( R(z)e^{iz}\right)= R(z)\cdot Re(e^{iz}) =R(z)\cdot \cos(z). \]
		Die Kontur besteht aus vier Teilen
		\[ \int_{-S}^S R(x)\cos x \mathrm{d}x \xrightarrow[S\to\infty]{} \int_{-\infty}^{\infty} R(x)\cos x\mathrm{d}x\]
		Nach dem Leibnitz-Kriterium konvergiert das uneigentliche Riemann-Integral auch, wenn $R$ nur eine 
		einfache Nullstelle bei $\infty$ hat:\\
		
		%... fehlende Skizze 26.6.2019
		
		Das Integral besteht abwechselnd aus Integralen über positive und negative Teilstücke der Länge 
		$\pi$ und für große $x$ fallen die Absolutbeträge der Teilintegrale monoton.
		\[ -\int_{-S}^S R(x+iS)e^{i(x+iS)}\mathrm{d}x=\underbrace{-e^{-S}}_{\text{fällt exponentiell}}
		\underbrace{\int_{-S}^S R(x+iS) e^{ix}\mathrm{d}x}_{\text{wächst höchstens polynomial für }S\to\infty}
		\xrightarrow[S\to\infty]{}0 \]
		\[ \int_0^S R(S+ix)e^{i(S+ix)}i\mathrm{d}x = \underbrace{e^{iS}}_{\text{Betrag $1$}}
		\int_0^S \underbrace{R(S+ix)}_{|\%|\leq\frac{C}{|S+ix|}\leq\frac{C}{S}}e^{-x}i\mathrm{d}x 
		\overset{S\to\infty}{\to}0, \]
		dieto für die linke Rechteck-Kante. \\
		\textsl{1. Fall:} $F$ hat keine reellen Polstellen.
		\begin{eqnarray*}
			\int_{-\infty}^{\infty} R(x)\cos x\mathrm{d}x &=& \lim_{S\to\infty} Re\left(\int_{c_S}R(z)e^{iz}
			\mathrm{d}z\right) \\
			&=& Re\left( 2\pi i\sum_{z_0\in\mathbb{H}}Res_{z_0}\left(R(z)e^{iz}\right)\right) \\
			&=& -2\pi Im\left( 2\pi i\sum_{z_0\in\mathbb{H}}Res_{z_0}\left(R(z)e^{iz}\right)\right) 
		\end{eqnarray*}
		Polstellen auf $\mathbb{R}$? Die reellen Integrale sind wohldefiniert, denn
		\[ \underbrace{\frac{1}{z-n\cdot\pi}\sin(z)}_{\text{$"$hebbar$"$}} 
		=Im\left(\underbrace{\frac{1}{z-n\cdot\pi}e^{iz}}_{\text{hat Pol mit reellem Residuum}}\right) \]
		bzw
		\[ \frac{\cos z}{z-(n+\frac{1}{2})\pi} = Re\left( \frac{e^{iz}}{z-(n+\frac{1}{2})\pi}\right) \]
		Ersetze lokal den Integrationsweg:\\
		
		%... fehlende Skizze 26.6.2019
		
		Betrachte das Integral über den Halbbogen:
		\begin{eqnarray*}
			\int_0^{\pi} R(z_0+\varepsilon e^{i(\pi-\varphi)})e^{iz_0+ie^{i(\pi-\varphi)}
			\cdot\varepsilon}\cdot(-i)\varepsilon e^{i(\pi-\varphi)}\mathrm{d}\varphi \\
			= \bigintss_0^{\pi} \underbrace{\left( \frac{Res_{z_0}R(z)}{\varepsilon\cdot e^{i(\pi-\varphi)}}+ 
			O(1)\right)}_{\to Res_{z_0} R(z)}
			\left( (\pm i) \underbrace{e^{\varepsilon i\underbrace{e^{i(\pi-\varphi)}}_{\text{Betrag $1$}}}}_
			{\to 1\text{ für }\varepsilon\to 0}\right) (-i\varepsilon) e^{i(\pi-\varphi)}\mathrm{d}\varphi \\
			\xrightarrow[\varepsilon\to 0]{} -i\pi \cdot Res_{z_0}R(z)\cdot e^{iz_0}
		\end{eqnarray*}
		Wir haben berechnet:
		\[ \bigintsss_{-\infty}^{\infty} R(x)\cos x\mathrm{d}x +\pi Im\left(\sum_{z_0\in\mathbb{R}}Res_{z_0}\left(
		R(z_0)e^{iz_0}\right)\right)
		= -2\pi Im\left( \sum_{z_0\in\mathbb{H}}Res_{z_0}\left(R(z_0)e^{iz_0}\right)\right). \]
	\end{proof}
	
	Analog für $\int R(x)\sin x\mathrm{d}x$.
	
	\begin{example}
		\begin{eqnarray*}
			\int_0^{\infty} \underbrace{\frac{\sin x}{x}}_{\text{gerade}}\mathrm{d}x
			&=& \frac{1}{2} \int_{-\infty}^{\infty} \frac{\sin x}{x}\mathrm{d}x \qquad (\text{Pol bei $0$}) \\
			&=& \frac{1}{2} Res_0 \left(\frac{e^{ix}}{x}\right) = \frac{\pi}{2}\cdot \frac{1}{1} =\frac{\pi}{2}.
		\end{eqnarray*}
	\end{example}
	
	\begin{definition}[Hauptzweig]
		Der Hauptwert des Logarithmus (Hauptzweig) ist die auf $\mathbb{C}\setminus (-\infty,0]$ definierte 
		Stammfunktion von $z\mapsto \frac{1}{z}$ mit Wert $0$ an der Stelle $1$. Wir schreiben dafür 
		auch $Log$ (anstelle $\log$). $Log(z)=\log |z|+i\ arg\ z$, $arg:\mathbb{C}\setminus(-\infty,0]\to(-\pi,\pi)$.
	\end{definition}
	
	\begin{corollary}
		Es sei $\lambda\in (0,1)$ und $R$ ein reeller rationaler Ausdruck, wohldefiniert für alle $x\geq 0$, mit 
		Nullstelle der Ordnung $\geq 1$ bei $\infty$. Dann gilt 
		\[ \int_0^{\infty} x^{\lambda -1}R(x)\mathrm{d}x = -\frac{\pi}{\sin(2\pi)}\sum_{z_0\in\mathbb{C}\setminus
		(-\infty,0]}Res_{z_0} \left(\underbrace{e^{(\lambda -1)Log(z) }}_
		{\text{$"$}z^{\lambda -1}\text{$"$}}R(-z)\right) \]
	\end{corollary}
	
	\begin{proof}
		Betrachte $f(z)=e^{(\lambda -1)Log(z)}R(-z)$ und für $\varphi >0$, $S>0$ den Zykel $c$ \\
		
		%... fehlende Skizze 26.6.2010
		
		Äußerer Kreis:
		\[ \bigintsss_{\varphi-\pi}^{\pi-\varphi}\underbrace{e^{(\lambda -1)Log(S\cdot e^{i\vartheta})}}_
		{\underbrace{S^{(\lambda -1)}}_{\to 0} \cdot \underbrace{ e^{(\lambda -1)i\vartheta}}_{|\%|=1}}
		\cdot \underbrace{ R\left( -S\cdot e^{i\vartheta}\right)}_{\leq \frac{C}{S}}\cdot 
		iS\underbrace{e^{i\vartheta}}_
		{\text{Betrag $1$}}\mathrm{d}\vartheta \underset{S\to\infty}{\to}0.\]
		Innerer Kreis:
		\[ \bigintsss_{\varphi-\pi}^{\pi-\varphi} \underbrace{\underbrace{e^{(\lambda -1)Log(\frac{1}{S}
		e^{i\vartheta})}}_
		{S^{1-\lambda}\cdot e^{(\lambda -1)i\vartheta}}\cdot \underbrace{R\left(-\frac{1}{S}e^{i\vartheta}\right)}_
		{\to R(0)} \cdot \frac{i}{S}}_{S^{-\lambda\to0}}\underbrace{e^{i\vartheta}}_{\text{Betrag $1$}}\mathrm{d}
		\vartheta \underset{S\to\infty}{\to}0. \]
		Für die geraden Stücke erhalten wir
		\begin{eqnarray*}
			-\int_{\frac{1}{S}}^S e^{(\lambda -1)Log(xe^{i(\pi-\varphi)})}R(\underbrace{-xe^{i(\pi-\varphi)}}_
			{\to x})\underbrace{e^{i(\pi-\varphi)}}_{\to -1}\mathrm{d}x
			+ \int_{\frac{1}{S}}^S e^{(\lambda -1)Log(xe^{i(\varphi-\pi)})}R(\underbrace{-xe^{i(\varphi-\pi)}}_
			{\to x})\underbrace{e^{i(\varphi-\pi)}}_{\to -1}\mathrm{d}x \\
			\xrightarrow[\varphi\to0]{} \int_{\frac{1}{S}}^S x^{\lambda -1}e^{i(\lambda -1)\pi}R(x)\mathrm{d}x
			-\int_{\frac{1}{S}}^S x^{\lambda -1}e^{-i(\lambda -1)\pi}R(x)\mathrm{d}x
		\end{eqnarray*}
		Nach Grenzübergang $S\to\infty$, $\varphi\to0$ erhalten wir
		\begin{eqnarray*}
			\int_0^{\infty} x^{\lambda -1}(2i\cdot \underbrace{\sin((\lambda -1)\pi)}_{-\sin(2\pi)})R(x)\mathrm{d}x
			= 2\pi i\sum_{z_0\in\mathbb{C}\setminus (-\infty,0]}Res_{z_0}(e^{(\lambda -1)z}R(-z)) \\
			\Rightarrow \qquad \int_0^{\infty} x^{\lambda -1}R(x)\mathrm{d}x
			=-\frac{\pi}{\sin(2\pi)}\sum_{z_0\in\mathbb{C}\setminus (-\infty,0]} 
			Res_{z_0}(e^{(\lambda -1)z}R(-z)).
		\end{eqnarray*}
	\end{proof}
	
	\begin{example}
		Sei $\lambda\in (0,2)$, $\lambda\neq 1$.
		\begin{eqnarray*} 
			\int_0^{\infty} \frac{x^{\lambda -1}}{(1+x)^2}\mathrm{d}x &=& -\frac{\pi}{\sin(2\pi)}(\lambda -1) \\
			&=& Res_1 \left( e^{(\lambda -1)Log(z)}\frac{1}{(1-z)^2}\right) \\
			&=& \left( e^{(\lambda -1)Log(z)}\right)_{z=1}^{'} \\
			&=& (\lambda -1)e^{(\lambda -1)Log(z)} \cdot \frac{1}{z}\vert_{z=1} =(\lambda -1)
		\end{eqnarray*}
	\end{example}
		
	\subsection{Das Null- und Polstellen zählende Integral}
	
	\begin{definition}
		Es sei $\Omega$ ein Gebiet. Eine Funktion heißt holomorph auf $\Omega$ bis auf isolierte 
		Singularitäten, wenn es eine diskrete Teilmenge $A\subset\Omega$ gibt, sd.
		\[ f:\Omega\setminus A\to\mathbb{C}\]
		holomorph ist. Wir nennen $f$ meromorph, wenn $f$ höchstens Pole besitzt.
	\end{definition}
	
	\begin{example}
		$\Omega =\mathbb{C}$
		\begin{itemize}
			\item $f(z)=\frac{1}{\sin(z)}$, $A=\pi\mathbb{Z}:=\{\pi n\ \vert\ n\in\mathbb{Z}\}$
			
			\item $f(z)=\frac{1}{z}$, $A=\{0\}$
			
			\item $f(z)=\frac{\sin(z)}{z}$, $A=\{0\}$
		\end{itemize}
		(kann auch $f(0)=1\quad\to\quad A=\emptyset$)
	\end{example}
	
	\begin{example}[Gegenbeispiele]
		\begin{itemize}
			\item $\log(t)$, $A=\{ x\in\mathbb{R}\ \vert\ x\leq 0\}$
			
			\item $f(z)=e^{\frac{1}{z}}$ (wesentliche Singularität)
			
			\item $f(z)=|z|$, $A=\mathbb{C}\setminus\{0\}$
		\end{itemize}
	\end{example}
	
	\begin{remark}
		Es sei $\Omega$ ein Gebiet ($\Omega =\hat{\mathbb{C}}$ auch oke). Dann ist 
		\[ \mathfrak{M}(\Omega)=\{f:\Omega\to\hat{\mathbb{C}}\ \vert\  f\text{ meromorph}\} \]
		ein Körper. \textsl{Vergleich:} $\mathbb{Z}\hookrightarrow \mathbb{Q}$ mit $H(\Omega)
		\hookrightarrow M(\Omega)$.
	\end{remark}
	
	Sei $f$ bei $w$ meromorph und habe eine Darstellung 
	\[ f(z)=(z-w)^{ord(t)}\cdot g(z),\]
	wobei $g(w)\neq 0$ und $g$ holomorph bei $w$ ist.
	
	\begin{definition}
		Logarithmische Ableitung:
		\[ \frac{f'(z)}{f(z)}=\frac{ord_w(f)}{z-w}+\frac{g'(z)}{g(z)} \]
		\[ \Rightarrow \qquad \qquad Res_w\left(\frac{f'}{f}\right)=ord_w(f). \]
		\textsl{Warum?} Für $|z-w|$ klein genug können wir schreiben
		\[ f(z)=\sum_{n=-N}^{\infty} a_n(z-w)^n \]
		für ein $N<\infty$. $N=ord_w(f) \Leftrightarrow a_{-N}\neq 0$.
		\[ f(z)=(z-w)^{-N}\sum_{n=-N}^{\infty} a_n(z-w)^{n+N} \underset{m:=n+N}{=}
		(z-w)^{-N} \underbrace{\sum_{m=0}^{\infty}a_{m-N}	(z-w)^m}_{g(z)}	\]
		\[ \Rightarrow \qquad \int_{|z-w|=\varepsilon} \frac{f'(z)}{f(z)} \mathrm{d}z =
		ord_w(f) \int \frac{1}{z-w}\mathrm{d}z =2\pi i \cdot ord_w(f). \]
	\end{definition}
	
	\begin{theorem}[vom Null- und Polstellen zählenden Integral]
		Es sei $\Omega$ ein Gebiet, $f\in\mathfrak{M}(\Omega)$ meromorph und $c$ ein nullhomologer 
		Zykel (der die Null- und Polstellen von $f$ nicht trifft). Dann gilt
		\[ \frac{1}{2\pi i} \int_c \frac{f'(z)}{f(z)}\mathrm{d}z =\sum_{z\in\Omega} n_z(c)\cdot ord_z(f) \]
	\end{theorem}
	
	\begin{example}
		\begin{itemize}
			\item $n_q(\gamma)=0$
			
			\item $n_q(\gamma)=+1$
			
			\item $n_q(\gamma)=-1$
			
			\item $n_q(\gamma)=+2$
		\end{itemize}
	\end{example}
	
	\begin{proof}
		Lokale Darstellung
		\[ \frac{f'}{f} =\frac{ord_w(f)}{z-w}+\frac{g'(z)}{g(z)} \]
		holomorph
		\[ \Rightarrow \qquad \qquad Res_w\left(\frac{f'}{f}\right)=ord_w(f).\]
	\end{proof}
	
	\begin{remark}[Argumentprinzip]
		$\gamma:S^1\to\mathbb{C}$ Kurve und $f\circ\gamma:S^1\to\mathbb{C}^*$ Kurve. Dann gilt
		\[ n_0(f\circ \gamma)=\frac{1}{2\pi i}\int_{\gamma} \frac{f'(z)}{f(z)}\mathrm{d}z. \]
		\textsl{Warum?}
		\[ w =f(z) \quad \Rightarrow \quad \frac{1}{2\pi i} \int \frac{\mathrm{d}}{w}=\frac{1}{2\pi i} 
		\int \frac{\frac{\mathrm{d}t}{\mathrm{d}z} \mathrm{d}z}{f(z)}.\]
	\end{remark}
	
	\begin{definition}
		Es sei $\Omega\subset\mathbb{C}$ ein Gebiet, $c\in Z(\Omega)$ Zykel. Wir sagen, dass 
		$c$ den Rand $(\partial \Omega)$ von $\Omega$ darstellt, wenn es eine Darstellung 
		\[ c=\sum_{i=1}^k n_i[\gamma_i] \]
		gibt, mit $\gamma_i(t)\in\partial\Omega =\overline{\Omega}\setminus\Omega$ für alle $i$ und alle 
		$t_i$ und
		\[ n_w(c)=\frac{1}{2\pi i}\int \frac{\mathrm{d}z}{z-w} =\begin{dcases}1&w\in\Omega \\ 0&w\in
		\mathbb{C}\setminus\Omega \end{dcases} \]
	\end{definition}
	
	\begin{example}
		Sei $0<r<R$, $c=S_r$ als Rand von $B_r$ und $\partial\Omega =S_R -S_r$.
		
		%... fehlende Skizze 1.7.2019
		
	\end{example}
	
	\begin{definition}
		Für eine Funktion $f:\Omega\setminus A\to\mathbb{C}$ holomorph, $f$ stetig am Rand gilt
		\[ \int_{\partial\Omega} f(z)\mathrm{d}z =\int_c f(z)\mathrm{d}z =2\pi i\sum_{z\in\Omega}
		\underbrace{n_z(c)}_{=1}Res_z(f), \]
		wobei $c$ den Rand von $\Omega$ darstellt.
	\end{definition}
	
	\begin{definition}
		Wir sagen, dass $f$ den Wert $w$ an der Stelle $z$ von Ordnung $k$ annimmt, wenn $ord_z(f-w)=k$ 
		gilt. 
		\begin{eqnarray*}
			N(f,\Omega,w)&=&\sum_{z\in f^{-1}(w)}ord_z(f-w) \\
			N(f,\Omega,\infty)&=& \sum_{z\in f^{-1}(\infty)}(-ord_z(f))
		\end{eqnarray*}
		Und wir sagen, dass $f$ in $\Omega$ den Wert $w$ genau $N(f,\Omega,w)$-mal annimmt.
	\end{definition}
	
	\begin{example}
		$f(z)=(z-w)^k$, $N(f,\mathbb{C},0)=k$ für $k\geq 0$ und $N(f,\mathbb{C},\infty)=0$.
	\end{example}
	
	\begin{corollary}\label{coroll:Umlauf}
		Es gilt 
		\[ N(f,\Omega,w)-N(f,\Omega,\infty) =\frac{1}{2\pi i} \int_{\gamma} \frac{f'(z)}{f(z)-w}\mathrm{d}z=
		n_w (f\circ\gamma).\]
	\end{corollary}
	
	\begin{definition}
		Eine Abbildung $\hat{f}:\hat{\mathbb{C}}\to\hat{\mathbb{C}}$ heißt meromorph, wenn 
		\[ z\mapsto f(z), \qquad w\mapsto f\left(\frac{1}{w}\right) \]
		für $f=\hat{f}|_{\mathbb{C}}$ meromorph sind.
	\end{definition}
	
	\begin{corollary}
		Es sei $f\in\mathfrak{M}(\hat{\mathbb{C}})$ nicht konstant, dann nimmt $f$ alle Werte 
		$w\in\hat{\mathbb{C}}$ gleich oft an.
	\end{corollary}
	
	\begin{example}
		$\hat{f}(z)=z$ hat einen Pol bei $\infty$.
	\end{example}
	
	\begin{proof}
	
		%... fehlende Skizze 1.7.2019
		
		
		Man betrachte die Riemannsche Zahlenkugel mit Polen von $f$ und die Kurve $\gamma$ in der 
		Ebene. Dann kann die Kurve $\gamma$ sowohl um die Pole von $f$ liegen, als auch in ihnen.
		Zu Letzterem folgt, dass die Umlaufzahl der Pole $0$ ist und damit 
		$N(f,\Omega,w)=N(f,\Omega,\infty)$ gilt.
	\end{proof}
	
	\begin{corollary}[Satz von Rouché]
		Es seien $f,g$ in einer Umgebung von $\overline{\Omega}$ holomorph und es sei $c$ ein Zykel, 
		der $\Omega$ umrandet. Wenn $|g|<|f|$ auf $\partial\Omega$, dann haben $f$ und $f+g$ 
		gleich viele Nullstellen.
	\end{corollary}
	
	\begin{proof}
		Da $|f|>|g|$ auf $\partial\Omega$ folgt $\gamma_s := (f+s\cdot g)\circ c$ für $s\in [0,1]$ und 
		$|f+s\cdot g|\geq |f|-s|g| >0$. Damit folgt dann, dass $\gamma_s:S^1\to\mathbb{C}^*$ für alle 
		$s\in [0,1]$. Seien $\gamma_0=f\circ c$ und $\gamma_1=(f+g)\circ c$. Dann sind 
		$\gamma_0$ und $\gamma_1$ homotop in $\mathbb{C}^*$ und es gilt mit \ref{coroll:Umlauf}
		\[ n_0(f\circ c) =n_0((f+g)\circ c = N(f,\Omega,0)-0=N(f+g,\Omega,0)-0. \]
	\end{proof}
	
	\begin{corollary}
		Seien $f(z)=z^n$, $g(z)=a_{n-1}z^{n-1}+\ldots+a_0$ und $\Omega=B_R(0)$ für $R$ groß. Dann 
		haben $z^n=f$ und $z^n+a_{n-1}z^{n-1}+\ldots+a_0$ gleich viele Nullstellen (nämlich $n$).
	\end{corollary}
	
	\subsection{Holomorpher Funktionalkalkül}
	
	\begin{definition}
		Sei $V$ ein Vektorraum, $A:V\to V$ linear, dann ist
		\[ Spec(A):=\{ z\in\mathbb{C}\ \vert\  z\mathds{1} -A\text{ nicht invertierbar}\}. \]
	\end{definition}
	
	\begin{example}
		$V=\mathbb{C}^n$, $A\in M_n(\mathbb{C})$
		\[ \Rightarrow \qquad Spec(A)=\{\lambda\in\mathbb{C}\ \vert\ \lambda\text{ Eigenwert von }A\}\]
		\textsl{Warum?}
		\[ z\mathds{1}-A\text{ nicht invertierbar} \quad\Leftrightarrow\quad det(z\mathds{1}-A)=0 
		\quad\Leftrightarrow\quad z\text{ Eigenwert von A}\]
	\end{example}
		
	Sei $f:B_R(0)\to\mathbb{C}$ holomorph. $A\in M_n(\mathbb{C})$, $Spec(A)\subset B_R(0)$. \\
	Wir wollen:
	\[  f(A):= \sum_{n=0}^{\infty} a_n A^n ,\]
	wenn $f(z)=\sum a_n z^n$ ist.
	
	\begin{example}
		\begin{enumerate}
			\item $A=\theta B$, $B=\begin{smallmatrix} 0&-1\\ 1&0\end{smallmatrix}$, $\theta\in\mathbb{R}$ 
			(oder $\mathbb{C}$), $f(z)=e^z =\sum_{n=0}^{\infty} \frac{z^n}{n!}$. Mit
			\begin{eqnarray}
				B^2=\mathds{1} \quad\Rightarrow\quad B^{2n}=(-1)^n\mathds{1},\ B^{2n+1}=B(-1)^n 
				\label{exp:matr;1}
			\end{eqnarray}
			folgt dann
			\begin{eqnarray*}
				f(A)=\sum_{n=0}^{\infty} \frac{\theta^n B^n}{n!} 
				&=&\sum_{n=0}^{\infty} \frac{\theta^{2n}B^{2n}}{(2n)!}+\sum_{n=0}^{\infty}
				\frac{\theta^{2n+1}B^{2n+1}}{(2n+1)!} \\
				&\underset{\ref{exp:matr;1}}{=}&  \sum_{n=0}^{\infty} \frac{\theta^{2n}(-1)^n\mathds{1}}
				{(2n)!}+\sum_{n=0}^{\infty} \frac{\theta^{2n+1}(-1)^n}{(2n+1)!}B \\
				&=& \cos(\theta)\mathds{!} +\sin(\theta)B \\
				&=& \begin{pmatrix} \cos(\theta) & -\sin(\theta) \\ \sin(\theta) & \cos(\theta) \end{pmatrix}
			\end{eqnarray*}
			
			\item Sei
			\[ A=\begin{pmatrix} \lambda_1 & 0 & \cdots & 0 \\
						       0 & \lambda_2 & \ddots & \vdots \\
						       \vdots & \ddots & \ddots & 0 \\
						       0 & \cdots & 0 & \lambda_n
				\end{pmatrix} \]
			dann ist
			\[ f(A)=\begin{pmatrix} f(\lambda_1) & 0 & \cdots & 0 \\
						       0 & (\lambda_2) & \ddots & \vdots \\
						       \vdots & \ddots & \ddots & 0 \\
						       0 & \cdots & 0 & (\lambda_n)
				\end{pmatrix} \]
			\textsl{Warum?}
			\[ A^k=\begin{pmatrix} \lambda_1^k & 0 & \cdots & 0 \\
						       0 & \lambda_2^k & \ddots & \vdots \\
						       \vdots & \ddots & \ddots & 0 \\
						       0 & \cdots & 0 & \lambda_n^k
				\end{pmatrix} \]
			daraus folgt
			\[ \sum_{k=0}^{\infty}a_k A^k=
				\begin{pmatrix} \sum_{k=0}^{\infty}a_k\lambda_1^k & 0 & \cdots & 0 \\
						       0 & \sum_{k=0}^{\infty}a_k\lambda_2^k & \ddots & \vdots \\
						       \vdots & \ddots & \ddots & 0 \\
						       0 & \cdots & 0 & \sum_{k=0}^{\infty}a_k\lambda_n^k
				\end{pmatrix} \]
			
			\item Sei $A=PDP^{-1}$, $A^n=PD^nP^{-1}$ $\Rightarrow$ $f(A)=Pf(D)P^{-1}$
		\end{enumerate}
	\end{example}
	
	\begin{example}[Operator ohne Eigenwerte]
		\begin{enumerate}
			\item$V=\{\text{ Polynome }p(z)\}$, $A:V\to V$ mit $(Ap)(z):=z\cdot p(z)$, $p(z)=a_nz^n+
			\ldots+a_0$ und 
			$(Ap)(z)=a_nz^{n+1}+\ldots+a_0 z$. Keine Eigenwerte! $z\cdot p(z)=\lambda p(z)$, 
			$p\not\equiv 0\ \Rightarrow\ z=\lambda\ \Rightarrow\ z$ ist konstant.
			
			\item Seien $V=\bigotimes_{n=0}^{\infty} V_n$, $V_n =\mathbb{C}\{z^n\}$ und 
			$A:V_n\to V_{n+1}$. Brauchen, dass $dim\ V=\infty$.
		\end{enumerate}
	\end{example}
	
	\begin{example}[Jordan Normalform]
		Sei $A\in M_n(\mathbb{C}$ mit $A=PJP^{-1}$, wobei
		\[ T= \begin{pmatrix} J_1 & 0 & \cdots & 0 \\
						0 & J_2 & \ddots & \vdots \\
						\vdots & \ddots & \ddots & 0 \\
						0 & \cdots & 0 & J_k
			\end{pmatrix}, \qquad
		J_i= \begin{pmatrix} \lambda_i & 1 & 0 & 0 \\
						0 & \lambda_i & \ddots & 0 \\
						\vdots & \ddots & \ddots & 1 \\
						0 & \cdots & 0 & \lambda_i
			\end{pmatrix} \]
		also gilt $A^k=PJ^kP^{-1}$ und damit folgt $f(A)=Pf(J)P^{-1}$ \\
		\textsl{Behauptung:} Es gilt
		\[ f(J_i)= \begin{pmatrix} \frac{f(\lambda_i)}{0!} & \frac{f'(\lambda_i)}{1!} 
							& \frac{f''(\lambda_i)}{2!} & \cdots & \cdot \\
						     0 & \ddots & \ddots & \ddots & \vdots \\
						     \vdots & \ddots & \ddots & \ddots & \frac{f''(\lambda_i)}{2!} \\
						     \vdots & \ddots & \ddots & \ddots & f'(\lambda_i) \\
						     0 & \cdots & \dots & 0 & f(\lambda_i)
				\end{pmatrix} \]
		\textsl{Kein Beweis}
	\end{example}
	
	Falls wir eine holomorphe Funktion $f$ haben mit 
	\[ f(a) = \frac{1}{2\pi i} \int_{\gamma} \frac{f(z)}{z-a}\mathrm{d}z \]
	für $a\in\Omega$. Dann stellt sich die Frage was $f(A)$ ist.
	
	\begin{definition}[Resolvente]
		Sei $A:V\to V$ linear und $z\notin Spec(A)$ dann ist $ R(A,z)=(z\mathds{1}-A)^{-1}$ die Resolvente.
	\end{definition}
	
	\begin{theorem}
		Es sei $A\in M_n(\mathbb{C})$, $f:\Omega\to\mathbb{C}$ holomorph und $Spec(A)\subset\Omega$.
		$\gamma$ sei (in $\Omega$) nullhomologer Zykel in $\Omega\setminus Spec(A)$ und 
		$n_{\lambda}(\gamma)=\pm 1$ für alle $\lambda\in Spec(A)$. Dann gilt
		\[ f(A)=\frac{1}{2\pi i}\int_{\gamma} R(A,z)f(z)\mathrm{d}z \text{$"="$} \frac{1}{2\pi i}
		\int_{\gamma} \frac{f(z)}{z-A}\mathrm{d}z. \]
	\end{theorem}
	
	\begin{example}
		$f:S^1\to\mathbb{C}$ ,$f=f(\theta)=f(\theta +2\pi k)$ und $\triangle f(\theta):=-f''(\theta)$.
		$\triangle$ hat Eigenwerte $n^2$ für $n\in\mathbb{Z}$ und Eigenfunktionen 
		$f_n(\theta)=e^{in\theta}$ für $n\in\mathbb{Z}$. \\
		\textsl{Bemerkung:} $"$$det(\triangle)=\prod_{n\in\mathbb{Z}} n^2$$"$, 
		$"$$det(\triangle +1)=\prod_{n\in\mathbb{Z}} (n^2 +1)$$"$ \\
		Können $\triangle$ als $"$Diagonalmatrix$"$ darstellen
		\[ \begin{pmatrix} \ddots & \ddots & \ddots & \ddots & \ddots & \ddots & \ddots & \ddots & \ddots \\
					  \ddots & 9 & 0 & \cdots & \cdots & \cdots & \cdots & 0 & \ddots \\
					  \ddots & 0 & 4 & \ddots & \ddots & \ddots & \ddots & \vdots & \ddots \\
					  \ddots & \vdots & \ddots & 1 & \ddots & \ddots & \ddots & \vdots & \ddots \\
					  \ddots & \vdots & \ddots & \ddots & 0 & \ddots & \ddots & \vdots & \ddots \\
					  \ddots & \vdots & \ddots & \ddots & \ddots & 1 & \ddots & \vdots & \ddots \\
					  \ddots & \vdots & \ddots & \ddots & \ddots & \ddots & 4 & 0 & \ddots \\
					  \ddots & 0 & \cdots & \cdots & \cdots & \cdots & 0 & 9 & \ddots \\
					  \ddots & \ddots & \ddots & \ddots & \ddots & \ddots & \ddots & \ddots & \ddots
		\end{pmatrix} \]
		Wir wollen für $f(z)=e^{-z}$ wissen, was $f(t\triangle)$ ist für $t\geq 0$. Es gilt 
		\[ f(t\triangle)f_n =\sum_{k=0}^{\infty} \frac{(t\triangle)^k}{k!} f_n =\sum_{k=0}^{\infty} 
		\frac{(tn^2)^k}{k!}f_n \]
		\[\Rightarrow\qquad e^{-t \triangle}f_n =e^{-tn^2}f_n \]
		Sei $F:S^1\to\mathbb{C}$ mit $F(\theta)=\sum_{n\in\mathbb{Z}} a_n f_n(\theta)$ eine beliebige 
		Fourierreihe. Dann gilt
		\[  e^{-t\triangle} F(\theta)=\sum_{n\in\mathbb{Z}}a_n e^{-tn^2}f_n(\theta)=: \tilde{F}.\]
		Wollen aber auch, dass gilt 
		\[ 2\pi i e^{-t\triangle}=\lim_{m\to\infty} \int_{\gamma_m} R(t\triangle,z)e^{-z}\mathrm{d}z, \]
		wobei
		
		%... fehlende Skizze 3.7.2019
		
		\[ 2\pi i e^{-t\triangle}f_n=\lim_{m\to\infty}\int_{\gamma_m} e^{-z}R(t\triangle,z)f_n\mathrm{d}z \]
		\[ R(t\triangle,z)f_n=(z-t\triangle)^{-1}f_n = (z-tn^2)^{-1}f_n \]
		Daraus folgt nun
		\[ 2\pi i e^{-t\triangle}=\lim_{m\to\infty}\int_{\gamma_m}\frac{e^{-z}}{z-tn^2}f_n(\theta)\mathrm{d}z 
		= 2\pi i e^{-tn^2}f_n(\theta). \]
	\end{example}
	
	\text{Warum?} Wir haben $F:S^1\to\mathbb{C}$ und definieren
	\[ \tilde{F}(t,\theta)=e^{t\triangle} F(\theta) \]
	Da $\tilde{F}$ die Gleichung $\tilde{F}|_{t=0}=F$ löst ist dann $\triangle\tilde{F}+\partial_t \tilde{F}=0$ 
	(die Wärmeleitungsgleichung) $\tilde{F}(t=0,\theta)=F(\theta)$.
	
	\section{Riemannscher Abbildungssatz}
	
	\textsl{Ziel:} alle einfach zusammenhängenden Gebiete in $\mathbb{C}$, mit Ausnahme von $\mathbb{C}$ 
	selbst, sind zueinander biholomorph.
	
	\subsection{Der Riemannsche Abbildungssatz als Maximierungsproblem}
	
	\begin{theorem}[Riemannscher Abbildungssatz] \label{thm:RA}
		Jedes einfach zusammenhängende Gebiet $\Omega\subset\mathbb{C}$, $\Omega\neq\emptyset,\mathbb{C}$ 
		lässt sich biholomorph auf $B_1(0)$ abbilden.
	\end{theorem}
	
	\begin{remark}
		Es gibt keine biholomorphe Abbildung $\mathbb{C}\to B_1(0)$.
	\end{remark}
	
	\begin{proof}
		Nach dem Satz von Liouville (Satz \ref{thm:LV}) ist jede holomorphe Abbildung $\mathbb{C}\to B_1(0)$ 
		konstant.
	\end{proof}
	
	\begin{remark} \label{rem:bihol}
		\begin{itemize}
			\item Abbildungen der Form $h_{z_0}:B_1(0)\to B_1(0)$ mit 
			\[z\mapsto \frac{z-z_0}{\overline{z_0}z-1}\] 
			für $z_0\in B_1(0)$, sind biholomorph. Vergleich Beispiel \ref{exp:moebius} 
			(Möbiustransformationen) und überprüfe
			\[ \frac{1}{\sqrt{1-|z_0|^2}}\begin{pmatrix} 1 & -z_0 \\ \overline{z_0} & 1 \end{pmatrix}\in U(1,1). \]
			
			\item Nach Proposition \ref{coroll:moebius} ist jede biholomorphe Abbildung 
			$B_1(0)\to B_1(0)$ eine Möbiustransformation.
		\end{itemize}
	\end{remark}
	
	\begin{corollary}[aus Satz \ref{thm:RA}]
		Es sei $\Omega\subset\mathbb{C}$, $\Omega\neq\emptyset,\mathbb{C}$, 
		ein einfach zusammenhängendes Gebiet, $z_0\in\Omega$ und 
		$w\in S_1(0)$. Dann existiert genau eine biholomorphe Abbildung $f:\Omega\to B_1(0)$ mit 
		$f(z_0)=0$ und $f'(z_0)=w|f'(z_0)|$.
	\end{corollary}
	
	\begin{proof}
		\textsl{Existenz:} Nach Satz \ref{thm:RA} existiert $g:\Omega\to B_1(0)$ biholomorph. Sei $x:=g(z_0)$. 
		Wir suchen eine biholomorphe Abbildung $B_1(0)\to B_1(0)$ mit $x\mapsto 0$. Betrachte dazu 
		$z\mapsto \frac{z-x}{\overline{x}z-1}$ (nach Bemerkung \ref{rem:bihol} biholomorph) und setze 
		\[ f(z)=c\cdot \frac{g(z)-x}{\overline{x}g(z)-1} \]
		mit $c\in S_1(0)$, sd. $f'(z_0)$ ein positives reelles Vielfaches von $w$ wird. \\
		
		\textsl{Eindeutigkeit:} Sei $h$ eine weitere biholomorphe Abbildung mit $h(z_0)=0$ und 
		$h'(z_0)=w|h'(z_0)|$. Dann gilt $h\circ f^{-1}:B_1(0)\to B_1(0)$ mit $0\mapsto 0$ und 
		\[ \left(h\circ f^{-1}\right)'(0)=\frac{h'(z_0)}{f'(z_0)}=\frac{ w|h'(z_0)|}{w|f'(z_0)|} \in(0,\infty). \]
		Nach \ref{coroll:moebius} ist $h\circ f^{-1}$ eine Möbiustransformation, dh.
		\[ \left(h\circ f^{-1}\right)(z)=\frac{az+b}{cz+d} \quad\text{mit}\quad 
		\begin{pmatrix} a&b\\c&d \end{pmatrix}\in U(1,1). \]
		Es ist $b=0$, daher $c=0$. Also $(h\circ f^{-1})(z)=\frac{a}{d}\cdot z$ mit $|\frac{a}{d}|=1$ und 
		$(h\circ f^{-1})'(0)=\frac{a}{d}\in (0,\infty)$. Es folgt $\frac{a}{d}=1$, also $h\circ f^{-1}=id$, dh. 
		$h=f$.
	\end{proof}
	
	\begin{proposition} \label{rem:log}
		Sei $f$ eine Funktion ohne Nullstellen auf einem einfach zusammenhängenden Gebiet 
		$\Omega\subset\mathbb{C}$. Dann besitzt $f$ einen Logarithmus und eine Quadratwurzel auf 
		ganz $\Omega$.
	\end{proposition}
	
	\begin{proof}
		Wir suchen eine Abbildung $g:\Omega\to\mathbb{C}$ holomorph mit $f(z)=exp(g(z))$. Da 
		$\Omega$ einfach zusammenhängend und $\frac{f'}{f}$ holomorph ist, existiert nach 
		\ref{coroll:stammfkt} eine Stammfunktion $F$ von $\frac{f'}{f}$. 
		Sei $G:\Omega\to\mathbb{C}$ mit
		\[ z\mapsto \frac{exp(F(z))}{f(z)}. \]
		\textsl{Behauptung:} $G'(z)=0$ für alle $z\in\Omega$. \\
		\textsl{Beweis:}
		\[ G'(z)=\frac{exp(F(z))\cdot \frac{f'(z)}{f(z)}\cdot f(z) -exp(F(z))\cdot f'(z)}{f(z)^2}=0. \]
		Also folgt $exp(F(z))=C\cdot f(z)$ mit $C\neq 0$. Wir wählen $C\in \mathbb{C}$ mit $C=exp(c)$. \\
		\textsl{Behauptung:} $g(z):=F(z)-c$ hat gewünschte Eigenschaft. \\
		\textsl{Beweis:}
		\[ exp(F(z)-c)=\frac{exp(F(z))}{C}=f(z) \]
		für alle $z\in\Omega$. \\
		Für die Quadratwurzel wähle $h(z)=exp(\frac{1}{2}g(z))$. Dann ist
		\[ h(z)^2 =exp\left(\frac{1}{2}g(z)\right)^2 =exp(g(z))=f(z). \]
	\end{proof}
	
	\begin{proposition}[Schritt 1] \label{prop:schritt1}
		Jedes einfach zusammenhängende Gebiet $\Omega\subset\mathbb{C}$, $\Omega\neq\emptyset,\mathbb{C}$ 
		lässt sich biholomorph auf 
		eine Teilmenge von $B_1(0)$ abbilden, die den Nullpunkt enthält.
	\end{proposition}
	
	\begin{proof}
		\textsl{Idee:}
		
		%... fehlende Skizze 8.7.2019
		
		Sei $\Omega\subset\mathbb{C}$, $\Omega\neq\emptyset,\mathbb{C}$ einfach zusammenhängend. \\
		\textsl{Behauptung 1:} $\Omega$ lässt sich biholomorph auf $\Omega_1\subset\mathbb{C}$ 
		abbilden, sd. im Komplent $\mathbb{C}\setminus\Omega_1$ eine volle Kreisscheibe 
		enthalten ist. \\
		\textsl{Beweis 1:} Nach Voraussetzung existiert ein $b\in\mathbb{C}\setminus\Omega$. Die 
		Funktion $f:\Omega\to\mathbb{C}$, $z\mapsto z-b$ ist holomorph und hat keine Nullstelle. 
		Nach Bemerkung \ref{rem:log} existiert eine Abbildung $g:\Omega\to\mathbb{C}$ mit 
		$(g(z))^2=z-b$. \\
		\textsl{Beobachtung:} $g$ ist injektiv. \\
		Falls $g(z_1)=g(z_2)$ $\Rightarrow$ $g(z_1)^2=g(z_2)^2$ $\Rightarrow$ $z_1=z_2$. Das heißt 
		$g:\Omega\to\Omega_1:=im(g)$ ist biholomorph. Aus der Beobachtung folgt auch, dass 
		$z_1=z_2$, falls $g(z_1)=-g(z_2)$ ist. In anderen Worten gilt, wenn $0\neq w\in im(g)$, so ist 
		$-w\notin im(g)$. Da $\Omega_1=im(g)$ nach dem Satz über Gebietstreue offen ist, finden wir 
		$B_r(a)$, $r>0$, mit $0\notin B_r(a)$ und $B_r(a)\subset\Omega_1$. Somit gilt $B_r(-a)\subset
		\mathbb{C}\setminus\Omega$. Also folgt Behauptung 1.\\
		\textsl{Behauptung 2:} $\Omega_1$ lässt sich biholomorph auf $\Omega_2\subset B_1(0)$, mit 
		$0\in \Omega_2$, abbilden. \\
		\textsl{Beweis 2:} Die Abbildung $\tilde{h}:\Omega_1\to\mathbb{C}$, $z\mapsto \frac{1}{z-a'}$ mit 
		$a':=-a$ bildet 
		biholomorph auf ein beschränktes Gebiet $\tilde{\Omega}_2$ ab. Für $z\in \Omega_1$ 
		ist $z\notin B_r(a')$, dh. $|z-a'|>r$, also $\frac{1}{|z-a'|}<\frac{1}{r}$, dh. $\tilde{h}(z)\in B_1(0)$. 
		Durch geeignete Translation $(z\mapsto z+a)$ erhalten wir eine biholomorphe Abbildung 
		$\hat{h}:\Omega_1\to\hat{\Omega}_2$ mit $0\in \hat{\Omega}_2$ und durch geeignetes 
		Stauchen $(z\mapsto \varphi z, 0<\varphi<1)$ schließlich $h:\Omega_1\to\Omega_2$ 
		biholomorph mit $0\in\Omega_2\subset B_1(0)$.\\
		Damit folgt Behauptung 2 und die Aussage folgt nun aus den Behauptungen.
	\end{proof}
	
	\begin{example}
		Wir betrachten $\Omega=\mathbb{C}\setminus(-\infty,0]$. \\
		Auf $\Omega$ ist durch $z\mapsto \sqrt{z}:=e^{\frac{1}{2}\log(z)}$ eine Wurzel mit Werten in 
		$(0,\infty)\times\mathbb{R}\subset \mathbb{R}^2=\mathbb{C}$ definiert. So erhalten wir 
		eine biholomorphe Abbildung $f:\Omega\to B_1(0)$ mit
		\[z\mapsto \frac{\sqrt{z}-1}{\sqrt{z}+1}.\]
	\end{example}
	
	\begin{proposition}[Schritt 2] \label{prop:schritt2}
		Sei $\Omega\subset B_1(0)$, $0\in\Omega\neq B_1(0)$, 
		ein einfach zusammenhängendes Gebiet. Dann existiert eine 
		Funktion $f:\Omega\to B_1(0)$ mit $f(0)=0$ und $|f'(0)|>1$.
	\end{proposition}
	
	\textsl{Achtung:} Für $\Omega=B_1(0)$ gilt dies nicht! (Lemma von Schwarz \ref{coroll:SL})
	
	\begin{proof}
		Sei $a\in B_1(0)\setminus\Omega$. Betrachte
		\[ h(z)=\frac{z-a}{\overline{a}z-1} .\]
		Die Funktion $h$ ist holomorph und hat in $\Omega$ keine Nullstelle. Nach Bemerkung 
		\ref{rem:log} existiert daher $H:\Omega\to\mathbb{C}$ mit $(H(z))^2=h(z)$. Dann ist 
		$H:\Omega\to B_1(0)$ injektiv. Verwende Bemerkung \ref{rem:bihol} erneut, verkette mit $H$ 
		und erhalte mit $\psi:\Omega\to B_1(0)$,
		\[ \psi(z)=\frac{H(z)-H(0)}{\overline{H(0)}H(z)-1}. \]
		\textsl{Behauptung:} $\psi(0)=0$ und $|\psi'(0)|>1$. \\
		\textsl{Beweis:} $\psi(0)=0$ ist klar. Es gilt 
		\[ H(z)^2=\frac{z-a}{\overline{a}z-1}, \]
		also $2H(0)\cdot H'(0)=|a|^2-1$. Außerdem gilt $|H(0)|^2=|a|$, also $|H(0)|=\sqrt{|a|}$. 
		Damit folgt
		\[ \left| \psi'(0)\right|=\frac{|H'(0)|}{|H(0)^2-1|}=\frac{|a|^2 -1}{2\cdot\sqrt{|a|}}\cdot\frac{1}{|a|-1}
		=\frac{|a|+1}{2\cdot\sqrt{|a|}}>1 .\]
		Wobei die letzte Ungleichung mit der Binomischen Formeln aus $\left(\sqrt{|a|}-1\right)^2>0$ folgt.
	\end{proof}
	
	\begin{corollary} \label{coroll:injsurj}
		Sei $\Omega\subsetneq B_1(0)$ einfach zusammenhängend, $0\in \Omega$, sei $F:\Omega\to B_1(0)$ 
		injektiv, sd. 
		$F(0)=0$ und für alle injektiven $f:\Omega\to B_1(0)$ mit $f(0)=0$ gilt
		\[ |F'(0)|\geq |f'(0)| ,\]
		dann ist $F$ surjektiv.
	\end{corollary}
	
	\begin{proof}
		Falls $F$ nicht surjektiv ist, existiert $a\in B_1(0)\setminus im(F)$. Also finden wir mit Proposition
		 \ref{prop:schritt2} 
		eine Abbildung $g: im(F)\to B_1(0)$ mit $g(0)=0$ und $|g'(0)|>1$. Sei $f=g\circ F$, dann folgt
		\[ |f'(0)|= \underbrace{|g'(\underbrace{F(0)}_{0})|}_{>1} \cdot |F'(0)| > |F'(0)| \]
		was ein Widerspruch ist. Also ist $F$ surjektiv.
	\end{proof}
	
	\textsl{Ziel:} Finde solch ein $F$. Dann können wir zeigen, dass $F$ sowohl injektiv als auch surjektiv und 
	holomorph, also biholomorph, ist von $\Omega\to B_1(0)$. Zusammen mit Proposition \ref{prop:schritt1} erhalten 
	wir die gesuchte Abbildung für Satz \ref{thm:RA}. \\
	\textsl{Idee:} Finde Folge $(f_n)_n$, $f_n:\Omega\to B_1(0)$ mit $f_n(0)=0$ und
	\[ \lim_{n\to\infty} |f_n'(0)|=\sup \{ |f'(0)|\ \vert\ f:\Omega\to B_1(0)\text{ injektiv, } f(0)=0 \} \]
	\textsl{Frage:} Existiert eine brauchbare Grenzfunktion?
	
	\subsection{Folgen holomorpher Funktionen}
	
	\textsl{Ziel:} 
	\begin{itemize}
		\item Konvergenzsatz
		
		\item Eigenschaften der Grenzfunktion
		
		\item Version von Arzela-Ascoli (mit schwächeren Voraussetzungen)
	\end{itemize}
	Es sei also $f_n:\Omega_n\to\mathbb{C}$ eine Folge von Funktionen auf $\Omega_n\subset\mathbb{C}$. \\
	Wir suchen also dann eine Grenzfunktion auf einem Gebiet
	\[ \Omega = \{ z\in\mathbb{C}\ \vert\ \text{Ex. $r>0$ und $n_0\in\mathbb{N}$, sd. $B_r(z)\in\Omega_n$ für alle 
	$n\geq n_0$} \} .\]
	
	%... Fehlende Skizze 10.7.2019
	
	
	Sei $K\subset\Omega$ kompakt, dann exsitiert sogar $r>0$, $n_0\in\mathbb{N}$, sd. $B_r(z)\subset\Omega_n$ 
	für alle $n\geq n_0$ und alle $z\in K$. \\
	\textsl{Beweis:} zu jedem $z\in K$ existiert $r_z,n_0(z)$, sd. $B_{2r}(z)\subset\Omega_n$ für alle $n\geq n_0$. 
	Für endlich viele $z_1,\ldots,z_N$ folgt
	\[ K\subset \bigcup_{i=1}^N B_{r(z_i)}(z_i). \]
	
	
	%... Fehlende Skizze 10.7.2019
	
	
	Für alle $w\in B_{r(z_i)}(z_i)$ folgt $B_{r(z_i)}(w)\subset B_{2r(z_i)}(z_i)\subset \Omega_n$. Setze jetzt
	\[ r=\min_{1\leq i\leq N} r(z_i), \qquad n_0=\max_{1\leq i\leq N} n_0(z_i). \]
	Wir hätten gern, dass $\Omega$ wir oben zusammenhängend ist. Falls nicht, betrachten wir eine
	Zusammenhangskomponente und nennen diese $\Omega$.
	
	\begin{definition}[Kompakte Konvergenz von Funktionenfolgen]
		Eine Folge $f_n:\Omega_n\to\mathbb{C}$ konvergiert kompakt gegen $f:\Omega\to\mathbb{C}$, wenn für 
		jedes Kompaktum $K\subset\Omega$ ein $n_0\in\mathbb{N}$ existiert mit $K\subset\Omega_n$ für alle 
		$n\geq n_0$ und die Folge $f_n|_K$ gleichmäßig gegen $f|_K$ konvergiert.
	\end{definition}
	
	\begin{remind}
		$f_n\to f$ gleichmäßig, wenn für alle $\varepsilon>0$ ein $n_0\in\mathbb{N}$ exisitert, sd. 
		$|f_n(x)-f(x)|<\varepsilon$ für alle $x$ und alle $n\geq n_0$. \\
		Alle $f_n$ stetig und gleichmäßige Konvergenz $\Rightarrow$ $f$ stetig. \\
		Aber gleichmäßige Konvergenz überall ist zu viel verlangt. Gleichmäßige Konvergenz auf $\overline{B_r(z)}$ 
		reicht schon um die Stetigkeit von $f$ auf $B_r(z)$ zu erhalten.
	\end{remind}
	
	\begin{theorem}[Konvergenzsatz von Weierstraß] \label{thm:Weierstr}
		Eine Folge $f_n:\Omega_n\to\mathbb{C}$ konvergiere kompakt gegen $f:\Omega\to\mathbb{C}$. Wenn alle 
		$f_n$ holomorph sind, ist auch $f$ holomorph und die Folge $f_n'$ konvergiert kompakt gegen $f'$.
	\end{theorem}
	
	\begin{proof}
		Nach dem Satz von Morera \ref{thm:morera} reicht es zu zeigen, dass das Kurvenintegral von $f$ über den 
		Rand von beliebigen Dreiecken $\triangle\subset\Omega$ stets verschwindet. Wegen kompakter 
		Konvergenz ist $f$ stetig. Außerdem sei $\triangle\subset\Omega$ eine Dreiecksfläche, dann folgt
		\[ \int_{\partial\triangle}f(z)\mathrm{d}z =\lim_{n\to\infty} \int_{\partial\triangle} f_n(z)\mathrm{d}z=0. \]
		Da $\triangle$ kompakt ist, haben wir gleichmäßige Konvergenz auf $\partial\triangle$, also konvergieren 
		die Integrale. Aus der Holomorphie der $f_n$ folgt dann, dass das Integral verschwindet. 
		Also ist $f$ holomorph nach dem Satz von Morera. \\
		Sei jetzt $z\in\Omega$, $\overline{B_{2r}(z)}\subset\Omega$, dann gilt für alle $w\in B_r(z)$, dass
		\begin{eqnarray*}
			f'(w)&=&\frac{1}{2\pi i} \int_{S_{2r}(z)} 
			\frac{f(\zeta)}{\underbrace{(\zeta -w)}_{|\%|\geq r>0}\ ^2}\mathrm{d}\zeta \\
			&=& \lim_{n\to\infty}\frac{1}{2\pi i} \int_{S_{2r}(z)}\frac{f_n(\zeta)}{(\zeta-w)^2}\mathrm{d}\zeta \\
			&=& \lim_{n\to\infty} f_n'(w)
		\end{eqnarray*}
		Sei jetzt $K\subset\Omega$ kompakt, dann gibt es endlich viele $z_i\in K$, sd. $\overline{B_{2r}(z_i)}$ und
		\[ K\subset \bigcup_{i=1}^N B_{r_i}(z_i) .\]
		Also erhalten wir auch gleichmäßige Konvergenz auf $K$, dh. $f_n'\to f'$ kompakt.
	\end{proof}
	
	Für die höheren Ableitungen erhalten wir kompakte Konvergenz entweder induktiv, oder indem wir Satz 1.30 
	für höhere Ableitungen benutzen. \\
	Zur gleichmäßigen Konvergenz. Es sei $|f_n(\zeta)-f(\zeta)|<\varepsilon$ für alle $n\geq n_0$ auf $S_{2r}(z)$
	\[ \Rightarrow \quad \left| \frac{1}{2\pi i}\int_{S_{2r}(z)}\frac{f_n(\zeta)}{(\zeta-w)^2}\mathrm{d}\zeta - 
	\frac{1}{2\pi i} \int_{S_{2r}(z)} \frac{f(\zeta)}{(\zeta-w)^2}\mathrm{d}\zeta \right| 
	\leq \frac{1}{2\pi i} \left| \int_{S_{2r}(z)} \frac{\varepsilon \mathrm{d}\zeta}{\underbrace{(\zeta-w)^2}_{\geq r}}\right|
	\leq \frac{2r\varepsilon}{r^2} \]
	
	\begin{remind}
		\[ N(f,\Omega,w)=\sum_{z\in f^{-1}(w)\cap\Omega} ord_z(f-w) \]
		Für injektive Funktionen ist $N(f,\Omega,w)\leq 1$.
	\end{remind}
	
	\begin{theorem}[Hurwitz] \label{thm:Hur}
		Es konvergieren $f_n:\Omega_n\to\mathbb{C}$ holomorph, kompakt gegen $f:\Omega\to\mathbb{C}$. Falls 
		$N(f_n,\Omega_n,w)\leq k$ für alle $n\geq n_0$, dann ist entweder $f$ konstant, oder es gilt 
		$N(f,\Omega,w)\leq k$. \\
	\end{theorem}
	
	Die Blätterzahl kann sich also im Grenzfall nicht erhöhen. Insbesondere ist die Grenzfunktion einer Folge 
	injektiver Funktionen wieder injektiv.
	
	\begin{proof}
		Fall $f$ konstant $w$ ist, sind wir fertig. \\
		Anderfalls liegt $f^{-1}(w)$ diskret in $\Omega$, denn falls nicht, wäre $z\mapsto f(z)-w$ identisch $0$ 
		nach dem Identitätssatz für holomorphe Funktionen. Wir nehmen an, dass es $k+1$ Urbilder 
		$z_0,\ldots,z_k\in f^{-1}(w)$ gibt. Dann finden wir $r>0$ so, dass $\overline{B_r(z_i)}\subset\Omega$ für 
		alle $i$ und $\overline{B_r(z_i)}\cap\overline{B_r(z_j)}=\emptyset$ für $i\neq j$.
		
		%... Fehlende Skizze 10.7.2019
		
		Außerdem dürfen wir annehmen (wegen Diskretheit des Urbildes $f^{-1}(w)$), dass $f(z)\neq w$ für alle 
		$z\in S_r(z_i)$ und alle $i=0,\ldots,k$. Dann existiert $\varepsilon>0$, sd. $|f(z)-w|>\varepsilon$ für alle 
		$z\in S_r(z_i)$ wegen Kompaktheit der $S_r(z_i)$ und Stetigkeit von $f$. Wähle wegen kompakter 
		Konvergenz $n_0$ so, dass $|f_n(z)-f(z)|<\varepsilon$ für alle $n\geq n_0$ und alle $z\in S_r(z_i)$. 
		Aus dem Satz von Rouché folgt (mit $g=f_n -f$), dass
		\[ N(f_n,\bigcup_{i=0}^k B_r(z_i),w)=N(f_n -w, \bigcup_{i=0}^k B_r(z_i),0) 
		= \sum_{i=0}^k \underbrace{N(f-w, B_r(z_i),0)}_{\geq 1} \geq k+1 .\]
		Insgesamt erhalten wir einen Widerspruch, denn
		\[ k\geq N(f_n,\Omega_n,w)\geq N(f_n,\bigcup B_r(z_i),w)\geq k+1. \]
	\end{proof}
	
	\begin{proposition} \label{prop:beschr;hol}
		Sei $\Omega\subset\mathbb{C}$ ein Gebiet, $C,\varepsilon>0$ und $K\subset\Omega$ kompakt. Dann 
		existiert $\delta>0$, sd. für alle Funktionen $f:\Omega\to\mathbb{C}$ mit $|f(z)|<C$ für alle $z\in\Omega$ 
		und für alle $z,w\in K$ gilt:
		\[ |z-w|<\delta \qquad \Rightarrow\qquad |f(z)-f(w)|<\varepsilon .\]
	\end{proposition}
	
	\textsl{Das heißt:} Beschränkte holomorphe Funktionen auf einem vorgegebenen Gebiet $\Omega$, mit fester 
	Schranke $C$, sind auf jedem Kompaktum gleichgradig stetig.
	
	\begin{proof}
		Zu jedem $z\in\Omega$ existiert ein $r>0$, sd. $\overline{B_{2r}(z)}\subset\Omega$, da $\Omega$ offen ist. 
		Dann existiert $r>0$, sd. $\overline{B_{2r}(z)}\subset\Omega$ für alle $z\in K$. 
		
		
		%... fehlende Skizze 15.7.2019
		
		
		Falls nicht, existiert 
		eine Folge $z_n\in K$, sd. $B_{\frac{2}{n}}(z_n)\nsubseteq\Omega$. Sei $z_0$ ein Häufungspunkt und 
		$r$ zu $z_0$ wie oben, dann erhalten wir einen Widerspruch: $\frac{2}{n}<r$ für alle $n>n_0$ und 
		$|z_0 -z_n|<r$ für alle $n>n_0$ in der Teilfolge. \\
		
		
		%fehlende Skizze 15.7.2019
		
		
		Seien jetzt $z,w\in K$, $|z-w|<r$.
		\begin{eqnarray*}
			|f(z)-f(w)|&=&\left| \frac{1}{2\pi i}\int_{S_{2r}(z)}\frac{f(\zeta)\mathrm{d}\zeta}{\zeta -z} -
			\frac{1}{2\pi i} \int_{S_{2r}(z)}\frac{f(\zeta)\mathrm{d}\zeta}{\zeta -w} \right| \\
			&=& \frac{1}{2\pi i} \left| \int_{\underbrace{S_{2r}(z)}_{\text{Länge }4\pi r}}
			 \frac{\overbrace{f(\zeta)}^{|\%|\leq C}\overbrace{(z-w)}^{|\%|<\delta}\mathrm{d}\zeta}
			 {\underbrace{(\zeta -z)}_{|\%|=2r}\underbrace{(\zeta -w)}_{|\%|>r}} \right| \\
			 &<& \frac{4\pi r\cdot \delta \cdot C}{2\pi \cdot 2r \cdot r}=\underbrace{C\cdot \frac{\delta}{r}}
			 _{=\varepsilon}.
		\end{eqnarray*}
		Also wähle $\delta <r$ so, dass $\frac{C\delta}{r}\leq\varepsilon$.
	\end{proof}
	
	\begin{definition}[lokal gleichmäßig beschränkt]
		Eine Folge $f_n:\Omega\to\mathbb{C}$ heißt lokal gleichmäßig beschränkt, wenn jeder Punkt $z\in\Omega$ 
		eine Umgebung $U\subset\Omega$ ein $n_0\in\mathbb{N}$ und eine Konstante $C>0$ besitzt, sd. 
		\[ |f_n(w)|<C \]
		für alle $n\geq n_0$ und alle $w\in U$.
	\end{definition}
	
	\begin{theorem}[Montel]
		Es sei $f_n:\Omega\to\mathbb{C}$ eine lokal gleichmäßig beschränkte Folge von Funktionen. Dann existiert 
		eine auf $\Omega$ kompakt konvergente Teilfolge.
	\end{theorem}
	
	\begin{proof}
		Wähle eine Folge $z_k$ in $\Omega$, deren Bild direkt in $\Omega$ liegt, z. B., indem wir alle Punkte 
		$x+iy\in\mathbb{C}$ mit $x,y\in\mathbb{Q}$ abzählen und nun diese in $\Omega$ betrachten. \\
		Nach Voraussetzung ist für festes $k$ die Folge $(f_n(z_k))_n$ beschränkt, hat also eine konvergente 
		Teilfolge. \\
		Wähle eine konvergente Teilfolge $f_{n_1}(z_0),f_{n_2}(z_0),\ldots$ für $z_0$, nenne diese Teilfolge 
		$(f_{0,n})_n$. Dann wähle konvergente Teilfolge hiervon, $f_{0,n_1}(z_1),f_{0,n_2}(z_1),\ldots$ für $z_1$, 
		nenne sie $(f_{1,n})_n$ usw. \\
		Definiere jetzt
		\[ g_k = f_{m,m}:\Omega\to\mathbb{C},\]
		dann ist die Folge $(g_m)_m$ ab $m\geq k$ eine Teilfolge der Folge $(f_{k,n})_n$, also existiert
		\[ \lim_{n\to\infty} g_m(z_k). \]
		Das heißt, wir dürfen annehmen, dass die Folge $f_n$ auf einer dichten abzählbaren Teilmenge 
		$A\subset\Omega$ punktweise konvergiert. \\
		\textsl{Zu zeige:} Die neue Folge $f_n$ konvergiert kompakt. \\
		Sei dazu $K\subset\Omega$ kompakt und $\varepsilon >0$. Dann suchen wir ein $n_0\in\mathbb{N}$, sd. 
		$|f_{n_1}(z)-f_{n_2}(z)|<\varepsilon$ für alle $n_1,n_2\geq n_0$ und alle $z\in K$. Dann existiert eine 
		Grenzfunktion (nach Cauchy-Kriterium) und die Folge $f_n|_K$ konvergiert gleichmäßig. \\
		Zu jedem $z\in K$ wähle $U_z\subset\Omega$, $C_z>0$, $n_z$ so, dass $z\in U_z$ und 
		$|f_n(w)|<C_z$ für alle $w\in U_z$ und alle $n\geq n_z$. Da $K$ kompakt ist, reichen endlich viele $U_z$, 
		um $K$ zu überdecken. Setze also
		\begin{eqnarray*}
			U&=&U_{z_1}\cup\cdots\cup U_{z_N} \supset K, \\
			n_0&=& \max (n_{z_1},\ldots,n_{z_N}) \\
			C_0&=&\max (C_{z_1},\ldots,C_{z_N})<\infty. 
		\end{eqnarray*}
		Dann ist $f_n$ für alle $n\geq n_0$ auf $U$ durch $C$ beschränkt. Jetzt wähle $\delta>0$ zu 
		$\frac{\varepsilon}{3}$ gemäß Proposition \ref{prop:beschr;hol}, dann folgt 
		\[ |f_n(z)-f_n(w)|<\frac{\varepsilon}{3} \]
		für alle $z,w\in K$ mit $|z-w|<\delta$ für alle $n\geq n_0$. \\
		Ohne Einschränkung sei $K$ eine endliche Vereinigung abgeschlossener Bälle. Dann wir $K$ überdeckt 
		von den Bällen $B_{\delta}(a)$, für $a\in A\cap K$. Davon reichen endlich viele:
		\[ K\subset B_{\delta}(a_1)\cup \cdots\cup B_{\delta}(a_N), \qquad a_1,\ldots,a_N\in A. \]
		Also existiert $n'\geq n_0$, sd.
		\[ |f_{n_1}(a_i)-f_{n_z}(a_i)|<\frac{\varepsilon}{3} \]
		für alle $n_1,n_2\geq n'$ und alle $1\leq i\leq N$. Es folgt die Cauchy-Bedingung auf $K$: \\
		Sei $z\in K$, dann existiert $i$ mit $z\in B_{\delta}(a_i)$, also
		\[ |z-a_i|<\delta \quad \Rightarrow\quad |f_{n_1}(z)-f_{n_1}(a_i)|,\ |f_{n_2}(z)-f_{n_2}(a_i)|,\ 
		|f_{n_1}(a_i)-f_{n_2}(a_i)|<\frac{\varepsilon}{3} \]
		für alle $n_1,n_2\geq n'$. Also gilt
		\[ |f_{n_1}(z)-f_{n_2}(z)|\leq |f_{n_1}(z)-f_{n_1}(a_i)|+|f_{n_1}(a_i)-f_{n_2}(a_i)|+|f_{n_2}(z)-f_{n_2}(a_i)| < 
		\varepsilon. \]
	\end{proof}
	
	\begin{proof}
		Zu Satz \ref{thm:RA}: \\
		Wir haben bereits eine biholomorphe Abbildung $h:\Omega\to U\subset B_1(0)$. Wir betrachten 
		$f_n: U\to B_1(0)$ injektiv mit $f_n(0)=0$ und
		\[ \lim_{n\to\infty} |f_n'(0)|=\sup \{ |f'(0)|\ \vert\ f:\Omega\to B_1(0)\text{ injektiv, } f(0)=0 \} \]
		Da die Folge $f_n$ durch $1$ bescrhänkt ist, existiert eine kompakte konvergente Teilfolge. 
		Die Grenzfunktion $F$ ist wieder holomorph mit $f(0)=0$ und erfüllt
		\[ |F'(0)|=\sup \{ |f'(0)|\ \vert\ f:\Omega\to B_1(0)\text{ injektiv, } f(0)=0 \} \]
		Dazu schreiben wir
		\[F'(0)=\frac{1}{2\pi i} \int_{S_r(0)} \frac{F(\zeta)}{\zeta^2}\mathrm{d}\zeta =\lim_{n\to\infty} \frac{1}{2\pi i} 
		\int_{S_r(0)}\frac{f_n(\zeta)}{\zeta^2}\mathrm{d}\zeta = \lim_{n\to\infty} f_n'(0). \]
		Nach dem Satz von Hurwitz \ref{thm:Hur} ist $F$ injektiv. Nach dem Maximumprinzip \ref{thm:Maxprin} 
		gilt $|F(z)|<1$ für alle 
		$z\in U$. Nach Folgerung \ref{coroll:injsurj} 
		ist $F:U\to B_1(0)$ surjektiv. Also ist $F:U\to B_1(0)$ biholomorph, somit auch 
		$F\circ h:\Omega\to B_1(0)$.
	\end{proof}

	\section{Konstruktion holomorpher und meromorpher Funktionen}
	
	\subsection{Der Satz von Mittag-Leffler}
	
	\textsl{Motivation:} Für Reihen von Funktionen kombinieren wir absolute und kompakte Konvergenz.
	
	\begin{definition}[normale Konvergenz] \label{def:normKonv}
		Eine Reihe $\sum_{n=0}^{\infty}f_n$ von Funktionen konvergiert normal auf $\Omega$, wenn für alle 
		$z\in\Omega$ eine Umgebung $U$ von $z$ und eine absolut konvergente Reihe
		$\sum_{n=0}^{\infty} C_n$ reeller Zahlen existiert, sd.
		\[ |f_n(w)|<C_n \]
		für alle $w\in U$.
	\end{definition}
	
	\begin{corollary}[aus dem Satz von Weierstraß \ref{thm:Weierstr}]
		Es sei $\sum_{n=0}^{\infty}f_n$ eine auf $\Omega$ normal konvergente Reihe holomorpher Funktionen. 
		Dann ist die Grenzfunktion holomorph und ihre Ableitung wird durch die normal konvergente Reihe
		\[ \sum_{n=0}^{\infty} f_n' \]
		gegeben.
	\end{corollary}
	
	\begin{proof}
		\textsl{Zu zeigen:} die Folge der Partialsummen $\sum_{n=0}^N f_n$ konvergiert kompakt. \\
		Sei dazu $K\subset\Omega$ kompakt. Jeder Punkt $z\in K$ besitzt eine Umgebung $U_z$ wie in Definition 
		\ref{def:normKonv} und eine Reihe $\sum_{n=0}^{\infty} C_{z,n}$. Endlich viele dieser Umgebungen 
		überdecken $K$, etwa zu $z_1,\ldots,z_k$. Dann erfüllt die Reihe auf $K$ eine zu Definition 
		\ref{def:normKonv} analoge Bedinung zur reellen, abosolut konvergenten Reihe 
		\[ \sum_{n=0}^{\infty} (C_{z_1,n}+\ldots+C_{z_k,n}). \]
		Also konvergiert die Partialsummenfolge gleichmäßig auf $K$, also kompakt auf $\Omega$. \\
		\textsl{Noch zu zeigen:} $\sum_{n=0}^{\infty} f_n'$ konvergiert normal. \\
		Zu $z\in\Omega$ wähle $r>0$ so, dass $\overline{B_{2r}(z)}\subset\Omega$. Dann finde eine Reihe 
		$\sum_{n=0}^{\infty} C_n$ zu $\overline{B_{2r}(z)}$ wie oben. Für $w\in B_r(z)$ betrachte
		\[ f_n'(w)=\left|\frac{1}{2\pi i} \int_{S_{2r}(z)}\frac{f_n(\zeta)}{(\zeta-w)^2}\mathrm{d}\zeta\right| 
		< \frac{4\pi r}{2\pi}\cdot \frac{C_n}{r^2} =\frac{2C_n}{r} .\]
		Also wähle zu $z$ die Umgebung $B_r(z)$ und die reelle Reihe $\sum_{n=0}^{\infty}\frac{2C_n}{r}$, 
		um normale Konvergenz zu zeigen.
	\end{proof}
	
	\begin{example} \label{exp:5.3}
		Die Funktion $\frac{1}{\sin^2 z}$ hat bei $n\pi\in\pi\mathbb{Z}$ Singularitäten mit Hauptteil 
		$\frac{1}{(z-n\pi)^2}$, denn es gibt ein Pol zweiter Ordnung und $\frac{1}{\sin^2(z-n\pi)}$ ist gerade in $z$. \\
		\textsl{Behauptung:} 
		\begin{eqnarray}
			\sum_{n\in\mathbb{Z}}\frac{1}{(z-n\pi)^2}=\frac{1}{\sin^2 z}. \label{exp:5.3;1}
		\end{eqnarray}
		
		
		%fehlende Skizze 17.7.2019
		
		
		Die linke Seite, genauer die Partialsummen, konvergiert kompakt auf $\mathbb{C}\setminus\pi\mathbb{Z}$.
		Sei $z\notin\pi\mathbb{Z}$, ohne Einschränkung $0\leq Re\  z\leq\pi$
		\begin{eqnarray*}
			\left| \sum_{n=1}^{\infty} \frac{1}{(z-n\pi)^2} \right| &\underset{z=x+iy}{\leq}& \sum_{n=1}^{\infty} 
			\frac{1}{|x-n\pi|^2+|y|^2} \\
			&\leq& \left| \frac{1}{(z-\pi)^2}\right| + \sum_{n=2}^{\infty} \frac{1}{(n-1)^2 \pi^2}
		\end{eqnarray*}
		Genauer für $\sum_{n=0}^{-\infty}<\infty$. Zeige auch
		\[ \sum_{n\in\mathbb{Z}} \frac{1}{(x+iy-n\pi)^2} \to 0 \]
		für $y\to\pm\infty$.
		Die Funktion
		\[ \sum_{n\in\mathbb{Z}} \frac{1}{(z-n\pi)^2}-\frac{1}{\sin^2 z} \]
		ist holomorph auf $\mathbb{C}\setminus \pi\mathbb{Z}$ mit hebbaren Sinuglaritäten, periodisch, d. h. $z$ 
		und $z+\pi$ liefern den gleichen Wert, ist also beschränkt mit Grenzwert $0$ für $Im\ z\to\pm\infty$ und somit 
		konstant $0$ nach dem Satz von Liouville \ref{thm:LV}. \\
		\textsl{Noch zu zeigen:} Die rechte Seite konvergiert gleichmäßig gegen $0$ für $Im\ z\to\pm\infty$. \\
		Dazu sei $z=x+iy$, ohne Einschränkung $x\in [0,\pi]$ (beide Seiten von \ref{exp:5.3;1} ergeben für $z$ 
		und $z+k\pi$ den gleichen Wert). 
		
		
		%fehlende Skizze 17.7.2019
		
		
		Für $n\geq 1$ gilt
		\[ |(x-n\pi)+iy| \geq \frac{1}{2}(|y|+(n-1)\pi) \]
		oder für $n\leq 0$ analog.
		\[ \sum_{n\in\mathbb{Z}}\frac{1}{|z-\pi n|^2} \leq 4\sum_{n=1}^{\infty} \frac{1}{(|y|+(n-1)\pi)^2}+4
		\sum_{n=0}^{\infty} \frac{1}{(|y|+n\pi)^2} \leq 8 \sum_{n=k}^{\infty} \frac{1}{n^2\pi^2} \]
		für $k\leq \frac{|y|}{\pi}$, $k\in\mathbb{Z}$. Da die Reihe $\sum \frac{1}{n^2 \pi^2}$ absolut konvergiert, 
		konvergiert
		\[ \sum_{n=k}^{\infty} \frac{1}{n^2\pi^2} \to 0 \]
		für $k\to\infty$, also für $|y|\to\infty$. 
		\[ \sum_{i=0}^{\infty} \frac{1}{\underbrace{(2i+1)^2}_{1+2\mathbb{Z}}} =\frac{\pi^2}{8}
		\sum_{n\in\mathbb{Z}}^{\infty} \frac{1}{\underbrace{(\frac{\pi}{2}-\pi n)^2}_{\in \frac{\pi}{2}+\pi\mathbb{Z}}}
		=\frac{\pi^2}{8}\cdot \frac{1}{\sin^2 \frac{\pi}{2}}=\frac{\pi^2}{8}. \]
		für $z=\frac{\pi}{2}$.
	\end{example}
	
	\begin{theorem}[Mittag-Leffler]
		Es sei $A\subset\mathbb{C}$ eine diskrete Menge von Punkten. Für 
		alle $a\in A$ sei $f_a:\mathbb{C}\to\mathbb{C}$ holomorph mit $f_a(0)=0$. Dann existiert eine 
		holomorphe Funktion $f:\mathbb{C}\setminus A\to \mathbb{C}$, sd. für alle $a\in A$ die Funktion
		\[ f(z) -f_a\left(\frac{1}{z-a}\right) \]
		bei $a$ eine hebbare Singularität hat.
	\end{theorem}
	
	\textsl{$"$diskret$"$ heißt:} $"$ohne Häufungspunkte$"$. \\
	Es gelte
	\[ f_a(w)=\sum_{n=1}^{\infty} b_n w^n \]
	(beachte: $f_a(0)=0$). Dann soll also $f_n$ bei $a$ eine Singularität mit Hauptteil
	\[ \sum_{n=1}^{\infty} \frac{b_n}{(z-a)^n} \]
	haben. In Beispiel \ref{exp:5.3} ist $A=\pi\mathbb{Z}$ und $f_a(w)=w^2$.
	
	\begin{proof}
		\textsl{Idee:} addiere die Funktionen $f_a\left(\frac{1}{z-a}\right)$ auf. Um Konvergenz zu gewährleisten, ziehen 
		die ein geeignetes Taylorpolynom ab. Das ändert nichts an den Hauptteilen. Da $A$ keine Häufungspunkte 
		hat, dürfen wir dem Betrag nach sortieren:
		\[ A=\{ a_0,a_1,\ldots\} \]
		mit $|a_0|,\leq|a_1|\leq \ldots$ und $\lim_{k\to\infty}|a_k|=\infty$. \\
		Für $a_k\neq 0$ schreibe
		\[ f_{a_k} \left(\frac{1}{z-a_k}\right) =\sum_{n=0}^{\infty} C_{k,n}\cdot z^n \]
		da die Funktion bei $z=0$ holomorph ist. Diese Entwicklung hat Konvergenzradius $|a_k|$. Also konvergiert 
		die Reihe gleichmäßig auf $B_{\frac{|a_k|}{2}}(0)$ und wir wählen $n_k$ so, dass
		\[ \left| f_{a_k}\left(\frac{1}{z-a_k}\right) -\underbrace{\sum_{n=0}^{n_k} C_{k,n} z^n}_{T_k(z) \text{( ein Polynom 
		in $z$)}}\right| < 2^{-k} \]
		Setze
		\[ f(z)=f_{a_0}\left(\frac{1}{z-a_0}\right) +\sum_{k=1}^{\infty}\left(f_{a_k}\left(\frac{1}{z-a_k}\right) -T_k(z)\right) \]
		\textsl{Behauptung:} Diese Reihe konvergiert normal auf $\mathbb{C}\setminus A$. \\
		Dazu betrachte für $R>0$ den Ball $\overline{B_R(0)}\setminus A$. Da $a_k|\to\infty$, existiert $k_0$, sd. 
		$|a_k|>2R$ für alle $k\geq k_0$. Es folgt 
		\[ \sum_{k=k_0}^{\infty}\underbrace{\left(f_{a_k}\left(\frac{1}{z-a_k}\right)-T_k(z)\right)}_{|\%|<2^{-k}}
		< 2^{1-k_0} \]
		auf ganz $\overline{B_R(0)}$. Die vorderen Summanden, die wir weggelassen haben, ändern nichts 
		an der Konvergenz und liefern die gesuchten Singularitäten auf $\overline{B_R(0)}$.
		
		
		%... fehlende Skizze 17.7.2019
		
		
	\end{proof}
	
	In der Praxis versucht man die Polynome $T_n$ weniger großzügig zu wählen. (siehe Beispiel \ref{exp:5.3})
	
	\begin{example}
		Wir möchten Singularitäten der Form $\frac{1}{z-n\pi}$ für alle $n\in\mathbb{Z}$.
		
		
		%... fehlende Skizze 17.7.2019
		
		
		\[ \frac{1}{z}+\sum_{n\neq 0} \underbrace{\left(\frac{1}{z-n\pi}+\frac{1}{n\pi}\right)}_{\frac{n\pi +z-n\pi}
		{(z-n\pi)n\pi}=\frac{z}{(z-n\pi)n\pi}} 
		= \frac{1}{z}+\sum_{n=1}^{\infty} \underbrace{\left( \frac{1}{z-n\pi}+\frac{1}{z+n\pi}\right)}_
		{\frac{2z}{z^2 -n^2\pi^2}} =\cot z. \]
		Beide Reihen konvergieren normal. \\
		\textsl{Denn:} Ableiten liefert die Reihe aus Beispiel \ref{exp:5.3} (bis auf Vorzeichen) und
		\[ \cot' =\frac{-\sin^2 -\cos^2}{\sin^2}=-\frac{1}{\sin^2}.\]
		Außerdem sind sowohl $\cot$ als auch die Reihe ungerade (d. h. $f(-z)=-f(z)$).
	\end{example}
	
	\subsection{Die $\Gamma$-Funktion}
	
	\textsl{Ziel:} Interpoliere $n!$ (genauer: $(n-1)!$)
	
	\begin{definition}
		Für $Re\ z>0$ definieren wir
		\[ \Gamma(z) =\int_0^{\infty} \underbrace{t^{z-1}}_{e^{(z-1)\log t}}e^{-t}\mathrm{d}t \]
		Für $Re\ z>0$ ist $|t^{z-1}|=|t^{Re\ z-1}|$ noch integrierbar bei $t\to 0$. Diese Funktion ist holomorph, da 
		wir die Ableitung ins Integral hineinziehen dürfen.
	\end{definition}
	
	\begin{theorem} \label{thm:Gamma}
		Die $\Gamma$-Funktion besitzt eine eindeutige meromorphe Fortsetzung auf ganz $\mathbb{C}$ mit Polen 
		$z=-n$, $n\in\mathbb{N}_0$. Für $z\in\mathbb{C}\setminus (-\mathbb{N}_0)$ gilt 
		\[ \Gamma(z+1)=z\cdot \Gamma(z). \]
		Für $0\neq n\in\mathbb{N}$ gilt $\Gamma(n)=(n-1)!$. \\
		(Zur Erinnerung: meromorph $\Leftrightarrow$ keine wesentlichen Singularitäten)
	\end{theorem}
	
	\begin{proof}
		Für $Re\ z>1$ gilt mit partieller Integration:
		\begin{eqnarray*}
			\Gamma(z)&=&\int_0^{\infty} t^{z-1}\underbrace{e^{-t}}_
			{-\frac{\mathrm{d}}{\mathrm{d}t}(e^{-t})}\mathrm{d}t \\
			&=& -\underbrace{\left.\left( t^{z-1}e^{-t}\right)\right\vert_{t=0}^{\infty}}_{\genfrac{}{}{0pt}{1}{\text{$\to0$ für 
			$t\to0$}}{\text{$\to0$ für 
			$t\to\infty$}}}
			+(z-1)\int_0^{\infty} t^{z-2}e^{-t}\mathrm{d}t \\
			&=&(z-1)\Gamma(z-1)
		\end{eqnarray*}
		Benutze diese Gleichung, um $\Gamma$ induktiv auszudehnen auf 
		\[ \{ z\in\mathbb{C} \ \vert\ Re(z)>-n\text{ \& }-z\notin\mathbb{N}_0 \} \]
		durch 
		\begin{eqnarray}
			\Gamma(z-1)=\frac{\Gamma(z)}{z-1}. \label{proof;Gamma1}
		\end{eqnarray}
		Dabei benutzen wir den Eindeutigkeitssatz, um \ref{proof;Gamma1} auf dem jeweiligen Definitionsbereich 
		zu zeigen.
		
		
		%... fehlende Skizze 22.7.2019
		
		
		Außerdem gilt
		\[ \Gamma(1)=\int_0^{\infty}e^{-t}\mathrm{d}t=-\left.(e^{-t})\right\vert_{t=0}^{\infty} =1 =0! \]
		Für $n\in\mathbb{N}$ folgt induktiv:
		\[ \Gamma(n+1)=n\cdot \Gamma(n)= n\cdot (n-1)! = n! \]
		Für $z\in -\mathbb{N}_0$ erhalten wir Polstellen:
		\[ z\cdot \Gamma(z) = \underbrace{\Gamma(z+1)}_{=1 \text{ bei }z=0} \quad \text{ nahe }z=0.\]
	\end{proof}
	
	Wir schauen uns die Pole noch genauer an.
	
	\begin{corollary}[Zerlegung nach Prym] \label{coroll:Prym}
		Für alle $z\in\mathbb{C}$ gilt
		\[ \Gamma(z) =\sum_{n=0}^{\infty} \underbrace{\frac{(-1)^n}{n!(z+n)}}_
		{\genfrac{}{}{0pt}{1}{\text{Polstelle bei $z=-n$ von erster}}
		{\text{Ordnung mit Residuum $\frac{(-1)^n}{n!}$}}} +\underbrace{\int_1^{\infty} t^{z-1}e^{-t}\mathrm{d}t}
		_{\text{holom. auf ganz $\mathbb{C}$}}. \]
	\end{corollary}
	
	\begin{proof}
		Für $Re\ z>0$ berechnen wir
		\begin{eqnarray*}
			\int_0^1 t^{z-1}e^{-t}\mathrm{d}t &=& \int_0^1 t^{z-1} \left( \sum_{n=0}^{\infty} (-1)^n \cdot 
			\frac{ t^n}{n!}\right)\mathrm{d}t \\
			&=& \sum_{n=0}^{\infty} \int_0^1 t^{z+n-1} \cdot \frac{(-1)^n}{n!}\mathrm{d}t \\
			&=& \sum_{n=0}^{\infty} \frac{(-1)^n}{n!} \underbrace{\left.\left(\frac{t^{z+n}}{z+n}\right)\right\vert_{t=0}^1}_
			{\genfrac{}{}{0pt}{1}{\text{$\to 0$ für $t\to 0$}}{\text{$=\frac{1}{z+n}$ für $ t=1$}}} 
			= \sum_{n=0}^{\infty} \frac{(-1)^n}{n!(z+n)}.
		\end{eqnarray*}
	\end{proof}
	
	\begin{theorem}[Wielandt]
		Es sei $\Omega\subset\mathbb{C}$ ein Gebiet mit $[1,2)\times\mathbb{R}\subset\Omega$. Es sei 
		$f:\Omega\to\mathbb{C}$ beschränkt und holomorph mit
		\begin{eqnarray}
			f(z+1)=z\cdot f(z) \label{thm:Wielandt;1}
		\end{eqnarray}
		für alle $z\in\Omega$ mit $z+1\in\Omega$. Dann gilt 
		\[ f(z)=f(1)\cdot \Gamma(z) \]
		auf ganz $\Omega$. \\
		Wir überprüfen Beschränktheit von $\Gamma$ für $z=x+iy$, $x\in [1,2)$:
		\begin{eqnarray*}
			|\Gamma(z)| &=&\left| \int_0^{\infty} e^{(x+iy-1)\log t}\cdot e^{-t}\mathrm{d}t \right| \\
			&\leq& \int_0^{\infty} \left| e^{(x-1)\log t}\right|\cdot\underbrace{\left| e^{iy\cdot \log t} \right|}_{=1}\cdot e^{-t}
			\mathrm{d}t \\
			&\leq& \underbrace{ \int_0^1 \underbrace{ t^{x-1}e^{-t}}_{\leq 1} \mathrm{d}t}_{\leq 1} +
			\underbrace{ \int_1^{\infty} \underbrace{ t^{x-1}}_{\leq t}e^{-t}\mathrm{d}t}_{< C<\infty}
		\end{eqnarray*}
		
		
		%... fehlende Skizze 22.7.2019
		
		
	\end{theorem}
	
	\begin{proof}
		Betrachte auf $\Omega$ die Funktion
		\[ h(z)=f(z)-f(1)\Gamma(z). \]
		Sie ist holomorph, beschränkt auf $[1,2)\times\mathbb{R}$ und erfüllt \ref{thm:Wielandt;1}.
		Jetzt betrachte
		\[ g(z)=h(z)h(n-z) \]
		Dann gilt
		\begin{eqnarray}
			g(z+1)=h(z+1)h(n-1-z)=z\cdot h(z)\cdot \frac{h(n-z)}{n-1-z} \label{thm:Wielandt;2}
		\end{eqnarray}
		Wähle also $n=1$ dann folgt
		\[\ldots =- h(z)\cdot h(1-z)=-g(z) \]
		Wie im Beweis von Satz \ref{thm:Gamma} dehnen wir $f$ und $h$ aus auf den Bereich 
		$Re\ z\geq 0$, mit Pol bei $z=0$. Dann können wir $g$ auf einem Gebiet definieren, das $[0,1]\times 
		\mathbb{R}\subset\mathbb{C}$ enthält, mit eventuellen Polen bei $z=0,1$. \\
		\textsl{Aber:} es gilt
		\[ h(1)=f(1)-f(1)\Gamma(1)=0 \]
		und somit hat $h$ auch bei $0$ einen Pol, denn
		\[ h(z)=0+a_1\cdot (z-1)+0((z-1)^2), \]
		also ist $h(0)=a_1$ beschränkt. Dann hat auch $g$ keinen Pol. Wegen Kompaktheit ist $g$ also beschränkt 
		auf $[0,1]\times[-1,1]\subset\mathbb{C}$
		
		
		%... fehlende Skizze 22.7.2019
		
		
		Für $z\in [1,2)\times\mathbb{R}$ ist $f$ beschränkt durch $C$. Für $z\in [0,1]\times[1,\infty)$ folgt 
		\[ |f(z)|=\frac{|f(z+1)|}{\underbrace{|z|}_{>1}} \leq \frac{C}{1}=C \]
		Analog für $z\in [0,1)\times (-\infty,-1]$, dito für $\Gamma$. Somit ist $h$ auf $[0,1]\times\mathbb{R}$ 
		beschränkt. Da $1-z\in [0,1]\times\mathbb{R}$ für alle $z\in [0,1]\times\mathbb{R}$, ist 
		\begin{eqnarray}
			g(z)=h(z)h(1-z) \label{thm:Wielandt;3}
		\end{eqnarray}
		ebenfalls auf $[0,1]\times\mathbb{R}$ beschränkt. Wegen \ref{thm:Wielandt;2} ist $g$ 
		auf ganz $\mathbb{C}$ beschränkt und somit nach Liouville \ref{thm:LV} konstant.
		Es folgt
		\[ g(1)=\underbrace{h(1)}_{0}\cdot \underbrace{h(0)}_{\in\mathbb{C}} =0. \]
		Aus $g=0$ folgt $h(z)=0$ oder $h(1-z)=0$, also $h=0=f-f(1)\cdot\Gamma$.
	\end{proof}
	
	Was wäre, wenn wir den Trick \ref{thm:Wielandt;3} direkt auf $\Gamma$ anwenden? \\
	Setze also
	\[ f(z)=\Gamma(z)\Gamma(1-z), \]
	dann hat $f$ Pole an allen $z\in \mathbb{Z}$.
	
	\begin{theorem}[Ergänzungssatz von Euler]
		Es gilt auf ganz $\mathbb{C}$:
		\[ \Gamma(z)\Gamma(1-z) =\frac{\pi}{\sin(\pi z)}. \]
	\end{theorem}
	
	\begin{proof}
		Die Funktion $f(z)=\Gamma(z)\Gamma(1-z)$ erfüllt
		\[ f(z+1)=-f(z) \]
		wie \ref{thm:Wielandt;2}. Nahe $z=0$ gilt nach Prym \ref{coroll:Prym} und Satz \ref{thm:Gamma}:
		\[ f(z) =\left( \frac{(-1)^0}{0!z}+\mathcal{O}(z^0)\right)\cdot \underbrace{\Gamma(1-z)}_{(1+\mathcal{O}(z))} 
		=\frac{1}{z} +\mathcal{O}(z^0). \]
		Das gleiche gilt für
		\[ \frac{\pi}{\sin(\pi z)}=\frac{ \pi}{(\pi z)}\cdot \underbrace{(1+\mathcal{O}(z^2))}_{\text{holomorph}} 
		= \frac{1}{z} +\mathcal{O}(z^0). \]
		Somit ist
		\[ g(z)=\Gamma(z)\Gamma(1-z)-\frac{\pi}{\sin(\pi z)} \]
		nahe $z=0$ beschränkt, hat also eine hebbare Singularität. Da $g(z+1)=-g(z)$ gilt, hat $g$ dann überhaupt 
		keine Pole. Daher ist $g$ auf $[0,1]\times[-1,1]\subset\mathbb{C}$ beschränkt, wegen 
    		\ref{thm:Wielandt;2} also auf $\mathbb{R}\times [-1,1]$.
		
		
		%... fehlende Skizze 22.7.2019
		
		
		Aus dem letzten Beweis wissen wir, dass $f(z)$ auch für $|Im\ z|\geq 1$ gleichmäßig beschränkt ist. 
		Wegen
		\begin{eqnarray*}
			\sin(x+iy)&=&\sin x\cdot \underbrace{\cos(iy)}_{\cosh(-y)} +\cos x\cdot \sin(iy) \\
			&=& \sin x\cdots \underbrace{\cosh y}_{\to \infty \text{ für }y\to \infty} -i\cos x \underbrace{\sinh y}_
			{\to \infty \text{ für }y\to \infty}
		\end{eqnarray*}
		Wegen $\sin^2 +\cos^2 =1$ ist mindestens $\sin x \geq \frac{1}{\sqrt{2}}$ oder $\cos x\geq\frac{1}{\sqrt{2}}$, 
		also $|\sin z|\to\infty$ für $y\to\infty$, d. h. $\frac{\pi}{\sin(\pi z)}$ ist beschränkt für $|Im\ z|\geq 1$. 
		Also ist $g$ auf ganz $\mathbb{C}$ beschränkt also konstant. Außerdem gilt \ref{thm:Wielandt;2}, also ist 
		$g=0$.
	\end{proof}
	
	Setze $z=\frac{1}{2}$ ein:
	\[ \Gamma\left(\frac{1}{2}\right)^2 = \frac{\pi}{\sin \frac{\pi}{2}}=\pi \quad \Rightarrow \quad \Gamma\left(\frac{1}
	{2}\right)=\sqrt{\pi}.\]
	\begin{eqnarray*}
		\Gamma\left(\frac{1}{2}\right)^2 &=& \left( \int_0^{\infty} t^{-\frac{1}{2}}e^{-t}\mathrm{d}t \right)^2 
		=\left(\int_0^{\infty} 2e^{-x^2}\mathrm{d}x\right)^2 \\
		&=& \left( \int_{-\infty}^{\infty}e^{-x^2}\mathrm{d}x\right)^2 = \int_{\mathbb{R}^2} e^{-x^2-y^2}\mathrm{d}x
		\mathrm{d}y \\
		&=& \int_0^{2\pi}\underbrace{\int_0^{\infty}r e^{-r^2} \mathrm{d}r}_{\frac{1}{2}}\mathrm{d}\varphi =\pi.
	\end{eqnarray*}
	
	
\end{document}